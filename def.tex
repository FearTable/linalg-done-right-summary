
% BEGIN custom geometry ************************************
\setlength{\abovedisplayskip}{0pt}
\setlength{\belowdisplayskip}{0pt}
\setlength{\abovedisplayshortskip}{0pt}
\setlength{\belowdisplayshortskip}{0pt}
\setlength{\jot}{0pt}

\titlespacing{\chapter}{0pt}{6pt}{3pt}
\titlespacing{\section}{0pt}{25pt}{3pt}
\titlespacing{\subsection}{0pt}{25pt}{3pt}
\titlespacing{\subsubsection}{0pt}{6pt}{3pt}
\titlespacing*{\chapter}{0pt}{-19pt}{0pt}
% END custom geometry **************************************


% BEGIN custom therem and proof environments ***************
\newtheoremstyle{mytheoremstyle} % name
{0.8em}                       % Space above {\topsep} 
{.2em}                        % Space below
{}                            % Body font
{0em}                         % Indent amount
{}                            % Theorem head font {\ttfamily\fontseries{b}\selectfont}
{{\textbf{:} \newline}}       % Punctuation after theorem head
{.3em}                        % Space after theorem head
{{\textbf{\scshape{\thmname{#1}\thmnumber{ #2}}}{\normalfont \thmnote{ \itshape (#3)}}}} 
                              % Theorem head spec (can be left empty, meaning ‘normal’)

% Define 'thm', 'thm-non', 'mydef', 'mydef-non', 'example', 'example-non'
% environments
\theoremstyle{mytheoremstyle}
\newtheorem{thm}{Theorem}[chapter]
\newtheorem*{thm-non}{Theorem}
\theoremstyle{mytheoremstyle}
\newtheorem{mydef}[thm]{Definition}
\newtheorem*{mydef-non}{Definition}
\theoremstyle{mytheoremstyle}
\newtheorem{example}[thm]{Example}
\newtheorem*{example-non}{Example}

\newtheoremstyle{indented-prf}
{.2em}        % space before
{0em}         % space after
{
 \setlength{\leftskip}{\mathindent}
 \setlength{\parindent}{0pt}
 \addtolength{\linewidth}{-\mathindent-3em}
 \setlist[enumerate]{align=parleft,left=\mathindent..\mathindent+\mathindent-30pt}
 \setlist[description]{leftmargin=\mathindent,labelindent=\mathindent-6pt-27pt} 
 \addtolength{\mathindent}{\mathindent}
}
{0em}         % indent
{\bfseries}   % header font
{\textbf{:}}  % punctuation
{.3em}        % after theorem header
{{\textbf{\scshape{\thmname{#1}\thmnumber{ #2}}}{\normalfont \thmnote{ (#3)}}}}

% define 'prf'-environment
\theoremstyle{indented-prf}
\newtheorem*{prf}{Proof}

\def\endprf{
  {\setlength{\leftskip}{\mathindent}\setlength{\rightskip}{3em}\setlength{\parindent}{0pt}\addtolength{\linewidth}{-\mathindent-3em}\setlist[enumerate]{align=parleft,left=\mathindent..\mathindent+\mathindent-30pt}\setlist[description]{leftmargin=\mathindent,labelindent=\mathindent-6pt-27pt} \addtolength{\mathindent}{\mathindent} \hfill$\blacksquare$}
}

\renewenvironment{proof}
{{\textbf{Proof:}}}
{\hfill $\Box$}
% END custom proof and thereom environments ****************


% BEGIN custom commands ************************************
%\DeclarePairedDelimiter\norm{\lVert}{\rVert}
%\DeclarePairedDelimiter\innerproduct{\langle}{\rangle}
\newcommand{\ip}[2]{\left \langle #1, #2 \right \rangle}
\newcommand{\norm}[1]{\left \lVert #1 \right \rVert}
\newcommand{\gt}{>}
\newcommand{\lt}{<}

% May be obsolete since you can use '$' inside equation environment
\newcommand\mytext[1]{\;\;\, \text{#1}\;\;}
\newcommand\myand{\;\;\, \text{and}\;\;}
\newcommand\myor{\;\;\, \text{or}\;\;}
\newcommand\where{\;\;\, \text{where}\;\;}
\newcommand\whereEach{\;\;\, \text{where each}\;\;}
\newcommand\dx{\, dx}
\newcommand\dy{\, dy}
\newcommand\dz{\, dz}
\newcommand\du{\, du}

% BFEMPH
\newcommand{\bfemph}[1]{{\scshape #1}}

% natural numbers, integers, real numbers, complex numbers:
\newcommand{\nat}{\mathbb{N}}
\newcommand{\integer}{\mathbb{N}}
\newcommand{\real}{\mathbb{R}}
\newcommand{\compl}{\mathbb{C}}
\newcommand{\myF}{\mathbb{F}}

% polynomial symbol:
\newcommand{\polyn}{\mathcal{P}}

% degree operator:
\def\degree{{\operatorname{deg} \,}}

% Matrix symbol M:
\newcommand{\mmatrix}{\mathcal{M}}

% use \span{..} for span(..)
\newcommand{\myspan}[1]{\operatorname{span} (#1)}

%Fast way to write v_1 ... v_n
\newcommand{\oneTillN}[1]{#1_1, \dots, #1_n}
\newcommand{\onetilln}[1]{#1_1, \dots, #1_n}

%Fast way to write v_1 ... v_m
\newcommand{\oneTillM}[1]{#1_1, \dots, #1_m}
\newcommand{\onetillm}[1]{#1_1, \dots, #1_m}

% fast way to write v_1 ... v_{#2}
% usage \onetill{v}{k-1} yields v_1 \dots v_{k-1}
\newcommand{\oneTill}[2]{#1_1, \dots, #1_{#2}}
\newcommand{\onetill}[2]{#1_1, \dots, #1_{#2}}

\newcommand{\kInOneTillM}{k \in \{1, \dots, m \}} % k e {1...m}
\newcommand{\kinonetillm}{k \in \{1, \dots, m \}} % k e {1...m}
\newcommand{\kInOneTillN}{k \in \{1, \dots, n \}} % k e {1...n}
\newcommand{\kinonetilln}{k \in \{1, \dots, n \}} % k e {1...n}
\newcommand{\kInOneTillP}{k \in \{1, \dots, p \}} % k e {1...p}
\newcommand{\kinonetillp}{k \in \{1, \dots, p \}} % k e {1...p}

% abbreviation for finite-dimensional vector space
\newcommand{\findimvecpac}{finite-dimensional vector space }
\newcommand{\findimvs}{finite-dimensional vector space }
\newcommand{\fdvs}{finite-dimensional vector space }

% abbreviation for linearly independent
\newcommand{\lid}{linearly independent}

% abbreviation for linearly independent
\newcommand{\ld}{linearly dependent }

% abbreviation for linearly independent
\newcommand{\vs}{vector space }
 
% abbreviation for finite-dimensional
\newcommand{\fd}{{finite-dimensional }}

% abbreviation for linear map
\newcommand{\lm}{{linear map }}

% abbreaviation for L(V,W)
\newcommand{\lvw}{{\mathcal{L}(V,W)}}

% use \linmap for L or \lin{arg1}{arg2} for L(arg1, arg2)
\newcommand{\linmap}{\mathcal{L}}
\newcommand{\lin}[2]{{\mathcal{L}(#1, #2)}}

% null and range
\newcommand{\mynull}{\operatorname{null}}
\newcommand{\myrange}{\operatorname{range}}

% even and odd
\newcommand{\even}{\operatorname{even}}
\newcommand{\odd}{\operatorname{odd}}

% legacy

% \newcommand{\basis}[2]{\overbrace{ \myspan{#1_1, \dots #1_{#2}}}^{\text{linearly independent}} }}
\def\myimpl{{\{black}{\implies}}

% END custom comands ***************************************

% BEGIN custom chapter format ******************************
\renewcommand*{\thechapter}{\arabic{chapter}}
\renewcommand*{\thesection}{\arabic{chapter}\Alph{section}}
\renewcommand*{\thesubsection}{\arabic{chapter}\Alph{section}\arabic{subsection}}

\titleformat{\chapter}[block]
{\bfseries\scshape}{{\relscale{1.1} chapter} {\relscale{1.1} \thechapter \newline} { } { }  \relscale{1.3} #1}{0.5em}{\selectfont\large}
\titleformat{\section}{\bfseries\relscale{1.2}\scshape}{}{0em}{\thesection { } #1} %Large
\titleformat{\subsection}{\relscale{1.1}\bfseries\scshape}{}{0em}{#1 (\thesubsection)} %large
% END custom chapter format ********************************