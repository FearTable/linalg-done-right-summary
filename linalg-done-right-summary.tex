\documentclass[11pt]{report}
%\usepackage[fleqn]{amsmath}
\usepackage[fleqn]{amsmath}
\usepackage{amssymb}
\usepackage{xcolor}

\usepackage[dvipsnames]{xcolor}
\usepackage[skip = 2pt, parfill]{parskip}
\usepackage{blkarray}
\usepackage{blindtext}
\usepackage{geometry}
\usepackage[explicit]{titlesec} %"big, medium, small, tiny".
\usepackage{xfakebold}
\usepackage{amsthm}
\usepackage{lmodern}
\usepackage{enumitem}
\usepackage{ragged2e}
\usepackage{kpfonts}
\usepackage{lipsum}
\usepackage{xfrac}
\usepackage{breqn}
\usepackage[colorlinks=false]{hyperref}
\usepackage{xr-hyper}
%\usepackage[fontsize=14pt]{scrextend}
\usepackage{relsize}
\usepackage{enumitem}
\setlist{nosep}

\externaldocument{../linalg-done-right-summary/linalgDoneRight}

\definecolor{darkteal}{RGB}{0, 41, 59}

%o\everymath{\color{darkteal}}
%\everydisplay{\color{darkteal}}
%\everydisplay{\color{Melon}}

%\hypersetup{
%	colorlinks,
%	linkcolor={red!50!black},
%	citecolor={blue!50!black},
%	urlcolor={blue!80!black}
%}

\hypersetup{
	colorlinks,
	linkcolor={black},
	citecolor={black},
	urlcolor={black}
}

\geometry{
	a4paper,
	total={170mm,257mm},
	left=17mm,
	right=17mm,
	top=15mm,
}

%\setcounter{secnumdepth}{0}
%\renewenvironment{proof}{\underline{\ttfamily Proof:}}{\hfill $\Box$ \\}
\renewenvironment{proof}
  {
    {
      \relscale{1.1}
      \bfseries
      \scshape
      \itshape
      \selectfont
      Proof:}
  }
  {
    \hfill $\Box$ \\
  }

%\titlespacing{\chapter}{0pt}{6pt}{3pt}
%\titlespacing{\section}{0pt}{10pt}{3pt}
%\titlespacing{\subsection}{0pt}{13pt}{3pt}
%\titlespacing{\subsubsection}{0pt}{6pt}{3pt}

\titlespacing{\chapter}{0pt}{6pt}{3pt}
\titlespacing{\section}{0pt}{25pt}{3pt}
\titlespacing{\subsection}{0pt}{25pt}{3pt}
\titlespacing{\subsubsection}{0pt}{6pt}{3pt}


%\renewcommand*{\thechapter}{Chapter \arabic{chapter}}
\renewcommand*{\thechapter}{\arabic{chapter}}
%\renewcommand*{\thesection}{\Roman{section}}
\renewcommand*{\thesection}{\arabic{chapter}\Alph{section}}
\renewcommand*{\thesubsection}{\arabic{chapter}\Alph{section}\arabic{subsection}}

%\titleformat*{\chapter}{\ttfamily\fontseries{b}\selectfont\large}
\titleformat{\chapter}[block]
{\bfseries\relscale{1.4}\scshape\selectfont}{\thechapter { } { } #1}{0.5em}{\selectfont\large}


\titlespacing*{\chapter}{0pt}{-19pt}{0pt}


% Old titleformat:
%\titleformat*{\section}{\fontseries{b}\selectfont\large}
%\titleformat*{\subsection}{\fontseries{b}\selectfont\normalsize}
%\titleformat*{\subsubsection}{\fontseries{b}\selectfont\normalsize}
%\titleformat*{\subsubsection}{\fontseries{b}\selectfont\normalsize}

\titleformat{\section}{\bfseries\relscale{1.3}\scshape\selectfont}{}{0em}{\thesection { } #1} %Large

\titleformat{\subsection}{\bfseries\relscale{1.2}\scshape\itshape\selectfont}{}{0em}{#1\  { } (\thesubsection)} %large

\titleformat{\subsubsection}{\relscale{1.1}\selectfont}{}{0em}{#1\  \thesection}


%\titlespacing{\chapter}{0pt}{\parskip}{-\parskip}
%\titlespacing{\section}{0pt}{\parskip}{-\parskip}
%\titlespacing{\subsection}{0pt}{\parskip}{-\parskip}
%\titlespacing{\subsubsection}{0pt}{\parskip}{-\parskip}

%\usepackage[skip=2pt plus1pt, indent=0pt]{parskip}
% \usepackage[skip=10pt plus1pt, indent=40pt]{parskip}
%\usepackage[small]{titlesec} % "big, medium, small, tiny".

\setlength{\abovedisplayskip}{0pt}
\setlength{\belowdisplayskip}{0pt}

\begin{document}

  %\raggedright
	
	%\linespread{1.2}
	
  \newcommand{\gt}{>}
\newcommand{\lt}{<}

%\def\deltaLow{{ \color{purple}{ \delta_{\text{low}} }}}
%\def\deltaSup{{ \color{purple}{ \delta_{\text{sup}} }}}
%\def\deltaHigh{{ \color{purple}{ \delta_{\text{high}} }}}
%\def\deltaVanilla{{ \color{purple}{ \delta }}}
%
%\def\tHigh{{ \color{olive}{ t_{\text{high}} }}}
%\def\tLow{{ \color{olive}{ t_{\text{low}} }}}
%
%\def\supremum{{ \color{RoyalBlue}{ \tilde s }}}
%\def\SSet{{ \color{RoyalBlue}{ S }}}
%
%\def\myEpsilon{{ \color{black}{\epsilon }}}
%\def\myspan{{ \color{black}{\text{span} }}}




\def\degree{{\operatorname{deg} \,}}

\newcommand{\myspan}[1]{\operatorname{span} (#1)}

%Fast way to write v_1 ... v_n
\newcommand{\oneTillN}[1]{#1_1, \dots, #1_n}
\newcommand{\onetilln}[1]{#1_1, \dots, #1_n}

%Fast way to write v_1 ... v_m
\newcommand{\oneTillM}[1]{#1_1, \dots, #1_m}
\newcommand{\onetillm}[1]{#1_1, \dots, #1_m}

% fast way to write v_1 ... v_{#2}
% usage \onetill{v}{k-1} yields v_1 \dots v_{k-1}
\newcommand{\oneTill}[2]{#1_1, \dots, #1_{#2}}
\newcommand{\onetill}[2]{#1_1, \dots, #1_{#2}}

\newcommand{\kInOneTillM}{k \in \{1, \dots, m \}}
\newcommand{\kinonetillm}{k \in \{1, \dots, m \}}
\newcommand{\kInOneTillN}{k \in \{1, \dots, n \}}
\newcommand{\kinonetilln}{k \in \{1, \dots, n \}}
\newcommand{\kInOneTillP}{k \in \{1, \dots, p \}}
\newcommand{\kinonetillp}{k \in \{1, \dots, p \}}

% abreviation for finite-dimensional vector space
\newcommand{\findimvecpac}{finite-dimensional vector space }
\newcommand{\findimvs}{finite-dimensional vector space }
\newcommand{\fdvs}{finite-dimensional vector space }

%abreviation for linearly independent
\newcommand{\lid}{linearly independent }

%abreviation for linearly independent
\newcommand{\ld}{linearly dependent }

%abreviation for linearly independent
\newcommand{\vs}{vector space }

%abreviation for finite-dimensional
\newcommand{\fd}{{finite-dimensional }}	

%abreviation for linear map
\newcommand{\lm}{{linear map }}	

%abbreaviation for L(V,W)
\newcommand{\lvw}{{\mathcal{L}(V,W)}}

\newcommand{\linmap}{\mathcal{L}}
\newcommand{\lin}[2]{{\mathcal{L}(#1, #2)}}

\newcommand{\mynull}{\operatorname{null}}

\newcommand{\myrange}{\operatorname{range}}

%\newcommand{\mmatrix}{\mathcal{M}}

% Natural numbers, integers, real numbers, complex numbers:
\newcommand{\nat}{\mathbb{N}}
\newcommand{\integer}{\mathbb{N}}
\newcommand{\real}{\mathbb{R}}
\newcommand{\compl}{\mathbb{C}}
\newcommand{\myF}{\mathbb{F}}

% Polynomial symbol:
\newcommand{\polyn}{\mathcal{P}}

% Matrix symbol:
\newcommand{\mmatrix}{\mathcal{M}}

%\newcommand{\bfemph}[1]{{\ttfamily\fontseries{b}\selectfont #1}}
\newcommand{\bfemph}[1]{{\fontseries{b}\selectfont #1}}

%\newcommand{\basis}[2]{\overbrace{ \myspan{#1_1, \dots #1_{#2}}}^{\text{linearly independent}} }}

\def\myimpl{{ \color{black}{\implies}}}


\def\bold#1{{\bf #1}}

\newtheoremstyle{mytheoremstyle} % name
%{\topsep}                    % Space above
{0.7em}                    % Space above
{0em}                        % Space below
{}                           % Body font
{0em}                           % Indent amount
%{\ttfamily\fontseries{b}\selectfont}                   % Theorem head font
{\fontseries{sb}\selectfont}                   % Theorem head font
{:\newline}                          % Punctuation after theorem head
{.3em}                       % Space after theorem head
{}  					     % Theorem head spec (can be left empty, meaning ‘normal’)

%\theoremstyle{mytheoremstyle}
%\newtheorem{thm}{Thm}[chapter]
%
%\theoremstyle{mytheoremstyle}
%\newtheorem{mydef}{Def}[chapter]
%
%\theoremstyle{mytheoremstyle}
%\newtheorem{Example}{Example}[chapter]


%\theoremstyle{mytheoremstyle}
%\newtheorem{thm}{Thm}[chapter]
%
%\theoremstyle{mytheoremstyle}
%\newtheorem{mydef}[thm]{Def}
%
%\theoremstyle{mytheoremstyle}
%\newtheorem{example}[thm]{Example}

\theoremstyle{mytheoremstyle}
\newtheorem{thm}{Theorem}[chapter]

\theoremstyle{mytheoremstyle}
\newtheorem{mydef}[thm]{Definition}
\newtheorem*{mydef-non}{Definition}

\theoremstyle{mytheoremstyle}
\newtheorem{example}[thm]{Example}

\setlength{\abovedisplayskip}{1pt}
\setlength{\belowdisplayskip}{1pt}


  \chapter{Vector Spaces}
  \section{$\mathbf{\real^n}$ and $\mathbf{\compl^n}$}

\subsection{Lists}
\setcounter{thm}{7}
\begin{mydef}
  A list of length $n$ is the same as an $n$ tuple. We write a list of vectors of length $m$ of vectors $(v_1, \ldots, v_m)$ as $v_1, \ldots, v_m$ throughout this document without closing parentheses. \\
  Since lists are the same as tuples, two lists are only equal if they have the same length and the same elements in the same order.
\end{mydef}

\begin{mydef}
  $\myF^n :\equiv \left \{ (x_1, \dots, x_n) \mid x_k \in \myF´, k \in \{1, \dots, n\} \right \}$
\end{mydef}
  \section{Definition of a vector space}

\setcounter{thm}{19}
\begin{mydef} [vector space]
  A ``vector space" is a set $V$ along with an addition on $V$ and a scalar multiplication on $V$ such that the following properties hold.
  \begin{itemize}
    \item \emph{Commutativity:}
    $ u+v = v+u \quad \forall u,v \in V$
    \item \emph{Associativity:}
    $(u+v)+w=u+(v+w)$ and $(ab)v=a(bv) \quad \forall u,v,w \in V$ and $\forall a,b \in \myF$
    \item \emph{Additive identity:}
    $\exists 0\in V:v+0=v \quad \forall v\in V$. ($\exists!$ is not a requirement but a property.)
    \item \emph{Additive inverse:}
    $\forall v\in V \quad \exists w\in V: v+w=0$
    \item \emph{Multiplicative identity:}
    $1v = v \quad \forall v\in V$
    \item \emph{Distributive properties:}\\
    $a(u+v) = au+av \myand$  \\
    $(a+b)v \, = av + bv \quad \forall a,b \in \myF \myand$
    $\forall u,v \in V$
  \end{itemize}
\end{mydef}

\setcounter{thm}{23}
\begin{mydef} [notation $\myF^S$]
  If $S$ is a set, then $\myF^S$ denotes the set of functions from $S$ to $\myF$. Let $f, g \in \myF^{S}$ and $\lambda \in \myF$:
  \begin{itemize}
    \item The ``sum" $f+g \in \myF^{S}$ is defined as follows: $(f+g)(x) :\equiv f(x)+ g(x) \quad \forall x\in S$
    \item The ``product" $\lambda f \in \myF^{S}$ is defined as follows: $(\lambda f)(x) :\equiv \lambda f(x) \quad \forall x \in S$.
  \end{itemize}
  The vector space $\myF^{n} :\equiv \myF^{\{1,2,\dots,n\}}$ is a special case, because each $(x_1, \dots, x_n) \in \myF^{n}$ can be thought of as a function $x$ from the set $\{1, 2, \dots, n\}$ to $\myF$ by writing $x(k)$ instead of $x_k$ for the $k^{\text{th}}$ coordinate of $(x_1, \dots, x_n)$.
\end{mydef}

\setcounter{thm}{25}
\begin{thm} [additive identity]
  A vector space has a unique additive identity.
\end{thm}
\begin{prf}
  $v\in V.$ Suppose $w,w' \in V$ are additive inverses of $v$. Then using associativity and the definition of the additive inverse we get
  \begin{equation}
    w = w + 0 = w + (v + w') = (w+v) + w' = 0 + w' = w'
  \end{equation}
  Thus $w = w'$, as desired^
\end{prf}

\begin{thm} [additive inverse]
  Every element in a vector space has a unique additive inverse.
\end{thm}

\begin{mydef}
  For $v,w\in V$, $-v$ denotes the additive inverse of $v$ and $w-v:\equiv w+(-v)$
\end{mydef}

\begin{mydef}
  For the rest of this summary, $V$ denotes a vector space over $\myF$.
\end{mydef}

% 1.30
\begin{thm} [the number $0$ times a vector]
  $0v = 0 \quad \forall v\in V$
\end{thm}
\begin{prf}
  $v \in V: 0v = (0+0)v = 0v +0v \implies 0 = 0v$. Using the definition of additive identity and the distributive property in $\myF$. After that, we subtract $0v$ on both sides of the equation.
\end{prf}

\begin{thm} [a number times the vector $0$]
  $a \vec0= \vec0 \quad \forall a\in \myF$
\end{thm}
\begin{prf}
  $a \in \myF: a0 = a(0+0) = a0 + a0 \implies 0 = a0$. Using the definition of additive identity and the distributive property for $0 \in V, a \in \myF$. After that, we subtract $0a$ on both sides of the equation.
\end{prf}

\begin{thm} [a number $-1$ times a vector]
  \label{thm: minus one times a vector}
  $(-1)v = -v \quad \forall v\in V$
\end{thm}
\begin{prf}
  $v \in V: v + (-1)v = 1v + (-1)v = (1 + (-1)v) = 0v = 0$ using the distributive property and the defintion of the multiplicative identity. This equation ($v+(-1)v=0$) tells us, that $(-1)v$ is the additive inverse of $v$, therefore $(-1)v = -v$.
\end{prf}

  \filbreak
  \section{Subspaces}

\begin{mydef} [subspace]
  A subset $U$ of $V$ ($U \subseteq V$) is called a \qt{subspace} of $V$ if $U$ is also a vector space with the same additive identity, addition, and scalar multiplication as on $V$.
\end{mydef}

\begin{thm} [conditions for a subspace]
  A subset $U$ of $V$ is a subspace of $V$ $\iff$ $U$ satisfies the following three conditions:
  \begin{itemize}
    \item \emph{Additive identity:}
    $0 \in U$
    \item \emph{Closed under addition:}
    $u,w \in U \implies u+w \in U$
    \item \emph{Closed under scalar multiplication:}
    $a \in \myF$ and $u \in U \implies au \in U$
  \end{itemize}
  The additive identity condition above could be replaced by the condition that $U$ is nonempty (because then taking $u \in U$ and multiplying it by $0$ would imply that $0\in U$).
\end{thm}
\begin{prf} We need to sho2 the proposition in two directions.
  \begin{description}
    \item{\qt{$\Rightarrow$ direction:}} If $U$ is a subspace of $V$, then $U$ satisfies the three conditions above by the definition of a vector space.
    \item{\qt{$\Leftarrow$ direction:}} The first condition ensures, that the additive identiy $0$ of $U$ is in $U$. The other two conditions make sure, that addition and scalar multiplication make sense on $U$.

    Thanks to \ref{thm: minus one times a vector} and the third condition, every $u \in U$ has an additive inverse $-u = (-1) u$.

    The other parts of the definition of a vector space, such associativity and commutativity, are automatically satisfied for $U$ because they hold on the larger space $V$. Thus $U$ is a vector space and hence a subspace of $V$.
  \end{description}
  \vspace{-1.1em}
\end{prf}

%\textbf{1.34} $U\subseteq V$ is a subspace of $V$ $\iff$
%\begin{itemize}
%	\item additive identity: $0 \in U$
%	\item closed under addition: $u,w\in U \implies u+w \in U$
%	\item closed under scalar multiplication: $a \in \myF$ and $u\in U$ $\implies$ $au \in U$
%\end{itemize}

\subsection{Sum of Subspaces}

% 1.36
\setcounter{thm}{35}
\begin{mydef}[sum of subspaces]
  If $V_1, \ldots, V_m$ are subspaces of $V$, the \qt{sum} of $V_1, \ldots, V_m$, denoted by $V_1 + \cdots + V_m$, is the set of all possible sums of elements of $V_1, \ldots, V_m$. More precisely,
\begin{equation}
    V_1 + \cdots + V_m :\equiv \{v_1 + \cdots + v_m \mid v_1 \in V_1, \, \dots \, , v_m \in V_m \}.
\end{equation}
\end{mydef}

%\textbf{1.40}
\setcounter{thm}{39}
\begin{thm} [sum of subspaces is the smallest containing subspace]
  \label{thm: sum of subspaces is the smallest containing subspace}
  $V_1 + \cdots + V_m$ is the smallest subspace of $V$ containing $V_1, \dots, V_m$.
\end{thm}
\begin{prf}
  First we need to show, that $V_1 + \cdots + V_m$ is a subspace of $V$. Clearly, the additive identity $0 = 0 + \cdots + 0 \in V_1 + \cdots + V_m \in V$.
  Let $u,w \in V_1 + \cdots + V_m$.  Then we have $u = v_1 + \cdots + v_m$ \st $v_1 \in V_1, \dots, v_m \in V_m$ and $w=v'_1 + \cdots + v'_m$ \st $v'_1, \dots, v'_m \in V_m$. Therefore
  \begin{equation}
    u+w = \underbrace{(v_1+v'_1)}_{\in V_1} + \cdots + \underbrace{(v_m+v'_m)}_{\in V_m} \in  V_1 + \cdots + V_m \in V.
  \end{equation}

  So $V_1 + \cdots + V_m$ is closed under addition. For $\lambda \in \myF$ we have that
  \begin{equation}
      \lambda u = \lambda(v_1 \cdots v_m) = \underbrace{\lambda v_1}_{\in V_1} + \cdots + \underbrace{\lambda v_m}_{\in V_m} \in V_1 + \cdots + V_m \in V,
  \end{equation}

  so it is closed under scalar multiplication. So we have shown that $V_1 + \cdots + V_m$ is a subspace of $V$. Now we have to show the minimality.
  The subspaces $V_1, \cdots, V_m$ are all contained in $V_1+\cdots+V_m$. Because $\forall \kInOneTillM:$
  \begin{equation}
    V_k = \{ v \mid v \in V_k \} \subseteq \{v_1 + \cdots + v_m \mid v_1 \in V_1,  \ldots , v_m \in V_m \} = V_1 + \cdots + V_m.
  \end{equation}

  Conversely, every subspace of $V$ containing $V_1, \ldots, V_m$ contains $V_1 + \cdots + V_m$, because its closed under addition. Thus $V_1 + \cdots + V_m$ is the smallest subspace of $V$ containing $V_1, \ldots, V_m$.
\end{prf}

\subsection{Direct Sums}

%\textbf{1.41}
\setcounter{thm}{40}
\begin{mydef}[direct sum, $\oplus$]
  \label{def: direct sum}
  %NO COMMA BEFORE IF ACC TO AXLER
  $V_1 + \cdots + V_m$ is called a \qt{direct sum} if each element of $V_1 +\cdots+V_m$ can be written in only one way as a sum $v_1 + \cdots + v_m$, where each $v_k \in V_k$. In this case we write:
  \begin{equation}
    V_1 \oplus \cdots \oplus V_m :\equiv V_1 + \cdots + V_m
  \end{equation}
\end{mydef}

\begin{example}
  Example: $\myF^3 =
  \left \{ \left ( x, y, 0 \right ) \in \myF^3 \mid x,y \in \myF \right \}
  \oplus
  \left \{ \left (  0, 0, z  \right ) \in \myF^3 \mid z \in \myF \right \}$
\end{example}


%\textbf{1.45}
\setcounter{thm}{44}
\begin{thm} [condition for a direct sum]
  \label{thm: condition for a direct sum}
  $V_1 + \cdots + V_m$ is a direct sum $\iff$ the only write $0$ as a sum $v_1 + \cdots + v_m$, where each $v_k \in V_k$, is by taking each $v_k$ equal to $0$.
\end{thm}
\begin{prf}
    Let  $V_1, \ldots, V_m$ be subspaces of $V$

    \qt{$\Rightarrow$ direction:} First suppose $V_1 + \cdots + V_m$ is a direct sum. Let $0 = v_1 + \cdots + v_m,$ where each $v_k \in V_k$.
    $\implies$ all the $v_k$'s are equal to $0$, because otherwise multiplying the equation above with a scalar $\lambda \in \myF$ would give a different sum (a different way to add them up).
    \begin{equation}
      0 = \lambda0= \lambda v_1 + \cdots + \lambda v_m, \whereEach \lambda v_k \in V_k
    \end{equation}

    Or a different way to argue, $0 = 0 + \cdots + 0$ is already a unique solution.

    \qt{$\Leftarrow$ direction:} Now suppose that the only way to write $0$ as a sum $v_1 + \cdots + v_m$ where each $v_k \in V_k$, is by taking each $v_k = 0$. Let
    \begin{equation}
      \begin{aligned}
        v&=v_1 + \cdots + v_m \where v_1 \in V_1, \ldots, v_m \in V_m \myand \\
        v&=u_1 + \cdots + u_m  \where u_1 \in V_1, \ldots, u_m \in V_m
      \end{aligned}
    \end{equation}

    Subtracting these two equations, we have
    \begin{equation}
      0=(v_1-u_1)+\cdots+(v_m-u_m).
    \end{equation}

    Because $v_1 - u_1 \in V_1, \ldots, v_m - u_m \in V_m$, the equation above implies that each $v_k -u_k$ equals $0$.
    \begin{equation}
      \implies v_1=u_1, \ldots, v_m=u_m
    \end{equation}

    Thus $u=v$, because $0$ has a unique representation.
\end{prf}

%\textbf{1.46}
\setcounter{thm}{45}
\begin{thm} [direct sum of two subspaces]
  \label{thm: intersection of direct sum of two subscpaces}
  $U+W$ is a direct sum $\iff$ $U \cap W = \{0\}$
\end{thm}
\begin{prf} Let $U$ and $W$ be subspaces of the same vector space.
  \begin{description}
    \item{\qt{$\Rightarrow$ direction:}} First suppose $U+W$ is a direct sum. If $v \in U \cap W$, then $v, -v \in U \myand v, -v \in W$. We also have that
    \begin{equation}
      0 = v + (-v), \where v\in U \myand (-v) \in W.
    \end{equation}
    $\implies v=0$, because $U+W$  is a direct sum. Therefore $U \cap W = \{0\}$

    \item{\qt{$\Leftarrow$ direction:}} Suppose $U\cap W = \{0\}$. Let  $u\in U$ and $w \in W$ such that
    \begin{equation}
      \label{eq: zero equals u plus w}
      0 = u+w
    \end{equation}
    By \ref{thm: condition for a direct sum} we know that our goal is to show that $u=w=0$. The equation \eqref{eq: zero equals u plus w} above implies that $u=-w\in W$  and $w=-u \in U$ for all arbitrary choices of $U$ and $W$. Thus $u,w \in U\cap W \implies u=w=0.$ Therefore, $U+W$ is a direct sum.
  \end{description}
  \vspace{-1em}
\end{prf}


  \chapter{Finite Dimensional Vector Spaces}
  \section{Span and Linear Independence}
\subsection{Linear Combinations and Span}

\setcounter{thm}{1}
\begin{mydef} [linear combination]
  $a_1 v_1 + \cdots + a_m v_m$ is called a \qt{linear combination} of a list of vectors $v_1, \dots, v_m$
\end{mydef}

\setcounter{thm}{3}
\begin{mydef} [span of a list of vectors]
	The \qt{span} of a list of vectors $\onetillm{v}$ is defined as follows:
  \begin{equation}
    \begin{aligned}
      \myspan {v_1, \dots, v_m} &:\equiv \{a_1 v_1 + \cdots + a_m v_m \mid a_1, \dots a_m \in \mathbb{F} \}\\
      \myspan{ } &:\equiv \varnothing
    \end{aligned}
  \end{equation}
\end{mydef}

\setcounter{thm}{5}
%\textbf{2.6}
\begin{thm} [the span of a list of vectors is a subspace]
  The  span of a list of vectors in $V$ is the smallest subspace of $V$ containing all vectors in the list.
\end{thm}

% 2.8
\begin{mydef} [spanning a vector space]
  If $V=\myspan{\oneTillM{v}}$, we say $\oneTillM{v}$ \qt{spans} $V$.
\end{mydef}

% 2.9
\setcounter{thm}{8}
\begin{mydef} [finite-dimensional vector space]
  Such a vector space is called finite\-/dimensional. We write
  \begin{equation}
    \dim V \neq \infty$ or $\dim V < \infty.
  \end{equation}
\end{mydef}


% 2.10
\begin{mydef}[polynomial, $\polyn(\myF)$]
  \phantom{.}
  \begin{itemize}
    \item $p: \myF \to \myF$ is called a \qt{polynomial} with coefficients in $\myF$, if there exists $a_0, \ldots, a_m \in \myF$ \st
    \begin{equation}
      p(z)=a_0 + a_1 + a_2z^2 + \cdots + a_m z^m \quad \forall z \in \myF.
    \end{equation}
    \item $\polyn (\myF)$ is the set of all polynomials with coefficients in $\myF$.
  \end{itemize}
\end{mydef}

% 2.11
\begin{mydef}[degree of a polynomial, $\deg p$]
  The degree $m$ of a polynomial $p=a_0+a_1z+a_2z^2+\cdots+a_mz^m$ is denoted with \begin{equation}
    \deg p :\equiv m.
  \end{equation}

  We also define for the zero polynomial $p=0$
  \begin{equation}
    \degree 0 : \equiv -\infty.
  \end{equation}
\end{mydef}

% 2.12
\begin{mydef}
  $\polyn_m (\myF )=\myspan{1,z,\dots,z^m}$ denotes the set of all polynomials with coefficients in $\myF$ and degree at most $m$ (One should actually use the lambda notation $\lambda z.z^m$ to speak of functions). Note that
  \begin{equation}
    \dim \polyn_m(\myF) = m+1
  \end{equation}
\end{mydef}

%2.13
\setcounter{thm}{12}
\begin{mydef}[infinite\-/dimensional]
  A vector space is called \qt{infinite\-/dimensional} if it is not \fd.
\end{mydef}


\subsection{Linear Independence}

\setcounter{thm}{14}
\begin{mydef} [linear independence]
  There are $2$ ways to think about \qt{linear independence}. Lets first start with the motivation. A list of vectors $\oneTillM{v}$ is \qt{\lid}, if every combination of these vectors $w = b_1v_1 + \dots+ b_mv_m$ $\in \myspan{\oneTillM{v}}$ has a unique choice of scalars $b_1, \dots, b_m \in \myF$.

  \bfemph{The actual definition of linear independence:} \\
  A list of vectors $\oneTillM{v}$ is called \lid, if the only way to combine them together and yield zero, is if we choose every coefficient $\lambda_i$ to be zero.
  \begin{equation}
    \lambda_1v_1 + \dots + \lambda_mv_m = 0 \iff \lambda_1 = \cdots = \lambda_m = 0
  \end{equation}

  Another way to put it:\\
  Let $w\in \myspan{v_1, \dots, v_m}$ sucht that for $a_1, \dots, a_m, c_1, \dots, c_m \in \myF$:
  \begin{equation}
    \begin{aligned}
      w & =a_1 v_1 + \cdots + a_m v_m \myand \\
      w & =c_1 v_1 + \cdots + c_m v_m \\
      & \iff \\
      0 & = \underbrace{(a_1 - c_1)}_{= \, 0\text{?}} v_1 + \cdots + \underbrace{(a_m -c_m)}_{= \, 0\text{?}} v_m
    \end{aligned}
  \end{equation}

  if $a_1 = c_1, a_2 = c_2, \, \dots \, , a_m = c_m$ is the only solution, the representation of $w$ as a linear combination of $v_1, \ldots, v_m$ is unique and the only way to add them up together equaling $0$ is $0v_1+\cdots+0v_m=0$. Both of these definitions are equivalent.

  The empty list $()$ is also defined to be linearly independent.
\end{mydef}


%\textbf{2.16}
\begin{mydef} [linear dependence]
  Otherwise, $\oneTillM{v}$ is called \qt{linearly dependent}.
\end{mydef}

%	\textbf{2.19} Linear dependence lemma:\\
%	Suppose $\oneTillM{v}\in V$ is a linearly dependent list.
%	$\implies \exists \kInOneTillM$ : $v_k \in \myspan{\oneTill{v}{k-1}}$


%\textbf{2.29}

\setcounter{thm}{18}
%\textbf{2.19}
\begin{thm} [linear dependence lemma]
  \label{thm: linear dependence lemma}
  Suppose $v_{1}, \dots, v_{m}\in V$ is a linearly dependent list.
  \begin{equation}
    \implies \exists k \in \{ 1, \dots, m \} : v_k \in \myspan {v_1, \dots, v_{k-1}}.
  \end{equation}
  Furthermore, if $k$ satisfies the conditions above and the $k^{\text{th}}$-term is removed from $v_1, \dots, v_m$, then the span of the remaining list equals $\myspan {v_1, \dots, v_{m}}$:
  \begin{equation}
    \myspan {v_1, \dots, v_{k-1}, v_{k+1}, \dots, v_m} = \myspan {v_1, \dots, v_m}
  \end{equation}
\end{thm}

\begin{prf}
  Because $\onetillm{v}$ are linearly dependent, we have
  \begin{equation}
    a_1 v_1 + \cdots + a_m v_m = 0
    $ where $a_1, \ldots, a_m \in \myF$ and $a_j \neq 0$ and $a_k \neq 0 $ for $j,k \in \{1, \ddd, m\}, j \neq k.
  \end{equation}

  So we know, at least $2$ $a$'s are not $0$.
  Let $k$ be the largest element of $\setonetillm$ such that $a_k \neq 0$. Then
  \begin{equation}
    v_k = - \frac{a_1}{a_n}v_1 - \cdots - \frac{a_{k-1}}{a_k}v_k,
  \end{equation}

  which proves that $v_k \in \myspan{v_1, \ddd, v_{k-1}}$.

  Now for the second claim, suppose $k$ is any element of $\setonetillm$ such that
  \begin{equation}
    v_k \in \myspan{v_1, \ddd, v_{k-1}}.
  \end{equation}

  Then we have
  \begin{equation}
    \label{eq: equation for v_k}
    v_k = b_1v_1 + \cdots + b_{k-1}v_{k-1}, \mytext{where} b_1, \ddd, b_{k-1} \in \myF.
  \end{equation}

  Suppose $u \in \myspan{v_1, \ddd, v_m}$. Then
  \begin{equation}
    \label{eq: linear comb for u}
    u = c_1v_1 + \cdots + c_kv_k + \cdots +c_{m}v_{m}, \mytext{where} c_1, \ddd, c_{m} \in \myF.
  \end{equation}

  Of course, for corner cases like $k=1$ or $k=2$ and such, the equations look different.

  Now we can replace $v_k$ with the right side of \eqref{eq: equation for v_k} in equation \eqref{eq: linear comb for u}.
  \begin{equation}
    \begin{aligned}
      u
      & = c_1v_1 + \cdots + c_k \cdot (b_1v_1 + \cdots + b_{k-1}v_{k-1}) + \cdots +c_{m}v_{m} \\
      & = (c_1 + c_k b_1)\cdot v_1 + (c_2 + c_k b_2) \cdot v_2 + \cdots + (c_{k-1}+c_k b_{k-1}) \cdot v_{k-1} + c_{k+1} v_{k+1} + \cdots + c_m v_m
    \end{aligned}
  \end{equation}

  Which shows $u \in \myspan{v_1, \ddd, v_{k-1}}$. Therefore $\myspan{v_1, \ddd, v_{k-1}, v_{k+1}, \ddd, v_m} = \myspan{v_1, \ddd, v_m}$.
\end{prf}

\setcounter{thm}{21}
%\textbf{2.22}
\begin{thm}  [length of linearly independent list $\mathbb{\leq}$ length of spanning list]
  \label{thm: length of linearly dependent list less or equal to length of spanning list}
  In a \fd vector space, the length of every linearly independent list of vectors is less than or equal to the length of every spanning list of vectors.
\end{thm}
\begin{prf}
  Let $u_1, \ldots, u_m$ be a linearly independent list of vectors in $V$ and let $w_1, \ldots, w_n$ such that $V=\myspan{w_1, \ldots, w_n}$. We need to prove that $m \leq n$. We do this with following algorithm:
  \begin{equation*}
      $Let $B_0 :\equiv  (w_1, \ldots, w_n)$. Note that $V=\myspan{B_0}.
  \end{equation*}

  \prooffont{Step 1:} Let $B_1^* :\equiv (u_1, w_1, \ldots, w_n)$, which yields a linearly dependent list of length $n+1$, because $u_1$ can be written as a linear combination of $w_1, \ldots, w_n$. \\
  In other words, the list
  \begin{equation*}
    (u_1, w_1, \ldots, w_n) = B_1^* $ is linearly dependent.$
  \end{equation*}

  Thus by the linear dependence lemma (\ref{thm: linear dependence lemma}), one of the vectors of $B_1^*$ can be written as a linear combination of the of the previous vectors in the list. Since $u_1 \neq 0$, \ref{thm: linear dependence lemma} implies that we can remove one of the $w$'s, say $w_j$, so that the new list
  \begin{equation*}
    B_1 :\equiv B_1^* - {w_j} = (u_1 w_1, \ldots, w_{j-1}, w_{j+1}, \ldots, w_n)
  \end{equation*}

  of length $n$ consisting of $u_1$ and the remaining $w$'s spans $V$.
  \begin{equation*}
    V=\myspan{B_1}
  \end{equation*}

  \prooffont{Step k, for k=2$\mathbf{, \ddd,}$m:} The list $B_{k-1}$ of length $n$ from previous step $k-1$ spans $V$. So we have, $u_k \in \myspan{B_{k-1}}$, because the span never changed by our algorithm. If we add $u_k$ to the list, our list becomes linearly dependent. Let
  \begin{equation*}
    B_k^* :\equiv (u_1, \ldots, u_{k-1}, u_k, \underbrace{\widetilde w_{1}, \ldots, \widetilde w_{n-(k-1)})}_{\text{remaining $w$'s}}
  \end{equation*}
  where $\widetilde w_{1}, \ldots, \widetilde w_{n-(k-1)}$ are our remaining $n-(k-1)$ $w$'s at our current step $k$, which are the same as from the previous list $B_{k-1}$$=$$(u_1, \ldots,u_{k-1},$$ \widetilde w_{1}, \ldots, \widetilde w_{n-(k-1)})$. Our new list has length $(n+1) = k + (n-(k-1))$. With the linear dependence lemma (\ref{thm: linear dependence lemma}), we can remove one vector of the list, since our new list $B_{k}^*$ was linearly dependent and spans $V$.
  \begin{equation*}
    V=\myspan{B_{k}^*}.
  \end{equation*}

  Because $u_1, \ldots, u_k$ are linearly independent, this vector can not be one of the $u's$. Hence there still must be at least one remaining $w$ at this step. $n-(k-1)$ $w's$ to be precise. So we can remove a $w$, lets say $\widetilde w_j$, so that the new list $B_k$ of length $n$ consisting of $u_1, \cdots, u_k$ and the remaining $w$'s spans $V$.
  \begin{equation*}
    B_k = B_k^* - \widetilde w_j$ such that $ V=\myspan{B_k}$ and length $|B_k|=n
  \end{equation*}

  \emph{\bfseries Conclusion:} After step $m$, we have added all the $u's$ and the precess stops. At each step $k$, as we add a $u$ to $B_{k-1}$, the linear dependence lemma implies that there is some $w$ to remove from $B_{k}^*$. Thus there are at least as many $w$'s as $u's$.
  \begin{equation*}
    \implies m \leq n
  \end{equation*}
  $\implies$ the length of $u_1, \ldots, u_m$ $\leq$ the length of $w_1, \ldots, w_n$.
\end{prf}

% 2.25
\setcounter{thm}{24}
\begin{thm} [finite-dimensional subspace]
  \label{thm: finite-dimensional subspace}
  Every subspace of a finite\-/dimensional vector space is finite\-/dimensional.
\end{thm}
\begin{prf}
  Let $\dim V \neq \infty$ and $U \subseteq V$.

  \emph{Step 1:} If $U = \{0\}$, the $U$ is finite\-/dimensional and we are done. If $u_1 \neq \{0\},$ then choose a nonzero vector $u_1 \in U$.

  \emph{Step k:} If $ U = \myspan{u_1, \ldots, u_{k-1}},$ then $U$ is \fd and we are done.

  If $U$ $\neq$ $\myspan{u_1, \ldots, u_{k-1}}$, then choose a vector $u_k \in U$ \st
  \begin{equation}
    u_k \notin \myspan{u_1, \ldots, u_{k-1}}
  \end{equation}

  After each stept, as long as the precess continues, we have constructed a list of vectors, such that no vector in the list is in the span of the previous vectors. Thus after each step we have constructed a linearly independent list, by the negation of of the linear dependence lemma (\ref{thm: linear dependence lemma}).
  This list can not be longer than any spanning list of $V$ (by \ref{thm: length of linearly dependent list less or equal to length of spanning list}).
  Thus the process eventually terminates, which means $U$ is \fd.
\end{prf}
  \section{Bases}


\setcounter{thm}{25}
%\textbf{2.26}
\begin{mydef} [basis]
  A \qt{basis} $\beta$ of $V$ is a list of vectors $\beta = \onetilln{v}$ in $V$ that is linearly independent and spans $V$.
  \begin{equation}
    V = \myspan{v_1, \dots, v_n} = \myspan{ \beta }
  \end{equation}
\end{mydef}

\setcounter{thm}{27}
\begin{thm} [condition for a basis]
  $\onetilln{v}$ is a basis of $V$ $\iff$ every $v \in V$ can be written uniquely in the form
  \begin{equation}
    v=a_1 v_1 + \dots + a_n v_n $, where $ a_1, \dots, a_n \in \myF
  \end{equation}
  Note that the definition of linear dependence has said nothing about uniqueness.
\end{thm}

\setcounter{thm}{29}
\begin{thm} [every spanning list contains a basis]
  \label{thm: every spanning list contains a basis}
  Every spanning list of a \vs can be reduced to a basis of the \vs.
\end{thm}
\begin{prf}
  Suppose $V=\myspan{v_1, \ldots, v_n}.$ Let $B_0 :\equiv (v_1, \ldots, v_n)$.

  \emph{\bfseries Step 1: } If $v_1 = 0$, delete $v_1$ from $B_0$:
  \begin{equation}
    $Let $B_1 :\equiv (v_2, \ldots, v_n).
  \end{equation}

  If $v_1 \neq 0$, then leave $B_0$ unchanged. $B_1 :\equiv B_0=(v_1, \ldots, v_n)$.

  \emph{\bfseries Step k: } If $v_k$ $\in$ $\myspan{v_1^*, \ldots, v_{k-1}^*}$, then delete $v_k$ from list $B_{k-1}$ sucht that:
  \begin{equation}
    B_k :\equiv B_{k-1} - v_k = (v_{1}^*, \ldots, v_{k-1}^*, v_{k+1}, \ldots, v_n)
  \end{equation}

  where $v_1^*, \ldots, v_{k-1}^*$ are not necessarily the original elements $v_1, \ldots, v_{k-1}$ from $B_0$.

  If $v_k \in \myspan{v_1, \ldots, v_{k-1}}$, leave the basis unchanged: $B_k :\equiv B_{k-1}$.

  Stop the process after step $n$, getting a list $B_n$. The list $B_n$ spans $V$ because our original list $B_0$ spanned $V$  have discarded only vectors that were already in the span of the previous vectors. The process ensures that no vector in $B$ is in the span of the previous ones. Thus $B$ is linearly independent. Hence $B$ is a basis of $V$.
  %TODO: Why does axler cite linear dependence lemma 2.19.
\end{prf}

\setcounter{thm}{30}
\begin{thm} [basis of finite\-/ dimensional vector space]
  \label{thm: every finite-dimensional vector space has a basis}
  Every \findimvs has a basis.
\end{thm}
\begin{prf}
  By definition, a \fdvs has a spanning list. The previous result tells us that each spanning list can be reduced to a basis.
\end{prf}

\begin{thm}
  \label{thm: every linearly independent list of vectors in a finite-dimensional vector space can be extended to a basis of the vector space}
  Every \lid list of vectors in a  \findimvs can be extended to a basis of the \vs.
\end{thm}
\begin{prf}
  Let $u_1, \ldots, u_m \in V$ be linearly independent. Let $V=\myspan{w_1, \ddd, w_n}.$
  \begin{equation}
    \implies V=\myspan{u_1, \ddd , u_m, w_1, \ddd , w_n }
  \end{equation}
  Applying the procedure of the proof of \ref{thm: every spanning list contains a basis} to reduce this list to a basis of $V$ produces a basis consisting of vectors $u_1, \cdots, u_m$ and some $w$'s.
\end{prf}

\begin{thm} [every subspace of $V$ is part of a direct sum equal to $V$]
  \label{thm: every subspace of V is part of a direct sum equal to V}
  If $V$ is \fd and $U$ is a subspace of $V$. Then there is another subspace $W$ of $V$ such that $V=U \oplus W.$
\end{thm}
\begin{prf}
  Suppose $U$ is a subspace of a \fdvs $V$.

  $\implies$ $U$ is also \fd (by \ref{thm: finite-dimensional subspace}).

  $\implies$ $\exists u_1, \ldots, u_m \in U $ such that $u_1, \ldots, u_m $ form a basis of $U$. (by \ref{thm: every finite-dimensional vector space has a basis})

  $\implies$ $u_1, \ldots, u_m$ can be extended to a basis $u_1, \ldots, u_m, w_1, \ldots, w_n$ of $V$. (by \ref{thm: every linearly independent list of vectors in a finite-dimensional vector space can be extended to a basis of the vector space})
  \begin{equation}
    $Let $W :\equiv \myspan{w_1, \ldots, w_n}.
  \end{equation}

  To prove $V=U\oplus W$, by $\ref{thm: sum and intersection of two subspaces}$ we only need to show that $V = U+W$ and $U \cap W = \{0\}$.


  \begin{description}

    \item{\bfemph{\slshape Proof of $V=U+W$:}} Let $v \in V$. Since $u_1, \ldots, u_m, w_1, \ldots, w_n$ is a basis of $V$, it spans $V$.
    \begin{equation}
      V = \myspan{u_1, \ldots, u_m, w_1, \ldots, w_n}.
    \end{equation}

    So we have $\exists a_1, \ldots, a_m, b_1, \ldots, b_n \in \myF$ \st
    \begin{equation}
      v = \underbrace{a_1 u_1 + \cdots a_m u_m}_{u} + \underbrace{b_1 w_1 + \cdots b_m w_m}_{w}.
    \end{equation}

    With $u$ and $w$ defined as above, we have $v=u+w$, where $u \in U$ and $w \in W$. Thus $v \in U+W,$ completing the proof that $V=U+W$.

    \item{\bfemph{\slshape Proof of $U \cap W = \{0\}$:}} Let $v \in U \cap W$. $\implies$ $\exists a_1, \ldots, a_m, b_1, \ldots, b_n \in \myF$ \st $v= a_1 u _1 + \cdots + a_m u_m$ and $v= b_1 w_1 + \cdots + b_n w_n$. Thus $a_1 u_1 + \cdots + a_m u_m - b_1 w_1 - \cdots - b_n w_n = 0$.

    Because $u_1, \ldots, u_m, w_1, \ldots, w_n$ is linearly independent, this implies that
    \begin{equation}
      a_1 = \cdots = a_m = b_1 = \cdots= b_n = 0.
    \end{equation}

    Thus $v=0$, completing the proof that $U\cap W = \{0\}.$

  \end{description}
  \vspace{-1em}
\end{prf}

\subsection{Exercises}
\setcounter{xrcs}{5}
% E 6
\begin{xrcs}
   Claim: If $v_1, v_2, v_3, v_4$ is a basis of $V$, then
  \begin{equation}
    v_1 + v_2, v_2 + v_3, v_3+v_4, v_4 $ is also a basis of $V.
  \end{equation}


  \begin{prf}
    Let $a_1, a_2, a_3, a_4 \in \myF$ s.t.
    \begin{equation}
      a_1 (v_1+v_2) + a_2(v_2+v_3) + a_3(v_3+v_4)+a_4 v_4 = 0.
    \end{equation}

    This is the case if and only if
    \begin{multline}
      a_1 v_1 + a_1 v_2 + a_2 v_2 + a_2 v_3 + a_3 v_3 + a_3 v_4 + a_4 v_4 \\
      = a_1 v_1 + (a_1 + a_2) v_2 + (a_2 + a_3) v_3 + (a_3 + a_4)v_4=0
    \end{multline}

    Since $v_1, \ldots, v_4$ are linearly independent, this is only the case if $0 = a_1 = a_1 + a_2 = a_2 + a_3 = a_3 + a_4$ and therefore $\iff$ $a_1 = \cdots = a_4 = 0$. So $v_1 + v_2, v_2 + v_3, v_3+v_4, v_4$ are linearly independent as well.

    Now we also have to show, that it spans $V$. Let $u \in V$. If with our old basis, $u$ would have been written as
    \begin{equation*}
      u = b_1 v_1 + \cdots + b_4 v_4.
    \end{equation*}

    Then with our new basis, we would write $a_1 = b_1.$ Next we would have $a_1 + a_2 = b_2$ and therefore $a_2 = b_2 - a_1 = b_2-b_1$. Next we would have $a_2+a_3=b_3$ and thus $a_3= b_3-a_2 = b_3 - (b_2-b_1)$. Lastly $a_4 = b_4-a_3 = b_4-(b_3 - (b_2-b_1))$. As we can see, our new basis also spans $V$.
  \end{prf}
\end{xrcs}

% E 8
\begin{xrcs}
  Claim: If $v_1, v_2, v_3, v_4$ is a basis of $V$ and $U \subseteq V$, such that $v_1, v_2 \in U$ and $v_3 \notin U$ and $v_4 \notin U$, then $v_1, v_2$ is a basis of $U$.
\end{xrcs}
\begin{prf}
  Any vector in $U$ can be represented as a unique linear combination of $v_1, \cdots, v_4$. Let $u\in U$ s.t. $u=a_1 v_1 + a_2 v_2 + a_3 v_3 + a_4 v_4$. In this case, $a_3$ must equal to $0$, because otherwise, due to closedness under addition and multiplication, we could have subtract $a_1 v_1 + a_2 v_2+a_4v_4$ from $u$ and multiply by $\sfrac{1}{a_3}$ to get $v_3$, which is not in $U$. The same goes for $a_4$, which must be $0$. So every vector in $U$ is of the form $u=a_1 v_1 + a_2 v_2$, since $v_1, v_2 \in U$ and $v_3, v_4 \notin U$. Since every basis has the same length and $v_1$ and $v_2$ are linearly independent, the list $v_1, v_2$ forms a basis of $U$.
\end{prf}

\begin{xrcs}
  Suppose $v_1, \cdots, v_m$ is a list of vectors in $V$. For $k \in \{1, \ldots, m\},$ let
  \begin{equation}
    w_k = v_1 + \cdots + v_k
  \end{equation}

  Show that $v_1, \ldots, v_m$ is a basis of $V$ if and only if $w_1, \ldots, w_m$ is a basis of $V$.
\end{xrcs}
\begin{prf}
  Let us first observe the following two patterns
  \begin{equation}
    \begin{aligned}
      w_1 &= v_1                    & \qquad v_1 &= w_1       \\
      w_2 &= v_1 + v_2              & \qquad v_2 &= w_2 - w_1 \\
      w_3 &= v_1 + v_2 + v_3        & \qquad v_3 &= w_3 - w_2 \\
      w_4 &= v_1 + v_2 + v_3 + v_4  & \qquad v_4 &= w_4 - w_3 \\
          &\;\;\vdots               &            &\;\;\vdots \\
      w_n &= v_1 + \cdots + v_n     & \qquad v_n &= w_n - w_{n-1}
    \end{aligned}
  \end{equation}
\end{prf}

  \section{Dimension}

% 2.34
\setcounter{thm}{33}
\begin{thm} [basis length] Any two bases of a \fdvs have the same length. So the the basis length does not depend on basis.
\end{thm}
\begin{prf}
  Let $B_1$ and $B_2$ be $2$ basis of $V$. Since both are linearly independent and both span $V$, we have $|B_1| \leq |B_2|$ and $|B_2| \leq |B_1|$ by using \ref{thm: length of linearly dependent list less or equal to length of spanning list}. Therefore $|B_1|=|B_2|.$
\end{prf}

\begin{mydef} [dimension]
  $\dim V :\equiv$ length of any basis of $V$.
\end{mydef}

% 2.37
\setcounter{thm}{36}
\begin{thm} [dimension of a subspace]
  \label{thm: dimension of a subspace}
  If $V$ is \fd and $U$ is a subspace of $V$, then
  \begin{equation}
    \dim U \leq \dim V
  \end{equation}
\end{thm}
\begin{prf}
  Since the length of everly linearly independent list is smaller or equal to every spanning list in $V$ according to
  \ref{thm: length of linearly dependent list less or equal to length of spanning list}, we have the situation that the length of the basis of $U$ is less or equal to the length of the basis of $V$.
\end{prf}


% 2.38
\begin{thm} [lineparly independent list of the right length]
  \label{thm: linearly independent list of the right length is a basis}
  Every linearly independent list of vectors of length $\dim V$ is a basis.
\end{thm}
\begin{prf}
  Let $\dim V = n$ and $v_1, \ldots, v_n$ be linearly independent in $V$. This list can be extended to a basis of $V$ by \ref{thm: every linearly independent list of vectors in a finite-dimensional vector space can be extended to a basis of the vector space}. However, every basis of $V$ has length $n$, so no elements are adjoined to $v_1, \ldots, v_n$ by such an extension process.
\end{prf}

% 2.39
\begin{thm} [subspace of full dimension]
  \label{thm: subspace of full dimension equals the whole space}
  If $U$ is a subspace of $V$ and $\dim U = \dim V \neq \infty$, then $U=V$.
\end{thm}
\begin{prf}
  Let $u_1, \ldots, u_n$ be a basis of $U$. Thus $n = \dim U$, and by the hypothesis we also have $n = \dim V$. Thus $u_1, \ldots, u_n$ is a linearly independent list of vectors in $V$ of length $\dim V$. So by our previous theorem \ref{thm: linearly independent list of the right length is a basis}, we see that $u_1, \ldots, u_n$ is a basis of $V$. In particular, every vector in $V$ is a linear combination of $u_1, \ldots, u_n$. Thus $U=V$.
\end{prf}

% 2.42
\setcounter{thm}{41}
\begin{thm} [spanning list of the right length]
  \label{thm: spanning list of the right length}
  Every spanning list of vectors in $V$ $(\dim V \neq \infty)$ with length $\dim V$ is a basis of $V$.
\end{thm}
\begin{prf}
  Suppose $\dim V = n$ and $v_1, \ldots, v_n$ spans $V$. The list $v_1, \ldots, v_n$ can be reduced to a basis of $V$ by \ref{thm: every spanning list contains a basis}. But our list already has length $n$, so no elements will get deleted. Thus $v_1, \ldots, v_n$ is a basis of $V$, as desired.
\end{prf}

% 2.43
\begin{thm} [dimension of a sum]
  \label{dimension of a sum}
  $V_1$ and $V_2$ are subspaces of a \fdvs we have
  \begin{equation}
    \implies \dim (V_1 + V_2) = \dim V_1 + \dim V_2 - \dim (V_1 \cap V_2)
  \end{equation}

  See exercise 2C19 and 2C20 for a even nicer result.
%  From exercise 2C19 and 2C20 we have the following result:
%  If $V_1$, $V_2$ and $V_3$ are subspaces of \fdvs we have
%  \begin{equation}
%    \begin{aligned}
%      \dim (V_1 &+ V_2 + V_3) = \dim V_1 + \dim V_2 + \dim V_3 \\
%      & \quad - \dim (V_1 \cap V_2) - \dim (V_1 \cap V_3) - \dim (V2 \cap V3) \\
%      & \quad + \dim (V_1 \cap V_2 \cap V_3)
%    \end{aligned}
%  \end{equation}
%
%as well as
%
%  \begin{equation}
%    \begin{aligned}
%      \dim (V_1 + V_2 + V_3)  = \dim V_1 + \dim V_2 + \dim V_3 \\
%        - \frac{\dim (V_1 \cap V_2) + \dim (V_1 \cap V_3) + \dim (V2 \cap V3)}{3} \\
%       - \frac{ \dim \left(  (V_1 + V_2) \cap V_3 \right) + \dim \left( (V_1 + V_3) \cap V_2 \right)+ \dim \left( (V2 + V3) \cap V_1 \right) }{3} \\
%    \end{aligned}
%  \end{equation}
\end{thm}



  \chapter{Linear Maps}
  \section{Vector of Linear Maps}

\subsection{Definition and Examples of Linear Maps}

% 3.1
\begin{mydef} [linear map]
  A \lm from $V$ to $W$ is a function $T:V\to W$ with following properties:
  \begin{itemize}
    \item additivity: $T(u+v)=Tu + Tv \qquad \forall u,v \in V$
    \item homogenity: $T(\lambda v)=\lambda (Tv) \qquad
    \forall \lambda \in \mathbb{F}, \; \forall v\in V$
  \end{itemize}
\end{mydef}

% 3.2
\begin{mydef} [notation $\linmap (V,W), \linmap (V)$]
  The set of all linear maps from $V$ to $W$ is denoted by $\mathcal{L}(V,W)$. For the set of all linear maps from $V$ to itself we use $\mathcal{L}(V) :\equiv \mathcal{L}(V,V)$
\end{mydef}

% 3.4
\setcounter{thm}{3}
\begin{thm}[linear map lemma]
  Suppose $\onetilln{v}$ is a basis of $V$ and $\onetilln{w}$ $\in$ $W$, where the $w$'s do not have any specific properties. Then there exist a unique \lm $T:V\to W$ or $T \in \linmap(V,W)$ such that
  \begin{itemize}
    \item[] $Tv_{k} = w_k \quad \forall \kinonetilln.$
  \end{itemize}
  It is of the form \begin{equation}
    T(c_1 v_1 + \cdots c_n v_n) \mapsto c_1 w_1 + \cdots + c_n w_n
  \end{equation}
\end{thm}

% 3.5
\setcounter{thm}{4}
\begin{mydef} [addition and scalar multiplication on $\linmap(V,W)$]
  Suppose $S, T \in \lvw$ and $\lambda \in \mathbb{F}.$ \\
  The sum $S+T \in \linmap (V,W)$ and the product $\lambda T \in \linmap (V,W)$ are defined by:
  \begin{itemize}
    \item $(S+T)(v) :\equiv Sv+Tv \quad \forall v \in V$
    \item $(\lambda T)(v) : \equiv \lambda (Tv) \quad \forall v \in V$
  \end{itemize}
\end{mydef}

% 3.6
\subsection{Algebraic Operations on $\linmap(V,W)$}
\setcounter{thm}{5}
\begin{thm} [$\lvw$]
  With these operations of addition and scalar multiplication as defined above, $\lvw$ is a \vs.
\end{thm}

% 3.7
\setcounter{thm}{6}
\begin{mydef}
  Let $T \in \lin{U}{V}$ and $S \in \lin{V}{W}$. We define the ``product'' $ST \in \lin{U}{W}$ as follows:
  \begin{itemize}
    \item[] $(ST)(u) :\equiv S(Tu) \quad \forall u \in U$
  \end{itemize}
  Thus $ST$ is just the usual composition  $S \circ T$ of two functions. But with linear functions, we have distributive and associative properties. 
\end{mydef}

% 3.8
\begin{thm} [algebraic properties of products of linear maps]
  With these definitions we have
  \begin{itemize}
    \item \bfemph{associativity:} $(T_1 T_2) T_3 = T_1 (T_2 T_3)$, whenever $T_3$ maps into the Domain of $T_2$ and $T_2$ maps into the Domain of $T_1$.
    \item \bfemph{identity:} $T I = I T = T$ for $T \in \lvw$ (The first $I$ is the identity operator on $V$ and the second $I$ the identity operator on $W$. \\
    We could also write $T I_V = I_W T$
    \item \bfemph{distributive properties:} For $T, T_1, T_2 \in \lin{U}{V}$ and $S, S_1, S_2 \in \lin{V}{W}$: \\ $(S_1 + S_2)T=S_1 T + S_2 T$ and $S(T_1 + T_2)=S T_1 + S T_2$
    \item \bfemph{non-commutative:} $ST \neq TS$ in general.
  \end{itemize}
\end{thm}

% 3.10
\setcounter{thm}{9}
\begin{thm} [linear maps take $0$ to $0$]
  $T\in \lvw \implies T(0)=0.$
\end{thm}
\begin{prf}
  $T(0) = T(0+0) = T(0) + T(0)$. Now subtract $T(0)$ on both sides.
\end{prf}
  \section{Null Spaces and Ranges}

\subsection{Null Space and Injectivity}

\begin{mydef} [null space, $\mynull T$]
  $\mynull T :\equiv \ker T :\equiv \{ v \in V \mid Tv=0\} \subseteq V$ for $T \in \lvw$
\end{mydef}

% 3.13
\setcounter{thm}{12}
\begin{thm} [null space is a subspace]
  For $T=\linmap(V, W)$, $\mynull T$ is a subspace of $V$
\end{thm}
\begin{prf}
  $0 \in \mynull T$ because $T(0) = 0$. Suppose $u,v \in \mynull T$ and $\lambda \in \myF.$ $\implies T(u+v)=Tu+Tv=0+0=0.$ Hence $u+v \in \mynull T$. We also have $T(\lambda u)= \lambda Tu = \lambda 0 = 0$, so $\lambda u \in \mynull T$.
\end{prf}

% 3.14
\setcounter{thm}{13}
\begin{mydef} [injectivity]
  \label{def: injectivity}
  A function $T: V \to W$ is called \qt{injective} if
  \begin{equation}
    Tu = Tv \implies u = v \quad \forall u,v \in V
  \end{equation}
  Or analogous:
  \begin{equation}
    u \neq v \implies Tu \neq Tv \quad \forall u,v \in V
  \end{equation}
\end{mydef}

% 3.15
\setcounter{thm}{14}
\begin{thm} [injectivity and null space property]
  \label{thm: injectivity iff null space equals zero-set}
  Let $T \in \linmap(V,W)$. Then
  \begin{equation}
    T $ is injective $ \iff \mynull T = \{0 \}.
  \end{equation}
\end{thm}

\subsection{Range and Surjectivity}

% 3.16
\setcounter{thm}{15}
\begin{mydef} [range]
  $\operatorname{range}T :\equiv \{Tv \mid v \in V\} \subseteq W$ for $T \in \lvw$
\end{mydef}

% 3.18
\setcounter{thm}{17}
\begin{thm} [range is a subspace]
  \label{thm: the range is a subspace}
  $T\in \linmap(V,W) \implies \myrange T$ is a subspace of $W$.
\end{thm}

%3.19
\setcounter{thm}{18}
\begin{mydef} [surjectivity]
  \label{def: surjectivity}
  If $\myrange T=W$, $T:V\to W$ is called \qt{surjektive} or \qt{onto}.
\end{mydef}

\subsection{Fundamental Theorem of Linear Maps}

  % 3.21
  \setcounter{thm}{20}
  \begin{thm} [{\slshape \scshape Fundamental Theorem of Linear Maps or ``Rank Nullity Theorem''}]
    \label{rank-nullity-theorem}
    \phantom{.} \\
    For $T \in \linmap(V,W)$, where $V$ is \fd, we have that $\myrange(T)$ is also \fd and
    \begin{equation}
      \dim V =
      \underbrace{ \dim\mynull T\phantom{g}\!\!\!\!\!}_{\text{nullity}}
      \; + \; \underbrace{\dim \myrange T}_{\text{rank}}.
    \end{equation}

    Note that $T \in \linmap(V,W)$ is not a typo.
  \end{thm}

  %3.22
  \setcounter{thm}{21}
  \begin{thm} [linear map to a lower-dimensional space is not injective]
    \label{thm: linear-map-to-a-lower-dimensional-space-is-not-injective}
    If $\dim V$ $>$ $\dim W$ \\
    $\implies$ No \lm from $V$ to $W$ is injective.
  \end{thm}
  \begin{prf} Let $T\in \linmap(V,W)$ such that $\dim V > \dim W$. Using \ref{rank-nullity-theorem}, we have
    \begin{equation}
      \dim \mynull T = \dim V - \dim \myrange T \geq \dim V - \dim W > 0,
    \end{equation}
    The first inequality follows from \ref{thm: dimension of a subspace}, because $\myrange T \subseteq W$. Now we have that $\dim \mynull T > 0.$ Thus $T$ is not injective by \ref{def: injectivity}.
  \end{prf}

  % 3.24
  \setcounter{thm}{23}
  \begin{thm} [linear map to a higher-dimensional space is not surjective]
    If $\dim V < \dim W$ $\implies$ No \lm from $V$ to $W$ is surjective.
  \end{thm}
  \begin{prf}
    Let $T\in \linmap (V,W)$ such that $\dim V < \dim W$. Then
    \begin{equation}
      \dim \myrange T = \dim V - \dim \mynull T \leq \dim < \dim W
    \end{equation}
    Thus we have $\dim \myrange T < \dim W$. So $\myrange T \neq W$. Thus $T$ is not surjective by \ref{def: surjectivity}.
  \end{prf}

  %3.26
  \setcounter{thm}{25}
  \begin{thm} [homogenenous system of linear equations]
    A homogeneous system of linear equations with more variables then equations has nonzero solutions.
  \end{thm}

  %\textbf{3.28}
  \setcounter{thm}{27}
  \begin{thm} [system of lienar equations with more equations than variables]
    An inhomogeneous system of linear equations with more equations then variables has no solutions for some choice of constant terms.
  \end{thm}

  \section{Matrices}
\subsection{Representing a Liner Map by a Matrix}

%3.31
\setcounter{thm}{30}
\begin{mydef}[matrix of a linear map, $\mmatrix(T)$]
  \label{def: matrix of a linear map}
  Let $T \in \linmap (L,W)$ and let $v_1, \dots, v_n$ be a basis of $V$ and let $w_1, \dots, w_m$ be a basis of $W$. The \qt{matrix of $T$} with respect to these bases is the $m$-by-$n$ matrix $\mmatrix(T)$ whose entries $A_{j,k}$ are defined by
  \begin{equation}
    T v_k = A_{1,k} w_1 + \cdots + A_{m,k} w_m \equiv \sum_{j=1}^{m} A_{j,k} w_j.
  \end{equation}
  If the bases are not clear from the context, one writes $\mmatrix(T, (v_1, \dots, v_n),(w_1, \dots, w_m))$. One remembers how to construct $\mmatrix(T)$, if one writes the bases on left hand side and on top of the matrix. It looks as follows:
  \begin{equation}
    \mathcal{M} (T) :\equiv
      \begin{blockarray}{cccccc}
                 & v_1 & \cdots & v_k     & \cdots & v_n \\
        \begin{block}{c(ccccc)}
          w_1    &     &        & A_{1,k} &        &     \\
          \vdots &     &        & \vdots  &        &     \\
          w_m    &     &        & A_{m,k} &        &     \\
        \end{block}
      \end{blockarray}
  \end{equation}
\end{mydef}

\subsection{Addition and Scalar Multiplication of Matrices}

% 3.35
\setcounter{thm}{34}
\begin{thm}[matrix of the sum of linear maps]
  $\mmatrix(S+T) = \mmatrix(S) + \mmatrix(T)$ for $S, T \in \linmap (V,W)$
\end{thm}

% 3.38
\setcounter{thm}{37}
\begin{thm}[matrix of a scalar times a linear map]
  $\mmatrix(\lambda T) = \lambda \mmatrix(T)$ for $\lambda \in \myF$ and $T \in \linmap (V,W)$
\end{thm}

\begin{mydef} [notation $\myF^{m,n}$]
  $\myF^{m,n}$ denotes the set of all $m$-by-$n$ matrices with entries in $\myF$, where $m,n \in \nat$
\end{mydef}

\begin{thm} % no []
  \label{thm: the dimension of the vector space of all m by n matrices is mn}
  $\myF^{m,n}$ is a vector space with $\dim \myF^{m,n}=mn$.
\end{thm}


\subsection{Matrix Multiplication}

Let $v_1, \ldots, v_n$ be a basis of $V$, $w_1, \ldots, w_m$ be a basis of $W$ and $u_1, \ldots, u_p$ be a basis of $U$.

Consider the linear maps $T: U \to V$ and $S: V \to W$. The composition $ST$ is a linear map from $U$ to $W$. Is it the case that
\begin{equation}
  \mmatrix(ST) = \mmatrix(S) \mmatrix(T) \; {\mathlarger ?}
\end{equation}

%One can choose a definition of matrix multiplication that forces this question to have a positive answer.

Later on, matrix multiplication will be defined accordingly such that this question has a positive answer.

Let $A :\equiv \mmatrix(S)\in \myF^{m,n}$ and $B :\equiv \mmatrix(T)\in \myF^{m,p}$. Here are the matrices of $A$ and $B$.
\begin{equation}
  A :\equiv \mathcal{M} (S) =
  \begin{blockarray}{cccccc}
    & v_1 & \cdots & v_r     & \cdots & v_n \\
    \begin{block}{c(ccccc)}
      w_1    &     &        & A_{1,r} &        &     \\
      \vdots &     &        & \vdots  &        &     \\
      w_m    &     &        & A_{m,r} &        &     \\
    \end{block}
  \end{blockarray},
  \quad
  B :\equiv \mathcal{M} (T) =
  \begin{blockarray}{cccccc}
    & u_1 & \cdots & u_k     & \cdots & u_p \\
    \begin{block}{c(ccccc)}
      v_1    &     &        & B_{1,k} &        &     \\
      \vdots &     &        & \vdots  &        &     \\
      v_n    &     &        & B_{n,k} &        &     \\
    \end{block}
  \end{blockarray}
\end{equation}

For $1 \leq k \leq p$, we have
\begin{equation}
  \begin{aligned}
    (ST) u_k
    = S (Tu_k)
    = S \left ( \sum_{r=1}^{n} B_{r,k} v_r \right )
    =  \sum_{r=1}^{n} B_{r,k} S v_r
    =  \sum_{r=1}^{n} B_{r,k} \sum_{j=1}^{m} A_{j,r}  w_j
    =  \sum_{j=1}^{m} \left( \sum_{r=1}^{n} A_{j,r}  B_{r,k} \right) w_j
  \end{aligned}
\end{equation}

\begin{minipage}{\linewidth-40pt}
  Thus $\mmatrix(ST)$ is  the $m$-by-$p$ matrix whose entry in row $j$, column $k$ equals
  $\sum_{r=1}^{n} A_{j,r}  B_{r,k}.$
\begin{equation}
  \mathcal{M} (ST) =
  \begin{blockarray}{cccccc}
             & u_1 & \cdots &  u_k      & \cdots & u_p \\
    \begin{block}{c(ccccc)}
      w_1    &     &        &
        \sum_{r=1}^{n} A_{1,r}  B_{r,k} &        &     \\
      \vdots &     &        &  \vdots   &        &     \\
      w_j    &     &        &
        \sum_{r=1}^{n} A_{j,r}  B_{r,k} &        &     \\
      \vdots &     &        &  \vdots   &        &     \\
      w_m    &     &        &
        \sum_{r=1}^{n} A_{m,r}  B_{r,k} &        &     \\
    \end{block}
  \end{blockarray}
\end{equation}
\end{minipage}


%3.41
\setcounter{thm}{40}
\begin{mydef} [matrix multiplication]
  \label{def: matrix multiplication}
  Let $A \in \myF^{m,n}$ and $B \in \myF^{n,p}$. Then $AB \in \myF^{m,p}$ whose entry in row $j$, column $k$, is given by the equation
  \begin{equation}
    (AB)_{j,k} :\equiv \sum_{r=1}^{n} A_{j,r} B_{r,k}.
  \end{equation} %for $A\in \myF^{m,n}$, $B\in \myF^{n,p}$, $AB \in \myF^{m,p}$
  \begin{minipage}{\linewidth}
    The octave code for calculating the entry $j,k$ of the result of $A, B$ with the \qt{sum-scheme} like above looks as follows

    \octave{standard_matrix_mult.m}
  \end{minipage}
\end{mydef}

% 3.43
\setcounter{thm}{42}
\begin{thm}[matrix of product of linear maps]
  If $T \in \linmap(U,V)$ and $S\in \linmap(V,W)$, then
  \begin{equation}
    \mmatrix(ST) = \mmatrix(S) \mmatrix(T).
  \end{equation}
\end{thm}
\begin{prf}
  See at the beginning of the sub-subsection.
\end{prf}

\begin{mydef} [notation: $A_{j, \mathsmaller{\bullet}}, A_{\mathsmaller{\bullet}, k}$ ] Let $A\in \myF^{m,n}$. Then
  \begin{itemize}
    \item If $1 \leq j \leq m$, then $A_{j, \mathsmaller{\bullet}}$ denotes the $1$-by-$n$ matrix consisting of row $j$ of $A$.
    \item If $1 \leq k \leq n$, then  $A_{\mathsmaller{\bullet}, k}$ denotes the $m$-by-$1$ matrix consisting of the column $k$ of $A$.
  \end{itemize}
\end{mydef}

% 3.46
\setcounter{thm}{45}
\begin{thm} [entry of matrix product equals row times column]
  If $A \in \myF^{m,n}$ and $B \in \myF^{n,p}$, then we have for $1 \leq j \leq m$ and $1 \leq k \leq p$
  \begin{equation}
    (AB)_{j,k} = A_{j, \mathsmaller{\bullet}} B_{\mathsmaller{\bullet}, k}
  \end{equation}

  \begin{minipage}{\linewidth}
    The octave code for multiplying rows and columns directly at entry $j,k$ looks as follows

    \octave{row_times_column_matrix_mult.m}
  \end{minipage}
\end{thm}

% 3.48
\setcounter{thm}{47}
\begin{thm}[column of matrix product equals matrix times column]
  If $A \in \myF^{m,n}$ and $B \in \myF^{n,p}$, then we have for $1 \leq k \leq p$
  \begin{equation}
    (AB)_{\mathsmaller{\bullet}, k} = AB_{\mathsmaller{\bullet}, k}
  \end{equation}

  In other words, column $k$ of $AB$ equals $A$ times column $k$ of $B$. The octave looks as follows

  \octave{matrix_times_columns_matrix_mult.m}

\end{thm}

\setcounter{thm}{49}
\begin{thm}
%  $b=\left(\begin{matrix}b_1\\ \vdots \\ b_n \end{matrix}\right)
  Suppose $A \in \myF^{m,n}$ and  $b = (b_1, \ldots, b_n)^{t}\in \myF^{n,1}$. Then
  \begin{equation}
    Ab = b_1 A_{\mathsmaller{\bullet}, 1} + \dots + b_n A_{\mathsmaller{\bullet},n}. \quad $See \ref{def: transpose} for the definition of $^t.
  \end{equation}
\end{thm}

\begin{thm}[matrix multiplication as linear comb. of columns or rows]
  \phantom{.} Let $C\in\myF^{m,c}$ and $R\in\myF^{c,n}$.

  \bfemph{Column-wise:} If $k \in \{1,\ldots,n\}$, then column $k$ of $CR\in \myF^{m,n}$ is a linear combination of the columns of $C$, with the coefficients of this linear combination coming from column $k$ of $R$ [$R_{\mathsmaller{\bullet}, k}$].
  \begin{equation}
    \begin{aligned}
      &CR = \\
      &\left(
      \underbrace{
      \left(
       \left(
          \begin{matrix}C_{1,1}
            \\ \vdots \\ C_{m,1}
          \end{matrix}
        \right) R_{1,1}
        + \cdots +
        \left(
          \begin{matrix}C_{1,c}
            \\ \vdots \\ C_{m,c}
          \end{matrix}
        \right) R_{c,1}
        \right)
      }_{\text{$1$st column}}
      \ldots
      \underbrace{
        \boxed{
          \left(
            \begin{matrix}C_{1,1}
              \\ \vdots \\ C_{m,1}
            \end{matrix}
          \right) R_{1,k}
          + \cdots +
          \left(
            \begin{matrix}C_{1,c}
              \\ \vdots \\ C_{m,c}
            \end{matrix}
          \right) R_{c,k}
        }
      }_{\text{$k$th column}}
      \ldots
      \underbrace{
        \left(
          \left(
            \begin{matrix}C_{1,1}
              \\ \vdots \\ C_{m,1}
            \end{matrix}
          \right) R_{1,n}
          + \cdots +
          \left(
            \begin{matrix}C_{1,c}
              \\ \vdots \\ C_{m,c}
            \end{matrix}
          \right) R_{c,n}
        \right)
      }_{\text{last column}}
      \right)
    \end{aligned}
  \end{equation}

\begin{minipage}{\linewidth}
  The octave\-/code for column-wise multiplication looks like follows.

  \octave{column_wise_matrix_mult.m}
\end{minipage}

  \bfemph{Row-wise:} If $j\in \{1, \ldots, m\}$, then row $j$ of $CR$ is a linear combination of the rows of $R$, with the coefficients of this linear combination comming from row $j$ of $C$ [$C_{j,\mathsmaller{\bullet}}$]
  \begin{equation}
    CR =
    \left(
      \begin{matrix}
        C_{1,1} (R_{1,1}, \ldots, R_{1,n}) + C_{1,2} (R_{2,1}, \ldots, R_{2,n}) + \cdots + C_{1,c} (R_{c,1}, \ldots, R_{c,n}) \\
        C_{2,1} (R_{1,1}, \ldots, R_{1,n}) + C_{2,2} (R_{2,1}, \ldots, R_{2,n}) + \cdots + C_{2,c} (R_{c,1}, \ldots, R_{c,n}) \\
        \vdots \\
        C_{j,1} (R_{1,1}, \ldots, R_{1,n}) + C_{j,2} (R_{2,1}, \ldots, R_{2,n}) + \cdots + C_{j,c} (R_{c,1}, \ldots, R_{c,n}) \\
        \vdots \\
        C_{m,1} (R_{1,1}, \ldots, R_{1,n}) + C_{m,2} (R_{2,1}, \ldots, R_{2,n}) + \cdots + C_{m,c} (R_{c,1}, \ldots, R_{c,n}) \\
      \end{matrix}
    \right)
  \end{equation}

  \begin{minipage}{\linewidth}
    The octave\-/code for row-wise multiplication looks like follows.

    \octave{row_wise_matrix_mult.m}
  \end{minipage}

\end{thm}

\subsection{Column-Row Factorization and Rank of a Matrix}

% 3.52
\setcounter{thm}{51}
\begin{mydef}[column rank, row rank]
  For $A \in \myF^{m,n}$, the \qt{column rank} of $A$ is the dimension of the span of the columns of $A$ in $\myF^{m,1}$. The \qt{row rank} of $A$ is the dimension of the span of the rows of $A$ in $\myF^{1,n}$.
\end{mydef}

% 3.54
\setcounter{thm}{53}
\begin{mydef}[transpose, $A^t$]
  \label{def: transpose}
  The \qt{transpose} of a matrix $A\in \myF^{m,n}$, denoted by $A^t\in \myF^{n,m}$, is obtained by interchanging rows and columns. Its entries are given by the equation
  \begin{equation}
    \left( A^{t} \right)_{k,j} = A_{j,k} \quad \forall k \in \{1, \ldots, m\}, j \in \{1, \ldots, n\}.
  \end{equation}
\end{mydef}

% 3.56
\setcounter{thm}{55}
\begin{thm}[column-row factorization]
  Suppose $A \in \myF^{m,n}$ with column rank $c \geq 1.$ $\implies \exists C\in \myF^{m,c}$ and $R\in \myF^{c,n}$ such that
  \begin{equation}
    A = CR
  \end{equation}
\end{thm}

 % 3.57}
\begin{thm}[column rank and row rank]
  $A\in \myF^{m,n} \implies$ column rank of $A$ $=$ row rank of $A$.
\end{thm}

\begin{mydef} [rank]
  The \qt{rank} of a matrix $A \in  \myF^{m,n}$ is the column rank of $A$.
\end{mydef}
  \section{Invertibility and Isomorphisms}

\subsection{Invertible Linear Maps}
  \setcounter{thm}{58}
  %\ 3.59
  \begin{mydef} [invertible, inverse]
    $S\in \linmap(W,V)$ with $ST=I_V \myand TS = I_W$ is called the \qt{inverse} of the \qt{invertible} linear map $T \in \linmap(V,W)$.
  \end{mydef}

  % 3.60
  \begin{thm} [inverse is unique]
    An invertible  linear map has a unique inverse.
  \end{thm}
  \begin{prf}
    Suppose $T\in \lvw$ is invertible and $S_1$ and $S_2$ are inverses of $T$. Hence,
    \begin{equation}
      S_1 = S_1 I = S_1 (T S_2) = (S_1 T) S_2 = I S_2 = S_2.
    \end{equation}

    Therefore, $S_1 = S_2$.
  \end{prf}

  % 3.61
  \begin{mydef} [notation $T^{-1}$]
    Since the inverse is unique, the inverse is denoted by $T^{-1}\in \linmap(W,V)$ for $T\in \linmap(V,W).$ We have
    \begin{equation}
      T^{-1}T=I_V$ and $T T^{-1} = I_W.
    \end{equation}
  \end{mydef}

  %\setcounter{thm}{62}
  %\textbf{3.63}
  \begin{thm} [invertibility, injectivity and surjectivity]
    A linear map is invertible $\iff$ it is injective and surjective.
  \end{thm}

  % 3.65
  \setcounter{thm}{64}
  \begin{thm} [injectivity is equivalent to surjectivity]
    \label{thm: injectivity is equivalent to surjectivity}
    For $T\in \linmap(V,W)$, if $V$ and $W$ are finite-dimensional, injectivity is equivalent to surjectivity:
    \begin{equation}
        T$ is invertible $\iff$ $T$ is injective $\iff$ $T$ is surjective.$
    \end{equation}
  \end{thm}
  \begin{prf}
    The rank-nullity theorem (\ref{rank-nullity-theorem}) or fundamental theorem of linear maps states that
    \begin{equation}
      \label{rank-nullity-nested-equation}
      \dim V = \dim \mynull T + \dim \myrange T
    \end{equation}

    If $T$ is injective, $\dim \mynull T = 0$ and therefore
    \begin{equation}
      \dim \myrange T = \dim V = \dim W,
    \end{equation}
    which means $T$ is surjective ($\myrange T = \dim W$). This is because every linearly independent list of vectors of of length $\dim W$ is a basis by
    \ref{thm: linearly independent list of the right length is a basis}. And thats how dimension is defined. (TODO)

    If $T$ is surjective, we have $\dim \myrange T = \dim W$ to begin with and therefore \eqref{rank-nullity-nested-equation} becomes
    \begin{equation}
      \dim \mynull T = \dim V - \dim \myrange T = \dim V - \dim W = 0.
    \end{equation}

    which makes $T$ also injective.
    Since injectivity and surjectivy together imply invertibility, this ends the proof.
  \end{prf}


  % 3.68
  \setcounter{thm}{67}
  \begin{thm} [$ST = I \iff TS=I$ on vector spaces of the same dimension]
    \label{thm: ST = I iff TS=I - on vector spaces of the same dimension}
    Suppoe $V$ and $W$ are finite-dimensinal vector spaces of the same dimension. Let $S \in \linmap (W,V)$, and $T \in \linmap(V,W)$. Then $ST = I$ $\iff$ $TS = I$.
  \end{thm}
  \begin{prf}
     \Rightarrowdirection Let $S \in \linmap (W,V)$, and $T \in \linmap(V,W)$ such that $ST = I$. If $v \in V$ and $Tv = 0$, then
     \begin{equation}
       v = Iv = (ST)v = S(Tv) = S(0) = 0.
     \end{equation}

     Thus, $\mynull T = \{0\}$ and therefore, $T$ is injective by \ref{thm: injectivity iff null space equals zero-set}. Because $V$ and $W$ have the same dimension, $T$ is invertible by \ref{thm: injectivity is equivalent to surjectivity}. Now, multiply both sides of the equation $ST = I$ by $T^{-1}$ to conclude:
     \begin{equation}
       S = T^{-1}.
     \end{equation}

     Hence, $TS = TT^{-1} = I$, as desired.

     \Leftarrowdirection Right above, we have shown that $ST = I$ implies that $TS=I$. To prove the implication in the other direction, simply reverse the roles of $S$ and $T$ and $V$ and $W$, showing that $TS=I$ implies that $ST=I$.
  \end{prf}

  \subsection{Isomorphic Vector Spaces}

  %\textbf{3.69}
  \begin{mydef}
    An isomorphism is an invertible linear map. Two vector spaces are called isomorphic if there is a isomorphism from one vector space to the other. $T:V\to W$
  \end{mydef}

  %\textbf{3.70}
  \begin{thm}[dimension shows whether vector spaces are isomorphic]
    \label{thm: dimension shows whether vector spaces are isomorphic}
    Two \fd vector spaces $U$, $V$ over $\mathbb{F}$ are isomorphic $\iff$ they have the same dimension.
    \begin{equation}
      \dim U = \dim V.
    \end{equation}
  \end{thm}

  %\textbf{3.71}
  \begin{thm}
    \label{thm: the space of linear maps and the space of all matrices are isomorphic}
    Suppose $\oneTillN{v}$ is a basis of $V$ and $\oneTillM{w}$ is a basis of $W$. Then $\mathcal{M}$ is an isomorphism between $\lvw$ and $\mathbb{F}^{m,n}$.
  \end{thm}

  %\textbf{3.72}
  \begin{thm}
    Suppose $V,W$ are \fd.
    \begin{equation}
      \implies \lvw $ is \fd and $\dim \lvw = (\dim V)\cdot(\dim W)
    \end{equation}
  \end{thm}
  \begin{prf}
    The desired result follows from \ref{thm: the space of linear maps and the space of all matrices are isomorphic}, \ref{thm: dimension shows whether vector spaces are isomorphic}, and \ref{thm: the dimension of the vector space of all m by n matrices is mn}. (Why?)
  \end{prf}

  \subsection{Linear Maps Thought of as Matrix Multiplication}

  \begin{mydef} [matrix of a vector, $\mmatrix(u)$]
    Let $u \in V$ and $\onetillm{v}$ be a basis of $V$ such that
  \begin{equation}
    u=b_1v_1+\dots+b_mv_m. $ We define $
        \mathcal{M}(u) :\equiv
      \left (
      \begin{matrix}
        b_1 \\ \vdots \\ b_m
      \end{matrix}
      \right ).
  \end{equation}

    The vector depends on the basis, but it is not included in the notation.
%    Recall:
%\begin{equation}
%    \mathcal{M} (T) :\equiv
%  \begin{blockarray}{cccc}
%    & v_1 \quad \cdots & v_k & \cdots \quad v_n \\
%    \begin{block}{c(ccc)}
%      w_1    & & A_{1,k} & \\
%      \vdots & & \vdots & \\
%      w_m    & & A_{m,k} & \\
%    \end{block}
%  \end{blockarray}
%\end{equation}

  \end{mydef}

  \setcounter{thm}{74}
  %\textbf{3.75}
  \begin{thm} [$\mmatrix(T)_{\mathsmaller{\bullet}, k} = \mmatrix(T v_k)$]
    Suppose $T \in \linmap(V,W)$ and $v_1, \ldots, v_n$ is a basis of $V$ and $w_1, \ldots, w_m$ is a basis of $W$. Let $1\leq k \leq n$. Then the $k^{\text{th}}$ column of $\mmatrix(T)$, which is denoted by $\mmatrix(T)_{\mathsmaller{\bullet}, k}$, equals $\mmatrix(Tv_k)$.
    \begin{equation}
      \mmatrix(T)_{\mathsmaller{\bullet}, k}=\mmatrix(Tv_k) \quad \forall k \in \{1, \ddd, n\}.
    \end{equation}
%    $\mmatrix_{\mathsmaller{\bullet}, k} = M(T v_k)$. The $k^{\text{th}}$ column of $\mmatrix_{\mathsmaller{\bullet}, k}$ equals $(A_{1,k}, \dots, A_{m,k})^\top$
  \end{thm}

  %3.76
  \begin{thm} [linear maps act like matrix multiplication]
    Suppose $T\in \lvw$ and $v\in V$. Suppose $\onetilln{v}$ is a basis of $V$ and $\onetillm{w}$ is a basis of $W$. Then
    \begin{equation}
      \mathcal{M} (Tv)=\mathcal{M}(T) \cdot \mathcal{M}(v)
    \end{equation}

    Or using different notation:
    \begin{equation}
      \mathcal{M} (Tv, (\onetilln{v}), (\onetillm{w}))=\mathcal{M}(T, (\onetilln{v}), (\onetillm{w})) \cdot \mathcal{M}(v)
    \end{equation}

  \end{thm}

  % 3.78
  \setcounter{thm}{77}
  \begin{thm} [dimension of $\myrange T$ and column rank of $\mmatrix(T)$]
    Suppose $\dim V < \infty$ and $\dim W < \infty$. Then for $T \in \linmap(V,W):$
    \begin{equation}
      \dim \myrange T = $ column rank of $ \mmatrix (T)
    \end{equation}
  \begin{mydef-non}
  $\mathcal{M}(T, (\onetilln{v})) :\equiv \mathcal{M}(T, (\onetilln{v}),(\onetilln{v}))$
\end{mydef-non}
  \end{thm}

  \subsection{Change of basis}


  % 3.80
  \setcounter{thm}{79}
  \begin{mydef}[invertible, inverse, $A^{-1}$]
    A square matrix $A$ is called invertible, if there is some square matrix B of the same size such that
    \begin{equation}
    	AB=BA=I.
    \end{equation}

    We call $B$ the inverse of $A$ denoted by $A^{-1} :\equiv B$. Rules:
    \begin{itemize}
      \item $(A^{-1})^{-1}=A$
      \item $(AC)^{-1} = C^{-1}A^{-1}$ (Because $(AC)(C^{-1}A^{-1})=I$ and $(C^{-1}A^{-1})(AC)=I$)
    \end{itemize}
  \end{mydef}

  % 3.81
  \begin{thm} [matrix of product of linear maps]
    Suppose $T\in \mathcal{L}(U,V)$ and $S\in \lvw$. If $\onetillm{u}$ is a basis of $U$, $\onetilln{v}$ is a basis of $V$ and $\onetill{w}{p}$ is a basis of $W$. Note that $\dim U = m$, $\dim V = n$, $\dim W = p$. Then we have:
    \begin{equation}
    \begin{aligned}
      &\mmatrix(ST, (\onetillm{u}), (\onetill{w}{p})) = \\
      &\qquad \mmatrix(S, (\onetilln{v}), (\onetill{w}{p} ))
      \cdot
      \mmatrix(T, (\onetillm{u}), (\onetilln{v}   ))
    \end{aligned}
    \end{equation}

    Or using different notation:
    \begin{equation}
      \mathcal{M}(ST) = \mathcal{M}(S) \cdot \mathcal{M}(T)
    \end{equation}
  \end{thm}

  % 3.84
  \setcounter{thm}{83}
  \begin{thm}[change-of-basis formula:]
    Suppose $T \in \linmap(V)$. Let
    \begin{equation}
          V = \myspan{\onetillm{u}}
          $ and $V= \myspan{\onetillm{v}}
    \end{equation}

  such that the $u$'s and $v$'s both form a basis. Let $A=\mmatrix(T, (\onetillm{u}))$ and $B=\mmatrix(T,(\onetillm{v})).$

   Let $C=\mmatrix(I, \onetillm{u}, \onetillm{v}).$ Then
    \begin{equation}
    A = C^{-1} B C
    \end{equation}
  \end{thm}

  % 3.86
  \setcounter{thm}{85}
  \begin{thm}
    If $\onetillm{v}$ is a basis $V$ and $T\in \mathcal{L}$ is invertible, then
    \begin{equation}
      \begin{aligned}
        \mmatrix(T^{-1}) & = (\mmatrix(T))^{-1} \; \text{or with different notation} \\
        \mmatrix(T^{-1}, (\onetillm{v})) & =\mmatrix(T, (\onetillm{v}))^{-1}
      \end{aligned}
    \end{equation}
  \end{thm}
  \section{Products and Quotients of Vector Spaces}
\subsection{Products of Vector Spaces}

\begin{mydef}The product, addition and scalar multiplication of vector spaces $V_1, \cdots, V_m$ is defined as follows:
  \begin{equation}
    \begin{aligned}
      V_1 \times \cdots \times V_m &:\equiv \{ (v_1, \dots, v_m) \mid v_1 \in V_1, \dots, v_m \in V_m\} \\
      (u_1, \dots, u_m) + (v_1, \dots, v_m) &:\equiv (u_1+v_1, \dots, u_m+v_m) \\
      \lambda (v_1, \dots, v_m) &:\equiv (\lambda v_1, \dots, \lambda v_m)
    \end{aligned}
  \end{equation}
\end{mydef}

\setcounter{thm}{88}
%\textbf{3.89}
\begin{thm}
  $V_1 \times \cdots \times V_m$ together with addition and scalar multiplication is a vector space over $\mathbb{F}$.
\end{thm}

\setcounter{thm}{91}
\begin{thm}
  $\dim (V_1 \times \cdots \times V_m) = \dim V_1 + \cdots + \dim V_m$
\end{thm}

%\textbf{3.93}
\begin{thm}
  Let $\Gamma: V_1 \times \cdots \times V_m \to V_1 + \cdots + V_m$ and
  $ \Gamma(v_1, \cdots, v_m) \mapsto v_1 + \cdots + v_m$ \\
  Then $v_1 + \cdots + v_m$ is a direct sum $\iff$ $\Gamma$ is injective.
\end{thm}

\begin{thm}$V_1 + \cdots + V_1$ is a direct sum $\iff$
  $\dim (V_1+\cdots+V_m) = \dim V_1 + \cdots + \dim V_m$
\end{thm}

\subsection{Quotient Spaces}

\setcounter{thm}{94}
%\textbf{3.95, 3.97}
\begin{mydef}
  $v+U :\equiv \{v+U \mid u\in U\}$ for $v\in V$ and $U\subseteq V$ is said to be ``a translate" of $U$.
\end{mydef}

\setcounter{thm}{98}
%\textbf{3.99}
\begin{mydef}
  $V/U :\equiv \{v+U \mid v\in V\}$ is called ``quotient space".
\end{mydef}

\begin{example}
  If $U=\{ (x,2x)\in \mathbb{R}^2 \mid x\in \mathbb{R} \} \implies \mathbb{R}/U$ is the set of all lines with slope $2$.
\end{example}

\begin{example}
  If $U$ is a plane in $\mathbb{R}^3$ $\implies$ $\mathbb{R}^3/U$ is the set of all planes parallel to $U$.
\end{example}


\setcounter{thm}{100}
%\textbf{3.101}
\begin{thm}
  $ U \subseteq V$ and $v,w\in V$. Then
  \begin{equation}
    v-w \in U \iff v+U = w + U \iff (v+U) \cap (w+U) \neq \varnothing
  \end{equation}
  That means, two translates of a subspaces are equal or disjoint.
\end{thm}

%\textbf{3.102}
\begin{mydef}
  Definition of addition and scalar multiplication on $V/U$.
  \begin{equation}
    \begin{aligned}
      (v+U)+(w+U) & :\equiv (v+w) + U \\
      \lambda (v+U) & :\equiv (\lambda v) + U \qquad \forall v,w \in V \text{ and } \forall \lambda \in \mathbb{F}
      \text{ and } U \subseteq V
    \end{aligned}
  \end{equation}
\end{mydef}


%\textbf{3.103}
\begin{thm}
  $V/U$ is a vector space with additive identity $0+U$ which is equal to $U$ and the additive inverse $(-v)+U$.
\end{thm}

\textbf{3.104} TODO \textbf{3.105} TODO \\
  
\section{Duality}
\subsection{Dual Space and Dual Map}

\setcounter{thm}{107}
%\textbf{3.108}
\begin{mydef}
  A ``linear functional'' $\phi$ is an element of $\mathcal{L}(V, \mathbb{F})$. So $\phi \in \linmap(V, \myF)$
\end{mydef}

\textbf{Examples:}
\begin{equation}
  \begin{array}{lll}
    \phi: \mathbb{R}^3  \to \mathbb{R}, &\phi (x,y,z)  & \mapsto 4x-5y-2z \\
    \phi: \mathbb{F}^n  \to \mathbb{F}, &\phi (x_1, \dots, x_n)
    & \mapsto c_1x_1 + \dots + c_nx_n  \\
    \phi: \mathcal{P} (\mathbb{R})  \to \mathbb{R},
    & \phi(p) & \mapsto 3p''(5) + 7p(4) \\
    \phi: \mathcal{P}(\mathbb{R}) \to \mathbb{R},
    & \phi(p)  &\mapsto \textstyle \int_{0}^{1} p(x) dx
  \end{array}
\end{equation}

\setcounter{thm}{109}
%\textbf{3.110}
\begin{thm}
  The dual space of $V$, denoted by $V'$ or $V^{*}$, is the vector space of all linear functionals on $V$.
  \begin{equation}
    V^{*} :\equiv \linmap(V, \myF)
  \end{equation}
\end{thm}

%\textbf{3.111}
\begin{thm}
  $\dim V^{*} = \dim V$.
\end{thm}
\begin{prf}
  $\dim V^{*} = \dim \mathcal{L}(V, \mathbb{F})=(\dim V) \cdot (\dim \mathbb{F}) = \dim V $
\end{prf}


%\textbf{3.112}
\begin{mydef}
  If $\onetillm{v}$ is a basis of $V$, then the ``dual basis'' of $\onetilln{v}$ is the list $\varphi_1, \dots, \varphi_m \in V^{*}$, where each $\varphi_j$ is the linear functional such that
  \begin{equation}
    \varphi_j(v_k) = \delta_{j,k} =
  \begin{cases}
    1,  & \text{if $k=j$} \\
    0, & \text{if $k \neq j$}
  \end{cases}
  \end{equation}
  Note that this is not the definition of each $\varphi_j$.
\end{mydef}

%\textbf{3.114}
\setcounter{thm}{113}
\begin{thm}
  Suppose $\onetillm{v}$ is a basis of $V$. $\onetillm{\varphi}$ is called ``a dual basis'' of $\onetillm{v}$ because for every $v \in V$ we have $v=\varphi_1 (v)v_1 + \cdots \varphi_m(v)v_m$
\end{thm}
\begin{prf}
  Let $V \ni v= c_1v_1 + \cdots c_mv_m$. Applying
  $\varphi_j$ on both sides gives $\varphi_j(v)=c_j$ $\forall j\in \{1, \dots, m\}$
  %    \iff$ $\varphi(v)=\phi(c_1v_1 + \cdots + c_mv_m)$ such that $\varphi_j(v)=c_j$, because $\varphi$ is linear and $\varphi_j(v_j)=1$ by definition.
\end{prf}

%\textbf{3.116}
\setcounter{thm}{115}
\begin{thm}
  The dual basis of a basis of $V$ is a basis of the dual space $V^{*}$
\end{thm}

%\textbf{3.118}
\setcounter{thm}{117}
\begin{mydef}

  The ``dual map'' $T^{*}$ of $T \in \linmap(V,W)$ is the linear map $T^{*} \in \lin{W^{*}}{V^{*}}$ defined like this:

  \begin{itemize}
    \item[] $T^{*}(\varphi) :\equiv \varphi \circ T \quad \forall \phi \in W^{*}$
    \item $\varphi \in W^{*}=\lin{W}{\mathbb{F}} \text{ and } T^{*}(\varphi) \in V^{*} = \lin{V}{\mathbb{F}}$.

    yo
    So $T^{*}$ is indeed a map from $W^{*}$ to $V^{*}$


    \item $\varphi, \psi \in W^{*} \implies T^{*} (\varphi + \psi) = (\varphi + \psi) \circ T = \varphi \circ T + \psi \circ T = T^{*} (\varphi) + T^{*} (\psi)$
    \item $\lambda \in \mathbb{F}, \varphi \in W \implies T^{*} (\lambda \varphi) = (\lambda \varphi) \circ T = \lambda (\varphi \circ T) = \lambda T^{*} (\varphi)$
  \end{itemize}

\end{mydef}

\setcounter{thm}{119}
\begin{thm}
  \label{algebraic-properties-of-dual-maps}
  Algebraic properties of dual maps:  \\
  $T \in \linmap(V,W)$ $\implies$
  \begin{enumerate}
    \item $(S+T)' = S' + T' \quad \forall S \in \linmap(V,W)$
    \item $(\lambda T)' = \lambda T' \quad \quad \forall \lambda \in \myF$
    \item $(ST)' = T'S' \quad \; \forall S \in \linmap (W,U)$
  \end{enumerate}
\end{thm}



\subsection{Dual Space and Range of Dual of Linear Map}

\begin{mydef} [annihilator, $U^0$]
  \label{def: annihiltator}
  For $U \subseteq V$, the ``annihilator'' of $U$, denoted by $U^{0}$, is defined by
  \begin{equation}
    U^0 = \{ \varphi \in \dual{V} \; \mid \; \varphi (u) = 0 \quad \forall u \in U \}
  \end{equation}
\end{mydef}

% 3.124
\setcounter{thm}{123}
\begin{thm}[the annihilator is a subspace]
  \label{thm: the annihilator is a subspace}
  $U \subseteq V \implies U^{0}$ is a subspace of $V{\mathlarger{\mathlarger{'}}}$
\end{thm}

% 3.125
\setcounter{thm}{124}
\begin{thm} [dimension of the annihilator]
    $U\subseteq V$ and $\dim V < \infty \implies$
    \begin{equation}
      \dim U^0 = \dim V - \dim U.
    \end{equation}
\end{thm}

% 3.127
\setcounter{thm}{126}
\begin{thm} [condition for the annihilator to equal $\{0\}$ or the whole space]
  If $\dim V < \infty$ and $U\subseteq V$. Then
  \begin{enumerate}
    \item $U^0 = \{0 \} \iff U = V$;
    \item $U^0 = \dual{V} \iff U = \{0\}$;
   \end{enumerate}
\end{thm}

% 3.128
\begin{thm}[the null space of $\dual{T}$]
  If $\dim V < \infty$, $\dim W<\infty$ and $T \in \linmap (V,W),$ then
  \begin{enumerate}
    \item $\mynull \dual{T} = (\myrange T)^0$;
    \item $\dim \mynull \dual{T} = \dim \mynull T + \dim W - \dim V$
  \end{enumerate} 
\end{thm}

% 3.129
\begin{thm}[$T$ surjective, $\dual{T}$ injective]
    If $\dim V < \infty$, $\dim W<\infty$ and $T \in \linmap (V,W),$ then
  \begin{equation}
    T$ is surjective $ \iff \dual{T} $ is injective$
  \end{equation}
\end{thm}

% 3.130
\begin{thm}[the range of $\dual{T}$]
    If $\dim V < \infty$, $\dim W<\infty$ and $T \in \linmap (V,W),$ then
    \begin{enumerate}
      \item $\dim \myrange \dual{T} = \dim \myrange T$;
      \item $\myrange \dual{T} = (\mynull T)^0$;
    \end{enumerate}
\end{thm}

% 3.131
\begin{thm} [$T$ injective, $\dual{T}$ surjective]
    If $\dim V < \infty$, $\dim W<\infty$ and $T \in \linmap (V,W),$ then
    \begin{equation}
      T $ is injective $ \iff \dual{T} $ is surjective. $
    \end{equation}
\end{thm}

\subsection{Matrix of Dual of Linear Map}

% 3.132
\setcounter{thm}{131}
\begin{thm}
  Suppose $V, W$ are finite-dimensional and $T \in \linmap(V,W)$.
  \begin{equation}
    \implies \mmatrix(T^{*}) = (\mmatrix(T))^{\top}
  \end{equation}
\end{thm}

%3.133
\begin{thm}
  If $A \in myF^{m,n}$, then the column rank of $A$ equals the row rank of $A$.
\end{thm}

  \chapter{Polynomials}
  
$\mathbb{F}$ denotes $\mathbb{R}$ or $\mathbb{C}$.

\section{Zeros of polynomials}
$p: \mathbb{F} \to \mathbb{F}, \, p(z) = a_0 + a_1z + \cdots + a_n z^m$

\setcounter{thm}{4}
\begin{mydef}[zero of a polynomial]
  $\lambda \in \mathbb{F}$ is called a \qt{zero}  (or \qt{root}) of a polynomial if $p(\lambda) = 0$.
\end{mydef}

\setcounter{thm}{5}
\begin{thm}[each zero of a polynomial corresponds to a degree\-/one factor]
  \label{factororing-out-zeros-of-a-polynomial-always-possible}
  $\forall p \in \mathcal{P}_m(\mathbb{F}): \; \exists \lambda \in \mathbb{F}$ s.t.
  \begin{equation}
  	p(\lambda) = 0 \iff \exists q \in \mathcal{P}_{m-1}: p(z) = (z-\lambda)q(z)
  \end{equation}
\end{thm}
\begin{prf} Let $p\in \polyn(\myF)_m$ for $m\in \nat$.
  
  ``$\Leftarrow$ direction:'' Suppose there is $q \in \polyn(\myF)$ s.t. $p(z) = (z-\lambda)q(z) \quad \forall z \in \myF$. Then
  \begin{equation}
    p(\lambda) = (\lambda -\lambda)q(\lambda) = 0,$ as desired.$
  \end{equation}
  
  ``$\Rightarrow$ direction:'' Now suppose $p(\lambda)=0$. Let $a_0, a_1, \ldots, a_m \in \myF"$ be s.t.
  \begin{equation}
    p(z) :\equiv a_0 + a_1z + \cdots a_mz^m \quad \forall z \in  \myF.
  \end{equation}
  
  Then
  \begin{equation}
    \label{eq: second formula of pz: pz = pz - plambda}
    p(z) = p(z)- p(\lambda) = a_1 (z-\lambda) + a_2 (z^2 - \lambda^2) + \cdots+ a_m (z^m- \lambda^m) \quad \forall z \in \myF.
  \end{equation}
  
  Using the binomic formula 
  $z^k - \lambda^k = (z-\lambda) \sum_{j=1}^{k} \lambda^{j-1}z^{k-j} \quad \forall k\in \setonetillm$, we get 
  \begin{equation}
    p(z) = (z-\lambda)
    \underbrace{
      \left(
        a_1 + a_2 (z+\lambda) + a_3(z^2 + \lambda z + \lambda^2) +
        a_4 (z^3 + \lambda z^2 + \lambda^2 z + \lambda^3) 
        + \cdots +
        a_m \sum_{j=1}^{m} \lambda^{j-1} z^{m-j}
      \right)}_{\text{a polynomial with degree $m-1$}}
  \end{equation}
  
  Thus $p$ equals $z-\lambda$ times some polyomial of degree $m-1$, as desired.
\end{prf}

In case you are wondering where the $a_0$ went by subtracting $p(\lambda)$ from $p(z)$, it is hidden int the roots! For example $p\in \polyn_1(\myF): p=a_0+a_1z = \left(z- \left(-\frac{a_0}{a_1}\right)\right)(a_1)$ where $\left(-\frac{a_0}{a_1}\right)$ is the only root!

\setcounter{thm}{7}
\begin{thm}
  A polynomial of degree $m$ has at most $m$ zeros.
\end{thm}
\begin{prf} 
  For $m=1$, $a_0+a_1z$ has only one zero, which is $-\frac{a_0}{a_1}$\\
  Now for $m>1$ let us assume the desired result holds for $m-1$. 
  
  If $p\in \mathcal{P}_{m}(\myF)$ has no zeros, we are done. 
  
  If $p$ has a zero $\lambda \in \myF$, by \ref{factororing-out-zeros-of-a-polynomial-always-possible}, there is a $q \in \mathcal{P}(\myF)_{m-1}$ such that 
  \begin{equation}
  	p(z)=(z-\lambda)q(z).
  \end{equation}
   
  Since $q$ has at most $m-1$ zeros, $p$ has at most $m$ zeros.
\end{prf}

\section{Division Algorithm for Polynomials}

\setcounter{thm}{8}
\begin{thm}
  \label{division-algorithm-for-polynomials}
  Suppose $p,s \in \mathcal{P} (\mathbb{F}), \, s\neq 0$. 
  Then $\exists q,r \in \mathcal{P} (\mathbb{F})$ s.t.
  \begin{equation}
    p=sq+r \myand \deg r < \deg s.
  \end{equation}
\end{thm}
\begin{prf}
  Let $n:\equiv \deg p$ and $m:\equiv \deg s$. 
  
  If $n<m$, then take $q=0$ and $r=p$.
  
  Thus now we assume $n \geq m$. The list
  \begin{equation}
    \label{eq: basis of P_n(F)}
    1, z, \ldots, z^{m-1}, s, zs, \ldots, z^{n-m}s
  \end{equation}
  
  is linearly independent because each polynomial in this list has a different degree.
  
  The list has length $1+(m-1)+1+(n-m)=n+1$, which equals $\dim \polyn_n(\myF)$. Hence \eqref{eq: basis of P_n(F)} is a basis of $\polyn_n(\myF)$ by \ref{thm: linearly independent list of the right length is a basis}. Therefore, we have $a_0, a_1, \ldots, a_{m-1} \in \myF$ and $b_0, b_1, \ldots, b_{n-m} \in \myF$ s.t.
  \begin{equation}
    \begin{aligned}
          p&=a_0 + a_1z + a_2z^2 + \cdots + a_{m-1} z^{m-1}
          + b_0 s + b_1 zs + b_2 z^2 s + \cdots + b_{n-m}z^{n-m}s \\
           &=\underbrace{\left(a_0 + a_1z + a_2z^2 + \cdots + a_{m-1} z^{m-1} \right)}_{r} + \underbrace{\Big(s\Big)}_{s} \underbrace{\left( b_0 + b_1 z + b_2 z^2 + \cdots + b_{n-m} z^{n-m} \right)}_{q} \\
           &=r+sq=sq+r, \mytext{with} \deg r = m-1 \leq \deg s = m,
    \end{aligned}
  \end{equation}
  as desired.
  
  The uniqueness of $q,r \in \polyn(\myF)$ follows from the uniqueness of the $a$'s and $b$'s, because \eqref{eq: basis of P_n(F)} is a basis of $\polyn_n(\myF)$.
\end{prf}
\section{Factorization of Polynomials over $\compl$}
\setcounter{thm}{11}
\begin{thm}[\emph{fundamental theorem of algebra}, first version]
  \label{fundamental-theorem-of-algebra-first-version}
  Every non\-/constant polynomial with complex coefficients has a zero in $\mathbb{C}$.
\end{thm}

\begin{thm} [\emph{fundamental theorem of algebra}, second version:]
  \label{fundamental-theorem-of-algebra-second-version}
  $p \in \polyn(\mathbb{C}), \; p$ is non-constant $\implies p$ has a unique factorization
  \begin{equation}
  	p(z)=c \cdot (z-\lambda_1) \cdots (z-\lambda_m) 
  \end{equation}
  where $c, \lambda_1, \dots, \lambda_m \in \compl$.
\end{thm}

\section{Factorization of Polynomials over $\real$}

% 4.14
\begin{thm}[polynomials with real coefficients have nonreal zeros in pairs]
  \label{thm: polynomials with real coefficients have nonreal zeros in pairs}
  Suppose $p\in \mathcal{P} (\mathbb{C})$ is a polynomial with real coefficients. If $\lambda \in \mathbb{C}$ is a zero of $p$, then so is $\overline{\lambda}$.
\end{thm}
\begin{prf}
  Let $p(z) :\equiv a_0 + a_1 z + \ldots + a_m z^m$, where $a_0, \ldots, a_m \in \real$. Suppose $\lambda \in \compl$ is a zero of $p$. Then 
  \begin{equation}
    \begin{aligned}
      a_0+a_1\lambda+\cdots+a_m\lambda^m&=0\\
      \overline{a_0+a_1\lambda+\cdots+a_m\lambda^m}&=\overline{0}\\
      \overline{a_0}+\overline{a_1\lambda}+\cdots+\overline{a_m\lambda^m}&=0\\
      a_0+a_1\overline{\lambda}+\cdots+a_m\overline{\lambda}^m&=0
    \end{aligned}
  \end{equation}
  
  The equation above shows that $\overline{\lambda}$ is a zero of $p$.
\end{prf}

\begin{thm} [factorization of a quadratic polynomial]
  \label{thm: factorization of a quadratic polynomial}
  Let $p(x)=x^2+bx+c \in \polyn_2(\real)$ such that $b,c \in \real$. Then $p$ can be written as
  $p(x)=x^2 + bx + c$ $=$ $(x-\lambda_1)(x-\lambda_2)$ with $\lambda_1, \lambda_2 \in \mathbb{R} \iff b^2 \geq 4c$.
\end{thm}
\begin{prf}
  Let $x^2+bx+c=(x+\sfrac{b}{2})^2+(c-\sfrac{b^2}{4})$. Therefore, if $b^2<4c$, the last term is always positive and no factoriza\-tion is possible because it has no zeros by \ref{factororing-out-zeros-of-a-polynomial-always-possible}.
  
  If $b^2 \geq 4c$, we define $d^2 :\equiv \sfrac{b^2}{4}-c$ which makes
  \begin{equation}
	  \begin{aligned}
	    p(x)&=x^2+bx+c
		  = \left (x+\sfrac{b}{2} \right )^2-d^2
		  = \left ( (x+\sfrac{b}{2}) +d \right) \cdot \left ( (x+\sfrac{b}{2}) - d \right) \\
		  &= \left (x-(-d -\sfrac{b}{2}) \right) \cdot \left (x-(d-\sfrac{b}{2})\right)
		  =\left(x-\lambda_1 \right) \cdot \left(x-\lambda_2\right)
	  \end{aligned}
  \end{equation}
  for 
  \begin{equation}
    \lambda_1 :\equiv -d-\sfrac{b}{2} \mytext{and} \lambda_2 :\equiv d-\sfrac{b}{2}.
  \end{equation}
  
  which is the same as the well known midnight formula:
  \begin{equation}
    \lambda_{1,2} = \frac{-b \pm \sqrt{b^2-4c}}{2}
  \end{equation}
  which also shows, that $b^2 \geq 4c$ is a necessary condition for real solutions for $\lambda_{1,2}$
\end{prf}

\begin{thm}
  Suppose $p \in \mathcal{P}(\mathbb{R})$ is non-constant. Then $p$ has a unique factorization:
  \begin{equation}
    p(x) = c(x-\lambda_1) \cdots (x-\lambda_m)(x^2+b_1x+c_1) \cdots (x^2+b_Mx+c_M)
  \end{equation}
  where $c, \lambda_1, \cdots, \lambda_m, b_1, \cdots, b_M, c_1, \cdots, c_M \in \mathbb{R}$
\end{thm}




  \chapter{Eigenvalues and Eigenvectors}
  \section{Invariant Subspaces}
\subsection{Eigenvalues}

\begin{mydef}
  A \lm from a \vs to itself is called an ``operator".
\end{mydef}

(Suppose $T\in \linmap(V)$, then may be $\left.T\right|_{V_{k}}$ is not an operator on a subspace $V_k$)

\begin{mydef}
  Let $T\in \linmap(V).$ $U \subseteq V$ is called ``invariant under $T$" if $\forall u \in U: Tu \in U.$ \\
  Thus $U$ is invariant under $T$ if $\left.T\right|_{U}$ is an operator on $U.$
\end{mydef}

\begin{example}
  Let $T\in \linmap(\mathcal{P}(\mathbb{R}))$ such that $Tp=p'.$ Let $U=\mathcal{P}_4(\mathbb{R}) \subseteq \mathcal{P}(\mathbb{R}).$ Then $U$ is invariant under $T$
  because if $p \in U$, $\deg p = 4$ and $\deg (p')=3$.
\end{example}

\begin{example}
  Let $T\in \linmap(V)$. Then $\{0\}, V, \operatorname{null} T, \operatorname{range} T$ are all invariant. \\
  (Sometimes, $\operatorname{null} T = \{0\}$ and $\operatorname{range} T=V$ if $T$ is invertible.)
\end{example}

\bigbreak

\bfemph{Invariant subspaces of dimension one:} \\
Take any $v\in V, v\neq 0$ and let $U :\equiv \{  \lambda v \mid \lambda \in \myF \} = \myspan{v}$, then $U$ is a one-dimensional subspace of $V$. \\
If $U$ is invariant under an operator $T \in \linmap (V)$, then $Tv  \in U$. $\implies \exists \lambda \in \myF: Tv = \lambda v$. \\
Conversely if $Tv = \lambda v$, $\lambda \in \myF$, then $\myspan{v}$ is a one-dimensional subspace of $V$ invariant under $T$.

\begin{mydef}
  Suppose $T\in \linmap (V)$. $\lambda \in \myF$ is called ``eigenvalue of $T$" if there exists $v \in V$ such that $v \neq 0$ and $Tv = \lambda v$
\end{mydef}

\setcounter{thm}{6}
\begin{thm}
  \label{equivalent-conditions-to-be-an-eigenvalue}
  The following are equivalent for $T \in \linmap(V)$ and $\lambda \in \myF$:
  \begin{enumerate}[label=(\alph*)]
    \item $\lambda$ is an eigenvalue of $T$. \label{first}
    \item $T-\lambda I$ is not injective. \label{second}
    \item $T-\lambda I$ is not surjective. \label{third}
    \item $T-\lambda I$ is not invertible. \label{forth}
  \end{enumerate}
\end{thm}
\begin{prf}
  Conditions \ref{first} and \ref{second} are equivalent because the eigenvector $v$ is a solution to 
  \begin{equation}
    Tv=\lambda v
  \end{equation} which is equivalent to 
  \begin{equation}
    (T-\lambda I)v=0.
  \end{equation} So there is a non-zero solution to $T-\lambda I$.
  \ref{second}, \ref{third} and \ref{forth} are equivalent by \ref{injectivity-is-equivalent-to-surjectivity}.
  
\end{prf}

\setcounter{thm}{7}
\begin{mydef}
  Let $T\in \linmap(V).$ A vector $v \in V$ is called ``an eigenvector" of $T$ corresponding to $\lambda$ if $v\neq 0$ and $Tv = \lambda v$.
  In other words:
  \\A vector $v\in V, v \neq 0$ is an eigenvector corresponding to $\lambda \iff v \in \operatorname{null}(T-\lambda I_V)$
\end{mydef}

\setcounter{thm}{10}
\begin{thm}[linearly independent eigenvectors]
  \label{thm: linearly independent eigenvectors}
  Every list of eigenvectors of $T$ corresponding to distinct eigenvalues of $T$ is linearly independent.
\end{thm}
\begin{prf}
  Suppose the desired result is false. Then there exists a smallest list of length $m$ of linearly depen\-dent eigenvectors $v_1, \dots, v_m$ with eigenvalues $\lambda_1, \dots, \lambda_m$ of $T$. Since an eigenvector is unequal to the zero vector, $m$ must be $\geq 2$.

  Because of the minimality of $m$ and becaue our list is linearly dependent: 
  \begin{equation}
    \exists a_1, \dots, a_m \neq 0$ such that $a_1 v_1 + \cdots + a_m v_m = 0. 
  \end{equation}
  
  Now we apply $T-\lambda_m I$ on both sides of the equation and get
  \begin{equation}
    \begin{gathered}
      a_1 \lambda_1 v_1 - a_1 \lambda_m v_1
      + \cdots + \\
      a_{m-1} \lambda_{m-1} v_{m-1} - a_{m-1} \lambda_{m} v_{m-1} +
      \underbrace{a_m \lambda_m v_m -a_m \lambda_m v_m}_{=0} =0
    \end{gathered}
  \end{equation}

  Which is the same as:
  \begin{equation}
    a_1 \underbrace{(\lambda_1 - \lambda_m)}_{\neq 0} v_1 + \cdots + a_{m-1} \underbrace{(\lambda_{m-1}-\lambda_{m})}_{\neq 0} v_{m-1}=0
  \end{equation}

  Which contradicts the minimality of $m$. Therefore, no such linearly dependent list of eigenvectors can exist.
  
\end{prf}

\begin{thm}
  Each operator on $V$ has at most $\dim V$ distinct eigenvalues.
  content
\end{thm}


\subsection{Polynomials applied to operators}

\setcounter{thm}{12}
\begin{mydef}
  Let $T \in \linmap(V)$ and $m\in \nat^{+}$
  \begin{itemize}
    \item $T^{m} :\equiv \underbrace{T \cdots T}_{\text{$m$ times}}$ or $T^{m} :\equiv T^{m-1} \cdot T$ such that $T^{m} \in \linmap(V)$
    \item $T^0 :\equiv I_V$
    \item If $T$ is invertible with inverse $T^{-1}$ then $T^{-m}\in \linmap(V)$ is defined by $T^{-m} :\equiv (T^{-1})^m$
  \end{itemize}
\end{mydef}
$\implies T^m T^n = T^{m+n}$ and $(T^m)^n=T^{mn}$ when $m,n \in \mathbb{Z}$ when $T$ is invertible. And $m,n \in \mathbb{N}$ if $T$ is not invertible.

\begin{mydef}
  For $p \in \mathcal{P} (\myF)$, $p(z) = a_0+a_1z+a_2z^2+\cdots+a_mz^m$
  $\forall z \in \compl$ and
  $T \in \linmap (V)$ we define: 
  \begin{equation}
    p(T) :\equiv a_0 I + a_1 T + a_2 T^2 + \cdots a_m T^m,$ $p(T) \in \linmap(V)
  \end{equation}
\end{mydef}

%TODO: example 5.15??

%TODO: example 5.16??

\setcounter{thm}{16}
\begin{thm} [multipliative properties]
  \label{multiplicative-properties}
  Suppose $p,q \in \mathcal{P} (\myF)$ and $T\in \linmap (V)$. Then \begin{equation}
    (p q)(T) = p(T) q(T) = q(T)p(T).
  \end{equation}
\end{thm}
\begin{prf} When a product of polynomials is expanded using the distributive property, it does not matter if the symbol is $z$ or $T$. Suppose $p(z) = \sum_{j=0}^{m} a_j z^j$ and $q(z)=\sum_{k=0}^{n} b_k z^k$ for all $z \in \myF$. Then  % wether?
  \begin{equation}
    (pq)(z) = \sum_{j=0}^{m} \sum_{k=0}^{n} a_j b_k z^{j+k} \myand
    (pq)(T) = \sum_{j=0}^{m} \sum_{k=0}^{n} a_j b_k T^{j+k}
            = \sum_{j=0}^{m} a_j T^j \sum_{k=0}^{n}  b_k T^k
  \end{equation}
  which is the same as $(pq)(T) = p(T)q(T)$ as well as $(pq)(T) = q(T)p(T)$
  
\end{prf}

\begin{thm} [null space and range of $p(T)$ are invariant under $T$]
  \label{thm: null space and range of p(T) are invariant under T}
  $T \in \linmap(V)$ and $p\in \mathcal{P} (\myF) \implies$
  $\operatorname{null} p(T)$ and $\operatorname{range} p(T)$ are invariant under $T$.
\end{thm}
\begin{prf}
  Suppose $u\in \operatorname{null} p(T)$
  \begin{equation}
    \implies p(T)u = 0. 
  \end{equation}
  
  Assoziativiy and distributivity of linear maps imply that 
  \begin{equation}
    (p(T))(Tu)=T(p(T)u)=T(0)=0.
  \end{equation}
  
  Which implies that
  \begin{equation}
    Tu \in \mynull p(T).
  \end{equation}

  Now suppose $u \in \myrange p(T)$
  \begin{equation}
    \begin{aligned}
    &\implies \exists v\in V: u=p(T)v \\
    &\implies Tu=T(p(T)v)=p(T)(Tv) \\
    &\implies Tu \in \myrange p(T)
    \end{aligned}
  \end{equation}
\end{prf}

\section{The Minimal Polynomial}
\subsection{Existence of Eigenvalues on Complex Vector Spaces}

\begin{thm} [existence of eigenvalues]
  \label{thm: existence of eigenvalues}
  Every operator on a finite-dimensional nonzero complex vector space has an eigenvalue.
\end{thm}
\begin{prf}
  Suppose $\dim(V)=n>0$ and $T\in \linmap(V).$ 
  
  Choose $v\in V, v\neq0$. Then 
  \begin{equation}
    v, Tv, T^2v, \dots, T^nv
  \end{equation} 
  is not linearly independent, 
  because the list has length $n+1$. Therefore, some linear combination of these vectors equals to $0$. 
  
  $\implies$ there exists a non-constant polynomial $p$ of smallest degree such that $p(T)v = 0$. By the first version of the fundamental theorem of algebra \ref{fundamental-theorem-of-algebra-first-version}, we have 
  \begin{equation}
    \begin{aligned}
      &\exists \lambda \in \compl: p(\lambda) = 0. \\  
      &\text{(\ref{factororing-out-zeros-of-a-polynomial-always-possible})}
      \implies \exists q \in \mathcal {P} (\compl): p(z) = (z-\lambda)  q(z) \quad \forall z \in \compl \\
      &\text{(\ref{multiplicative-properties})} \implies 0=p(T)v=(T-\lambda I) (q(T)v).
    \end{aligned} 
  \end{equation}
  
  Because $q$ has a smaller degree than $p \implies q(T)v \neq 0$. (Why?) 
  
  $\implies$ $\lambda$ is an eigenvalue of $T$ with eigenvector $q(T)v$.
  
\end{prf}

\subsection{Eigenvalues and the Minimal Polynomial}
\begin{mydef} [monic polynomials]
  A monic polynomial is a polynomial whose highest-degree coefficient equals $1$.
\end{mydef}
\begin{example}
  $p(z)=2+9z^2+z^7, \quad \deg p = 7$
\end{example}

\begin{thm}
  \label{unique-monic-polynomial-of-smallest-degree}
  Suppose $T\in \linmap(V)$. Then there exists a unique monic polynomial $p\in \mathcal{P}(\myF)$ of smallest degree such that 
  \begin{equation}
    p(T)=0. 
  \end{equation}
  Furthermore $\deg p \leq \dim V$
\end{thm}
\begin{prf}
  If $\dim V=0$, then $I$ is the zero-operator on $V$ and we let $p=1$ such that $1I\vec0=0$.           Now we use induction on $\dim V$ and we assume $\dim V > 0$ and that the theorem holds for all vector spaces of smaller dimension.
  
  Let $v\in V, v \neq 0$. The list \begin{equation}
    v, Tv, \dots, T^{\dim V}
  \end{equation}
  has length $1+\dim V.$
  $\implies$ linear dependence.

  By the linear dependence lemma (\ref{linear-dependence-lemma}), there is a smallest positive integer $m\leq \dim V$ such that
  \begin{equation}
    c_0 v + c_1 Tv + \cdots + c_{m-1} T^{m-1} v + 1\cdot T^m v = 0
  \end{equation}
  for some $c_0, c_1, \dots, c_{m-1} \in \myF$. 

  Let
  \[ q(z) :\equiv c_0 + c_1z + \cdots + c_{m-1} z^{m-1} +z^{m} \in \mathcal{P}_m (\myF) \]
  $\implies q(T) v=0$. Note that $q(z)$ is a monic polynomial.

  If $k \in \nat$, then
  \begin{equation}
    q(T)(T^kv)=T^k(q(T)v) =T^k (0) =0.
  \end{equation}
  By the linear dependence lemma (\ref{linear-dependence-lemma}) \begin{equation}
    \implies v, Tv, \dots, T^{m-1}v
  \end{equation}
  from before are linearly independent 
  \begin{equation}
    \begin{aligned}
    \implies &\dim \mynull q(T)   \geq m \\ 
    \implies &\dim \myrange q(T)  = \dim V - \dim \mynull q(T) 
                                \leq \dim V - m
    \end{aligned} 
  \end{equation}
  
  Because $\myrange q(T)$ is invariant under $T$ (by  \ref{thm: null space and range of p(T) are invariant under T}), we can apply our induction hypothesis to the operator $\left.T\right|_{\myrange q(T)}$. 
  
  So there exists monic $s \in \mathcal{P} (\myF):$
  \begin{equation}
    \deg s \leq \dim V - m \myand s(\left.T\right|_{\myrange q(T)})=0
  \end{equation}
  
  \begin{equation}
    \implies \forall v \in V: \left((sq)(T)\right)(v) = s(T) (q(T)v) = 0, 
  \end{equation}
  
  because $q(T)v \in \myrange q(T)$ and \begin{equation}
    \left.s(T)\right|_{\myrange q(T)}=s\left( \left.T\right|_{\myrange q(T)} \right ).
  \end{equation}
  Therefore, $sq$ is a monic polynomial such that $\deg sq \leq \dim V$ and $(sq)(T)=0$.

  Proof of uniqueness: Let $p\in \mathcal{P} (\myF)$ a monic polynomial of smallest degree such that 
  \begin{equation}
    p(T)=0.
  \end{equation} Let $r\in \mathcal{P} ( \myF)$ another monic polynomial of same degree such that $r(T)=0.$ 
  \begin{equation}
    \implies (p-r) (T) = 0
  \end{equation} 
  
  and also $\deg (p-r) < \deg p = \deg r$ If $p-r \neq 0$, we could devide $p-r$ by the coefficient of the highest order term in $p-r$ to get a monic polynomial that when applied to $T$ gives the $0$-operator. This polynomial would have a smaller degree than $p$ or $r$, which would be a contradiction. Therefore $p-r=0 \iff p = r$.

\end{prf}

\setcounter{thm}{23}

\begin{mydef} [minimal polynomial]
  Let $T\in \linmap (v)$. The ``minimal polynomial of $T$" is the unique monic polynomial $p\in \mathcal{P}(\myF)$ of smallest degree s.t. $p(T)=0$
\end{mydef}
\bfemph{Computation:} Find the smallest $m \in \nat$ such that: \\
$c_0I + c_1 T + \cdots + c_{m-1} T^{m-1} = -T^{m}$ has a solution $c_0, \dots, c_{m-1} \in \myF$. Solve for $m=1,2,\dots,\dim V$

Even faster (usually), pick $v \in V$ with $v \neq 0$ and consider the equation $c_0v + c_1Tv + \cdots + C_{\dim V-1}T^{\dim V-1}v=-T^{\dim V}v$.
If this equation has a unique solution, as happens most of the time $c_0, c_1, c_2, \dots, c_{\dim V-1}, 1$ are the coefficients of the minimal polynomial of $T$.
%TODO: do more.

\setcounter{thm}{26}
\begin{thm} [eigenvalues are the zeros of the minimal polynomial]
  \label{thm: eigenvalues are the zeros of the minimal polynomial}
  Let $T \in \linmap(V)$. Then
  \begin{enumerate}[label=(\alph*)]
    \item The zeros of the minimal polynomial of $T$ are the eigenvalues of $T$.
    \item If $V$ is a complex vector space, the minimal polynomial has the form 
    \begin{equation}
      (z-\lambda_1)\cdots(z-\lambda_m), \mytext{where} \lambda_1, \dots, \lambda_m
    \end{equation} are the eigenvalues of $T$, possibly with repetitions.
  \end{enumerate}
\end{thm}
\begin{prf} Let $p$ be the minimal polynomial of $T$.
  \begin{enumerate}[label=(\alph*)]
    \item Suppose $\lambda \in \myF$ is a zero of $p$. $\implies p(z)=(z-\lambda)q(z)$, whre $q$ is a monic polynomial with coefficients in $\myF$ (see \ref{factororing-out-zeros-of-a-polynomial-always-possible})
    \begin{equation}
      p(T)=0\implies 0=(T-\lambda I)(q(T)v) \quad \forall v\in V.
    \end{equation}
    Because $q$ is of lesser degree than $p$, there exists at least one vector $v\in V$ sucht that $q(T)v \neq 0$, which makes $q(T)v$ an eigenvector with eigenvalue $\lambda$.

    Suppose $\lambda \in \myF$ is an eigenvalue of $T$. Thus there exists $v\in V, v \neq 0$ such that $Tv=\lambda v$. Repeated applications of $T$ on both sides of this equation show that $T^kv =\lambda^k v \quad \forall k\in \nat$.
    $\implies p(T)v=p(\lambda)v$. Because $p$ is the minimal polynomial of $T$, we have $p(T)v=0$. $\implies p(\lambda) = 0$. $\implies$ $\lambda$ is a zero of $p$.

    \item use (a) and the second version of the fundamental theorem of algebra. (\ref{fundamental-theorem-of-algebra-second-version})
  \end{enumerate}
\end{prf}

\setcounter{thm}{28}
\begin{thm}[every ``zero-polynomial" is a multiple of the minimal polynomial]
  \label{thm: every zero polynomial is a multiple of the minimal polynomial}
  $T\in \linmap(V)$ and $q \in \mathcal{P} (\myF)$: 
  \begin{equation}
    q(T)=0 \iff q \text{is a multiple of the minimal polynomial of} T.
  \end{equation}
\end{thm}
\begin{prf}
  Let $p$ denote the minimal polynomial of $T$.

	\begin{description}
  
  \item{$\Rightarrow$ direction:}{
			Suppose $q(T)=0$.
			By (\ref{division-algorithm-for-polynomials}) there exists $s,r \in \mathcal{P} (\myF)$ such that
			\begin{equation}
				q=ps+r, \quad \deg r < \deg p
			\end{equation}
			We have
			\begin{equation}
				\label{aa}
				0 = q(T) = p(T)s(T) + r(T) = r(T).
			\end{equation}
			The equation above implies that $r=0$. Otherwise, dividing $r$ by its highest-degree coefficient would produce a monic polynomial that when applied to $T$ gives $0$. A contradiction because $\deg r < \deg p$ and $p$ is minimal. Thus \ref{aa} becomes the equation $q=ps$, as desired
		}
		\item{$\Leftarrow$ direction:}{
			Suppose $q=ps$ for $q,p,s \in \mathcal{P}(\myF)$. We have
			\begin{equation}
				q(T) = p(T)s(T)=0s(T)=0,
			\end{equation}
			as desired. 
      
		}
	\end{description}
  
\end{prf}

\setcounter{thm}{30}
\begin{thm}[minimal polynomial of a restriction operator]
  \label{thm: minimal polynomial of a restriction operator}
  If $U$ is a subspace of $V$, then the minimal polynomial of $T$ is a polynomial multiple of the minimal polynomial of $\left .T \right | _{ U}$
\end{thm}
\begin{prf}
  Suppose $p$ is the minimal polynomial of $T$.
  \begin{equation}
    \implies p(T)v=0 \quad \forall v \in V.
  \end{equation}
  In particular,
  \begin{equation}
    p(T)u=0 \quad \forall u\in U.
  \end{equation} Thus $p\left( \left.T\right|_{U} \right)=0.$ Now the previous theorem
  \autoref{thm: every zero polynomial is a multiple of the minimal polynomial} tells us, that $p$ is a polynomial multiple of the minimal polynomial of $\left. T \right |_U$.
  
\end{prf}

\begin{thm} [invertibility and the constant term of the minimal polynomial]
  $T \in \linmap (V):$ $T$ is not invertible $\mathsmaller{\iff}$ the constant term of the minimal polynomial of $T$ is $0$.
\end{thm}
\begin{prf}
  $T$ is not invertible $\mathsmaller{\overset{\text{\ref{equivalent-conditions-to-be-an-eigenvalue}}}{\iff}}$ $0$ is an eigenvalue of $T$ $\mathsmaller{\overset{\text{\ref{thm: eigenvalues are the zeros of the minimal polynomial}}}{\iff}}$ $0$ is a zero of $p$ $\mathsmaller{\iff}$ the constant term of $p$ is $0$.
  (In the first equivalence, we have actually used that $0$ is an eigenvalue of $T$ if and only if $T-0I$ is not invertible, according to \ref{equivalent-conditions-to-be-an-eigenvalue}.)
  
\end{prf}

\subsection{Eigenvalues on Odd-Dimensional Real Vector Spaces}
\begin{thm}[even-dimensional null space]
  Suppose $\myF = \real$ and $V$ is finite-dimensional. Suppose also that $T \in \linmap (V)$ and $b,c \in \real$ with $b^2 < 4c$. 
  \begin{equation}
    \implies \dim \mynull (T^2 +bT +cI) \mytext{is an even number}
  \end{equation}
\end{thm}

\begin{thm}[operators on odd-dimensional vectors spaces have eigenvalues]
  Every operator on an odd-dimensional vector space has an eigenvalue.
\end{thm}

\section{Upper-Triangular Matrices}

\setcounter{thm}{38}
\begin{thm} [conditions for upper-triangular matrix]
  \label{conditions for upper-triangular matrix}
  If $T \in \linmap (V)$ and $v_1, \dots, v_n$ is a basis of $V$. Then the following are equivalent.
  \begin{enumerate}[label=(\alph*)]
    \item The matrix of $T$ with respect to $v_1, \dots, v_n$ is upper triangular.
    \item $\myspan{v_1, \dots, v_k}$ is invariant under $T$ $\quad \forall k \in \{ 1, \dots, n\}$
    \item $T v_k \in \myspan{v_1, \dots, v_k} \quad \forall k \in \{1, \dots, n\}$
  \end{enumerate}
\end{thm}
\begin{prf}
  First suppose (a) holds. So like in \ref{matrix-of-linear-map}, the matrix-diagram $\mmatrix(T)$ looks like this
  \begin{equation}
  \mathcal{M} (T) \equiv
  \begin{blockarray}{cccccc}
             & v_1     & \cdots & v_j      & \cdots & v_n     \\
    \begin{block}{c(ccccc)}
      v_1    & A_{1,1} & \cdots & A_{1,j}  & \cdots & A_{1,n} \\
      \vdots &         & \ddots & \vdots   &   *    & \vdots  \\
      \vdots &         &        & A_{j,j}  &   *    & \vdots  \\
      \vdots &         &        &          & \ddots & \vdots  \\
      v_n    &         &        &          &        & A_{n,n} \\
    \end{block}
  \end{blockarray}
  \end{equation}
  
  Using a modified version of the equation in \ref{matrix-of-linear-map}, we have for $j\in \{1, \ldots n \}$
  \begin{equation}
    T v_j = A_{1,j} v_j + \cdots + A_{j,j} v_j = \sum_{l=1}^{j} A_{l,j} v_j
  \end{equation}
  
  which implies
  \begin{equation}
    T v_j \in \myspan{v_1, \ldots, v_j}
  \end{equation}
  
  Now suppose $j, k \in \{1 \ldots n \}$ such that $j \leq  k.$ Because
  \begin{equation}
    \myspan{v_1, \ldots, v_j} \subseteq \myspan{v_1, \ldots, v_k}
  \end{equation} 
  
  we see that
  \begin{equation}
    T v_j \in \myspan{v_1 \ldots v_k} \mytext{for each} j \in \{1 \ldots k\}.
  \end{equation}

  Thus $\myspan{v_1 \ldots v_k}$ is invariant under $T$, completing the proof that (a) implies (b)
  
  Now suppose (b) holds, so $\myspan{v_1, \ldots, v_k}$ is invariant under $T$ for each $k \in \{1 \ldots n\}.$ 
  
  In particular
  \begin{equation}
    T v_k \in \myspan{v_1, \ldots, v_k} \quad \forall k \in \{1 \ldots n\}
  \end{equation}
  
  Thus (b) implies (c).
  
  Now suppose (c) hods, so $T v_k \in \myspan{v_1, \ldots, v_k} \quad \forall k \in \{1 \ldots n\}$, which is the same as
  \begin{equation}
    T v_k = a_1 v_1 + \cdots + a_k v_k \where a_1, \dots a_k \in \myF
  \end{equation}
  
  Hence all the entries under the diagonal of $\mmatrix (T)$ are $0$, because $v_1, \ldots, v_n$ are linearly independent. Thus $\mmatrix(T)$ is an upper-triangular matrix, completing the proof that (c) implies (a).
  
  We have shown that (a) $\implies$ (b) $\implies$ (c) $\implies$ (a)
  
\end{prf}

\begin{thm}[equation satisfied by operator with upper-triangular matrix]
  \label{thm:equation-satisfied-by-operator-with-upper-triangular-matrix}
  Suppose $T\in \linmap(V)$ and $V$ has a basis with respect to which $T$ has an upper-triangular matrix with diagonal entries $\lambda_1, \dots, \lambda_n$.
  \begin{equation}
    \implies (T-\lambda_1I) \cdots (T-\lambda_nI)=0
  \end{equation}
\end{thm}
\begin{prf}
  Let $A :\equiv \mmatrix(T)$. Let $v_1, \ldots, v_n$ denote a basis of $V$ with respect to which $T$  has an upper-triangular matrix with diagonal entries $\lambda_1, \ldots, \lambda_n$.
  \begin{equation}
    A=\mmatrix(T) =
    \left( {\begin{array}{ccc}
        \lambda_1 &         &  * \\
                  &  \ddots &    \\
            0     &         & \lambda_n
    \end{array} } \right)
  \end{equation}
  
  Then
  \begin{equation}
    \label{i-need-a-ref}
    T v_1 = \lambda_1 v_1 \iff (T-\lambda_1 I)  v_1 = 0,
  \end{equation}

  which implies that
  \begin{equation}
    (T-\lambda_1 I) \cdots (T-\lambda_m I) v_1 = 0, \mytext{for} \mathbf{m = 1, \ldots, n}
  \end{equation}
  (using commutativity for linear maps)
  \bigbreak
  Note that $(T-\lambda_2 I) v_2 \in \myspan{v_1}$ because $T v_2 = A_{2,1} v_1 + \lambda_2 v_2$. Thus using (\ref{i-need-a-ref})
  \begin{dmath}
    \label{i-also-need-a-ref}
    (T- \lambda_1 I) (T- \lambda_2 I) v_2 
    = (T- \lambda_1 I) A_{2,1}v_1  
    = A_{2,1}(T- \lambda_1 I) v_1
    =0.
  \end{dmath}
  
  which implies that
  \begin{equation}
    (T-\lambda_1 I) (T-\lambda_2 I) \cdots (T-\lambda_m I) v_2 = 0, \mytext{for} \mathbf{m = 2, \ldots, n}
  \end{equation}
  (using commutativity for linear maps)
  \bigbreak
  
  Note that $(T-\lambda_3 I) v_3 \in \myspan{v_1, v_2}$ because $T v_3 = A_{3,1} v_1 +   A_{3,2} v_2 + \lambda_3 v_3$. \\
  Thus using (\ref{i-need-a-ref}) and (\ref{i-also-need-a-ref}), we get
  \begin{equation}
    \begin{aligned}
    (T- \lambda_1 I) (T- \lambda_2 I) (T- \lambda_3 I)v_3 
    &=(T- \lambda_1 I) (T- \lambda_2 I)(A_{3,1} v_1 +   A_{3,2} v_2)  \\
    &= (T- \lambda_1 I)A_{3,1} v_1(T- \lambda_2 I)+(T- \lambda_1 I)(T- \lambda_2 I)A_{3,2} v_2 \\
    &= A_{3,1}(T- \lambda_1 I) v_1(T- \lambda_2 I)+A_{3,2}(T- \lambda_1 I)(T- \lambda_2 I) v_2 \\
    &= 0
    \end{aligned}
  \end{equation}
  
  which implies that
  \begin{equation}
    (T-\lambda_1 I) (T-\lambda_2 I) \cdots (T-\lambda_m I) v_3= 0, \mytext{for} \mathbf{m = 3, \ldots, n}
  \end{equation}
  (using commutativity for linear maps)
  \bigbreak
  
  Continuing this pattern, we see that
  \begin{equation}
    (T-\lambda_1 I) \cdots (T- \lambda_n I) v_k = 0 \quad \forall k \in \{ 1\ldots n \}
  \end{equation}
  
  Thus $(T-\lambda_1 I) \cdots (T- \lambda_n I)$ is the $0$ operator because it is $0$ on each vector in a basis of $V$.
  
\end{prf}

\begin{thm}[determination of eigenvalues from upper-triangular matrix]
  \label{thm:determination-of-eigenvalue-from-upper-triangular-matrix}
  Suppose $T\in \linmap(V)$ has an upper\-/triangular matrix with respect to some basis of $V$. Then the eigenvalues of $T$ are precisely the entries on the diagonal of that upper-triangular matrix. 

\end{thm}
\begin{prf}
  Let $v_1, \ldots, v_n$ denote a basis of $V$ with respect to which $T$ has an upper-triangular matrix with diagonal entries $\lambda_1, \ldots, \lambda_n$.
  \begin{equation}
    \mmatrix(T) =
    \left( {\begin{array}{ccc}
        \lambda_1 &         &  * \\
                  &  \ddots &    \\
           0      &         & \lambda_n
    \end{array} } \right)
  \end{equation}  
  Because $T v_1 = \lambda_1 v_1$, we see that $\lambda_1$ is an eigenvalue of $T$.
  
  Suppose $k \in \{2 \ldots n\}.$ 
  
  Then $(T-\lambda_k I) v_k \in \myspan{v_1, \ldots, v_{k-1}}.$ 
  
  Thus $T-\lambda_k I$ maps $\myspan{v_1, \ldots, v_k}$ into $\myspan{v_1, \ldots, v_{k-1}}$.
  
  Because
  \begin{equation}
    \dim \myspan{v_1, \ldots, v_k} = k \myand \dim \myspan{v_1, \ldots, v_{k-1}} = k-1,
  \end{equation}
  
  this implies that $T-\lambda_k I$, which is restricted to $\myspan{v_1, \ldots, v_{k-1}}$, is not injective by \autoref{thm: linear-map-to-a-lower-dimensional-space-is-not-injective}. Thus 
  \begin{equation}
    \exists v \in \myspan{v_1, \ldots, v_n}: v\neq0$ and $(T-\lambda_k I)v=0. 
  \end{equation}
  Thus $\lambda_k$ is an eigenvalue of $T$. Hence we have shown that every entry of the diagonal of $\mmatrix(T)$ is an eigenvalue of $T$.
  
  To prove $T$ has no other eigenvalues, let
  \begin{equation}
    \begin{aligned}
      q &:\equiv (z-\lambda_1) \cdots (z-\lambda_2). \\
      &\implies q(T) = 0 \mytext{by \autoref{thm:equation-satisfied-by-operator-with-upper-triangular-matrix}}
    \end{aligned}
  \end{equation}
  
  Hence $q$ is a polynomial multiple of the minimal polynomial of $T$ by \ref{thm: every zero polynomial is a multiple of the minimal polynomial}. Thus every zero of the minimal polynomial is a zero of $q$.
  
  Because the zeros of the minimal polynomial of $T$ are the eigenvalues of $T$ by \ref{thm: eigenvalues are the zeros of the minimal polynomial}, this implies that every eigenvalue of $T$ is a zero of $q$. Hence the eigenvalues of $T$ are all contained in the list $\lambda_1, \ldots, \lambda_n$.
\end{prf}


\setcounter{thm}{43}
\begin{thm}[necessary and sufficient condition to have an upper-triangular-matrix]
  \label{thm:necessary and sufficient condition to have an upper-triangular-matrix}
  Let $T\in \linmap(V)$\footnotemark[1]. Then $T$ has an upper-triangular matrix in respect to some basis $V$ $\iff$ the min. polynomial of $T$ equals $(z-\lambda_1) \cdots (z-\lambda_m), \where \lambda_1, \dots \lambda_m \in \myF$
\end{thm}

\setcounter{thm}{46}
\begin{thm}[necessary condition for every operator on $V$ to have an upper-triangular matrix]
  \label{thm: necessary condition for every operator to have an upper-triangular matrix}
  Let $\myF = \compl$. Let $T\in \linmap (V)$\footnotemark[1]. Then $T$ has an upper-triangular matrix with respect to some basis of $V$.
\end{thm}

\setcounter{footnote}{1}
\footnotetext{$V$ is finite dimensional.}

\pagebreak

\section{Diagonalizable Operators}
\subsection{Diagonal Matrices}

\setcounter{thm}{47}
\begin{mydef} [diagonal matrix]
  A ``diagonal matrix" is a square matrix that is $0$ everywhere except possibly on \nopagebreak the diagonal
\end{mydef}

\setcounter{thm}{49}
\begin{mydef} [diagonalizable]
  An operator on $V$ is called ``diagonalizable" if the operator has a diagonal matrix with respect to some basis on $V$
\end{mydef}

\setcounter{thm}{51}
\label{eigenspace}
\begin{mydef} [eigenspace, $E(\lambda, T)$]
  Let $T \in \linmap(V)$ and $\lambda \in \myF$. The ``eigenspace" of $T$ corresponding to $\lambda$ is the subspace $E(\lambda, T)$ of $V$ defined by
  \begin{equation}
    E(\lambda, T) :\equiv  \mynull(T-\lambda I) = \{ v\in V \mid Tv = \lambda v\}
  \end{equation}
\end{mydef}

\setcounter{thm}{53}
\begin{thm} [sum of eigenspaces is a direct sum]
  \label{thm: sum of eigenspaces is a direct sum}
  Suppose $T\in \linmap (V)$ and $\lambda_1, \dots, \lambda_m$ are distinct eigenvalues of $T$. Then
  \begin{equation}
    E(\lambda_1, T) + \cdots + E(\lambda_m, T)
  \end{equation}
  is a direct sum. Furthermore, if $V$ is finite-dimensional, then
  \begin{equation}
    \begin{aligned}
      \dim E(\lambda_1, T) + \cdots + \dim E(\lambda_m, T)
      & = \dim \left( E(\lambda_1, T)  \oplus \cdots \oplus E(\lambda_m, T) \right) \\
      & \leq \dim V
    \end{aligned}
  \end{equation}
\end{thm}

\subsection{Conditions for Diagonalizability}
\setcounter{thm}{54}
\begin{thm} [conditions equivalent to diagonalizability]
  \label{thm: conditions equivalent to diagonalizability}
  Let $\lambda_1, \dots,\lambda_m$ denote the distinct eigenvalues of $T\in \linmap (V)$. Then
  \begin{enumerate}[label=(\alph*)]
    \item $T$ is diagonalizable.
    \item $V$ has a basis consisting of eigenvectors of $T$.
    \item $V=E(\lambda_1, T) \oplus \cdots \oplus E(\lambda_m, T).$
    \item $\dim V = \dim E(\lambda_1, T) + \cdots + \dim E(\lambda_m, T)$
  \end{enumerate}
\end{thm}

\setcounter{thm}{57}
\begin{thm} [enough eigenvalues implies diagonalizability]
  \label{thm: enough eigenvalues implies diagonalizability}
  $T\in \linmap(V)$ has $\dim V$ distinct eigenvalues $\implies$ $T$ is diagonalizable.
\end{thm}
\begin{prf}
  Let $\lambda_1, \ldots, \lambda_{\dim V}$ be the eigenvalues of the eigenvectors $v_1, \ldots v_{\dim V}$ which are linearly independent by \autoref{thm: linearly independent eigenvectors}. So we have a basis consisting of $\dim V$ eigenvectors. So $T$ is diagonalizable.
\end{prf}

\setcounter{thm}{61}
% 5.62
\begin{thm}[necessary and sufficient condition for diagonalizability]
  \label{thm: necessary and sufficient condition for diagonalizability}
  Let $T\in \linmap (V)$\footnotemark[1]. Then $T$ is diagonalizable $\iff$ the minimal polynomial of $T$ equals
  \begin{equation}
    (z-\lambda_1) \cdots (z-\lambda_m), \where \lambda_1, \dots, \lambda_m \in \myF \myand \lambda_1 \neq \cdots \neq \lambda_m
  \end{equation}
\end{thm}

\setcounter{thm}{64}
%5.65
\begin{thm}[restriction of diagonalizable operator to invariant subspace]
  \label{thm: restriction of diagonalizable operator to invariant subspace}
  $T\in \linmap(V)$. $T$ is diagonalizable and $U$ is a subspace of $V$ that is invariant under $T$. \\
  $\implies$ $\left.T\right|_U$ is a diagonalizable operator on $U$.
\end{thm}
\begin{prf}
  Diagonazability of $T$ $\mathsmaller{\overset{\text{\ref{thm: necessary and sufficient condition for diagonalizability}}}{\iff}}$ the minimal polynomial of $T$ equals 
  \begin{equation}
    (z-\lambda_1)\cdots(z-\lambda_m) \mytext{for} \lambda_1 \neq \cdots \neq \lambda_m.
  \end{equation} 
  
  By \ref{thm: minimal polynomial of a restriction operator}, the minimal polynomial of $T$ is a polynomial multiple of the minimal polynomial of $\left.T\right|_U$.
  
  Hence the minimal polynomial of $\left.T\right|_U$  has the form required by \ref{thm: necessary and sufficient condition for diagonalizability}, which shows that $\left.T\right|_U$ is diagonalizable. It consists of factors $(z-\lambda_1)$ or $(z-\lambda_2), \dots, (z-\lambda_m)$.
\end{prf}

% TODO: Gerhgorin Disk Theorem Def 5.66 and Theorem 5.67

\section{Commuting Operators}
\begin{mydef}
  Two operators or matrices $A$ and $B$ ``commute" if $ST=TS$
\end{mydef}

\end{document}
