\section{Matrices}
\subsection{Representing a Liner Map by a Matrix}

%3.31
\setcounter{thm}{30}
\begin{mydef}[matrix of a linear map, $\mmatrix(T)$]
  \label{def: matrix of a linear map}
  Let $T \in \linmap (L,W)$ and let $v_1, \dots, v_n$ be a basis of $V$ and let $w_1, \dots, w_m$ be a basis of $W$. The ``matrix of $T$'' with respect to these bases is the $m$-by-$n$ matrix $\mmatrix(T)$ whose entries $A_{j,k}$ are defined by
  \begin{equation}
    T v_k = A_{1,k} w_1 + \cdots + A_{m,k} w_m \equiv \sum_{j=1}^{m} A_{j,k} w_j
  \end{equation}
  If the bases are not clear from the context, one writes $\mmatrix(T, (v_1, \dots, v_n),(w_1, \dots, w_m))$. One remembers how to construct $\mmatrix(T)$ as follows:
  \begin{equation}
    \mathcal{M} (T) :\equiv
      \begin{blockarray}{cccccc}
                 & v_1 & \cdots & v_k     & \cdots & v_n \\
        \begin{block}{c(ccccc)}
          w_1    &     &        & A_{1,k} &        &     \\
          \vdots &     &        & \vdots  &        &     \\
          w_m    &     &        & A_{m,k} &        &     \\
        \end{block}
      \end{blockarray}
  \end{equation}
\end{mydef}

\subsection{Addition and Scalar Multiplication of Matrices}

% 3.35
\setcounter{thm}{34}
\begin{thm}[matrix of the sum of linear maps]
  $\mmatrix(S+T) = \mmatrix(S) + \mmatrix(T)$ for $S, T \in \linmap (V,W)$
\end{thm}

% 3.38
\setcounter{thm}{37}
\begin{thm}[matrix of a scalar times a linear map]
  $\mmatrix(\lambda T) = \lambda \mmatrix(T)$ for $\lambda \in \myF$ and $T \in \linmap (V,W)$
\end{thm}

\begin{mydef} [notation $\myF^{m,n}$]
  $\myF^{m,n}$ denotes the set of all $m$-by-$n$ matrices with entries in $F$, where $m,n \in \nat$
\end{mydef}

\begin{thm} % no []
  $\myF^{m,n}$ is a vector space with $\dim \myF^{m,n}=mn$.
\end{thm}


\subsection{Matrix Multiplication}

Let $v_1, \ldots, v_n$ be a basis of $V$ and $w_1, \ldots, w_m$ be a basis of $W.$ Suppose also, that $u_1, \ldots, u_p$ is a basis of $U$.

Consider the linear maps $T: U \to V$ and $S: V \to W$. The composition $ST$ is a linear map from $U$ to $W$. Does
\begin{equation}
  \mmatrix(ST) = \mmatrix(S) \mmatrix(T) \; {\mathlarger ?}
\end{equation}

One can choose a definition of matrix multiplication that forces this question to have a positive answer. Let $A :\equiv \mmatrix(S)$ and $B :\equiv \mmatrix(T)$. For $1 \leq k \leq p$, we have
\begin{equation}
  \begin{aligned}
    (ST) u_k 
    = S \left ( \sum_{r=1}^{n} B_{r,k} v_r \right )
    =  \sum_{r=1}^{n} B_{r,k} S v_r 
    =  \sum_{r=1}^{n} B_{r,k} \sum_{j=1}^{m} A_{j,r}  w_j 
    =  \sum_{j=1}^{m} \left( \sum_{r=1}^{n} A_{j,r}  B_{r,j} \right) w_j 
  \end{aligned}
\end{equation}

Thus $\mmatrix(ST)$ is the $m$-by-$p$ matrix whose entry in row $j$, columns $k$ equals 
  $\sum_{r=1}^{n} A_{j,r}  B_{r,k}.$


The Matrix of $ST$ denoted as $\mmatrix(ST)$ looks as follows:
\begin{equation}
  \mathcal{M} (ST) =
  \begin{blockarray}{cccccc}
             & v_1 & \cdots &  v_k      & \cdots & v_n \\
    \begin{block}{c(ccccc)}
      w_1    &     &        &  
        \sum_{r=1}^{n} A_{1,r}  B_{r,k} &        &     \\
      \vdots &     &        &  \vdots   &        &     \\
      w_j    &     &        &  
        \sum_{r=1}^{n} A_{j,r}  B_{r,k} &        &     \\
      \vdots &     &        &  \vdots   &        &     \\
      w_m    &     &        &  
        \sum_{r=1}^{n} A_{m,r}  B_{r,k} &        &     \\
    \end{block}
  \end{blockarray}
\end{equation}

%3.41
\setcounter{thm}{40}
\begin{mydef} [matrix multiplication]
  \label{def: matrix multiplication}
  Let $A \in \myF^{m,n}$ and $B \in \myF^{n,p}$. Then $AB \in \myF^{m,p}$ whose entry in row $j$, column $k$, is given by the equation
  \begin{equation}
    (AB)_{j,k} :\equiv \sum_{r=1}^{n} A_{j,r} B_{r,k}.
  \end{equation} %for $A\in \myF^{m,n}$, $B\in \myF^{n,p}$, $AB \in \myF^{m,p}$
\end{mydef}

% 3.43
\setcounter{thm}{42}
\begin{thm}[matrix of product of linear maps]
  If $T \in \linmap(U,V)$ and $S\in \linmap(V,W)$, then $\mmatrix(ST) = \mmatrix(S) \mmatrix(T).$
\end{thm}
\begin{prf}
  See at the beginning of the subsection.
\end{prf}

\begin{mydef} [notation: $A_{j, \mathsmaller{\bullet}}, A_{\mathsmaller{\bullet}, k}$ ] Let $A\in \myF{m,n}$. Then
  \begin{itemize}
    \item If $1 \leq j \leq m$, then $A_{j, \mathsmaller{\bullet}}$ denotes the $1$-by-$n$ matrix consisting of row $j$ of $A$. 
    \item If $1 \leq k \leq n$, then  $A_{\mathsmaller{\bullet}, k}$ denotes the $m$-by-$1$ matrix consisting of the column $k$ of $A$.
  \end{itemize}
\end{mydef}

% 3.46
\setcounter{thm}{45}
\begin{thm} [entry of matrix product equals row times column]
  $(AB)_{j,k} = A_{j, \mathsmaller{\bullet}} B_{\mathsmaller{\bullet}, k}$ if $1 \leq j \leq m$ and $1 \leq k \leq p$
\end{thm}

% 3.48
\setcounter{thm}{47}
\begin{thm}[column of matrix product equals matrix times column]
  If $A \in \myF{m,b}$ and $B \in \myF^{n,p}$, then
  \begin{equation}
    (AB)_{\mathsmaller{\bullet}, k} = AB_{\mathsmaller{\bullet}, k}
  \end{equation}
  
  if $1 \leq k \leq p$. In other words, column $k$ of $AB$ equals $A$ times column $k$ of $B$.
\end{thm}

\setcounter{thm}{49}
\begin{thm}
%  $b=\left(\begin{matrix}b_1\\ \vdots \\ b_n \end{matrix}\right)
  Suppose $A \in \myF^{m,n}$ and  $b = (b_1, \ldots, b_n)^{\top}\in \myF^{n,1}$. Then 
  \begin{equation}
    Ab = b_1 A_{\mathsmaller{\bullet}, 1} + \dots + b_n A_{\mathsmaller{\bullet},n}
  \end{equation}
\end{thm}

\begin{thm}[linear combinations of columns or rows]
  Let $C\in\myF^{m,c}$ and $R\in\myF^{c,n}$.
  \begin{enumerate}
    \item If $k \in \{1,\ldots,n\},$, then column $k$ of $CR\in \myF^{m,n}$ is a linear combination of the columns of $C$, with the coefficients of this linear combination coming from column $k$ of $R$.
    \begin{equation}
      \begin{aligned}
        CR = 
          \left ( \ldots \left(
            \begin{matrix}C_{1,1} 
              \\ \vdots \\ C_{m,1} 
            \end{matrix}
          \right) R_{1,k} 
          + \cdots + 
          \left(
            \begin{matrix}C_{1,c}
              \\ \vdots \\ C_{m,c} 
            \end{matrix}\right) R_{c,k}
          \ldots
        \right)
      \end{aligned}
    \end{equation}
    
  The octave\-/code for row-wise multiplication looks like this:
  
  \begin{center}    
    \begin{minipage}{\linewidth}
      \addtolength{\linewidth}{-7em}  
      \lstinputlisting[
      aboveskip=0.5em,
      frame=single,
      numbers=left,
      style=Matlab-bw,
      basicstyle=\scriptsize\ttfamily]{octave/row_wise_matrix_mult.m}
      \addtolength{\linewidth}{+7em}
    \end{minipage}
  \end{center}
  
    \item If $j\in \{1, \ldots, m\}$, then row $j$ of $CR$ is a linear combination of the rows of $R$, with the coefficients of this linear combination comming from row $j$ of $C$
    \begin{equation}
      CR = 
      \left(
        \begin{matrix}
          C_{1,1} (R_{1,1}, \ldots, R_{1,n}) + C_{1,2} (R_{2,1}, \ldots, R_{2,n}) + \cdots + C_{1,c} (R_{c,1}, \ldots, R_{c,n}) \\
          C_{2,1} (R_{1,1}, \ldots, R_{1,n}) + C_{2,2} (R_{2,1}, \ldots, R_{2,n}) + \cdots + C_{2,c} (R_{c,1}, \ldots, R_{c,n}) \\
          \vdots \\
          C_{j,1} (R_{1,1}, \ldots, R_{1,n}) + C_{j,2} (R_{2,1}, \ldots, R_{2,n}) + \cdots + C_{j,c} (R_{c,1}, \ldots, R_{c,n}) \\
          \vdots \\
          C_{m,1} (R_{1,1}, \ldots, R_{1,n}) + C_{m,2} (R_{2,1}, \ldots, R_{2,n}) + \cdots + C_{m,c} (R_{c,1}, \ldots, R_{c,n}) \\
        \end{matrix}
      \right)
    \end{equation}
    

    The octave\-/code for column-wise multiplication looks like this:
    
    \begin{center}
      \begin{minipage}{\linewidth}
        \addtolength{\linewidth}{-7em} 
        \lstinputlisting[
        aboveskip=0.5em,
        frame=single,
        numbers=left,
        style=Matlab-bw,
        basicstyle=\scriptsize\ttfamily]{octave/column_wise_matrix_mult.m}
        \addtolength{\linewidth}{+7em}
      \end{minipage}
    \end{center}
  \end{enumerate}
\end{thm}

\subsection{Column-Row Factorization and Rank of a Matrix}

% 3.52
\setcounter{thm}{51}
\begin{mydef}
  For $A \in \myF{m,n}$, the ``column rank'' of $A$ is the dimension of the span of the columns of $A$ in $\myF^{m,1}$. The ``row rank'' of $A$ is the dimension of the span of the rows of $A$ in $\myF{1,n}$.
\end{mydef}

% 3.54
\setcounter{thm}{53}
\begin{mydef}
  The ``transpose'' of a matrix $A\in \myF^{m,n}$, denoted by $A^t\in \myF^{n,m}$, is obtained by interchanging rows and columns. Its entries are given by the equation
  \begin{equation}
    \left( A^{t} \right)_{k,j} = A_{j,k}.
  \end{equation}
\end{mydef}

% 3.56
\setcounter{thm}{55}
\begin{thm}[column-row factorization]
  Suppose $A \in \myF^{m,n}$ with column rank $c \geq 1.$ $\implies \exists C\in \myF^{m,c}$ and $R\in \myF^{c,n}$ such that
  \begin{equation}
    A = CR
  \end{equation}
\end{thm}

 % 3.57}
\begin{thm}[column rank and row rank]
  $A\in \myF^{m,n} \implies$ column rank of $A$ $=$ row rank of $A$.
\end{thm}

\begin{mydef} [rank]
  The ``rank'' of a matrix $A \in  \myF^{m,n}$ is the column rank of $A$.
\end{mydef}