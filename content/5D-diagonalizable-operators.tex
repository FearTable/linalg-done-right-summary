\section{Diagonalizable Operators}
\subsection{Diagonal Matrices}

\setcounter{thm}{47}
\begin{mydef} [diagonal matrix]
  A \qt{diagonal matrix} is a square matrix that is $0$ everywhere except possibly on \nopagebreak the diagonal
\end{mydef}

\setcounter{thm}{49}
\begin{mydef} [diagonalizable]
  An operator on $V$ is called \qt{diagonalizable} if the operator has a diagonal matrix with respect to some basis on $V$
\end{mydef}

\setcounter{thm}{51}
\label{eigenspace}
\begin{mydef} [eigenspace, $E(\lambda, T)$]
  Let $T \in \linmap(V)$ and $\lambda \in \myF$. The \qt{eigenspace} of $T$ corresponding to $\lambda$ is the subspace $E(\lambda, T)$ of $V$ defined by
  \begin{equation}
    E(\lambda, T) :\equiv  \mynull(T-\lambda I) = \{ v\in V \mid Tv = \lambda v\}
  \end{equation}
\end{mydef}

\setcounter{thm}{53}
\begin{thm} [sum of eigenspaces is a direct sum]
  \label{thm: sum of eigenspaces is a direct sum}
  Suppose $T\in \linmap (V)$ and $\lambda_1, \dots, \lambda_m$ are distinct eigenvalues of $T$. Then
  \begin{equation}
    E(\lambda_1, T) + \cdots + E(\lambda_m, T)
  \end{equation}
  is a direct sum. Furthermore, if $V$ is finite-dimensional, then
  \begin{equation}
    \begin{aligned}
      \dim E(\lambda_1, T) + \cdots + \dim E(\lambda_m, T)
      & = \dim \left( E(\lambda_1, T)  \oplus \cdots \oplus E(\lambda_m, T) \right) \\
      & \leq \dim V
    \end{aligned}
  \end{equation}
\end{thm}

\subsection{Conditions for Diagonalizability}
\setcounter{thm}{54}
\begin{thm} [conditions equivalent to diagonalizability]
  \label{thm: conditions equivalent to diagonalizability}
  Let $\lambda_1, \dots,\lambda_m$ denote the distinct eigenvalues of $T\in \linmap (V)$. Then
  \begin{enumerate}[label=(\alph*)]
    \item $T$ is diagonalizable.
    \item $V$ has a basis consisting of eigenvectors of $T$.
    \item $V=E(\lambda_1, T) \oplus \cdots \oplus E(\lambda_m, T).$
    \item $\dim V = \dim E(\lambda_1, T) + \cdots + \dim E(\lambda_m, T)$
  \end{enumerate}
\end{thm}

\setcounter{thm}{57}
\begin{thm} [enough eigenvalues implies diagonalizability]
  \label{thm: enough eigenvalues implies diagonalizability}
  $T\in \linmap(V)$ has $\dim V$ distinct eigenvalues $\implies$ $T$ is diagonalizable.
\end{thm}
\begin{prf}
  Let $\lambda_1, \ldots, \lambda_{\dim V}$ be the eigenvalues of the eigenvectors $v_1, \ldots, v_{\dim V}$ which are linearly independent by \autoref{thm: linearly independent eigenvectors}. So we have a basis consisting of $\dim V$ eigenvectors. So $T$ is diagonalizable.
\end{prf}

\setcounter{thm}{61}
% 5.62
\begin{thm}[necessary and sufficient condition for diagonalizability]
  \label{thm: necessary and sufficient condition for diagonalizability}
  Let $T\in \linmap (V)$\footnotemark[1]. Then $T$ is diagonalizable $\iff$ the minimal polynomial of $T$ equals
  \begin{equation}
    (z-\lambda_1) \cdots (z-\lambda_m), \where \lambda_1, \dots, \lambda_m \in \myF \myand \lambda_1 \neq \cdots \neq \lambda_m
  \end{equation}
\end{thm}

\setcounter{thm}{64}
%5.65
\begin{thm}[restriction of diagonalizable operator to invariant subspace]
  \label{thm: restriction of diagonalizable operator to invariant subspace}
  $T\in \linmap(V)$. $T$ is diagonalizable and $U$ is a subspace of $V$ that is invariant under $T$. \\
  $\implies$ $\left.T\right|_U$ is a diagonalizable operator on $U$.
\end{thm}
\begin{prf}
  %  Diagonazability of $T$ $\mathsmaller{\overset{\text{\ref{thm: necessary and sufficient condition for diagonalizability}}}{\iff}}$ the minimal polynomial of $T$ equals 
  Diagonazability of $T$ $\iff_{(\ref{thm: necessary and sufficient condition for diagonalizability})}$ the minimal polynomial of $T$ equals 
  \begin{equation}
    (z-\lambda_1)\cdots(z-\lambda_m) \mytext{for} \lambda_1 \neq \cdots \neq \lambda_m.
  \end{equation} 
  
  By \ref{thm: minimal polynomial of a restriction operator}, the minimal polynomial of $T$ is a polynomial multiple of the minimal polynomial of $\left.T\right|_U$.
  
  Hence the minimal polynomial of $\left.T\right|_U$  has the form required by \ref{thm: necessary and sufficient condition for diagonalizability}, which shows that $\left.T\right|_U$ is diagonalizable. It consists of factors $(z-\lambda_1)$ or $(z-\lambda_2), \dots, (z-\lambda_m)$.
\end{prf}

% TODO: Gerhgorin Disk Theorem Def 5.66 and Theorem 5.67