\section{The Minimal Polynomial}
\subsection{Existence of Eigenvalues on Complex Vector Spaces}

\begin{thm} [existence of eigenvalues]
  \label{thm: existence of eigenvalues}
  Every operator on a \fd non\-zero, complex vector space has an eigenvalue.
\end{thm}
\begin{prf}
  Suppose $V$ is finite-dimensional with $\dim V \geq 1$ and $T\in \linmap(V)$. Let $v\in V$ be nonzero. Then the list
  \begin{equation}
    v, Tv, T^2v, \dots, T^nv
  \end{equation}

  of length $n+1$ cannot be linearly independent. Hence, some non-trivial linear combination of these vectors is zero, so there exists a non-constant polynomial $p \in \polyn(\compl)$ of minimal degree such that
  \begin{equation}
    p(T)v = 0.
  \end{equation}

  By the first version of the fundamental theorem of algebra \ref{fundamental-theorem-of-algebra-first-version}, there exists a root $\exists \lambda \in \compl$ with
  \begin{equation}
    p(\lambda) = 0.
  \end{equation}

  By \ref{factororing-out-zeros-of-a-polynomial-always-possible}, we know that $\exists q \in \mathcal {P} (\compl)$ such that we can factor out $(z-\lambda)$ and write
  \begin{equation}
    p(z) = (z-\lambda)  q(z) \quad \forall z \in \compl.
  \end{equation}

Applying this to \(T\) gives
\[
0 = p(T)v = (T-\lambda I)\bigl(q(T)v\bigr).
\]
If \(q(T)v\) were zero, then \(q\) (of strictly smaller degree) would also annihilate \(v\), contradicting the minimality of \(p\).  Thus
\[
q(T)v \neq 0
\]
and so \(\lambda\) is an eigenvalue of \(T\) with eigenvector \(q(T)v\).
%  Using the multiplicative properties of linear maps \ref{multiplicative-properties}, it follows that
%  \begin{equation}
%    0=p(T)v=((T-\lambda I)q(T))(v)=(T-\lambda I) (q(T)v).
%  \end{equation}
%
%  Because $q$ has a strictly smaller degree than $p$, it holds that
%  \begin{equation}
%    q(T)v \neq 0.
%  \end{equation}
%
%  Otherwise, $q(T)$ would also annihilate $v$, contradicting the minimality of $p$.
%  Thus, $\lambda$ is an eigenvalue of $T$ with eigenvector $q(T)v$.
\end{prf}


%\begin{thm}[Existence of eigenvalues]
%  \label{thm:existence-of-eigenvalues}
%  Every operator on a finite-dimensional, nonzero complex vector space has an eigenvalue.
%\end{thm}

%\begin{prf}
%  Suppose \(V\) is finite-dimensional with \(\dim V\ge1\) and \(T\in\Lin(V)\).  Let \(v\in V\) be nonzero.  Then the list
%  \[
%  v,\;T v,\;T^2 v,\;\dots,\;T^n v
%  \]
%  of length \(n+1\) cannot be linearly independent.  Hence some nontrivial linear combination of these vectors is zero, so there exists a nonconstant polynomial
%  \[
%  p(z)=a_0+a_1z+\cdots+a_nz^n\;\in\;\C[z]
%  \quad(\text{of minimal degree})
%  \]
%  such that
%  \[
%  p(T)v \;=\; 0.
%  \]
%
%  By the Fundamental Theorem of Algebra (first version) \ref{fundamental-theorem-of-algebra-first-version}, there is a root \(\lambda\in\C\) with
%  \[
%  p(\lambda)=0.
%  \]
%  By factorization of polynomials over \(\C\) \ref{factororing-out-zeros-of-a-polynomial-always-possible}, we write
%  \[
%  p(z)=(z-\lambda)\,q(z),
%  \quad q\in\C[z],\;\deg q<\deg p.
%  \]
%  Then
%  \[
%  0
%  \;=\;p(T)v
%  \;=\;(T-\lambda I)\bigl(q(T)v\bigr).
%  \]
%  If \(q(T)v\) were zero, then \(q\) (of strictly smaller degree) would also annihilate \(v\), contradicting the minimality of \(p\).  Thus \(q(T)v\neq0\), and so
%  \(\lambda\) is an eigenvalue of \(T\) with eigenvector \(q(T)v\).
%\end{prf}

\subsection{Eigenvalues and the Minimal Polynomial}

\begin{mydef} [monic polynomials]
  A \qt{monic polynomial} is a polynomial whose highest\-/degree coefficient equals $1$.
\end{mydef}
\begin{example}
  $p(z)=2+9z^2+z^7, \quad \deg p = 7$.
\end{example}

\begin{thm}[existence, uniqueness, and degree of minimal polynomial]
  \label{thm: unique monic polynomial of smallest degree}
  Suppose $T\in \linmap(V)$. Then there exists a unique monic polynomial $p\in \mathcal{P}(\myF)$ of smallest degree such that
  \begin{equation}
    p(T)=0.
  \end{equation}

  Furthermore, $\deg p \leq \dim V$.
\end{thm}
\begin{prf}
  Before beginning our inductive proof, we examine some base cases. We first prove the existence of a monic polynomial that annihilates $T$ and later establish its uniqueness; these two parts are independent.

  %Before we can start our proof with induction, let us look at some base cases. This proof will first look at the existence of such a monic polynomial and deal with the uniqueness of such at the end. Both proofs are completely independent.

  \prooffont{Base case:} If $\dim V=0$, then the identity operator $I$ is the zero-operator on $V$, and we let $p=1$ such that
  \begin{equation}
      1I\vec0=0.
  \end{equation}

  \prooffont{Another base case:} Now suppose that $\dim V=1$. For any nonzero $v \in V$, the list $v, Tv$ is linearly dependent (since it has length $2$), so there exists a scalar $c_0 \in \myF$ such that
  \begin{equation}
    c_0v + Tv = 0
  \end{equation}

  This follows from the linear dependence lemma \ref{thm: linear dependence lemma}. If we choose $q(z) := c_0+z \quad \forall z \in \compl$, then
  \begin{equation}
    q(T)v=(c_0 I+T)(v)=c_0v+Tv=0.
  \end{equation}

  In the case where $T$ is the zero operator, one may instead choose $q(z) = z$. So we have $\dim \mynull q(T) = 1$ and therefore $\dim \myrange q(T) = \dim V - \dim \myrange q(T) = 1-1= 0$. Thus, every vector in this one-dimensional space is a scalar multiple of $v$ and maps to zero by $q(T)$. Thus, $q(z)$ serves as a minimal polynomial for $T$.

  \prooffont{Yet another base case:} Suppose $\dim V=2$. For any nonzero $v \in V$, the list $v, Tv, T^2v$ is linearly dependent (since it has length $3$). According to the linear dependence lemma \ref{thm: linear dependence lemma}, we either have
  \begin{equation}
     c_0v + Tv = 0 \mytext{or}
     c_0v + c_1Tv + T^2v = 0,
  \end{equation}

  For some scalars $c_0, c_1 \in  \myF$. If we define $q(z) := c_0+z$ in the first case or $q(z) := c_0+c_1z+z^2$ in the second case accordingly, then $q(T)v=0$.

  In the first case where $(-1)c_0v=Tv$ and $q(z)=c_0+z$, the vector $v$ by itself is linearly independent and thus $\dim \mynull q(T) \geq 1$.  Therefore
  \begin{equation}
    \dim \myrange q(T) \leq \dim V - \dim \mynull q(T) \leq 1.
  \end{equation}

  Since $\myrange q(T)$ is at most one-dimensional and is invariant under $T$ (by \ref{thm: null space and range of p(T) are invariant under T}), we can restrict $T$ to this subspace. In this restricted setting $\myrange q(T)$, we know how to construct the desired polynomial; denote it by $s$, defined such that
  %  For a at most one-dimensional space like $\myrange q(T)$, we already know how to construct such a polynomial. Since $\myrange q(T)$ is invariant under $T$ by \ref{thm: null space and range of p(T) are invariant under T}, we can restrict $T$ to this subspace. So for the operator $\left. T \right |_{\myrange q(T)}$, we know how to construct such a polynomial. Let us call it $s$. Let $s$ be defined such that
  \begin{equation}
    \begin{aligned}
        &s \left(\left. T \right |_{\text{range $q(T)$}} \right)=0.
    \end{aligned}
  \end{equation}

  Just for clearity, the equation above means that $s$ is the zero-operator of $\left. T \right |_{\myrange q(T)}$. Therefore,
  \begin{equation}
    \forall u \in V: \left((sq)(T)\right)(u) = s(T)(q(T)u)=0
  \end{equation}

  Thus, $sq$ is a candidate to for our minimal polynomial of $T$.

  A similar reasoning goes for the case where $(-1)c_0v+(-1)c_1Tv=T^2v$ and $q(z) = c_0+c_1z+z^2.$
  Here we have $v$ and $Tv$ as linearly independent vectors. Note that if $k \in \nat$, then
  \begin{equation}
    \label{eq: nice equation for q(t)}
    q(T)(T^kv)=T^k(q(T)v) =T^k (0) =0,
  \end{equation}

  Hence the equation \eqref{eq: nice equation for q(t)} above tells us that $v\in \mynull q(T)$ and $Tv \in \mynull q(T)$. Here in this case we conclude that $\dim \mynull q(T) \geq 2$ and therefore $\dim \myrange q(T) = 0$.   Since $\dim \mynull q(T) = 2$, we have that
  \begin{equation}
    \mynull q(T) = V.
  \end{equation}

  This means that every vector of $V$ gets mapped to $0$ if $q(T)$ is applied. So we have found a minimal monic polynomial!

  \prooffont{General case:} As we can see a pattern now, we can use induction on $\dim V$ and assume $\dim V > 2$ and that the theorem holds for all vector spaces of smaller dimension than $\dim V$.

  Let $v\in V$ be nonzero. Then the list
  \begin{equation}
    \label{eq: v Tv T^dimV}
    v, Tv, \dots, T^{\dim V}
  \end{equation}

  (which has length $1+\dim V$) is linearly dependent.
  By the linear dependence lemma (\ref{thm: linear dependence lemma}), there is a smallest positive integer $m\leq \dim V$,  and some $c_0, \ldots, c_{m-1} \in \myF$ such that
  \begin{equation}
    \begin{aligned}
      &c_0 v + c_1 Tv + \cdots + c_{m-1} T^{m-1} v + T^m v = 0
    \end{aligned}
  \end{equation}

  Let us define
  \begin{equation}
    q(z) := c_0 + c_1z + \cdots + c_{m-1} z^{m-1} +z^{m} \in \mathcal{P}_m (\myF)
  \end{equation}

  accordingly. Thus, we have
  \begin{equation}
    q(T) v=0.
  \end{equation}

  Note that $q(z)$ is a monic polynomial and that $\deg q = m$. Observe that if $k \in \nat$, then
  \begin{equation}
    \label{eq: nice equation for q(T) again}
    q(T)(T^kv)=T^k(q(T)v) =T^k (0) =0.
  \end{equation}

  By the linear dependence lemma (\ref{thm: linear dependence lemma}) and the definition of $m$, the list
  \begin{equation}
    \label{eq: v, Tv, T^m-1v}
    v, Tv, \dots, T^{m-1}v
  \end{equation}

  from before is linearly independent (This is not the same list as \eqref{eq: v Tv T^dimV}). This fact together with equation \eqref{eq: nice equation for q(T) again} imply that the vectors of the list \eqref{eq: v, Tv, T^m-1v} all belong to $\mynull q(T)$. Hence,
  \begin{equation}
    \begin{aligned}
      &\dim \mynull q(T)   \geq m \\
    \end{aligned}
  \end{equation}

  Using the Rank-Nullity Theorem, this implies:
  \begin{equation}
    \dim \myrange q(T)  = \dim V - \dim \mynull q(T) \leq \dim V - m
  \end{equation}

  Because $\myrange q(T)$ is invariant under $T$ by  \ref{thm: null space and range of p(T) are invariant under T}, we can apply our induction hypothesis to the operator $\left.T\right|_{\myrange q(T)}$.
  Hence, there exists monic $s \in \mathcal{P} (\myF)$ such that $\deg s \leq \dim V - m$ and
  \begin{equation}
    s\left(\left.T\right|_{\myrange q(T)}\right)=0
  \end{equation}

  In other words, $\forall u \in \myrange q(T): s(T)(u) = 0$. So we can safely say that
  \begin{equation}
    \left.s(T)\right|_{\myrange q(T)}=s\left( \left.T\right|_{\myrange q(T)} \right ).
  \end{equation}

  Because $\forall u \in V: q(T)u \in \myrange q(T)$, we conclude:
  \begin{equation}
     \forall u \in V: \left((sq)(T)\right)(u) = s(T) (q(T)u) = 0
  \end{equation}


  Therefore, $sq$ is a monic polynomial such that
  \begin{equation}
    \deg sq \leq (\dim V -m ) + m = \dim V
  \end{equation}

  and $(sq)(T)=0$.

  \prooffont{Proof of uniqueness:} Let $p\in \mathcal{P} (\myF)$ a monic polynomial of smallest degree such that
  \begin{equation}
    p(T)=0.
  \end{equation} Let $r\in \mathcal{P} ( \myF)$ be another monic polynomial of the same degree such that $r(T)=0$. This implies
  \begin{equation}
    (p-r) (T) = 0.
  \end{equation}

  Moreover, $\deg (p-r) < \deg p = \deg r$. If $p-r \neq 0$, we could divide $p-r$ by the coefficient of the highest order term in $p-r$ to obtain a monic polynomial that, when applied to $T$, yields the zero-operator. This polynomial would have a smaller degree than $p$ or $r$, which would be a contradiction. Therefore, $p-r=0 \iff p = r$.
\end{prf}

\begin{thm}[Existence, uniqueness, and degree of minimal polynomial]
  \label{thm: unique monic polynomial of smallest degree}
  Suppose $T \in \linmap(V)$. Then there exists a unique monic polynomial $p \in \mathcal{P}(\myF)$, of minimal degree, such that
  \[
  p(T) = 0.
  \]
  Furthermore, $\deg p \leq \dim V$.
\end{thm}

\begin{prf}
  We prove existence first, and uniqueness second. The arguments are independent.

  \prooffont{Base case:} If $\dim V = 0$, then $V = \{ \vec{0} \}$, and any operator on $V$ is both the zero and identity operator. Let $p(z) := 1$; then $p(T) = I = 0$ on $V$, so $p(T)\vec{0} = 0$.

  \prooffont{Next base case:} Suppose $\dim V = 1$. Let $v \in V$, $v \ne 0$. Then the list $(v, Tv)$ is linearly dependent (length $2$ > dimension $1$), so there exists $c_0 \in \myF$ such that
  \[
  Tv + c_0 v = 0.
  \]
  Define $q(z) := z + c_0$; then $q(T)v = 0$. Since $v \ne 0$ and spans $V$, it follows that $q(T) = 0$ on all of $V$. If $T = 0$, we may instead use $q(z) := z$; in either case, $q$ is monic of degree $1$ and annihilates $T$.

  \prooffont{Another base case:} Suppose $\dim V = 2$. Let $v \in V$, $v \ne 0$; then the list $(v, Tv, T^2v)$ is linearly dependent. By the linear dependence lemma, either
  \[
  Tv + c_0 v = 0, \quad \text{or} \quad T^2v + c_1 Tv + c_0 v = 0
  \]
  for some $c_0, c_1 \in \myF$. Define
  \[
  q(z) :=
  \begin{cases}
    z + c_0, & \text{in the first case}, \\
    z^2 + c_1 z + c_0, & \text{in the second}.
  \end{cases}
  \]
  Then $q(T)v = 0$.

  If $v$ and $Tv$ are linearly dependent, then $\dim \mynull q(T) \geq 1$ and by rank-nullity,
  \[
  \dim \myrange q(T) \leq 1.
  \]
  Since $\myrange q(T)$ is $T$-invariant (Theorem~\ref{thm: null space and range of p(T) are invariant under T}), we restrict $T$ to this subspace, which has dimension $1$, and apply the previous case to obtain a monic polynomial $s$ such that
  \[
  s\left( \left. T \right|_{\myrange q(T)} \right) = 0.
  \]
  Then, for all $u \in V$,
  \[
  (sq)(T)u = s(T)(q(T)u) = 0.
  \]
  Thus $sq$ is a monic annihilating polynomial of degree at most $2$.

  If instead $v$ and $Tv$ are linearly independent, then
  \[
  q(T)(T^k v) = T^k(q(T)v) = 0 \quad \forall k \in \mathbb{N}.
  \]
  Hence both $v$ and $Tv$ lie in $\mynull q(T)$, so
  \[
  \dim \mynull q(T) \geq 2 = \dim V \quad \Rightarrow \quad q(T) = 0.
  \]

  In either case, we find a monic annihilating polynomial of degree $\leq 2$.

  \prooffont{Inductive step:} Assume the theorem holds for vector spaces of dimension less than $n$, and let $\dim V = n > 2$.

  Pick $v \in V$, $v \ne 0$. The list $(v, Tv, \dots, T^n v)$ has length $n+1$, so it is linearly dependent. Let $m \leq n$ be minimal such that
  \[
  T^m v + c_{m-1} T^{m-1}v + \cdots + c_0 v = 0.
  \]
  Define the monic polynomial
  \[
  q(z) := z^m + c_{m-1} z^{m-1} + \cdots + c_0.
  \]
  Then $q(T)v = 0$. Since
  \[
  q(T)(T^k v) = T^k(q(T)v) = 0 \quad \forall k \in \mathbb{N},
  \]
  the vectors $v, Tv, \dots, T^{m-1}v$ lie in $\mynull q(T)$. By minimality of $m$, this list is linearly independent, so
  \[
  \dim \mynull q(T) \geq m \quad \Rightarrow \quad \dim \myrange q(T) \leq n - m.
  \]

  As $\myrange q(T)$ is $T$-invariant, we restrict $T$ to it and apply the inductive hypothesis: there exists a monic polynomial $s$ of degree at most $n - m$ such that
  \[
  s\left( \left. T \right|_{\myrange q(T)} \right) = 0.
  \]
  Since $q(T)u \in \myrange q(T)$ for all $u \in V$, it follows that
  \[
  (sq)(T)u = s(T)(q(T)u) = 0.
  \]
  Hence $sq$ is a monic annihilating polynomial with
  \[
  \deg(sq) = \deg s + \deg q \leq n.
  \]

  \prooffont{Uniqueness:} Suppose $p$ and $r$ are two monic polynomials of minimal degree such that $p(T) = r(T) = 0$. Then
  \[
  (p - r)(T) = 0,
  \]
  but $\deg(p - r) < \deg p$ unless $p = r$. This contradicts the minimality of $p$ unless $p = r$. Therefore, the monic minimal polynomial is unique.
\end{prf}



\setcounter{thm}{23}

\begin{mydef} [minimal polynomial]
  Let $T\in \linmap (v)$. The \qt{minimal polynomial of $T$} is the unique monic polynomial $p\in \mathcal{P}(\myF)$ of smallest degree \st $p(T)=0$.
\end{mydef}
\bfemph{Computation:} Find the smallest $m \in \nat$ such that: \\
$c_0I + c_1 T + \cdots + c_{m-1} T^{m-1} = -T^{m}$ has a solution $c_0, \dots, c_{m-1} \in \myF$. Solve for $m=1,2,\dots,\dim V$

Even faster (usually), pick $v \in V$ with $v \neq 0$ and consider the equation
\begin{equation}
  c_0v + c_1Tv + \cdots + c_{\mydim V-1}T^{\mydim V-1}v=-T^{\mydim V}v.
\end{equation}
If this equation has a unique solution, as happens most of the time
\begin{equation}
  c_0, c_1, c_2, \dots, c_{\dim V-1}, 1
\end{equation}
are the coefficients of the minimal polynomial of $T$.
%TODO: do more.

\setcounter{thm}{26}
\begin{thm} [eigenvalues are the zeros of the minimal polynomial]
  \label{thm: eigenvalues are the zeros of the minimal polynomial}
  Let $T \in \linmap(V)$, $\dim v \neq \infty$. Then
  \begin{enumerate}[label=\textbf{(\alph*)}]
    \item The zeros of the minimal polynomial of $T$ are the eigenvalues of $T$.
    \item If $V$ is a complex vector space, the minimal polynomial has the form
    \begin{equation}
      (z-\lambda_1)\cdots(z-\lambda_m), \mytext{where} \lambda_1, \dots, \lambda_m
    \end{equation} are the eigenvalues of $T$, possibly with repetitions.
  \end{enumerate}
\end{thm}
\begin{prf} Let $p$ be the minimal polynomial of $T$.
  \begin{enumerate}[label=\textbf{(\alph*)}]
    \item \Rightarrowdirection Suppose $\lambda \in \myF$ is a zero of $p$. $\implies p(z)=(z-\lambda)q(z)$, where $q$ is a monic polynomial with coefficients in $\myF$ (see \ref{factororing-out-zeros-of-a-polynomial-always-possible})
    \begin{equation}
      p(T)=0\implies 0=(T-\lambda I)(q(T)v) \quad \forall v\in V.
    \end{equation}
    Because $q$ is of lesser degree than $p$, there exists at least one vector $u\in V$ sucht that
    \begin{equation}
      q(T)u \neq 0,
    \end{equation}

    which makes $q(T)u$ an eigenvector with eigenvalue $\lambda$.

    \Leftarrowdirection Conversely, suppose $\lambda \in \myF$ is an eigenvalue of $T$. Thus there exists $v\in V, v \neq 0$ such that $Tv=\lambda v$. Repeated applications of $T$ on both sides of this equation show that $T^kv =\lambda^k v \quad \forall k\in \nat$.
    $\implies p(T)v=p(\lambda)v$. Because $p$ is the minimal polynomial of $T$, we have $p(T)v=0$. $\implies p(\lambda) = 0$. $\implies$ $\lambda$ is a zero of $p$.

    \item use (a) and the second version of the fundamental theorem of algebra. (\ref{fundamental-theorem-of-algebra-second-version})
  \end{enumerate}
  \vspace*{-\baselineskip}
\end{prf}

\setcounter{thm}{28}
\begin{thm}[every \qt{zero-polynomial} is a multiple of the minimal polynomial]
  \label{thm: every zero polynomial is a multiple of the minimal polynomial}
  $T\in \linmap(V)$ and $q \in \mathcal{P} (\myF)$:
  \begin{equation}
    q(T)=0 \iff q \mytext{is a multiple of the minimal polynomial of} T.
  \end{equation}
\end{thm}
\begin{prf}
  Let $p$ denote the minimal polynomial of $T$.

  \begin{description}

    \item{\Rightarrowdirection}{
      Suppose $q(T)=0$.
      By (\ref{division-algorithm-for-polynomials}) there exists $s,r \in \mathcal{P} (\myF)$ such that
      \begin{equation}
        q=ps+r, \quad \deg r < \deg p
      \end{equation}
      We have
      \begin{equation}
        \label{eq: 0 = q(T) = p(T)s(T) + r(T) = r(T)}
        0 = q(T) = p(T)s(T) + r(T) = r(T).
      \end{equation}
      The equation above implies that $r=0$. Otherwise, dividing $r$ by its highest\-/degree coefficient would produce a monic polynomial that when applied to $T$ gives $0$. A contradiction because $\deg r < \deg p$ and $p$ is minimal. Thus \eqref{eq: 0 = q(T) = p(T)s(T) + r(T) = r(T)} becomes the equation $q=ps$, as desired
    }
    \item{\Leftarrowdirection}{
      Suppose $q=ps$ for $q,p,s \in \mathcal{P}(\myF)$. We have
      \begin{equation}
        q(T) = p(T)s(T)=0s(T)=0,
      \end{equation}
      as desired. Which proves both directions.
    }
  \end{description}
   \vspace*{-\baselineskip}
\end{prf}

\setcounter{thm}{30}
\begin{thm}[minimal polynomial of a restriction operator]
  \label{thm: minimal polynomial of a restriction operator}
  If $U$ is a subspace of $V$, then the minimal polynomial of $T$ is a polynomial multiple of the minimal polynomial of $\left .T \right | _{ U}$
\end{thm}
\begin{prf}
  Suppose $p$ is the minimal polynomial of $T\in \linmap(V)$ and $U \subseteq V$.
  \begin{equation}
    \implies p(T)v=0 \quad \forall v \in V.
  \end{equation}
  In particular,
  \begin{equation}
    p(T)u=0 \quad \forall u\in U.
  \end{equation} Thus $p\left( \left.T\right|_{U} \right)=0.$ Now the previous theorem
  \ref{thm: every zero polynomial is a multiple of the minimal polynomial} tells us, that $p$ is a polynomial multiple of the minimal polynomial of $\left. T \right |_U$. [This means that the degree of $p$ is at least as large].
\end{prf}

\begin{thm} [invertibility and the constant term of the minimal polynomial]
  Let $T \in \linmap (V)$. Then we have: $T$ is not invertible $\iff$ the constant term of the minimal polynomial of $T$ is $0$.
\end{thm}
\begin{prf}
  $T$ is not invertible $\iff_{\text{(\ref{thm: equivalent conditions to be an eigenvalue})}}$ $0$ is an eigenvalue of $T$ $\iff_{(\text{\ref{thm: eigenvalues are the zeros of the minimal polynomial})}}$ $0$ is a zero of $p$ $\iff$ the constant term of $p$ is $0$.
  (In the first equivalence, we have actually used that $0$ is an eigenvalue of $T$ if and only if $T-0I$ is not invertible, according to \ref{thm: equivalent conditions to be an eigenvalue}.)
\end{prf}

\subsection{Eigenvalues on Odd-Dimensional Real Vector Spaces}

\begin{thm}[even-dimensional null space]
  \label{thm: even-dimensional null space}
  Suppose $\myF = \real$ and $V$ is finite-dimensional. Suppose also that $T \in \linmap (V)$ and $b,c \in \real$ with $b^2 < 4c$. Then $\dim \mynull (T^2 +bT +cI)$ is an even number.
\end{thm}
\begin{prf}
  Recall that $\mynull (T^2 +bT +cI)$ is invariant under $T$, by \ref{thm: null space and range of p(T) are invariant under T}. Let
  \begin{align}
    N &:= \mynull (T^2 +bT +cI) \myand \\
    S &:= T|_N.
  \end{align}

  With these definitions above, we can assume that
  \begin{equation}
    \label{eq: equation for S}
    S^2 + bS + cI = 0,
  \end{equation}

  where $0$ means the zero operator. We now need to prove that $\dim N$ is even. Suppose $\lambda \in \real$ and $v \in N$ are such that $Sv = \lambda v$. Then
  \begin{equation}
    0 = (S^2 + bS + cI)v = (\lambda ^2 + b \lambda + c)v = \underbrace{\left( \left( \lambda + \tfrac{b}{2}\right)^2 + c - \tfrac{b^2}{4} \right)}_{\geq \; 0, \mytext{since} b^2 \leq 4c} v
  \end{equation}

  The term in the large parentheses above is a positive number. Thus, the equation above implies that $v = 0$. Hence, we have shown that $S$ has no eigenvalues.

  Let $U$ be a subspace of $N$ with even dimension, that is invariant under $S$ and has the largest dimension among all subspaces of $N$. Note that $U$ could be $\{0\}$. If $U = N$, then we are done.

  Assume that $U \neq N$. By maximality, $U$ is not properly contained in any larger even-dimensional $S$-invariant subspace of $N$. Hence, there exists $w \in N$ such that $w \notin U$. Now, let
  \begin{equation}
    W := \myspan {w, Sw}.
  \end{equation}

  Clearly, $W \subseteq N$, and since $w \notin U, W \subsetneq U$. We can rewrite \eqref{eq: equation for S} as $ S(Sw) = -bSw - cw $. Using this, we calculate for $x \in W$, written as $x = a_1 w + a_2 Sw$ for some scalars $a_1, a_2$:
  \begin{align}
    S(x) &= a_1 S w + a_2 S^2 w = a_1 S w - a_2 (b Sw + cw) \\
         &= (a_1 - a_2 b) Sw -  a_2 c  w \\
         & \in W
  \end{align}

  Thus, we see that $W$ is invariant under $S$ (remember, $S = T|_N$). Furthermore, $\dim W = 2$, because otherwise $w$ would be an eigenvalue of $S$, since $w$ and $Sw$ would be linearly dependent. This means that there would exist nonzero scalars $a_1$ and $a_2$, such that we have a nontrivial relation $a_1w + a_2 Sw = 0$. In this case, $-\sfrac{a_1}{a_2}$ would be an eigenvalue of $S$ with eigenvector $w$, which is impossible. Now, using \ref{thm: dimension of a sum of subspaces}:
  \begin{equation}
    \dim (U + W) = \dim U + \underbrace{\dim W}_{= \; 2} - \underbrace{\dim (U \cap W)}_{= \; 0} = \dim U + 2,
  \end{equation}

  where $U \cap W = \{0\}$, because otherwise, $U \cap W$ would be a one-dimensional $S$-invariant subspace providing an eigenvector for $S$, which is impossible. And the dimension of $U \cap W$ can not be $2$ either, because then we would have $U \cap W = W$, using \ref{thm: subspace of full dimension equals the whole space}. But then $W \subseteq U$, contradicting the fact that $w \notin U$. Since $U$ was assumed to be maximal among even-dimensional $S$-invariant subspaces, the existence of $U+W$, with $\dim (U+W)=\dim U+2$, contradicts the maximality of $U$.
  Thus, the assumption that $U \neq N$ was incorrect. Therefore, we conclude that $U = N$. Hence, $N= \mynull (S^2 +bS +cI)=\mynull (T|_N^2 +bT|_N +cI)=\mynull (T^2 +bT +cI)$ has even dimension $\dim N$.
\end{prf}
  %Since $U+W$ is invariant under $S$ because both $U$ and $W$ are invariant under $S$, the equation above shows that there exists a subspace of $N$ invariant under $S$ of even dimension larger than $\dim U$.

\begin{thm}[operators on odd-dimensional vectors spaces have eigenvalues]
  Every operator on an odd\-/dimensional vector space has an eigenvalue.
\end{thm}
\begin{prf}
  Suppose $\myF = \real$ and $V$ is finite-dimensional. Let $n := \dim V$, and suppose $n$ is an odd number. Let $T \in \linmap(V)$. We will use induction on $n$ in steps of size two to show that $T$ has an eigenvalue. Note that the desired result holds if $\dim V = 1$, because then every nonzero vector in $V$ is an eigenvector of $T$.

  Now, suppose $n \geq 3$ and the desired result holds for all operators on all odd-dimensional vector spaces of dimension less $n$, i.e. with dimension $\leq n-2$. Let $p$ denote the minimal polynomial of $T$. If $p$ is a polynomial multiple of $x-\lambda$ for some $\lambda \in \real$, then $\lambda$ is and eigenvalue of $T$ by theorem \ref{thm: eigenvalues are the zeros of the minimal polynomial} and we are done. Thus, we can assume that there exists $b, c \in \real$ such that $b^2 < 4c$ and $p$ is a polynomial multiple of $x^2 + bx + c$ by \ref{thm: factorization of a polynomial over R}. By looking at this thoerem, we can see that there exists a monic polynomial $q \in \polyn(\real)$ such that $p(x) = q(x)(x^2 + bx + c) \quad \forall x \in \real$. Now,
  \begin{equation}
    0 = p(T) = (q(T)) (T^2 + bT + cI),
  \end{equation}

  which means that $q(T)$ equals $0$ on $\myrange(T^2 + bT + cI)$. Or with other words,
  \begin{equation}
    q(T)|_{\myrange(T^2 + bT + cI)} = 0.
  \end{equation}

  Because $\deg q < \deg p$ and $p$ is the minimal polynomial of $T$, this implies that
  \begin{equation}
    \label{eq: subspace unequal to V}
    \myrange (T^2 + bT + cI) \neq V.
  \end{equation}

  The rank-nullity theorem \ref{rank-nullity-theorem} tells us that
  \begin{equation}
    \label{eq: equation for dim V}
    \dim V = \dim \mynull (T^2 + bT + cI) + \dim \myrange(T^2 + bT + cI)
  \end{equation}

  The previous theorem \ref{thm: even-dimensional null space} tells us that $\dim \mynull (T^2 + bT + cI)$ is even. Since $\dim V$ is odd, the equation \eqref{eq: equation for dim V} above and equation \eqref{eq: subspace unequal to V} shows that $\dim \myrange(T^2 + bT + cI)$ is odd as well and at least $2$ less than $\dim V$. We know that $\myrange(T^2 + bT + cI)$ is invariant under $T$ by \ref{thm: null space and range of p(T) are invariant under T}. Our induction hypothesis now implies that $T$ restricted to $\myrange(T^2 + bT + cI)$, which is a subset of $V$, has an eigenvalue; this also means that $T$ has an eigenvalue.
\end{prf}