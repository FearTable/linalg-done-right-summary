\chapter{Vector spaces}

%\section{Basic properties}
\section{Definition of a Vector Space}

\setcounter{thm}{19}
\begin{mydef}
    A ``vector space" is a set $V$ along with an addition on $V$ and a scalar multiplication on $V$ such that the following properties hold.
    \begin{itemize}
        \item \bfemph{Commutativity:} 
        $ u+v = v+u \quad \forall u,v \in V$
        \item \bfemph{Associativity:} 
        $(u+v)+w=u+(v+w)$ and $(ab)v=a(bv) \quad \forall u,v,w \in V$ and $\forall a,b \in \myF$
        \item \bfemph{Additive identity:} 
        $\exists 0\in V:v+0=v \quad \forall v\in V$. ($\exists!$ is not a requirement but a property.)
        \item \bfemph{Additive inverse:} 
        $\forall v\in V \quad \exists w\in V: v+w=0$
        \item \bfemph{Multiplicative identity:} 
        $1v = v \quad \forall v\in V$
        \item \bfemph{Distributive properties:}\\
        $a(u+v) = au+av$ and  \\
        $(a+b)v \, = av + bv \quad \forall a,b \in \myF$ and 
        $\forall u,v \in V$
    \end{itemize}
\end{mydef}

\setcounter{thm}{23}
\begin{mydef}
    If $S$ is a set, then $\myF^S$ denotes the set of functions from $S$ to $\myF$. Let $f, g \in \myF^{S}$ and $\lambda \in \myF$:
    \begin{itemize}
        \item The ``sum" $f+g \in \myF^{S}$ is defined as follows: $(f+g)(x) :\equiv f(x)+ g(x) \quad \forall x\in S$
        \item The ``product" $\lambda f \in \myF^{S}$ is defined as follows: $(\lambda f)(x) :\equiv \lambda f(x) \quad \forall x \in S$.
    \end{itemize}
    The vector space $\myF^{n} :\equiv \myF^{\{1,2,\dots,n\}}$ is a special case, because each $(x_1, \dots, x_n) \in \myF^{n}$ can be thought of as a function $x$ from the set $\{1, 2, \dots, n\}$ to $\myF$ by writing $x(k)$ instead of $x_k$ for the $k^{\text{th}}$ coordinate of $(x_1, \dots, x_n)$.
\end{mydef}

\setcounter{thm}{25}
\begin{thm}
    A vector space has a unique additive identity.
\end{thm}

\begin{thm}
    Every element in a vector space has a unique additive inverse.
\end{thm}

\begin{mydef}
    For $v,w\in V$, $-v$ denotes the additive inverse of $v$ and $w-v:\equiv w+(-v)$
\end{mydef}

\begin{mydef}
    For the rest of this summary, $V$ denotes a vector space over $\myF$.
\end{mydef}

\begin{thm}
    $0v = 0 \quad \forall v\in V$
\end{thm}

\begin{thm}
    $a \vec0= \vec0 \quad \forall a\in \myF$
\end{thm}

\begin{thm}
    $(-1)v = -v \quad \forall v\in V$
\end{thm}

\filbreak
\section{Subspaces}

\begin{mydef}
    A subset $U$ of $V$ ($U \subseteq V$) is called a ``subspace" of $V$ if $U$ is also a vector space with the same additive identity, addition, and scalar multiplication as on $V$.
\end{mydef}

\begin{thm}
    A subset $U$ of $V$ is a subspace of $V$ $\iff$ $U$ satisfies the following three conditions:
    \begin{itemize}
        \item \bfemph{Additive identity:} 
        $0 \in U$
        \item \bfemph{Closed under addition:} 
        $u,w \in U \implies u+w \in U$
        \item \bfemph{Closed under scalar multiplication:} 
        $a \in \myF$ and $u \in U \implies au \in U$
    \end{itemize}
\end{thm}


%\textbf{1.34} $U\subseteq V$ is a subspace of $V$ $\iff$ 
%\begin{itemize}
%	\item additive identity: $0 \in U$
%	\item closed under addition: $u,w\in U \implies u+w \in U$
%	\item closed under scalar multiplication: $a \in \myF$ and $u\in U$ $\implies$ $au \in U$
%\end{itemize}

\subsection{Sum of Subspaces}

\setcounter{thm}{35}

\begin{mydef}
    $V_1 + \cdots + V_m :\equiv \{v_1 + \cdots + v_m \mid v_1 \in V_1, \, \dots \, , v_m \in V_m \}$ for $V_i$'s beeing subspaces of $V$. It is called the ``sum from $V_1$ up to $V_m$."
\end{mydef} 

%\textbf{1.40} 
\setcounter{thm}{39}
\begin{thm}
    $V_1 + \cdots + V_m$ is the smalles subspace of V containing $V_1, \dots, V_m$
\end{thm}

\subsection{Direct Sum}

%\textbf{1.41} 
\setcounter{thm}{40}
\begin{thm}
    $V_1 + \cdots + V_m$ is called ``a direct sum", if each element of $V_1 +\cdots+V_m$ can be written in only one way as a sum $v_1 + \cdots + v_m$, where each $v_k \in V_k$. In this case: 
    \begin{equation}
    	V_1 \oplus \cdots \oplus V_m :\equiv V_1 + \cdots + V_m
    \end{equation}
\end{thm}

\begin{example}
   example: $\myF^3 = 
   \left \{ \left ( \begin{smallmatrix} x \\ y \\ 0 \end{smallmatrix} \right ) \in \myF \mid x,y \in \myF \right \} 
   \oplus 
   \left \{ \left ( \begin{smallmatrix} 0 \\ 0 \\ z \end{smallmatrix} \right ) \in \myF \mid z \in \myF \right \}$ 
\end{example}


%\textbf{1.45}
\setcounter{thm}{44}
\begin{thm}
    $V_1 + \cdots + V_m$ is a direct sum $\iff$ the only write $0$ as a sum $v_1 + \cdots + v_m$, where each $v_k \in V_k$, is by taking each $v_k$ equal to $0$.
\end{thm}


%\textbf{1.46} 
\setcounter{thm}{45}
\begin{thm}
    $U+W$ is a direct sum $\iff$ $U \cap W = \{0\}$
\end{thm}

