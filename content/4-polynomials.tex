
$\mathbb{F}$ denotes $\mathbb{R}$ or $\mathbb{C}$.

\section{Zeros of polynomials}
$p: \mathbb{F} \to \mathbb{F}, \, p(z) = a_0 + a_1z + \cdots + a_n z^m$ $\forall z \in \myF$.

\setcounter{thm}{4}
\begin{mydef}[zero of a polynomial]
  $\lambda \in \mathbb{F}$ is called a \qt{zero}  (or \qt{root}) of a polynomial $p$ if $p(\lambda) = 0$.
\end{mydef}

\setcounter{thm}{5}
\begin{thm}[each zero of a polynomial corresponds to a degree\-/one factor]
  \label{factororing-out-zeros-of-a-polynomial-always-possible}
  $\forall p \in \mathcal{P}_m(\mathbb{F}): \; \exists \lambda \in \mathbb{F}$ \st
  \begin{equation}
  	p(\lambda) = 0 \iff \exists q \in \mathcal{P}_{m-1}: p(z) = (z-\lambda)q(z)
  \end{equation}
\end{thm}
\begin{prf} Let $p\in \polyn(\myF)_m$ for $m\in \nat$.

  \Leftarrowdirection Suppose there is $q \in \polyn(\myF)$ \st $p(z) = (z-\lambda)q(z) \quad \forall z \in \myF$. Then
  \begin{equation}
    p(\lambda) = (\lambda -\lambda)q(\lambda) = 0,$ as desired.$
  \end{equation}

  \Rightarrowdirection Now suppose $p(\lambda)=0$. Let $a_0, a_1, \ldots, a_m \in \myF$ \st
  \begin{equation}
    p(z) :\equiv a_0 + a_1z + \cdots a_mz^m.
  \end{equation}

  Thus
  \begin{equation}
    \label{eq: second formula of pz: pz = pz - plambda}
    p(z) = p(z)- p(\lambda) = a_1 (z-\lambda) + a_2 (z^2 - \lambda^2) + \cdots+ a_m (z^m- \lambda^m) \quad \forall z \in \myF.
  \end{equation}

  Using the binomic formula
  $z^k - \lambda^k = (z-\lambda) \sum_{j=1}^{k} \lambda^{j-1}z^{k-j} \quad \forall k\in \setonetillm$, we get
  \begin{equation}
    p(z) = (z-\lambda)
    \underbrace{
      \left(
        a_1 + a_2 (z+\lambda) + a_3(z^2 + \lambda z + \lambda^2) +
        a_4 (z^3 + \lambda z^2 + \lambda^2 z + \lambda^3)
        + \cdots +
        a_m \sum_{j=1}^{m} \lambda^{j-1} z^{m-j}
      \right)}_{\text{a polynomial with degree $m-1$}}
  \end{equation}

  Thus $p$ equals $z-\lambda$ times some polyomial of degree $m-1$, as desired.
\end{prf}

In case you are wondering where the $a_0$ went by subtracting $p(\lambda)$ from $p(z)$, it is hidden in the roots! For example $p\in \polyn_1(\myF): p=a_0+a_1z = \left(z- \left(-\frac{a_0}{a_1}\right)\right)(a_1)$ where $\left(-\frac{a_0}{a_1}\right)$ is the only root!

\setcounter{thm}{7}
\begin{thm}
  \label{thm: a polynomial of degree m has at most m zeros}
  A polynomial $p \in \polyn_m(\myF)$ of degree $m$ has at most $m$ zeros in $\myF$.
\end{thm}
\begin{prf}
  For $m=1$, $a_0+a_1z$ has only one zero, which is $-\frac{a_0}{a_1}$\\
  Now for $m>1$ let us assume the desired result holds for $m-1$.

  If $p\in \mathcal{P}_{m}(\myF)$ has no zeros, we are done.

  If $p$ has a zero $\lambda \in \myF$, by \ref{factororing-out-zeros-of-a-polynomial-always-possible}, there is a $q \in \mathcal{P}(\myF)_{m-1}$ such that
  \begin{equation}
  	p(z)=(z-\lambda)q(z).
  \end{equation}

  Since $q$ has at most $m-1$ zeros, $p$ has at most $m$ zeros.
\end{prf}

\section{Division Algorithm for Polynomials}

\setcounter{thm}{8}
\begin{thm}
  \label{division-algorithm-for-polynomials}
  Suppose $p,s \in \mathcal{P} (\mathbb{F}), \, s\neq 0$.
  Then $\exists q,r \in \mathcal{P} (\mathbb{F})$ \st
  \begin{equation}
    p=sq+r \myand \deg r < \deg s.
  \end{equation}
\end{thm}
\begin{prf}
  Let $n:\equiv \deg p$ and $m:\equiv \deg s$.

  If $n<m$, then take $q=0$ and $r=p$.

  Thus now we assume $n \geq m$. The list
  \begin{equation}
    \label{eq: basis of P_n(F)}
    1, z, \ldots, z^{m-1}, s, zs, \ldots, z^{n-m}s
  \end{equation}

  is linearly independent because each polynomial in this list has a different degree.

  The list has length $1+(m-1)+1+(n-m)=n+1$, which equals $\dim \polyn_n(\myF)$. Hence \eqref{eq: basis of P_n(F)} is a basis of $\polyn_n(\myF)$ by \ref{thm: linearly independent list of the right length is a basis}. Therefore, we have $a_0, a_1, \ldots, a_{m-1} \in \myF$ and $b_0, b_1, \ldots, b_{n-m} \in \myF$  \st
  \begin{equation}
    \begin{aligned}
          p&=a_0 + a_1z + a_2z^2 + \cdots + a_{m-1} z^{m-1}
          + b_0 s + b_1 zs + b_2 z^2 s + \cdots + b_{n-m}z^{n-m}s \\
           &=\underbrace{\left(a_0 + a_1z + a_2z^2 + \cdots + a_{m-1} z^{m-1} \right)}_{r} + \underbrace{\Big(s\Big)}_{s} \underbrace{\left( b_0 + b_1 z + b_2 z^2 + \cdots + b_{n-m} z^{n-m} \right)}_{q} \\
           &=r+sq=sq+r, \mytext{with} \deg r = m-1 \leq \deg s = m,
    \end{aligned}
  \end{equation}
  as desired.

  The uniqueness of $q,r \in \polyn(\myF)$ follows from the uniqueness of the $a$'s and $b$'s, because \eqref{eq: basis of P_n(F)} is a basis of $\polyn_n(\myF)$.
\end{prf}
\section{Factorization of Polynomials over \texorpdfstring{$\compl$}{C}}
\setcounter{thm}{11}
\begin{thm}[{\slshape \scshape Fundamental Theorem of Algebra}, first version]
  \label{fundamental-theorem-of-algebra-first-version}
  Every non\-/constant polynomial with complex coefficients has a zero in $\mathbb{C}$.
\end{thm}

\begin{thm} [{\slshape \scshape Fundamental Theorem of Algebra}, second version:]
  \label{fundamental-theorem-of-algebra-second-version}
  If $p \in \polyn(\mathbb{C})$ is nonconstant polynomial, which means $\deg p \geq 1$, then $p$ has a unique factorization
  \begin{equation}
  	p(z)=c \cdot (z-\lambda_1) \cdots (z-\lambda_m)
  \end{equation}
  where $c, \lambda_1, \dots, \lambda_m \in \compl$. This factorization is unique except for the order of the factors.
\end{thm}
\begin{prf}
  Let $p \in \polyn(\compl)$ and let $m :\equiv \deg p$. If $m = 1$, then
  \[
    p = a_0 + a_1 z = a_0 \cdot \left(z-\tfrac{a_1}{a_0} \right),
  \]

  so the the desired factorization exists and is unique.

  So assume that $m > 1$ and that the desired factorization exists and is unqique for all polynomials of degree $m-1$.

  First we will show that the desired factorization of $p$ exists. By the first version of the fundamental theoreom of algebra \ref{fundamental-theorem-of-algebra-first-version}, $p$ has a zero $\lambda \in \compl$. By \ref{factororing-out-zeros-of-a-polynomial-always-possible}, there is polynomial $q$ of degree $m-1$ \st
  \[
    p(z) = (z-\lambda) q(z) \quad \forall z \in \compl.
  \]

  Our induction hypothesis implies that $q$ has the desired factorization, which when plugged into the equation above gives the desired factorization of $p$.

  \prooffont{Uniqueness:} The number $c$ is uniquely determined as the coefficient $a_m$ of $z^m$ in $p$. So we only need to show that except for the order, there is only one way to chosse $\lambda_1, \ddd, \lambda_m$. If we have some $\tau_1, \ddd, \tau_m \in \compl$ \st
  \begin{equation}
    \xcancel{\mathlarger{c}} \cdot (z - \lambda_1) \cdots (z-\lambda_m) = \xcancel{\mathlarger{c}} \cdot (z-\tau_1) \cdots (z-\tau_m) \quad \forall z \in \compl,
  \end{equation}


  then because the left side of the equation above equals $0$ when $z=\lambda_1$, one of the $\tau$'s on the right hide side equals $\lambda_1$, because every zero of a polynomial corresponds to a degree-one factor by \ref{factororing-out-zeros-of-a-polynomial-always-possible}. Without loss of generality, we can assume by relabeling, that $\tau_1 = \lambda_1$. Now if $z \neq \lambda_1$, we can use division to get
  \begin{equation}
    \label{eq: original equation for z-lambda_2 ...}
    \begin{aligned}
      \xcancel{(z - \lambda_1)} (z - \lambda_2) \cdots (z-\lambda_m) = \xcancel{(z-\tau_1)} (z-\tau_2) \cdots (z-\tau_m) \qquad \forall z \in \compl \\
      (z - \lambda_2) \cdots (z-\lambda_m) = (z-\tau_2) \cdots (z-\tau_m) \qquad  \forall z \in \compl \backslash \{ \lambda_1 \} \\
      0 = \Big((z - \lambda_2) \cdots (z-\lambda_m) \Big) -   \Big((z-\tau_2) \cdots (z-\tau_m)\Big) \qquad \forall z \in \compl \backslash \{ \lambda_1 \}
    \end{aligned}
  \end{equation}

  Not that the last equation holds for all $z \in \compl$ except possibly $z = \lambda_1$. Remember that our goal is to use induction on the degree of the polynomial. So it would come in handy if the remaining polynomials would be equal after cancelation and behave like any other polynomial. But sadly, we have derived the last formula in \eqref{eq: original equation for z-lambda_2 ...} by the assumtion that $z \neq \lambda_1$ using devision. Here comes the trick. Now lets assume that $(z - \lambda_2) \cdots (z-\lambda_m)$ and $(z - \tau_2) \cdots (z-\tau_m)$ are not always equal. Thus, with the standard laws of polynomial arithmetic we can form a new polynomial $s \in \polyn_{m-1}(\compl)$ (is it $m-1$?) with
  \begin{equation}
    s(z) :\equiv   \Big((z - \lambda_2) \cdots (z-\lambda_m)\Big) - \Big((z-\tau_2) \cdots (z-\tau_m)\Big), \mytext{where $s$ is not the $0$ polynomial.}
  \end{equation}

   Using the last equation in \eqref{eq: original equation for z-lambda_2 ...}, we know that $s(z) = 0 \quad \forall z \in \compl \backslash \{\lambda_1\}$. But we know that $s$ can have at most $m-1$ zeros. A contradiction, so it must be the case that $s = 0$ [the zero polynomial] and therefore $\forall z \in \compl: s(z) = 0$. The zero polynomial is the only polynomial that is allowed to have infinitely many roots. This is only the case if $\forall z \in \compl$:
   \[
    (z - \lambda_2) \cdots (z-\lambda_m) = (z - \tau_2) \cdots (z-\tau_m).
   \]

   So we get two identical polynomials after canceling ouut $(z-\lambda_1)$. By our induction hypothesis, the equation above implies that except for the order, the $\lambda$'s are the same as the $\tau$'s, comleting the proof of uniqueness.
\end{prf}

\section{Factorization of Polynomials over \texorpdfstring{$\real$}{R}}

% 4.14
\begin{thm}[polynomials with real coefficients have nonreal zeros in pairs]
  \label{thm: polynomials with real coefficients have nonreal zeros in pairs}
  Suppose $p\in \mathcal{P} (\mathbb{C})$ is a polynomial with real coefficients. If $\lambda \in \mathbb{C}$ is a zero of $p$, then so is $\overline{\lambda}$.
\end{thm}
\begin{prf}
  Let $p(z) :\equiv a_0 + a_1 z + \ldots + a_m z^m$, where $a_0, \ldots, a_m \in \real$. Suppose $\lambda \in \compl$ is a zero of $p$. Then
  \begin{equation}
    \begin{aligned}
      a_0+a_1\lambda+\cdots+a_m\lambda^m&=0\\
      \overline{a_0+a_1\lambda+\cdots+a_m\lambda^m}&=\overline{0}\\
      \overline{a_0}+\overline{a_1\lambda}+\cdots+\overline{a_m\lambda^m}&=0\\
      a_0+a_1\overline{\lambda}+\cdots+a_m\overline{\lambda}^m&=0
    \end{aligned}
  \end{equation}

  The equation above shows that $\overline{\lambda}$ is a zero of $p$ as well.
\end{prf}

\begin{thm} [factorization of a quadratic polynomial]
  \label{thm: factorization of a quadratic polynomial}
  Let $p(x)=x^2+bx+c \in \polyn_2(\real)$ such that $b,c \in \real$. Then $p$ can be written as
  $p(x)=x^2 + bx + c$ $=$ $(x-\lambda_1)(x-\lambda_2)$ with $\lambda_1, \lambda_2 \in \mathbb{R} \iff b^2 \geq 4c$.
\end{thm}
\begin{prf}
  Let $x^2+bx+c=\left(x+\tfrac{b}{2} \right)^2+\left(c-\tfrac{b^2}{4} \right)$. Therefore, if $b^2<4c$, the last term is always positive and no factoriza\-tion is possible because it has no zeros by \ref{factororing-out-zeros-of-a-polynomial-always-possible}.

  If $b^2 \geq 4c$, we define $d^2 :\equiv \tfrac{b^2}{4}-c$ which makes
  \begin{equation}
	  \begin{aligned}
	    p(x)&=x^2+bx+c
		  = \left (x+\tfrac{b}{2} \right )^2-d^2
		  = \left ( \left(x+\tfrac{b}{2} \right) +d \right) \cdot \left ( \left(x+\tfrac{b}{2} \right) - d \right) \\
		  &= \left (x-\left(-d -\tfrac{b}{2} \right) \right) \cdot \left (x-\left(d-\tfrac{b}{2} \right)\right)
		  =\left(x-\lambda_1 \right) \cdot \left(x-\lambda_2\right)
	  \end{aligned}
  \end{equation}
  for
  \begin{equation}
    \lambda_1 :\equiv -d-\tfrac{b}{2} \mytext{and} \lambda_2 :\equiv d-\tfrac{b}{2}.
  \end{equation}

  which is the same as the well known midnight formula:
  \begin{equation}
    \lambda_{1,2} = \frac{-b \pm \sqrt{b^2-4c}}{2}
  \end{equation}
  which also shows, that $b^2 \geq 4c$ is a necessary condition for real solutions for $\lambda_{1,2}$.
\end{prf}

The next result gives a factorization of a polynomial over $\real$. The second version of the fundamental theorem of algebra \ref{fundamental-theorem-of-algebra-second-version} gives a factorization of $p$ with complex coefficients. Complex but nonreal zeros of $p$ come in pairs, see \ref{thm: polynomials with real coefficients have nonreal zeros in pairs}. Multiplying terms of the form $(x-\lambda)$ and $(x-\overline{\lambda})$ together, we get
\[
  (x^2 - 2( \realpart \lambda) x + |\lambda|^2),
\]

which is a quadratic term of the required form. But \ref{thm: polynomials with real coefficients have nonreal zeros in pairs} does not state that these two factors appear the same number of times, as needed to make the idea work. However, the next proof works around this point.

In the next result, either $m$ or $M$ may equal to $0$. The numbers $\lambda_1, \ddd, \lambda_m$ are precisely the real zeros of $p$, for these are the only real values of $x$ for which the right side of the equation in the next result equals 0.

\begin{thm}[factorization of a polynomial over $\real$]
  \label{thm: factorization of a polynomial over R}
  Suppose $p \in \mathcal{P}(\mathbb{R})$ is non-constant. Then $p$ has a unique factorization:
  \begin{equation}
    p(x) = c(x-\lambda_1) \cdots (x-\lambda_m)(x^2+b_1x+c_1) \cdots (x^2+b_Mx+c_M)
  \end{equation}
  where $c, \lambda_1, \ddd, \lambda_m, b_1, \ddd, b_M, c_1, \ddd, c_M \in \mathbb{R}$, with $b_k^2 < 4c_k \quad \forall k \in \{1, \ddd, M\}.$
\end{thm}
\begin{prf}
  First we will prove that the desired factorization exist, and after that, we will prove uniqueness.

  Think of $p$ as an element of $\polyn(\compl)$, not $\polyn(\real)$, but with real coefficients like stated in the theorem. If all (complex) zeros of $p$ are real, then we have the desired factorization by the second version of the fundamental theorem of algebra \ref{fundamental-theorem-of-algebra-second-version}. Thus suppose $p$ has a complex zero $\lambda \in \compl$ with $\lambda \notin \real$. By \ref{thm: polynomials with real coefficients have nonreal zeros in pairs}, $\overline \lambda$ is a zero of $p$ as well. Thus we write $\forall x \in \real$
  \[
  \begin{aligned}
    p(x)
    &= (x - \lambda) (x-\overline \lambda) q(x) \\
    &= (x^2 - 2 (\realpart  \lambda)x +  |\lambda|^2) q(x),  \\
  \end{aligned}
  \]

  % page 129
  for some $q \in \polyn(\compl)$, with $\deg q = \deg p -2$. If we can prove that $q$ has real coefficients, then using induction on the degree of such polynomial completes the proof of the existence part of this result. If we solve for $q(x)$, we get
  \[
    q(x) = \tfrac{p(x)}{x^2 - 2(\realpart \lambda) x  + |\lambda|^2} \quad \forall x \in \real.
  \]

  The equation above implies that $q(x) \in \real \quad \forall x \in \real$. Let $n :\equiv \deg p$. Then we can reqrite $q$ as
  \[
    q(x) = a_0 + a_1 x + \cdots + a_{n-2} x^{n-2} \mytext{where} a_0, \ddd, a_{n-2} \in \compl,
  \]

  because we don't know if the $a's$ are all real yet. But we know that $\forall x \in \real: q(x) \in \real$. Thus we have $\forall x \in \real:$
  \[
    0 = \impart q(x) = (\impart a_0) + (\impart a_1)x + \cdots + (\impart a_{n-2})x^{n-2}
  \]

  Since according to \ref{thm: a polynomial of degree m has at most m zeros} a complex polynomial with degree $n-2$ can have at most $n-2$ zeros, this equation can not hold for all $x \in \real$. This implies that $\impart a_0, \ddd, \impart a_{n-2}$ all equal to $0$, because only the zero polynomial is allowed th have infinitely many zeros. Thus all the coefficients of $q$ are real, as desired. Hence the wanted factorization exists by induction.

  \prooffont{Uniqueness:} A factor of $p$ of the form $x^2 + b_kx + c_k$ with $b_k^2 < 4c_k$ can be uniquely wirtten as $(x-\lambda_k)(x-\overline{\lambda_k})$ with $\lambda_k \in \compl$. Two different factorizations of $p$ as an element of $\polyn(\real)$ would lead to two different factorizations of $p$ as an element of $\polyn(\compl)$, contradicting using again the second version of the fundamental theorem of algebra $\ref{fundamental-theorem-of-algebra-second-version}$.
\end{prf}
