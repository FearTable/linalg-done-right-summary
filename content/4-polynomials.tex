
$\mathbb{F}$ denotes $\mathbb{R}$ or $\mathbb{C}$.

\subsection{Zeros of polynomials}
$p: \mathbb{F} \to \mathbb{F}, \, p(z) = a_0 + a_1z + \cdots + a_n z^m$

\setcounter{thm}{4}
\begin{thm}
  $\lambda \in \mathbb{F}$ is called a zero (or root) of a polynomial if $p(\lambda) = 0$.
\end{thm}

\setcounter{thm}{5}
\begin{thm}[each zero of a polynomial corresponds to a degree\-/one factor]
  % each zero of a polynomial corresponds to a degree-one factor
  \label{factororing-out-zeros-of-a-polynomial-always-possible}
  $\forall p \in \mathcal{P}_m(\mathbb{F}) \; \exists \lambda \in \mathbb{F}$ such that 
  \begin{equation}
  	p(\lambda) = 0 \iff \exists q \in \mathcal{P}_{m-1}: p(z) = (z-\lambda)q(z)
  \end{equation}
\end{thm}

\setcounter{thm}{7}
\begin{thm}
  A polynomial of degree $m$ has at most $m$ zeros.
\end{thm}
\begin{prf} 
  For $m=1$, $a_0+a_1z$ has only one zero, which is $-\frac{a_0}{a_1}$\\
  Now for $m>1$ and assuming the desired result holds for $m-1$. 
  
  If $p\in \mathcal{P}_{m}(\myF)$ has no zeros, we are done. 
  
  If $p$ has a zero $\lambda \in \myF$, by \ref{factororing-out-zeros-of-a-polynomial-always-possible}, there is a $q \in \mathcal{P}(\myF)_{m-1}$ such that 
  \begin{equation}
  	p(z)=(z-\lambda)q(z).
  \end{equation}
   
  Since $q$ has at most $m-1$ zeros, $p$ has at most $m$ zeros.
\end{prf}

\section{Division Algorithm for Polynomials}

\setcounter{thm}{8}
\begin{thm}
  \label{division-algorithm-for-polynomials}
  Suppose $p,s \in \mathcal{P} (\mathbb{F}), \, s\neq 0$. \\
  Then $\exists q,r \in \mathcal{P} (\mathbb{F})$ such that
  \begin{equation}
  	p=sq+r \myand \deg r < \deg s.
  \end{equation}
\end{thm}

\section{Factorization of Polynomials over $\compl$}
\setcounter{thm}{11}
\begin{thm}[Fundamental theorem of algebra, first version]
  \label{fundamental-theorem-of-algebra-first-version}
  Every non\-/constant polynomial with complex coefficients has a zero in $\mathbb{C}$.
\end{thm}

\begin{thm} [Fundamental theorem of algebra, second version:]
  \label{fundamental-theorem-of-algebra-second-version}
  $p \in \polyn(\mathbb{C}), \; p$ is non-constant $\implies p$ has a unique factorization
  \begin{equation}
  	p(z)=c \cdot (z-\lambda_1) \cdots (z-\lambda_m) 
  \end{equation}
  where $c, \lambda_1, \dots, \lambda_m \in \compl$.
\end{thm}

\section{Factorization of Polynomials over $\real$}
\begin{thm}
  Suppose $p\in \mathcal{P} (\mathbb{C})$ is a polynomial with real coefficients. If $\lambda \in \mathbb{C}$ is a zero of $p$, then so is $\overline{\lambda}$.
\end{thm}

\begin{thm} [Factorization of a quadratic polynomial]
  Let $p(x)=x^2+bx+c \in \polyn_2(\real)$ such that $b,c \in \real$. Then $p$ can be written as
  $p(x)=x^2 + bx + c$ $=$ $(x-\lambda_1)(x-\lambda_2)$ with $\lambda_1, \lambda_2 \in \mathbb{R} \iff b^2 \geq 4c$.
\end{thm}
\begin{prf}
  Let $x^2+bx+c=(x+\sfrac{b}{2})^2+(c-\sfrac{b^2}{4})$. Therefore, if $b^2<4c$, the last term is always positive and no factoriza\-tion is possible because it has no zeros (\ref{factororing-out-zeros-of-a-polynomial-always-possible}). \\
  If $b^2 \geq 4c$, we define $d^2 :\equiv \sfrac{b^2}{4}-c$ which makes
  \begin{equation}
	  \begin{aligned}
	    p(x)&=x^2+bx+c
		  = \left (x+\sfrac{b}{2} \right )^2-d^2
		  = \left ( (x+\sfrac{b}{2}) +d \right) \cdot \left ( (x+\sfrac{b}{2}) - d \right) \\
		  &= \left (x-(-d -\sfrac{b}{2}) \right) \cdot \left (x-(d-\sfrac{b}{2})\right)
		  =\left(x-\lambda_1 \right) \cdot \left(x-\lambda_2\right)
	  \end{aligned}
  \end{equation}
  for 
  \begin{equation}
    \lambda_1 :\equiv -d-\sfrac{b}{2} \mytext{and} \lambda_2 :\equiv d-\sfrac{b}{2}.
  \end{equation}
  
  which is the same as the well known midnight formula:
  \begin{equation}
    \lambda_{1,2} = \frac{-b \pm \sqrt{b^2-4c}}{2}
  \end{equation}
  which also shows, that $b^2 \geq 4c$ is a necessary condition for real solutions for $\lambda_{1,2}$
\end{prf}

\begin{thm}
  Suppose $p \in \mathcal{P}(\mathbb{R})$ is non-constant. Then $p$ has a unique factorization:
  \begin{equation}
    p(x) = c(x-\lambda_1) \cdots (x-\lambda_m)(x^2+b_1x+c_1) \cdots (x^2+b_Mx+c_M)
  \end{equation}
  where $c, \lambda_1, \cdots, \lambda_m, b_1, \cdots, b_M, c_1, \cdots, c_M \in \mathbb{R}$
\end{thm}

