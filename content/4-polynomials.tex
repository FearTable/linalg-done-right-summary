
$\mathbb{F}$ denotes $\mathbb{R}$ or $\mathbb{C}$.

\section{Zeros of polynomials}
$p: \mathbb{F} \to \mathbb{F}, \, p(z) = a_0 + a_1z + \cdots + a_n z^m$

\setcounter{thm}{4}
\begin{mydef}[zero of a polynomial]
  $\lambda \in \mathbb{F}$ is called a \qt{zero}  (or \qt{root}) of a polynomial if $p(\lambda) = 0$.
\end{mydef}

\setcounter{thm}{5}
\begin{thm}[each zero of a polynomial corresponds to a degree\-/one factor]
  \label{factororing-out-zeros-of-a-polynomial-always-possible}
  $\forall p \in \mathcal{P}_m(\mathbb{F}): \; \exists \lambda \in \mathbb{F}$ s.t.
  \begin{equation}
  	p(\lambda) = 0 \iff \exists q \in \mathcal{P}_{m-1}: p(z) = (z-\lambda)q(z)
  \end{equation}
\end{thm}
\begin{prf} Let $p\in \polyn(\myF)_m$ for $m\in \nat$.
  
  ``$\Leftarrow$ direction:'' Suppose there is $q \in \polyn(\myF)$ s.t. $p(z) = (z-\lambda)q(z) \quad \forall z \in \myF$. Then
  \begin{equation}
    p(\lambda) = (\lambda -\lambda)q(\lambda) = 0,$ as desired.$
  \end{equation}
  
  ``$\Rightarrow$ direction:'' Now suppose $p(\lambda)=0$. Let $a_0, a_1, \ldots, a_m \in \myF"$ be s.t.
  \begin{equation}
    p(z) :\equiv a_0 + a_1z + \cdots a_mz^m \quad \forall z \in  \myF.
  \end{equation}
  
  Then
  \begin{equation}
    \label{eq: second formula of pz: pz = pz - plambda}
    p(z) = p(z)- p(\lambda) = a_1 (z-\lambda) + a_2 (z^2 - \lambda^2) + \cdots+ a_m (z^m- \lambda^m) \quad \forall z \in \myF.
  \end{equation}
  
  Using the binomic formula 
  $z^k - \lambda^k = (z-\lambda) \sum_{j=1}^{k} \lambda^{j-1}z^{k-j} \quad \forall k\in \setonetillm$, we get 
  \begin{equation}
    p(z) = (z-\lambda)
    \underbrace{
      \left(
        a_1 + a_2 (z+\lambda) + a_3(z^2 + \lambda z + \lambda^2) +
        a_4 (z^3 + \lambda z^2 + \lambda^2 z + \lambda^3) 
        + \cdots +
        a_m \sum_{j=1}^{m} \lambda^{j-1} z^{m-j}
      \right)}_{\text{a polynomial with degree $m-1$}}
  \end{equation}
  
  Thus $p$ equals $z-\lambda$ times some polyomial of degree $m-1$, as desired.
\end{prf}

In case you are wondering where the $a_0$ went by subtracting $p(\lambda)$ from $p(z)$, it is hidden int the roots! For example $p\in \polyn_1(\myF): p=a_0+a_1z = \left(z- \left(-\frac{a_0}{a_1}\right)\right)(a_1)$ where $\left(-\frac{a_0}{a_1}\right)$ is the only root!

\setcounter{thm}{7}
\begin{thm}
  A polynomial of degree $m$ has at most $m$ zeros.
\end{thm}
\begin{prf} 
  For $m=1$, $a_0+a_1z$ has only one zero, which is $-\frac{a_0}{a_1}$\\
  Now for $m>1$ let us assume the desired result holds for $m-1$. 
  
  If $p\in \mathcal{P}_{m}(\myF)$ has no zeros, we are done. 
  
  If $p$ has a zero $\lambda \in \myF$, by \ref{factororing-out-zeros-of-a-polynomial-always-possible}, there is a $q \in \mathcal{P}(\myF)_{m-1}$ such that 
  \begin{equation}
  	p(z)=(z-\lambda)q(z).
  \end{equation}
   
  Since $q$ has at most $m-1$ zeros, $p$ has at most $m$ zeros.
\end{prf}

\section{Division Algorithm for Polynomials}

\setcounter{thm}{8}
\begin{thm}
  \label{division-algorithm-for-polynomials}
  Suppose $p,s \in \mathcal{P} (\mathbb{F}), \, s\neq 0$. 
  Then $\exists q,r \in \mathcal{P} (\mathbb{F})$ s.t.
  \begin{equation}
    p=sq+r \myand \deg r < \deg s.
  \end{equation}
\end{thm}
\begin{prf}
  Let $n:\equiv \deg p$ and $m:\equiv \deg s$. 
  
  If $n<m$, then take $q=0$ and $r=p$.
  
  Thus now we assume $n \geq m$. The list
  \begin{equation}
    \label{eq: basis of P_n(F)}
    1, z, \ldots, z^{m-1}, s, zs, \ldots, z^{n-m}s
  \end{equation}
  
  is linearly independent because each polynomial in this list has a different degree.
  
  The list has length $1+(m-1)+1+(n-m)=n+1$, which equals $\dim \polyn_n(\myF)$. Hence \eqref{eq: basis of P_n(F)} is a basis of $\polyn_n(\myF)$ by \ref{thm: linearly independent list of the right length is a basis}. Therefore, we have $a_0, a_1, \ldots, a_{m-1} \in \myF$ and $b_0, b_1, \ldots, b_{n-m} \in \myF$ s.t.
  \begin{equation}
    \begin{aligned}
          p&=a_0 + a_1z + a_2z^2 + \cdots + a_{m-1} z^{m-1}
          + b_0 s + b_1 zs + b_2 z^2 s + \cdots + b_{n-m}z^{n-m}s \\
           &=\underbrace{\left(a_0 + a_1z + a_2z^2 + \cdots + a_{m-1} z^{m-1} \right)}_{r} + \underbrace{\Big(s\Big)}_{s} \underbrace{\left( b_0 + b_1 z + b_2 z^2 + \cdots + b_{n-m} z^{n-m} \right)}_{q} \\
           &=r+sq=sq+r, \mytext{with} \deg r = m-1 \leq \deg s = m,
    \end{aligned}
  \end{equation}
  as desired.
  
  The uniqueness of $q,r \in \polyn(\myF)$ follows from the uniqueness of the $a$'s and $b$'s, because \eqref{eq: basis of P_n(F)} is a basis of $\polyn_n(\myF)$.
\end{prf}
\section{Factorization of Polynomials over $\compl$}
\setcounter{thm}{11}
\begin{thm}[\emph{fundamental theorem of algebra}, first version]
  \label{fundamental-theorem-of-algebra-first-version}
  Every non\-/constant polynomial with complex coefficients has a zero in $\mathbb{C}$.
\end{thm}

\begin{thm} [\emph{fundamental theorem of algebra}, second version:]
  \label{fundamental-theorem-of-algebra-second-version}
  $p \in \polyn(\mathbb{C}), \; p$ is non-constant $\implies p$ has a unique factorization
  \begin{equation}
  	p(z)=c \cdot (z-\lambda_1) \cdots (z-\lambda_m) 
  \end{equation}
  where $c, \lambda_1, \dots, \lambda_m \in \compl$.
\end{thm}

\section{Factorization of Polynomials over $\real$}

% 4.14
\begin{thm}[polynomials with real coefficients have nonreal zeros in pairs]
  \label{thm: polynomials with real coefficients have nonreal zeros in pairs}
  Suppose $p\in \mathcal{P} (\mathbb{C})$ is a polynomial with real coefficients. If $\lambda \in \mathbb{C}$ is a zero of $p$, then so is $\overline{\lambda}$.
\end{thm}
\begin{prf}
  Let $p(z) :\equiv a_0 + a_1 z + \ldots + a_m z^m$, where $a_0, \ldots, a_m \in \real$. Suppose $\lambda \in \compl$ is a zero of $p$. Then 
  \begin{equation}
    \begin{aligned}
      a_0+a_1\lambda+\cdots+a_m\lambda^m&=0\\
      \overline{a_0+a_1\lambda+\cdots+a_m\lambda^m}&=\overline{0}\\
      \overline{a_0}+\overline{a_1\lambda}+\cdots+\overline{a_m\lambda^m}&=0\\
      a_0+a_1\overline{\lambda}+\cdots+a_m\overline{\lambda}^m&=0
    \end{aligned}
  \end{equation}
  
  The equation above shows that $\overline{\lambda}$ is a zero of $p$.
\end{prf}

\begin{thm} [factorization of a quadratic polynomial]
  \label{thm: factorization of a quadratic polynomial}
  Let $p(x)=x^2+bx+c \in \polyn_2(\real)$ such that $b,c \in \real$. Then $p$ can be written as
  $p(x)=x^2 + bx + c$ $=$ $(x-\lambda_1)(x-\lambda_2)$ with $\lambda_1, \lambda_2 \in \mathbb{R} \iff b^2 \geq 4c$.
\end{thm}
\begin{prf}
  Let $x^2+bx+c=(x+\sfrac{b}{2})^2+(c-\sfrac{b^2}{4})$. Therefore, if $b^2<4c$, the last term is always positive and no factoriza\-tion is possible because it has no zeros by \ref{factororing-out-zeros-of-a-polynomial-always-possible}.
  
  If $b^2 \geq 4c$, we define $d^2 :\equiv \sfrac{b^2}{4}-c$ which makes
  \begin{equation}
	  \begin{aligned}
	    p(x)&=x^2+bx+c
		  = \left (x+\sfrac{b}{2} \right )^2-d^2
		  = \left ( (x+\sfrac{b}{2}) +d \right) \cdot \left ( (x+\sfrac{b}{2}) - d \right) \\
		  &= \left (x-(-d -\sfrac{b}{2}) \right) \cdot \left (x-(d-\sfrac{b}{2})\right)
		  =\left(x-\lambda_1 \right) \cdot \left(x-\lambda_2\right)
	  \end{aligned}
  \end{equation}
  for 
  \begin{equation}
    \lambda_1 :\equiv -d-\sfrac{b}{2} \mytext{and} \lambda_2 :\equiv d-\sfrac{b}{2}.
  \end{equation}
  
  which is the same as the well known midnight formula:
  \begin{equation}
    \lambda_{1,2} = \frac{-b \pm \sqrt{b^2-4c}}{2}
  \end{equation}
  which also shows, that $b^2 \geq 4c$ is a necessary condition for real solutions for $\lambda_{1,2}$
\end{prf}

\begin{thm}
  Suppose $p \in \mathcal{P}(\mathbb{R})$ is non-constant. Then $p$ has a unique factorization:
  \begin{equation}
    p(x) = c(x-\lambda_1) \cdots (x-\lambda_m)(x^2+b_1x+c_1) \cdots (x^2+b_Mx+c_M)
  \end{equation}
  where $c, \lambda_1, \cdots, \lambda_m, b_1, \cdots, b_M, c_1, \cdots, c_M \in \mathbb{R}$
\end{thm}

