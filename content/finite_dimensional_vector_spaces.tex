\chapter{Finite Dimensional Vector Spaces}
\section{The definition of span}
\setcounter{thm}{4}
\begin{mydef} The ``span" of a list of vectors $\onetillm{v}$ is defined as follows:
 \begin{equation}
     \myspan {v_1, \dots, v_m} :\equiv \{a_1 v_1 + \cdots + a_m v_m \mid a_1, \dots a_m \in \mathbb{F} \}
 \end{equation}   
\end{mydef}

%\textbf{2.6} 

\begin{thm}
    The  span of a list of vectors in $V$ is the smallest subspace of $V$ containing all vectors in the list.
\end{thm}

%\textbf{2.7} 
\begin{thm}
    If $\myspan{\oneTillM{v}} = V$, we say $\oneTillM{v}$ ``spans" $V$.    
\end{thm}

\setcounter{thm}{8}
%\textbf{2.9} 
\begin{thm}
    Such a vector space is called finite-dimensional. $\dim V \neq \infty$
\end{thm}

%\textbf{2.10, 2.11, 2.12} 
\begin{thm}
    $\mathcal{P} (\mathbb{F} )=\myspan{1,z,\dots,z^m}$ denotes the set of all polynomials with coefficients in $\myF$ and degree at most $m$. The degree of a polynomial $p$ is denoted with $\degree p$. We also define $\degree 0 : \equiv -\infty$
\end{thm}


\section{Linear Independence}
\setcounter{thm}{14}
\begin{mydef}
    There are $2$ ways to define linear independence. A list of vectors $\oneTillM{v}$ is called ``\lid", if every combination of these vectors has a unique representation $w = b_1v_1 + \dots+ b_mv_m \in \myspan{\oneTillM{v}}$. (The list $\onetillm{v}$ is \lid, if the choice of $b$'s to yield $w$ is unique.) \\
    Another way to put it: A list of vectors $\oneTillM{v}$ is also called \lid, if the only way to combine them together and yield zero $\lambda_1v_1 + \dots + \lambda_mv_m = 0$, is if we choose every coefficient $\lambda_i$ to be zero.
    \begin{equation}
        \begin{aligned}
            w & =a_1 v_1 + \cdots + a_m v_m \; \text{and} \\ 
            w & =c_1 v_1 + \cdots + c_m v_m \\
            & \iff \\
            0 & = \underbrace{(a_1 - c_1)}_{= \, 0\text{?}} v_1 + \cdots + \underbrace{(a_m -c_m)}_{= \, 0\text{?}} v_m
        \end{aligned}
    \end{equation}
    
    if $a_1 = c_1, a_2 = c_2, \, \dots \, , a_m = c_m$ is the only solution, the representation of $w$ is unique and the only way to add them up together equaling $0$ is $0v_1+\cdots+0v_m=0$. Both of these definitions are equivalent.
\end{mydef}


%\textbf{2.16} 
\begin{mydef}
    Otherwise, $\oneTillM{v}$ is called ``linearly dependent".
\end{mydef}

%	\textbf{2.19} Linear dependence lemma:\\
%	Suppose $\oneTillM{v}\in V$ is a linearly dependent list.
%	$\implies \exists \kInOneTillM$ : $v_k \in \myspan{\oneTill{v}{k-1}}$


%\textbf{2.29} 

\setcounter{thm}{18}
%\textbf{2.19} 
\begin{thm}
    \label{linear-dependence-lemma}
    \bfemph{Linear dependence Lemma:} \\
    Suppose $v_{1}, \dots, v_{m}\in V$ is a linearly dependent list. 
    $\myimpl \exists k \in \{ 1, \dots, m \} : v_k \in \myspan {v_1, \dots, v_{k-1}}$. \\
    Furthermore, if $k$ satisfies the conditions above and the $k^{\text{th}}$-term is removed from $v_1, \dots, v_m$, then the span of the remaining list equals $\myspan {v_1, \dots, v_{k-1}}$:
    \begin{equation}
        \myspan {v_1, \dots, v_{k-1}, v_{k+1}, \dots, v_m} = \myspan {v_1, \dots, v_m}
    \end{equation}
\end{thm}

\setcounter{thm}{21}
%\textbf{2.22} 

\begin{thm}
    \bfemph{Length of linearly independent list $\mathbb{\leq}$ length of spanning list:}\\
    In a finite dimensional vector space, the length of every linearly independent list of vectors is less then or equal to the length of every spanning list of vectors. From now on, every vector space we consider is finite dimension if not explicitly mentioned otherwise.
\end{thm}


\section{Bases}

\setcounter{thm}{25}
%\textbf{2.26} 
\begin{mydef}
    A basis of $V$ is a list of vectors $\onetillm{v}$ in $V$ that is linearly independent and spans $V$.
\end{mydef}


\setcounter{thm}{27}
\begin{mydef} 
    $\onetillm{v}$ is a basis of $V$ $\overset{\text{def}}{\iff}$ every $v \in V$ can be written uniquely in the form $v=a_1 v_1 + \dots + a_n v_n$. 
\end{mydef}

\setcounter{thm}{29}
\begin{thm} Every spanning list of a \vs can be reduced to a basis of the \vs.
\end{thm}

\setcounter{thm}{30}
\begin{thm} Every \findimvs has a basis.\end{thm}

\begin{thm} Every \lid list of vectors in a  \findimvs can be reduced to a basis of the \vs. \end{thm}

\begin{thm} Suppose $V$ is \fd and $U$ is a subspace of $V$. Then there is a subspace $W$ of $V$ such that $V=U \oplus W$. \end{thm}

\section{Dimensions}

\setcounter{thm}{33}
\begin{thm} Any two bases of a \fdvs have the same length (Another wording: basis length does not depend on basis, by 2.22)
\end{thm}

\begin{mydef} $\dim V :\equiv$ length of any basis of $V$
\end{mydef}

\setcounter{thm}{36}
\begin{thm} If $V$ is \fd  and $U$ is a subspace of $V$, $\implies \dim U \leq \dim V$ (by 2.22)
\end{thm}

\begin{thm} Every \lid list of vectors of length $\dim V$ is a basis. (by 2.32)   
\end{thm}

\begin{thm} If $U$ is a subspace of $V$ and $\dim U = \dim V$, then $U=V$. (by 2.38)
\end{thm}

\setcounter{thm}{41}
\begin{thm}
Every spanning list of vectors in $V$ of length $\dim V$ is a basis of $V$
\end{thm}

\begin{thm} $V_1$ and $V_2$ are subspaces of a \fdvs
\begin{equation}
	\implies \dim (V_1 + V_2) = \dim V_1 + \dim V_2 - \dim (V_1 \cap V_2)
\end{equation}
\end{thm}