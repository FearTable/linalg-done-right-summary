
\section{Duality}
\subsection{Dual Space and Dual Map}

\setcounter{thm}{107}
%\textbf{3.108}
\begin{mydef}
  A ``linear functional'' $\phi$ is an element of $\mathcal{L}(V, \mathbb{F})$. So $\phi \in \linmap(V, \myF)$
\end{mydef}

\textbf{Examples:}
\begin{equation}
  \begin{array}{lll}
    \phi: \mathbb{R}^3  \to \mathbb{R}, &\phi (x,y,z)  & \mapsto 4x-5y-2z \\
    \phi: \mathbb{F}^n  \to \mathbb{F}, &\phi (x_1, \dots, x_n)
    & \mapsto c_1x_1 + \dots + c_nx_n  \\
    \phi: \mathcal{P} (\mathbb{R})  \to \mathbb{R},
    & \phi(p) & \mapsto 3p''(5) + 7p(4) \\
    \phi: \mathcal{P}(\mathbb{R}) \to \mathbb{R},
    & \phi(p)  &\mapsto \textstyle \int_{0}^{1} p(x) dx
  \end{array}
\end{equation}

\setcounter{thm}{109}
%\textbf{3.110}
\begin{thm}
  The dual space of $V$, denoted by $\dual{V}$ or $V^{*}$, is the vector space of all linear functionals on $V$.
  \begin{equation}
    \dual{V} :\equiv \linmap(V, \myF)
  \end{equation}
\end{thm}

%\textbf{3.111}
\begin{thm}
  $\dim \dual{V} = \dim V$.
\end{thm}
\begin{prf}
  $\dim \dual{V} = \dim \mathcal{L}(V, \mathbb{F})=(\dim V) \cdot (\dim \mathbb{F}) = \dim V $
\end{prf}


%\textbf{3.112}
\begin{mydef}
  If $\onetillm{v}$ is a basis of $V$, then the ``dual basis'' of $\onetilln{v}$ is the list $\varphi_1, \dots, \varphi_m \in V^{*}$, where each $\varphi_j$ is the linear functional such that
  \begin{equation}
    \varphi_j(v_k) = \delta_{j,k} =
  \begin{cases}
    1,  & \text{if $k=j$} \\
    0, & \text{if $k \neq j$}
  \end{cases}
  \end{equation}
  Note that this is not the definition of each $\varphi_j$.
\end{mydef}

%\textbf{3.114}
\setcounter{thm}{113}
\begin{thm}
  Suppose $\onetillm{v}$ is a basis of $V$. $\onetillm{\varphi}$ is called ``a dual basis'' of $\onetillm{v}$ because for every $v \in V$ we have $v=\varphi_1 (v)v_1 + \cdots \varphi_m(v)v_m$
\end{thm}
\begin{prf}
  Let $V \ni v= c_1v_1 + \cdots c_mv_m$. Applying
  $\varphi_j$ on both sides gives $\varphi_j(v)=c_j$ $\forall j\in \{1, \dots, m\}$
  %    \iff$ $\varphi(v)=\phi(c_1v_1 + \cdots + c_mv_m)$ such that $\varphi_j(v)=c_j$, because $\varphi$ is linear and $\varphi_j(v_j)=1$ by definition.
\end{prf}

%\textbf{3.116}
\setcounter{thm}{115}
\begin{thm}
  The dual basis of a basis of $V$ is a basis of the dual space $V^{*}$
\end{thm}

%\textbf{3.118}
\setcounter{thm}{117}
\begin{mydef}

  The ``dual map'' $T'$ of $T \in \linmap(V,W)$ is the linear map $T' \in \lin{W'}{V'}$ defined like this:

  \begin{itemize}
    \item[] $T'(\varphi) :\equiv \varphi \circ T \quad \forall \phi \in W'$
    \item $\varphi \in W'=\lin{W}{\mathbb{F}} \text{ and } T'(\varphi) \in V' = \lin{V}{\mathbb{F}}$.

    yo
    So $T'$ is indeed a map from $W'$ to $V'$


    \item $\varphi, \psi \in W' \implies T' (\varphi + \psi) = (\varphi + \psi) \circ T = \varphi \circ T + \psi \circ T = T' (\varphi) + T'(\psi)$
    \item $\lambda \in \mathbb{F}, \varphi \in W \implies T' (\lambda \varphi) = (\lambda \varphi) \circ T = \lambda (\varphi \circ T) = \lambda T' (\varphi)$
  \end{itemize}

\end{mydef}

\setcounter{thm}{119}
\begin{thm}
  \label{algebraic-properties-of-dual-maps}
  Algebraic properties of dual maps:  \\
  $T \in \linmap(V,W)$ $\implies$
  \begin{enumerate}
    \item $\dual{(S+T)} = \dual{S} + \dual{T} \quad \forall S \in \linmap(V,W)$
    \item $\dual{(\lambda T)} = \lambda \dual{T} \quad \quad \forall \lambda \in \myF$
    \item $\dual{(ST)} = \dual{T}\dual{S} \quad \; \forall S \in \linmap (W,U)$
  \end{enumerate}
\end{thm}



\subsection{Dual Space and Range of Dual of Linear Map}

\begin{mydef} [annihilator, $U^0$]
  \label{def: annihiltator}
  For $U \subseteq V$, the ``annihilator'' of $U$, denoted by $U^{0}$, is defined by
  \begin{equation}
    U^0 = \{ \varphi \in \dual{V} \; \mid \; \varphi (u) = 0 \quad \forall u \in U \}
  \end{equation}
\end{mydef}

% 3.124
\setcounter{thm}{123}
\begin{thm}[the annihilator is a subspace]
  \label{thm: the annihilator is a subspace}
  $U \subseteq V \implies U^{0}$ is a subspace of $\dual{V}$
\end{thm}

% 3.125
\setcounter{thm}{124}
\begin{thm} [dimension of the annihilator]
    $U\subseteq V$ and $\dim V < \infty \implies$
    \begin{equation}
      \dim U^0 = \dim V - \dim U.
    \end{equation}
\end{thm}

% 3.127
\setcounter{thm}{126}
\begin{thm} [condition for the annihilator to equal $\{0\}$ or the whole space]
  If $\dim V < \infty$ and $U\subseteq V$. Then
  \begin{enumerate}
    \item $U^0 = \{0 \} \iff U = V$;
    \item $U^0 = \dual{V} \iff U = \{0\}$;
   \end{enumerate}
\end{thm}

% 3.128
\begin{thm}[the null space of $\dual{T}$]
  If $\dim V < \infty$, $\dim W<\infty$ and $T \in \linmap (V,W),$ then
  \begin{enumerate}
    \item $\mynull \dual{T} = (\myrange T)^0$;
    \item $\dim \mynull \dual{T} = \dim \mynull T + \dim W - \dim V$
  \end{enumerate} 
\end{thm}

% 3.129
\begin{thm}[$T$ surjective, $\dual{T}$ injective]
    If $\dim V < \infty$, $\dim W<\infty$ and $T \in \linmap (V,W),$ then
  \begin{equation}
    T$ is surjective $ \iff \dual{T} $ is injective$
  \end{equation}
\end{thm}

% 3.130
\begin{thm}[the range of $\dual{T}$]
    If $\dim V < \infty$, $\dim W<\infty$ and $T \in \linmap (V,W),$ then
    \begin{enumerate}
      \item $\dim \myrange \dual{T} = \dim \myrange T$;
      \item $\myrange \dual{T} = (\mynull T)^0$;
    \end{enumerate}
\end{thm}

% 3.131
\begin{thm} [$T$ injective, $\dual{T}$ surjective]
    If $\dim V < \infty$, $\dim W<\infty$ and $T \in \linmap (V,W),$ then
    \begin{equation}
      T $ is injective $ \iff \dual{T} $ is surjective. $
    \end{equation}
\end{thm}

\subsection{Matrix of Dual of Linear Map}

% 3.132
\setcounter{thm}{131}
\begin{thm}
  Suppose $V, W$ are finite-dimensional and $T \in \linmap(V,W)$.
  \begin{equation}
    \implies \mmatrix(\dual{T}) = (\mmatrix(T))^{\top}
  \end{equation}
\end{thm}

%3.133
\begin{thm}
  If $A \in \myF^{m,n}$, then the column rank of $A$ equals the row rank of $A$.
\end{thm}