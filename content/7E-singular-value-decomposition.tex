\section{Singular Value Decomposition}

\subsection{Singular Values}

\mce{64}
\begin{thm}[properties of $T*T$]
  \label{thm: properties of T^*T}
  Suppose $T \in \linmap(V,W)$. Then
  \begin{enumerate}[label=\textbf{(\alph*)}]
    \item $T^*T$ is a positive operator on $V$;
    \item $\mynull T^*T = \mynull T$;
    \item $\myrange T^*T = \myrange T^*$;
    \item $\dim \myrange T = \dim \myrange T^* = \dim  \myrange T^*T$;
  \end{enumerate}
\end{thm}

\begin{mydef}[singular values]
  Suppose $T \in \linmap (V,W)$. The \qt{singular values} of $T$ are the nonegative square roots of the eigenvalues of $T^*T$, listed in decreasing order, each included as many times as the dimension of the corresponding eigencspace of $T^*T$.
\end{mydef}

\mce{68}
\begin{thm}[role of positive singular values]
  Suppose that $T\in \linmap (V,W)$. Then
  \begin{enumerate} [label=\textbf{(\alph*)}]
    \item $T$ is injective $\iff$ $0$ is not a singular value of $T$;
    \item the number of positive singular values of $T$ equals $\dim \myrange T$;
    \item $T$ is surjective $\iff$ number of positive singular values of $T$ equals $\dim W$.
  \end{enumerate}
\end{thm}

The table below compares eigenvalues with singular values.

\begin{tabular}{|p{.45 \textwidth}|p{.45 \textwidth}|}
  \hline
  \emph{List of eigenvalues} & \emph{List of singular values} \\
  \hline
  context: vector spaces & context: inner product spaces \\
  \hline
  defined only for linear maps from a vector space to itself & defined for linear maps from an inner producte space to a possibly different inner product space \\
  \hline
  can be arbitrary real numbers (if $\myF = \real$) or complex numbers (if $\myF=\compl$) & are nonnegative numbers \\
  \hline
  can be the emptly list if $\myF = \real$ & length of list equals dimension of domain \\
  \hline
  includes $0$ $\iff$ operator is not invertible & includes $0$ $\iff$ linear map is not injective \\
  \hline
  no standard order, especially if $\myF = \compl$ & always listed in decreasing order \\
  \hline
\end{tabular}

\mce{69}
\begin{thm}[isometries characterized through singular values]
  \label{thm: isometries characterized by having all singular a values equal 1}
  Suppose $S \in \linmap (V,W)$. Then
  \[
    S$ is an isometry $ \iff $ all singular values of $S$ equal $1.
  \]
\end{thm}

\subsection{SVD for Linear Maps and for Matrices}

\mce{70}
\begin{thm}[singular value decomposition]
  Let $T\in \linmap (V,W)$ and the positive singular values of $T$ are $s_1, \ddd, s_m$. Then there exist two orthonormal lists $e_1, \ddd, e_m$ in $V$ and $f_1, \ddd, f_m$ in $W$ \st
  \[
    Tv = s_1 \ip{v}{e_1} f_1 + \cdots + s_m \ip{v}{e_m}f_m \quad \forall v \in V.
  \]
\end{thm}

\mce{74}
\begin{mydef}[diagonal matrix]
  An $M$-by-$N$ matrix $A$ is called a \qt{diagonal matrix} if all entries of the matrix are $0$ except possibly $A_{k,k}$ for $k \in \big\{1, \ddd, \operatorname{min} \{M,N\} \big \}$.
\end{mydef}

The table below compares the spectral theorem with the singular value decomposition

\begin{tabular}{|p{.45 \textwidth}|p{.45 \textwidth}|}
  \hline
  \emph{Spectral theorem} & \emph{Singular value decomposition} \\
  \hline
  describes only self-adjoint operators (when $\myF = \real$) or normal operators (when $\myF = \compl$) & describes arbitrary linear maps from an inner product space to a possibly different inner product space \\
  \hline
  produces a single orthonormal basis &
  produces two orthonormal lists, one for domain space and one for range space, that are not necessarily the same even when range space equals domain space \\
  \hline
  different proofs depending on wether $\myF = \real$ or $\myF = \compl$ & same proof works regardless of whether $\myF = \real$ or $\myF = \compl$ \\
  \hline
\end{tabular}

\begin{thm}[singular value decomposition of adjoint and pseudoinverse]
  \label{thm: singular value decomposition of adjoint and pseudoinverse}
  Suppose $T \in \linmap(V,W)$ and the positive singular values of $T$ are $s_1, \ddd, s_m$. Suppose $e_1, \ddd, e_m$ and $f_1, \ddd, f_m$ are orthonormal lists in $V$ and $W$ such that we have the following SVD:
  \[
    Tv = s_1 \ip{v}{e_1} f_1 + \cdots + s_m \ip{v}{e_m}f_m \quad \forall v \in V.
  \]

  Then we have $\forall w \in W$:
  \begin{itemize}
    \item $T^*w = s_1 \ip{w}{f_1}e_1 + \cdots + s_m \ip{w}{f_m}e_m$
    \item $T^\dagger w = \tfrac{\ip{w}{f_1}}{s_1} e_1 + \cdots + \tfrac{\ip{w}{f_m}}{s_m} e_m$
  \end{itemize}
\end{thm}

\begin{thm}[matrix version of SVD]
  Suppose $A$ is a $p$-by-$n$ matrix of rank $m \geq 1$. Then there exists a $p$-by-$n$ matrix $B$ with orthonormal columns, and $m$-by-$m$ diagonal matrix $D$ with positive numbers on the diagonal and an $n$-by-$m$ matrix $C$ with orthonormal columns \st
  \[
    A = BDC^*.
  \]
\end{thm}