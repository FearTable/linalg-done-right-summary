\section{Isometries, Unitary Operators, and Matrix Factorization}
\subsection{Isometries}

\begin{mydef}[isometry]
  $S \in \lin{V}{W}$ is called an \qt{isometry} if
  \[
  \norm{Sv} = \norm{v} \quad \forall v \in V.
  \]

  In other words, a linear map is an isometry if it preserves norms.

  If $S \in \linmap(V,W)$ is an isometry an $v \in V$ is \st $Sv = 0$, then
  \[
  \norm{v} = \norm{Sv} = \norm{0} = 0,
  \]

  which implies that $v=0$. Thus every isometry is injective.
\end{mydef}

(The Greek word \emph{isos} means \qt{equal}; the Greek word \emph{metron} means \qt{measure}. Thus \emph{isometry} literraly means \qt{equal measure}.)

\mce{49}
\begin{thm}[characterizations of isometeries]
  \label{thm: characterizations of isometeries}
  Suppose $S \in \linmap (V,W)$. Suppose $e_1, \ddd, e_n$ is an orthonormal basis of $V$ and $f_1, \ddd, f_m$ is an orthonormal basis of $W$. Then the following are equivalent.
  \begin{enumerate}[label=\textbf{(\alph*)}]
    \item $S$ is an isometry.
    \item $S^*S = I$.
    \item $\ip{Su}{Sv} = \ip{u}{v} \quad \forall u,v \in V$.
    \item $Se_1, \ddd, Se_n$ is an orthonormal list in $W$.
    \item The columns of $\mmatrix(S, (e_1, \ddd, e_n), (f_1, \ddd, f_m))$ form an orthonormal list in $\myF^m$ with respect to the Euclidean inner product.
  \end{enumerate}
\end{thm}
\begin{prf}
  We will proof $(a) \Rightarrow (b) \Rightarrow (c) \Rightarrow (d) \Rightarrow (e)  \Rightarrow (a)$

  \qt{First step:} First suppose (a) holds, so $S$ is an isometry. If $v\in V$, then
  \[
  \ip{(I-S^*S)v}{v} = \ip{v}{v} - \ip{S^*Sv}{v} = \norm{v}^2-\ip{Sv}{Sv} = \norm{v}^2 - \norm{Sv}^2 = 0.
  \]

  Since $I-S^*S$ is a self adjoint operator because $(I-S^*S)^* = I-S^*S$, we can use \ref{thm: T self-adjoint and <Tv,v>=0 for all v <=> T=0} to conclude that $I-S^*S$ equals the zero operator. Thus $S^*S = I$, proving $(a) \implies (b)$.

  \qt{Second step:} Now suppose (b) holds, so $S^*S=I$. If $u,v \in V$ then
  \[
  \ip{Su}{Sv} = \ip{S^*Su}{v} = \ip{Iu}{v} = \ip{u}{v}, \; $ proving $ (b) \implies (c)
  \]

  \qt{Third step:} Now suppose that (c) holds, so $\ip{Su}{Sv} = \ip{u}{v} \quad \forall u,v \in V$. Thus if $j,k \in \{1, \ddd, n\}$, then
  \[
  \ip{Se_j}{Se_k} = \ip{e_j}{e_k}.
  \]

  Hence $Se_1, \ddd, Se_n$ is an orthonormal list in $W$, proving that $(c) \implies (d)$.
\end{prf}

\subsection{Unitary Operators}

\mce{51}
\begin{mydef}[unitary operator]
  An operator $S \in \linmap (V)$ is called \qt{unitary} if $S$ is an ivertible isometry.
\end{mydef}

Note: Every isometry is injective. Every injective operator on a finite-dimensional vector space is invertible. So \qt{unitary} and \qt{isometry} mean the same thing for operators on finite-dimensional inner product spaces. But remember: A unitary operator maps a vector space to itself, while an isometry maps a vector space to another (possibly different) vector space.

\begin{example}[rotation of $\real^2$]
  Suppose $\theta \in \real$ and $S$ is the operator on $\myF^2$ whose matrix $\mmatrix(S)$ with respect to the standard basis of $\myF^2$ is
  \[
    \left(\begin{array}{cc}
      \cos \theta & - \sin \theta \\
      \sin \theta & \cos \theta
    \end{array} \right)
    = \mmatrix(S).
  \]

  The two columns of this matrix form an orthonormal list in $\myF^2$. Therefore, $S$ is an isometry by the equivalence of (a) and (e) in \ref{thm: characterizations of isometeries}. Thus $S$ is a unitary operator.

  If $\myF=\real$, then $S$ is the counterclockwise rotation by $\theta$ radians around to origin of $\real^2$. So it preserves norms.
\end{example}

\mce{53}
\begin{thm}[characterizations of unitary operators]
  \label{thm: characterizations of unitary operators}
  Suppose $S \in \linmap(V)$. Suppose $e_1, \ddd, e_n$ is an orthonormal basis of $V$. Then
  \begin{enumerate}[label=\textbf{(\alph*)}]
    \item $S$ is a unitary operator.
    \item $S^*S = SS^* = I$.
    \item $S$ is invertible an $S^{-1} = S^*$.
    \item $Se_1, \ddd, Se_n$ is an orthonormal basis of $V$.
    \item The rows of $\mmatrix(S,(e_1, \ddd, e_n))$ form an orthonormal basis of $\myF^n$ with respect to the Euclidean inner product.
    \item $S^*$ is a unitary operator.
  \end{enumerate}
\end{thm}

\mce{54}
\begin{thm}[eigenvalues of unitary operators]
  \label{thm: eigenvalues of unitary operators}
  Suppose $\lambda$ is an eigenvalue of a unitary operator. Then $|\lambda| = 1$.
\end{thm}

\mce{55}
\begin{thm}[description of unitary operators on complex inner product spaces]
  \label{thm: description of unitary operators on complex inner product spaces}
  Suppose $\myF = \compl$ and $S \in \linmap(V)$. Then the following are equivalent.
  \begin{enumerate}[label=\textbf{(\alph*)}]
    \item $S$  is a unitary operator.
    \item There is an orthonormal basis of $V$ consisting of eigenvertors of $S$ whose corresponding eigenvalues all have absolute value $1$.
  \end{enumerate}
\end{thm}

\mce{56}
\section{QR Factorization}
\begin{mydef}[unitary matrix]
  An $n$-by-$n$ matrix is called \qt{unitary} if its columns form an orthnormal list in $\myF^n$.
\end{mydef}

\begin{thm}[characterizations of unitary matrices]
  Let $Q \in \myF^{n,n}$. Then the following are equivalent.
  \begin{enumerate}[label=\textbf{(\alph*)}]
    \item $Q$ is a unitary matrix.
    \item The rows of $Q$ form an orthonormal list in $\myF^{n}$.
    \item $\norm{Qv} = \norm{v} \quad v \in \myF^n$.
    \item $Q^*Q=QQ^*=I$, the $n$-by-$n$ matrix with $1$'s on die diagonal and $0$'s elswhere.
  \end{enumerate}
\end{thm}

\begin{thm}[QR factorization]
  Suppose $A$ is a square matrix with linearly independent columns. Then there exists unique matrices $Q$ and $R$ such that $Q$ is uniatry, $R$ is upper triangular with only positive numbers on its diagonal, and
  \[
    A=QR.
  \]
\end{thm}


\subsection{Cholesky Factorization}

\mce{61}
\begin{thm}[positive invertible operator]
  \label{thm: positive invertible operator}
  A self-adjoint operator $T\in \linmap(V)$ is a positive invertible operator $\iff$ $\ip{Tv}{v} >0 \quad \forall v \in V$, $v \neq 0$.
\end{thm}

\begin{mydef}[positive definite]
  A matrix $B \in \myF^{n,n}$ is called \qt{positive definite} if $B^* = B$ and
  \[
    \ip{Bx}{x} > 0 \quad \forall x \in \myF^n, \, x \neq 0.
  \]
\end{mydef}

\begin{thm}[Cholesky factorization]
  Suppose $B$ is a positive definite matrix. Then there exists a unuqie upper triangular matrix $R$ with only positive numbers on its diagonal \st
  \[
    B = R^*R.
  \]
\end{thm}

