\section{Definition of a Vector Space}

\setcounter{thm}{19}
\begin{mydef} [vector space]
  A \qt{vector space} is a set $V$ along with an addition on $V$ and a scalar multiplication on $V$ such that the following properties hold.
  \begin{itemize}
    \item \emph{Commutativity:}
    $ u+v = v+u \quad \forall u,v \in V$
    \item \emph{Associativity:}
    $(u+v)+w=u+(v+w)$ and $(ab)v=a(bv) \quad \forall u,v,w \in V$ and $\forall a,b \in \myF$
    \item \emph{Additive identity:}
    $\exists 0\in V:v+0=v \quad \forall v\in V$.

    $\exists!$ is not a requirement, but a property of vector spaces. See \ref{thm: unique additive identity}.
    \item \emph{Additive inverse:}
    $\forall v\in V \quad \exists w\in V: v+w=0$
    \item \emph{Multiplicative identity:}
    $1v = v \quad \forall v\in V$
    \item \emph{Distributive properties:}
    $
      \alignLongunderstack{
        &a(u+v) = au+av \quad \forall u,v \in V \myand  \\
        &(a+b)v \, = av + bv \quad \forall a,b \in \myF \myand v \in V.
      }
    $
  \end{itemize}
\end{mydef}

\setcounter{thm}{23}
\begin{mydef} [notation $\myF^S$]
  If $S$ is a set, then $\myF^S$ denotes the set of functions from $S$ to $\myF$. Let $f, g \in \myF^{S}$ and $\lambda \in \myF$:
  \begin{itemize}
    \item The \qt{sum} $f+g \in \myF^{S}$ is defined as follows: $(f+g)(x) :\equiv f(x)+ g(x) \quad \forall x\in S$
    \item The \qt{product} $\lambda f \in \myF^{S}$ is defined as follows: $(\lambda f)(x) :\equiv \lambda f(x) \quad \forall x \in S$.
  \end{itemize}
  The vector space $\myF^{n} :\equiv \myF^{\{1,2,\dots,n\}}$ is a special case, because each $(x_1, \dots, x_n) \in \myF^{n}$ can be thought of as a function $x$ from the set $\{1, 2, \dots, n\}$ to $\myF$ by writing $x(k)$ instead of $x_k$ for the $k^{\text{th}}$ coordinate of $(x_1, \dots, x_n)$.
\end{mydef}

\setcounter{thm}{25}
\begin{thm} [unique additive identity]
  \label{thm: unique additive identity}
  A vector space has a unique additive identity.
\end{thm}
\begin{prf}
  Suppose $0$ and $0'$ are both additive identities. Then $0' = 0'+ 0 = 0 + 0' = 0.$
\end{prf}

\begin{thm} [unique additive inverse]
  Every element in a vector space has a unique additive inverse.
\end{thm}
\begin{prf}
  Let $v\in V.$ Suppose $u,w \in V$ are additive inverses of $v$. Then using associativity and the definition of the additive inverse we get
  \begin{equation}
    u = u + 0 = u + (v + w) = (u+v) + w = 0 + w =w
  \end{equation}
  Thus $u = w$, as desired.
\end{prf}

\begin{mydef}
  For $v,w\in V$, $-v$ denotes the additive inverse of $v$ and $w-v:\equiv w+(-v)$
\end{mydef}

\begin{mydef}
  For the rest of this summary, $V$ denotes a vector space over $\myF$.
\end{mydef}

% 1.30
\begin{thm} [the number $0$ times a vector]
  $0v = 0 \quad \forall v\in V$
\end{thm}
\begin{prf}
  $v \in V: 0v = (0+0)v = 0v +0v \implies 0 = 0v$. Using the definition of additive identity and the distributive property in $\myF$. After that, we subtract $0v$ on both sides of the equation.
\end{prf}

\begin{thm} [a number times the vector $0$]
  $a \vec0= \vec0 \quad \forall a\in \myF$
\end{thm}
\begin{prf}
  $a \in \myF: a\vec0 = a(\vec0+\vec0) = a\vec0 + a\vec0 \implies \vec0 = a\vec0$. Using the definition of additive identity and the distributive property for $\vec0 \in V, a \in \myF$. After that, we subtract $\vec0a$ on both sides of the equation.
\end{prf}

\begin{thm} [a number $-1$ times a vector]
  \label{thm: minus one times a vector}
  $(-1)v = -v \quad \forall v\in V$
\end{thm}
\begin{prf}
  Let $v \in V: v + (-1)v = 1v + (-1)v = (1 + (-1))v = 0v = 0$ using the distributive property and the defintion of the multiplicative identity. This equation [$v+(-1)v=0$] tells us, that $(-1)v$ is the additive inverse of $v$, therefore $(-1)v = -v$.
\end{prf}
