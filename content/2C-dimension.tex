\section{Dimension}

% 2.34
\setcounter{thm}{33}
\begin{thm} [basis length] Any two bases of a \fdvs have the same length. So the the basis length does not depend on basis.
\end{thm}
\begin{prf}
  Let $B_1$ and $B_2$ be $2$ basis of $V$. Since both are linearly independent and both span $V$, we have $|B_1| \leq |B_2|$ and $|B_2| \leq |B_1|$ by using \ref{thm: length of linearly dependent list less or equal to length of spanning list}. Therefore $|B_1|=|B_2|.$
\end{prf}

\begin{mydef} [dimension]
  $\dim V :\equiv$ length of any basis of $V$.
\end{mydef}

% 2.37
\setcounter{thm}{36}
\begin{thm} [dimension of a subspace]
  \label{thm: dimension of a subspace}
  If $V$ is \fd and $U$ is a subspace of $V$, then
  \begin{equation}
    \dim U \leq \dim V
  \end{equation}
\end{thm}
\begin{prf}
  Since the length of everly linearly independent list is smaller or equal to every spanning list in $V$ according to
  \ref{thm: length of linearly dependent list less or equal to length of spanning list}, we have the situation that the length of the basis of $U$ is less or equal to the length of the basis of $V$.
\end{prf}


% 2.38
\begin{thm} [lineparly independent list of the right length]
  \label{thm: linearly independent list of the right length is a basis}
  Every linearly independent list of vectors of length $\dim V$ is a basis.
\end{thm}
\begin{prf}
  Let $\dim V = n$ and $v_1, \ldots, v_n$ be linearly independent in $V$. This list can be extended to a basis of $V$ by \ref{thm: every linearly independent list of vectors in a finite-dimensional vector space can be extended to a basis of the vector space}. However, every basis of $V$ has length $n$, so no elements are adjoined to $v_1, \ldots, v_n$ by such an extension process.
\end{prf}

% 2.39
\begin{thm} [subspace of full dimension]
  \label{thm: subspace of full dimension equals the whole space}
  If $U$ is a subspace of $V$ and $\dim U = \dim V \neq \infty$, then $U=V$.
\end{thm}
\begin{prf}
  Let $u_1, \ldots, u_n$ be a basis of $U$. Thus $n = \dim U$, and by the hypothesis we also have $n = \dim V$. Thus $u_1, \ldots, u_n$ is a linearly independent list of vectors in $V$ of length $\dim V$. So by our previous theorem \ref{thm: linearly independent list of the right length is a basis}, we see that $u_1, \ldots, u_n$ is a basis of $V$. In particular, every vector in $V$ is a linear combination of $u_1, \ldots, u_n$. Thus $U=V$.
\end{prf}

% 2.42
\setcounter{thm}{41}
\begin{thm} [spanning list of the right length]
  \label{thm: spanning list of the right length}
  Every spanning list of vectors in $V$ $(\dim V \neq \infty)$ with length $\dim V$ is a basis of $V$.
\end{thm}
\begin{prf}
  Suppose $\dim V = n$ and $v_1, \ldots, v_n$ spans $V$. The list $v_1, \ldots, v_n$ can be reduced to a basis of $V$ by \ref{thm: every spanning list contains a basis}. But our list already has length $n$, so no elements will get deleted. Thus $v_1, \ldots, v_n$ is a basis of $V$, as desired.
\end{prf}

% 2.43
%\begin{thm} [dimension of a sum]
%  \label{thm: dimension of a sum of subspaces}
%  If $V_1$ and $V_2$ are subspaces of a \fdvs $V$, then
%  \begin{equation}
%    \dim (V_1 + V_2) = \dim V_1 + \dim V_2 - \dim (V_1 \cap V_2)
%  \end{equation}
%\end{thm}
%\begin{prf}
%  Let $r_1, \ddd, r_n \in V$ be a basis of $V_1 \cap V_2$, where $n \in \nat_{\geq 0}$, meaning the basis can also be the empty list and $V_1 \cap V_2$ might equal to $\{ \vec 0 \}$. Note that the empty list is defined to be linearly independent and the span of the empty list is defined to be equal to $\{ \vec 0 \}$. By this definition, we have
%  \[
%    \dim(V_1 \cap V_2) = n.
%  \]
%
%  Since $r_1, \ddd, r_n$ is linearly independent in $V$ as well as in $V_1$ and $V_2$, this list can be extended to a basis of $V_1$ and $V_2$. So let
%  \begin{itemize}
%    \item $r_1, \ddd, r_n, u_1, \ddd, u_j$ be a basis of $V_1$ and
%    \item $r_1, \ddd, r_n, w_1, \ddd, w_k$ be a basis of $V_2$ for $j,k \in \nat_{\geq 0}$.
%  \end{itemize}
%
%  Then we have $\dim (V_1) = m+j$ and $\dim(V_2) = m+k$. Note that the extensions might be empty and $j,k \in \nat_{\geq 0}$, meaning $V_1$ or $V_2$ could be $V_1 \cap V_2$. If both would equal $V_1 \cap V_2$ this would mean that $V_1 = V_2$. The reader can easily verify that the formula still holds in this case ($\dim (V_2+V_2) = \dim (V_2) = 2 \cdot \dim (V_2) - \dim (V_2 \cap V_2)$).
%
%  We will show that
%  \begin{equation}
%    \label{eq: basis with r's, u's and w's}
%    r_1, \ddd, r_n, u_1, \ddd, u_j, w_1, \ddd, w_k
%  \end{equation}
%
%  is a basis of $V_1 + V_2$. This will complete the proof, because then we will have
%  \begin{equation}
%    \begin{aligned}
%      \dim (V_1 + V_2) &= m + j + k \\
%      &= (m+j) + (m+k) - m \\
%      &= \dim (V_1) + \dim (V_2) - \dim (V_1 \cap V_2)
%    \end{aligned}
%  \end{equation}
%
%  The list \eqref{eq: basis with r's, u's and w's} is contained in $V_1 + V_2$. The span of this list contains $V_1$ and contains $V_2$ and hence is equal to $V_1 + V_2$. Thus to show that \eqref{eq: basis with r's, u's and w's} is a basis of $V_1 + V_2$ we only need to show that is linearly independent. Therefore, let $a_1, \ddd, a_n, b_1, \ddd, b_j, c_1, \ddd, c_k \in \myF$, such that
%\begin{equation}
%  \label{eq: linear independence formula for r's, u's and w's}
%  \begin{aligned}
%    0 = \underbrace{a_1 r_1 + \cdots + a_n r_n}_{\substack{\in \; V_1 \cap V_2 \\ (\text{common intersection})}}
%    + \underbrace{b_1 u_1 + \cdots + b_j u_j}_{\substack{\in \; V_1 \\ (\text{extension for } V_1 )}}
%    + \underbrace{c_1 w_1 + \cdots + c_k w_k}_{\substack{\in \; V_2 \\ (\text{extension for } V_2 ) }}.
%  \end{aligned}
%\end{equation}
%
%We need to prove that all the $a$'s, $b$'s, and $c$'s equal $0$. The equation above can be rewritten as
%\begin{equation}
%  c_1 w_1 + \cdots + c_k w_k = - a_1 r_1 - \cdots - a_m r_m - b_1 u_1 - \cdots - b_j u_j
%\end{equation}
%
%which shows that $c_1 w_1 + \cdots + c_k w_k \in V_1$ because $r_1, \ddd, r_n, u_1, \ddd, u_j$ is a basis of $V_1$. But all the the $w$'s are in $V_2$ as well, so this implies that $c_1 w_1 + \cdots + c_k w_k \in V_1 \cap V_2$. Because $r_1, \ddd, r_n$ is a basis of $V_1 \cap V_2$, we have
%\begin{equation}
%  c_1 w_1 + \cdots + c_k w_k = d_1 r_1 + \cdots + d_m r_m, $ for some $ d_1, \ddd, d_m \in \myF. $ (which are newly introduced here) $
%\end{equation}
%
%But $r_1, \ddd, r_n, w_1, \ddd, w_k$ is linearly independent and the equation above can be rewritten to $0 = c_1 w_1 + \cdots + c_k w_k - d_1 r_1 - \cdots - d_m r_m$, so all the $c$'s and $d$'s must equal to $0$. Thus \eqref{eq: linear independence formula for r's, u's and w's} becomes the equation
%\begin{equation}
%  a_1 r_1 + \cdots + a_m r_m + b_1 u_1 + \cdots + b_j u_j = 0
%\end{equation}
%
%Because $r_1, \ddd, r_n, u_1, \ddd, u_j$ is linearly independent, the equation above implies that all the $a$'s and $b$'s are $0$. So we have show that in equation \eqref{eq: linear independence formula for r's, u's and w's} that all coefficients ($a$, $b$, $c$) are $0$, more precisely $  0 = a_1 = \cdots = a_n = b_1 = \cdots =  b_j = c_1 \cdots =  c_k$,
%completing the proof.
%\end{prf}


\begin{thm} [dimension of a sum]
\label{thm: dimension of a sum of subspaces}
If $V_1$ and $V_2$ are subspaces of a \fdvs $V$, then
\begin{equation}
  \dim (V_1 + V_2) = \dim V_1 + \dim V_2 - \dim (V_1 \cap V_2).
\end{equation}
\end{thm}
\begin{prf}
Let $r_1, \ddd, r_n \in V$ be a basis of $V_1 \cap V_2$, where $n \in \nat_{\geq 0}$. This means the basis can also be the empty list, and $V_1 \cap V_2$ might equal $\{ \vec 0 \}$. Note that the empty list is defined to be linearly independent, and the span of the empty list is defined to be equal to $\{ \vec 0 \}$. By this definition, we have
\[
\dim(V_1 \cap V_2) = n.
\]

Since $r_1, \ddd, r_n$ is linearly independent in $V$ as well as in $V_1$ and $V_2$, this list can be extended to a basis of $V_1$ and $V_2$. So let
\begin{itemize}
  \item $r_1, \ddd, r_n, u_1, \ddd, u_j$ be a basis of $V_1$, and
  \item $r_1, \ddd, r_n, w_1, \ddd, w_k$ be a basis of $V_2$ for $j,k \in \nat_{\geq 0}$.
\end{itemize}

Then we have $\dim (V_1) = n+j$ and $\dim(V_2) = n+k$. Note that the extensions might be empty ($j,k \in \nat_{\geq 0}$), meaning $V_1$ or $V_2$ could coincide with $V_1 \cap V_2$. If both subspaces equal $V_1 \cap V_2$, this would imply $V_1 = V_2$. The reader can easily verify that the formula still holds in this case: $\dim (V_2+V_2) = \dim (V_2) = 2 \cdot \dim (V_2) - \dim (V_2 \cap V_2)$.

We will show that
\begin{equation}
  \label{eq: basis with r's, u's and w's}
  r_1, \ddd, r_n, u_1, \ddd, u_j, w_1, \ddd, w_k
\end{equation}
is a basis of $V_1 + V_2$. This will complete the proof because then we will have
\begin{equation}
  \begin{aligned}
    \dim (V_1 + V_2) &= n + j + k \\
    &= (n+j) + (n+k) - n \\
    &= \dim (V_1) + \dim (V_2) - \dim (V_1 \cap V_2).
  \end{aligned}
\end{equation}

The list \eqref{eq: basis with r's, u's and w's} is contained in $V_1 + V_2$. The span of this list contains $V_1$ and $V_2$; therefore, it equals $V_1 + V_2$. Thus, to show that \eqref{eq: basis with r's, u's and w's} is a basis of $V_1 + V_2$, we only need to show that it is linearly independent. Therefore, let $a_1, \ddd, a_n, b_1, \ddd, b_j, c_1, \ddd, c_k \in \myF$, such that
\begin{equation}
  \label{eq: linear independence formula for r's, u's and w's}
  \begin{aligned}
    0 = \underbrace{a_1 r_1 + \cdots + a_n r_n}_{\substack{\in \; V_1 \cap V_2 \\ (\text{common intersection})}}
    + \underbrace{b_1 u_1 + \cdots + b_j u_j}_{\substack{\in \; V_1 \\ (\text{extension for } V_1 )}}
    + \underbrace{c_1 w_1 + \cdots + c_k w_k}_{\substack{\in \; V_2 \\ (\text{extension for } V_2 ) }}.
  \end{aligned}
\end{equation}

We need to prove that all the $a$'s, $b$'s, and $c$'s equal $0$. The equation above can be rewritten as
\begin{equation}
  c_1 w_1 + \cdots + c_k w_k = - a_1 r_1 - \cdots - a_n r_n - b_1 u_1 - \cdots - b_j u_j,
\end{equation}
which shows that $c_1 w_1 + \cdots + c_k w_k \in V_1$ because $r_1, \ddd, r_n, u_1, \ddd, u_j$ is a basis of $V_1$. But all the $w$'s are in $V_2$ as well, so this implies that $c_1 w_1 + \cdots + c_k w_k \in V_1 \cap V_2$. Because $r_1, \ddd, r_n$ is a basis of $V_1 \cap V_2$, we have
\begin{equation}
  c_1 w_1 + \cdots + c_k w_k = d_1 r_1 + \cdots + d_n r_n, \text{ for some } d_1, \ddd, d_n \in \myF. \text{ (which are newly introduced here)}
\end{equation}

But $r_1, \ddd, r_n, w_1, \ddd, w_k$ is linearly independent, and the equation above can be rewritten to $0 = c_1 w_1 + \cdots + c_k w_k - d_1 r_1 - \cdots - d_n r_n$, so all the $c$'s and $d$'s must equal $0$. Thus \eqref{eq: linear independence formula for r's, u's and w's} becomes the equation
\begin{equation}
  a_1 r_1 + \cdots + a_n r_n + b_1 u_1 + \cdots + b_j u_j = 0.
\end{equation}

Because $r_1, \ddd, r_n, u_1, \ddd, u_j$ is linearly independent, the equation above implies that all the $a$'s and $b$'s are $0$. So we have shown that in equation \eqref{eq: linear independence formula for r's, u's and w's} all coefficients ($a$, $b$, $c$) are $0$, more precisely $  0 = a_1 = \cdots = a_n = b_1 = \cdots =  b_j = c_1 \cdots =  c_k$,
completing the proof.
\end{prf}

