\section{Dimension}

% 2.34
\setcounter{thm}{33}
\begin{thm} [basis length] Any two bases of a \fdvs have the same length. So the the basis length does not depend on basis.
\end{thm}
\begin{prf}
  Let $B_1$ and $B_2$ be $2$ basis of $V$. Since both are linearly independent and both span $V$, we have $|B_1| \leq |B_2|$ and $|B_2| \leq |B_1|$ by using \ref{thm: length of linearly dependent list less or equal to length of spanning list}. Therefore $|B_1|=|B_2|.$
\end{prf}

\begin{mydef} [dimension]
  $\dim V :\equiv$ length of any basis of $V$.
\end{mydef}

% 2.37
\setcounter{thm}{36}
\begin{thm} [dimension of a subspace]
  \label{thm: dimension of a subspace}
  If $V$ is \fd and $U$ is a subspace of $V$, then
  \begin{equation}
    \dim U \leq \dim V
  \end{equation}
\end{thm}
\begin{prf}
  Since the length of everly linearly independent list is smaller or equal to every spanning list in $V$ according to
  \ref{thm: length of linearly dependent list less or equal to length of spanning list}, we have the situation that the length of the basis of $U$ is less or equal to the length of the basis of $V$.
\end{prf}


% 2.38
\begin{thm} [lineparly independent list of the right length]
  \label{thm: linearly independent list of the right length is a basis}
  Every linearly independent list of vectors of length $\dim V$ is a basis.
\end{thm}
\begin{prf}
  Let $\dim V = n$ and $v_1, \ldots, v_n$ be linearly independent in $V$. This list can be extended to a basis of $V$ by \ref{thm: every linearly independent list of vectors in a finite-dimensional vector space can be extended to a basis of the vector space}. However, every basis of $V$ has length $n$, so no elements are adjoined to $v_1, \ldots, v_n$ by such an extension process.
\end{prf}

% 2.39
\begin{thm} [subspace of full dimension]
  \label{thm: subspace of full dimension equals the whole space}
  If $U$ is a subspace of $V$ and $\dim U = \dim V \neq \infty$, then $U=V$.
\end{thm}
\begin{prf}
  Let $u_1, \ldots, u_n$ be a basis of $U$. Thus $n = \dim U$, and by the hypothesis we also have $n = \dim V$. Thus $u_1, \ldots, u_n$ is a linearly independent list of vectors in $V$ of length $\dim V$. So by our previous theorem \ref{thm: linearly independent list of the right length is a basis}, we see that $u_1, \ldots, u_n$ is a basis of $V$. In particular, every vector in $V$ is a linear combination of $u_1, \ldots, u_n$. Thus $U=V$.
\end{prf}

% 2.42
\setcounter{thm}{41}
\begin{thm} [spanning list of the right length]
  \label{thm: spanning list of the right length}
  Every spanning list of vectors in $V$ $(\dim V \neq \infty)$ with length $\dim V$ is a basis of $V$.
\end{thm}
\begin{prf}
  Suppose $\dim V = n$ and $v_1, \ldots, v_n$ spans $V$. The list $v_1, \ldots, v_n$ can be reduced to a basis of $V$ by \ref{thm: every spanning list contains a basis}. But our list already has length $n$, so no elements will get deleted. Thus $v_1, \ldots, v_n$ is a basis of $V$, as desired.
\end{prf}

% 2.43
\begin{thm} [dimension of a sum]
  \label{thm: dimension of a sum of subspaces}
  $V_1$ and $V_2$ are subspaces of a \fdvs we have
  \begin{equation}
    \implies \dim (V_1 + V_2) = \dim V_1 + \dim V_2 - \dim (V_1 \cap V_2)
  \end{equation}

  See exercise 2C19 and 2C20 for a even nicer result.
%  From exercise 2C19 and 2C20 we have the following result:
%  If $V_1$, $V_2$ and $V_3$ are subspaces of \fdvs we have
%  \begin{equation}
%    \begin{aligned}
%      \dim (V_1 &+ V_2 + V_3) = \dim V_1 + \dim V_2 + \dim V_3 \\
%      & \quad - \dim (V_1 \cap V_2) - \dim (V_1 \cap V_3) - \dim (V2 \cap V3) \\
%      & \quad + \dim (V_1 \cap V_2 \cap V_3)
%    \end{aligned}
%  \end{equation}
%
%as well as
%
%  \begin{equation}
%    \begin{aligned}
%      \dim (V_1 + V_2 + V_3)  = \dim V_1 + \dim V_2 + \dim V_3 \\
%        - \frac{\dim (V_1 \cap V_2) + \dim (V_1 \cap V_3) + \dim (V2 \cap V3)}{3} \\
%       - \frac{ \dim \left(  (V_1 + V_2) \cap V_3 \right) + \dim \left( (V_1 + V_3) \cap V_2 \right)+ \dim \left( (V2 + V3) \cap V_1 \right) }{3} \\
%    \end{aligned}
%  \end{equation}
\end{thm}

