\section{Dimension}

\setcounter{thm}{33}
\begin{thm} Any two bases of a \fdvs have the same length. So the the basis length does not depend on basis.
\end{thm}
\begin{prf}
  Let $B_1$ and $B_2$ be $2$ basis of $V$. Since both are linearly independent and both span $V$, we have $|B_1| \leq |B_2|$ and $|B_2| \leq |B_1|$ by using \ref{length-of-linearly-dependent-list-less-or-equal-length-of-spanning-list}. Therefore $|B_1|=|B_2|.$
\end{prf}

\begin{mydef} [Dimension]
	$\dim V :\equiv$ length of any basis of $V$
\end{mydef}

\setcounter{thm}{36}
\begin{thm} [dimension of a subspace]
  \label{dimension-of-a-subspace}
  $\dim U \leq \dim V$ if $V$ is \fd  and $U$ is a subspace of $V$.
\end{thm}

\begin{prf}
  Since the length of everly linearly independent list is smaller or equal to every spanning list in $V$ (according to \ref{length-of-linearly-dependent-list-less-or-equal-length-of-spanning-list}), we have the situation that the length of the basis of $U$ is less or equal to the length of $V$.
\end{prf}

\begin{thm}
  \label{every-lid-list-of-length-dim-v-is -a-basis-of-v}
  Every \lid list of vectors of length $\dim V$ is a basis. (by 2.32)
\end{thm}

\begin{thm} If $U$ is a subspace of $V$ and $\dim U = \dim V$, then $U=V$. (by 2.38)
\end{thm}

\setcounter{thm}{41} 
\begin{thm}
  Every spanning list of vectors in $V$ of length $\dim V$ is a basis of $V$
\end{thm}

\begin{thm} $V_1$ and $V_2$ are subspaces of a \fdvs we have
  \begin{equation}
    \implies \dim (V_1 + V_2) = \dim V_1 + \dim V_2 - \dim (V_1 \cap V_2)
  \end{equation}

  From exercise 2C19 and 2C20 we have the following result:
  If $V_1$, $V_2$ and $V_3$ are subspaces of \fdvs we have
  \begin{equation}
    \begin{aligned}
      \dim (V_1 &+ V_2 + V_3) = \dim V_1 + \dim V_2 + \dim V_3 \\
      & \quad - \dim (V_1 \cap V_2) - \dim (V_1 \cap V_3) - \dim (V2 \cap V3) \\
      & \quad + \dim (V_1 \cap V_2 \cap V_3)
    \end{aligned}
  \end{equation}

as well as

  \begin{equation}
    \begin{aligned}
      \dim (V_1 + V_2 + V_3)  = \dim V_1 + \dim V_2 + \dim V_3 \\
        - \frac{\dim (V_1 \cap V_2) + \dim (V_1 \cap V_3) + \dim (V2 \cap V3)}{3} \\
       - \frac{ \dim \left(  (V_1 + V_2) \cap V_3 \right) + \dim \left( (V_1 + V_3) \cap V_2 \right)+ \dim \left( (V2 + V3) \cap V_1 \right) }{3} \\
    \end{aligned}
  \end{equation}
\end{thm}

