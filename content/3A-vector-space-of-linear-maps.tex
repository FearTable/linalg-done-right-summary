\section{Vector of Linear Maps}

\subsection{Definition and Examples of Linear Maps}

% 3.1
\begin{mydef} [linear map]
  A \lm from $V$ to $W$ is a function $T:V\to W$ with following properties:
  \begin{itemize}
    \item \emph{additivity:} $T(u+v)=Tu + Tv \qquad \forall u,v \in V$
    \item \emph{homogenity:} $T(\lambda v)=\lambda (Tv) \qquad
    \forall \lambda \in \mathbb{F}, \; \forall v\in V$
  \end{itemize}
\end{mydef}

% 3.2
\begin{mydef} [notation $\linmap (V,W), \, \linmap (V)$]
  The set of all linear maps from $V$ to $W$ is denoted by $\mathcal{L}(V,W)$. For the set of all linear maps from $V$ to itself we use $\mathcal{L}(V) :\equiv \mathcal{L}(V,V)$
\end{mydef}

% 3.4
\setcounter{thm}{3}
\begin{thm}[linear map lemma]
  \label{thm: linear map lemma}
  Suppose $\onetilln{v}$ is a basis of $V$ and $\onetilln{w}$ $\in$ $W$, where the $w$'s do not have any specific properties. Then there exist a unique \lm $T:V\to W$ or $T \in \linmap(V,W)$ such that
  \begin{itemize}
    \item[] $Tv_{k} = w_k \quad \forall \kinonetilln.$
  \end{itemize}
  It is of the form \begin{equation}
    T(c_1 v_1 + \cdots + c_n v_n) = c_1 w_1 + \cdots + c_n w_n.
  \end{equation}
\end{thm}

% 3.5
\setcounter{thm}{4}
\begin{mydef} [addition and scalar multiplication on $\linmap(V,W)$]
  Suppose $S, T \in \lvw$ and $\lambda \in \mathbb{F}.$ \\
  The sum $S+T \in \linmap (V,W)$ and the product $\lambda T \in \linmap (V,W)$ are defined by:
  \begin{itemize}
    \item $(S+T)(v) :\equiv Sv+Tv \quad \forall v \in V$
    \item $(\lambda T)(v) : \equiv \lambda (Tv) \quad \forall v \in V$
  \end{itemize}
\end{mydef}

% 3.6
\subsection{Algebraic Operations on \texorpdfstring{$\linmap(V,W)$}{L(V,W)}}
\setcounter{thm}{5}
\begin{thm} [$\lvw$]
  With these operations of addition and scalar multiplication as defined above, $\lvw$ is a \vs.
\end{thm}

% 3.7
\setcounter{thm}{6}
\begin{mydef}
  Let $T \in \lin{U}{V}$ and $S \in \lin{V}{W}$. We define the \qt{product} $ST \in \lin{U}{W}$ as follows:
  \begin{itemize}
    \item[] $(ST)(u) :\equiv S(Tu) \quad \forall u \in U$
  \end{itemize}
  Thus $ST$ is just the usual composition  $S \circ T$ of two functions. But with linear functions, we have distributive and associative properties.
\end{mydef}

% 3.8
\begin{thm} [algebraic properties of products of linear maps]
  With these definitions we have
  \begin{itemize}
    \item \bfemph{associativity:} $(T_1 T_2) T_3 = T_1 (T_2 T_3)$, whenever $T_3$ maps into the domain of $T_2$ and $T_2$ maps into the Domain of $T_1$.
    \item \bfemph{identity:} $T I = I T = T$ for $T \in \lvw$ (The first $I$ is the identity operator on $V$ and the second $I$ the identity operator on $W$. \\
    We could also write $T I_V = I_W T$.
    \item \bfemph{distributive properties:} For $T, T_1, T_2 \in \lin{U}{V}$ and $S, S_1, S_2 \in \lin{V}{W}$: \\ $(S_1 + S_2)T=S_1 T + S_2 T$ and $S(T_1 + T_2)=S T_1 + S T_2$.
    \item \bfemph{non-commutative:} $ST \neq TS$ in general.
  \end{itemize}
\end{thm}

% 3.10
\setcounter{thm}{9}
\begin{thm} [linear maps take $0$ to $0$]
  \label{thm: linear maps take 0 to 0}
  $T\in \lvw \implies T(0)=0.$
\end{thm}
\begin{prf}
  $T(0) = T(0+0) = T(0) + T(0)$. Now subtract $T(0)$ on both sides.
\end{prf}