\section{Positive (Semidefinite) Operators}

\mce{34}
\begin{mydef}[positive (semidefinite) operators]
  An operator $T \in \linmap (V)$ is called \qt{positive (semidefinite)}, if $T$ is self-adjoint and
  \[
  \ip{Tv}{v} \geq 0 \quad \forall v \in V.
  \]

  One can either use the term \qt{positive operator} or \qt{poisitve semidefinite operator}, which means the same.
\end{mydef}

\mce{36}
\begin{mydef}[square root]
  An operator $R$ is called \qt{square root} of an operator $T$ if $R^2=T$.
\end{mydef}

\mce{38}
\begin{thm}[charecterizations of positive operators]
  Let $T \in \linmap(V)$. Then the following are equivalent.
  \begin{enumerate}[label=\textbf{(\alph*)}]
    \item $T$ is a positive operator.
    \item $T$ is self-adjoint and all eigenvalues of $T$ are nonegative
    \item With respect to some orthonormal basis of $V$, the matrix $\mmatrix(T)$ of $T$ is a diagonal matrix with only nonnegative numbers one the diagonal.
    \item $T$ has a positive square root.
    \item $T$ has a self-adjoint square root.
    \item $T = R^* R$ for some $R \in \linmap(V)$.
  \end{enumerate}
\end{thm}
\begin{prf}
  We will proof $(a) \Rightarrow (b) \Rightarrow (c) \Rightarrow (d) \Rightarrow (e) \Rightarrow (f) \Rightarrow (a)$

  \qt{First step:} If (a) holds, that means that $T$ is also self-adjoint. Then we have for an eigenvalue $\lambda \in \myF$
  \[
  0 \leq \ip{Tv}{v} = \ip{\lambda v}{v} = \lambda \ip{v}{v}.
  \]
  \[
  \implies 0 < \lambda
  \]

  \qt{Second step:} Suppose (b) holds. By the spectral theorem, there is an orthonormal basis $e_1, \ddd, e_n$ of $V$ consisting of eigenvectors of $T$. Let $\lambda_1, \ddd, \lambda_n$ be the eigenvalues of $T$ corresponding to $e_1, \ddd, e_n$. By (a), we know that they are all nonegative. So $(b) \implies (c)$.

  \qt{Third step:} Now suppose (c) holds. Then there is a orthonormal basis $e_1, \ddd, e_n$ of $V$ such that
  $\mmatrix(T) = \operatorname{diag}(\lambda_1,  \ddd, \lambda_n), \; \lambda_i > 0$. The linear map lemma \ref{thm: linear map lemma} implies that $\exists R \in \linmap (V)$:
  \[
  R e_k = \sqrt{\lambda_k}e_k \in V \quad \forall k \in \{1, \ddd,n\}.
  \]
  \[
  R^2 e_k = R R e_k = R \sqrt{\lambda_k}e_k = \sqrt{\lambda_k} R e_k = \sqrt{\lambda_k} \sqrt{\lambda_k} e_k =  \lambda_k e_k = T e_k \quad \forall k \in \{1, \ddd,n\}.
  \]

  which implies that $R^2 = T$. Thus $R$ is a positive (semidefinite) square root of $T$. Hence (d) holds, which shows $(c) \implies (d)$.

  \qt{Fourth step:} Every positive (semidefinite) operator is self-adjoint by the definition of a positive (semidefinite) operator. Thus $(d) \implies (e)$.

  \qt{Fifth step:} Now suppose (e) holds, meaning that there exists a self-adjoint operator $R=R^* \in \linmap(V)$ on $V$ \st $T=R^2$. Then $T=R^*R$. Hence $(e) \implies (f)$.

  \qt{Sixth step:} Finally, suppose (f) holds. Let $R \in \linmap (V)$ be such that $T=R^*R$. Then $T^* =  (R^* R)^* = R^* ( R^*)^* = R^*R = T$. Hence $T$ is self-adjoint. Note that $\forall v \in V$:
  \[
  \ip{Tv}{v} = \ip{R^* Rv}{v} = \ip{Rv}{(R^*)^*v} = \ip{Rv}{Rv} \geq 0.
  \]

  Thus $T$ is positive (semidefinite), showing that $(f) \implies (a)$.
\end{prf}

\begin{thm}[uniqueness of square root]
  \label{thm: uniqueness of square root}
  Each positive operator on $V$ has a unique positive square root.
\end{thm}

\begin{mydef}[notation: $\sqrt{T}$]
  For $T$ a positive operator, $\sqrt{T}$ denotes the unique positive square root of $T$.
\end{mydef}

\begin{thm}
  \label{thm: T positive and <Tv,v> = 0 => Tv = 0}
  Suppose $T$ is a positive operator on $V$ and $v \in V$ \st $\ip{Tv}{v} = 0$.
  \[
  \implies Tv = 0.
  \]
\end{thm}
\begin{prf}
  We have $0 = \ip{Tv}{v}= \ip{\sqrt{T}\sqrt{T}v}{v} = \ip{\sqrt{T}v}{\sqrt{T}v} = \norm{\sqrt{T}v}^2.$ Hence $\sqrt{T}v = 0$. Thus $Tv = \sqrt{T}\left(\sqrt{T}v\right)=0$, as desired.
\end{prf}