\section*{Exercises about Duality}

\exercise{1}

\begin{xrcs}
  Explain why each linear functional is surjective or the zero map.
  \begin{xsol}
    $\varphi \in \dual{V} = \lin{V}{\myF}$ implies that for every $\lambda \in \myF$, and every $v \in V$,
    \begin{equation}
      \varphi (\lambda v) = \lambda \varphi(v).
    \end{equation}

    If $\varphi$ is not the zero map, then $\varphi$ evaluated at $v \in V$ or $\lambda v \in V$ is a scalar in  $\myF$ for some $v \in V$ (i.e, $\varphi(v) \in \myF$ and $ \varphi(\lambda v) = \lambda \varphi(v) \in \myF$). Hence, $\myrange \varphi = \myF$, since $\lambda  \in \myF$ was arbitrary.
  \end{xsol}
\end{xrcs}

\exercise{3}
\begin{xrcs}
  Suppose $V$ is finite-dimensional and $v \in V$ is nonzero. Prove that there exists $\varphi \in \dual{V}$ such that $\varphi(v) = 1$.
  \begin{xsol}
    Let $v_1, \ddd, v_n$ be a basis of $V$. Let $v \in V$ be nonzero such that $v = a_1 v_1 + \cdots + a_n v_n$, where $a_i \in \myF$. Since $v$ is nonzero, where exists a $k$ such that $a_k \neq 0$. Define $\varphi: V \to \real$ by
    \begin{equation}
      \varphi (u) = \frac{c_i}{a_i},
    \end{equation}

    whenever $u = c_1 v_1 + \dots + c_n v_n \in V$. Note that this definition corresponds to the use of the linear map lemma (\ref{thm: linear map lemma}), where we define $\varphi$ on each basis vector as follows:
    \begin{equation}
      \varphi (v_1) = 0, \;  \ddd, \;
      \varphi(v_{k-1}) = 0, \;
      \varphi(v_k) = \frac{1}{a_i}, \;
      \varphi(v_{k+1}) = 0, \;  \ddd, \;
      \varphi(v_n) = 0.
    \end{equation}

    Hence, $\varphi(a_1 v_1 + \cdots + a_n v_n) = \frac{a_i}{a_i} = 1$. It is easy to check that $\varphi$ is a linear map.
  \end{xsol}
\end{xrcs}

\exercise{5}
\begin{xrcs}
  Suppose $V$ is finite dimensional and $U$ is a subspace to $V$, where $U \neq V$. Prove that there exists a $\varphi \in \dual{V}$ such that $\varphi (u) = 0$ for every $u \in U$ but $\varphi \neq 0$.
  \begin{xsol}
    Choose $W \subset V$ such that $V = U \oplus W$. Hence, there is a basis
    \begin{equation}
      u_1, \ddd, u_n, w_1, \ddd, w_m
    \end{equation}
    of $V$ such that the $u$'s and $w$'s form a basis of $U$ and $W$ respectively. We define $\varphi \in \dual{V}$ such that for $v \in V$, written as
    \begin{equation}
      v = a_1 u_1 + \cdots + a_n u_n + b_1 w_1 + \cdots + b_m w_m,
    \end{equation}

    we have
    \begin{equation}
      \varphi (v) = b_1 + \cdots + b_m.
    \end{equation}

    It is easy to see that $\varphi (u) = 0$ vor every $u \in U$ (since $U \cap W = \{0\}$, by \ref{thm: sum and intersection of two subspaces}) and that $\varphi$ is linear.
  \end{xsol}
\end{xrcs}

\exercise{10}
\begin{xrcs}
  Suppose $m$ is positive integer.
  \begin{enumerate}
    \item Show that $1, x-5, \ddd, (x-5)^m$ is a basis of $\polyn_m(\real)$.
    \item What is the dual basis of the basis in (a)?
  \end{enumerate}
  \begin{xsol}
    We first start the exercise by solving it for $\polyn_4(\real)$ straight forward before solving it more abstract and generic for $\polyn_m(\real)$. Recall that $e_0 = x^0, e_1 = x^1, \ddd, e_4 = x^4$ is the standard basis of $\polyn_4(\real)$. We want to show that $v_0 = 1, v_1 = x-5, \ddd, v_4 = (x-5)^4$ is a basis as well. If we do some rewiting and expansion of the formulas, we get
    \begin{equation}
      \begin{aligned}
        v_0 = (x^1-5x^0)^0 &= 1x^0 \;,\\
        v_1 = (x^1-5x^0)^1 &= -5x^0+x^1 \;,\\
        v_2 = (x^1-5x^0)^2 &= 25x^0-10x^1+x^2 \;,\\
        v_3 = (x^1-5x^0)^3 &= -125x^0+75x^1-15x^2+x^3 \;,\\
        v_4 = (x^1-5x^0)^4 &= 625x^0-500x^1+150x^2-20x^3+x^4 \;.\\
      \end{aligned}
    \end{equation}

    Now, for a given $p \in \polyn_4(\real)$ so that $p(x) = b_0 x^0 + b_1 x^1 + \cdots + b_4 x^4$ ($b_j \in \myF$) we want to show that there exists $a_i \in \myF$ such that $\sum_{j=0}^{4} b_j e_j = \sum_{j=0}^{4} a_j v_j$. So let's expand
    \begin{equation}
      \begin{aligned}
        \textstyle
        \sum_{j=0}^{4} a_j v_j
          &= a_0 (1x^0) + a_1 (-5x^0+x^1) + a_2 (25x^0-10x^1+x^3) \\
          &  \qquad + a_3 (-125x^0+75x^1-15x^2+x^3) + a_4 (625x^0-500x^1+150x^2+x^4) \\
          &= (a_0 - 5a_1 + 25 a_2 - 125 a_3 + 625 a_4) + (a_1 -10 a_2 + 75 a_3 - 500 a_4) x\\
          & \qquad + (a_2 - 15 a_3 + 150 a_4) x^2 + (a_3 - 20 a_4) x^3 + a_4 x^4 \\
          &= b_0 x^0 + b_1 x^1 + \cdots + b_4 x^4 = \textstyle \sum_{j=0}^{4} b_j e_j.
      \end{aligned}
    \end{equation}

    We can solve this for $a_j$ by back substitution:
    \begin{equation}
      \begin{aligned}
      a_4 &= b_4, \\
      a_3 &= b_3 + 20 a_4, \\
      a_2 &= b_2 + 15 a_3 - 150 a_4, \\
      a_1 &= b_1 + 10 a_2 + 75 a_3 - 500 a_4, \text{ and finally } \\
      a_0 &= b_0 + 5a_1 - 25 a_2 + 125 a_3 - 625 a_4.
      \end{aligned}
    \end{equation}

    Hence, $v_1, \ddd, v_4$ is indeed a basis of $\polyn_4(\real)$. Now, we want to find the dual basis $\psi_0, \psi_1, \ddd, \psi_4$ of $v_1, \ddd, v_4$. Recall that the condition
    \begin{equation}
      \phi_j (v_k) = \delta_{j,k} = \begin{cases}
        1,\quad j=k \\ 0, \quad j \neq k
      \end{cases}
    \end{equation}

    must hold, and we thus want to define each $\phi_j$ on the standard basis $e_0, \ddd, e_4$. So let's start with $\psi_0$, where $\psi_0 (v_0) = 1$ and zero on all other $v$'s.
    \begin{equation}
      \begin{aligned}
      \psi_0(v_0) &= 1\psi_0(x^0) = 1 \;. \implies \psi_0(x^0) = 1.\\
      \psi_0(v_1) &= -5\psi_0(x^0)+\psi_0(x^1) = 0. \implies \psi_0(x^1) = 5 \;.\\
      \psi_0(v_2) &= 25psi_0(x^0)-10\psi_0(x^1)+\psi_0(x^2)= 0. \implies \phi(x^2) = 25 \;.\\
      \psi_0(v_3) &= -125\psi_0(x^0)+75\psi_0(x^1)-15\psi_0(x^2)+\psi_0(x^3) = 0. \implies \psi_0(x^3) = 125.\\
      \psi_0(v_4) &= 625\psi_0(x^0)-500\psi_0(x^1)+150\psi_0(x^2)-20\psi_0(x^3)+\psi_0(x^4) = 0 . \implies \psi_0(x^4) = 625 .\\
      \end{aligned}
    \end{equation}

    Now, the same for $\psi_1$:
    \begin{equation}
      \begin{aligned}
      \psi_1(v_0) &= 1\psi_1(x^0) = 0 \;. \implies \psi_0(x^0) = 0.\\
      \psi_1(v_1) &= -5\psi_1(x^0)+\psi_1(x^1) = 1. \implies \psi_1(x^1) = 1 \;.\\
      \psi_1(v_2) &= 25psi_0(x^0)-10\psi_1(x^1)+\psi_1(x^2)= 0. \implies \psi_1(x^2) = 10 \;.\\
      \psi_1(v_3) &= -125\psi_1(x^0)+75\psi_1(x^1)-15\psi_1(x^2)+\psi_1(x^3) = 0. \implies \psi_1(x^3) = 75.\\
      \psi_1(v_4) &= 625\psi_1(x^0)-500\psi_1(x^1)+150\psi_1(x^2)-20\psi_1(x^3)+\psi_1(x^4) = 0 . \implies \psi_1(x^4) = 500 .\\
      \end{aligned}
    \end{equation}

    Now, $\psi_2$:
    \begin{equation}
      \begin{aligned}
      \psi_2(v_0) &= 1\psi_2(x^0) = 0 \;. \implies \psi_2(x^0) = 0.\\
      \psi_2(v_1) &= -5\psi_2(x^0)+\psi_2(x^1) = 0. \implies \psi_2(x^1) = 0 \;.\\
      \psi_2(v_2) &= 25\psi_2(x^0)-10\psi_2(x^1)+\psi_2(x^2)= 1. \implies \psi_2(x^2) = 1 \;.\\
      \psi_2(v_3) &= -125\psi_2(x^0)+75\psi_2(x^1)-15\psi_2(x^2)+\psi_2(x^3) = 0. \implies \psi_2(x^3) = 15.\\
      \psi_2(v_4) &= 625\psi_2(x^0)-500\psi_2(x^1)+150\psi_2(x^2)-20\psi_2(x^3)+\psi_2(x^4) = 0 . \implies \psi_2(x^4) = 150 .\\
      \end{aligned}
    \end{equation}

   For $\psi_3$ we get $\psi_3(x^0) = \psi_3(x_1) = \psi_3(x^2) = 0$ and $\psi_3 (x^3) = 1$. Now what is left to calculate is
   \begin{equation}
     \psi_3(v_4) = 625\psi_2(x^0)-500\psi_2(x^1)+150\psi_2(x^2)-20\psi_2(x^3)+\psi_2(x^4) = 0.
   \end{equation}

    Hence, $\psi_4(v_4) = 20.$ For $\psi_4$ we get $\psi_4(x^0) = \psi_4(x_1) = \cdots = \psi_4(x^3) = 0$, and $\psi_4(v_4) = 1$, and thus, $\psi_4(x^4) = 1$. Now, let's take a more generic approach making extensive use of the binomic formula
    \begin{equation}
      (a+b)^n = \sum_{i=0}^{n} \binom{n}{i} a^{i}b^{n-i}, \quad (a,b \in \real, n\in \nat),
    \end{equation}

    and Pascal's identity
    \begin{equation}
      \binom{n}{k} + \binom{n}{k+1} = \binom{n+1}{k+1}, \quad (n,k \in \nat).
    \end{equation}

    We start by rewriting once again
    \begin{equation}
      \begin{aligned}
        v_0 = (x^1-5x^0)^0 &= \binom{0}{0}(-5)^0 e_0 = e^0\;,\\
        v_1 = (x^1-5x^0)^1 &= \binom{1}{0}(-5)^1 e_0 + \binom{1}{1}(-5)^0 e_1 \;,\\
        v_2 = (x^1-5x^0)^2 &= \binom{2}{0}(-5)^2 e_0 + \binom{2}{1}(-5)^1 e_1 + \binom{2}{2}(-5)^0 e_2\;,\\
        & \; \vdots \\
        v_m = (x^1-5x^0  )^m &= \binom{m}{0}(-5)^m e_0 + \binom{m}{1}(-5)^{m-1} e_1 + \cdots + \binom{m}{m}(-5)^0 e_m
                           = \sum_{i=0}^{m} \binom{m}{i} (-5)^{m-i} e_{i}.
      \end{aligned}
    \end{equation}

    Now, $\psi_j(e_k) = \binom{k}{j} 5^{k-j}$, taylor polynomials. (why?).
    ChatGTP: Now, why is $\psi_j(e_k) = \binom{k}{j} \cdot 5^{k-j}$?
    We want to define the dual basis $\{\psi_j\}_{j=0}^m$ of $\{v_k\}_{k=0}^m$, meaning we require
    \[
    \psi_j((x-5)^k) = \delta_{j,k}.
    \]
    Using the binomial theorem, we can expand each $(x-5)^k$ as
    \[
    (x-5)^k = \sum_{i=0}^k \binom{k}{i} (-5)^{k-i} e_i.
    \]
    To satisfy the condition $\psi_j((x-5)^k) = \delta_{j,k}$, it is enough to define $\psi_j$ on the standard basis $\{e_k\}$ by
    \[
    \psi_j(e_k) = \binom{k}{j} \cdot 5^{k-j}.
    \]
    This works because $\psi_j$ corresponds to applying the $j$-th derivative at $x=5$, divided by $j!$, to polynomials:
    \[
    \psi_j(p(x)) = \frac{1}{j!} p^{(j)}(5),
    \]
    and
    \[
    \frac{1}{j!} \cdot \frac{d^j}{dx^j} x^k \Big|_{x=5} = \binom{k}{j} \cdot 5^{k-j}.
    \]
    Thus, $\{\psi_j\}_{j=0}^m$ is indeed the dual basis of $\{v_k\}_{k=0}^m$.
  \end{xsol}
\end{xrcs}


\exercise{11}
\begin{xrcs}
  Suppose $v_1, \ddd, v_n$ is a basis of $V$ and $\varphi_1, \ddd, \varphi_n$ is the corresponding dual basis of $\dual{V}$. Suppose $\psi \in \dual{V}$. Prove that
  \begin{equation}
     \label{eq: equation for psi}
     \psi = \psi(v_1) \varphi_1 + \cdots +  \psi(v_n) \varphi_n.
  \end{equation}
  \begin{xsol}
    For a given $\psi \in \dual{V}$, let $\Gamma \in \dual{V}$ defined by
    \begin{equation}
      \Gamma := \psi(v_1) \varphi_1 + \cdots +  \psi(v_n) \varphi_n
    \end{equation}

    juste like in \eqref{eq: equation for psi}. We want to show that $\Gamma = \psi$. Observe that $\Gamma$ is a linear functional, since it is the linear combination of some linear  functionals $\varphi_1, \ddd, \varphi_n$ (which are the dual basis of $V$). In this case, $\psi(v_1), \ddd, \psi(v_n)$ are the scalars of this sum, which do not change for a given $\psi$ and a given basis of $V$. If we now apply $\Gamma$ to each $v_k$ (for $k = 1, \ddd, n$), we get
    \begin{equation}
      \begin{aligned}
        \Gamma(v_k) &= (\psi(v_1) \varphi_1 + \cdots +  \psi(v_n) \varphi_n) (v_k) \\
         &= \psi(v_1) \varphi_1 (v_k) + \cdots +  \psi(v_k) \varphi_n (v_k) + \cdots +  \psi(v_n) \varphi_n (v_k) \\
         &= \psi(v_1) \cdot 0 + \cdots +  \psi(v_k) \cdot 1 + \cdots +  \psi(v_k) \cdot 0 \\
         &= \psi(v_k).
      \end{aligned}
    \end{equation}

    Hence, $\Gamma$ and $\psi$ agree on every basis vector $v_k$. Therefore, $\Gamma = \psi$.
  \end{xsol}
\end{xrcs}

\exercise{14}
\begin{xrcs}
  Define $T: \real^3 \to \real^2$ by
  \begin{equation}
    T(x, y, z) := (4x + 5y + 6z, 7x + 8y + 9z).
  \end{equation}

  Suppose $\varphi_1, \varphi_2$ denotes the dual basis of the standard basis of $\real^2$ and $\psi_1, \psi_2, \psi_3$ denotes the dual basis of the standard basis of $\real^3$.
  \begin{enumerate}
    \item Describe the linear functionals $T'(\varphi_1)$ and $T'(\varphi_2)$.
    \item Write $T'(\varphi_1)$ and $T'(\varphi_2)$ as a linear combination of $\psi_1, \psi_2, \psi_3$.
  \end{enumerate}

  \begin{xsol}
    Let's start by getting comfortable about the notation. Let $e_1, e_2, e_3$ denote the standard basis of $\real^3$ and $e_1, e_2$ denote the standard basis of $\real^2$. In this case, $T(x, y, z) = (4x + 5y + 6z, 7x + 8y + 9z)$ actually means
    \begin{equation}
      T(x e_1 + y e_2 + z e_3) = (4x + 5y + 6z) e_1 + (7x + 8y + 9z) e_2
    \end{equation}

    Thus, we have $T e_1 = 4e_1 + 7 e_2$, $T e_2 = 5e_1 + 8e_2$, and $T e_3 = 6e_1 + 9 e_2$. The Matrix $\mmatrix(T)$ of $T$ would look like this:
    \begin{equation}
      \mathcal{M} (T) =
      \begin{blockarray}{cccc}
        & e_1 & e_2 & e_3  \\
        \begin{block}{c(ccc)}
          e_1 & 4   & 5   & 6    \\
          e_2 & 7   & 8   & 9    \\
        \end{block}
      \end{blockarray}
    \end{equation}

    Now, $\dual{T} (\varphi_1) = \varphi_1 \circ T$ and $\dual{T} (\varphi_2) = \varphi_2 \circ T$. Hence,
    \begin{equation}
      \begin{aligned}
        &\dual{T} (\varphi_1) (x e_1 + y e_2 + z e_3) = \varphi_1 ((4x + 5y + 6z) e_1 + (7x + 8y + 9z) e_2 ) = 4x + 5y + 6z \;, \; \text{ and } \\
        &\dual{T} (\varphi_2) (x e_1 + y e_2 + z e_3) = \varphi_2 ((4x + 5y + 6z) e_1 + (7x + 8y + 9z) e_2 ) = 7x + 8y + 9z,
      \end{aligned}
    \end{equation}

    which answers question (a). For question (b), we use the result \eqref{eq: equation for psi} from question 11 and write
    \begin{equation}
      \begin{aligned}
        &\dual{T} (\varphi_1) =  \sum_{k=1}^{3} \dual{T} (\varphi_1) (e_k) \psi_k = 4 \psi_1 + 5 \psi_2 + 6 \psi_3 \; , \text{ and }  \\
        &\dual{T} (\varphi_2) =  \sum_{k=1}^{3} \dual{T} (\varphi_2) (e_k) \psi_k = 7 \psi_1 + 8 \psi_2 + 9 \psi_3 \;
      \end{aligned}
    \end{equation}


    \prooffont{Variation of Qustion 14:} The question becomes more interesting if we use another basis for $\real^2$ and $\real^3$, say $v_1 = (1,1), v_2=(2,1)$ for $\real^2$ and $w_1 = (2,2,0), w_2 = (4,2,0), w_3 = (0,0,1)$ for $\real^3$. In this case, let $\phi_1, \phi_2$ denote the dual basis of $v_1, v_2$ and $\psi_1, \psi_2, \psi_3$ denote the dual basis of $w_1, w_2, w_3$. Now, we have to calculate $\phi$ and $\psi$ first. Per definition, we have
    \begin{equation}
      \phi(v) = \phi (a_1 v_1 + a_2 v_2) = a_1 \phi_1 (v_1) + a_2 \phi_2 (v_2).
    \end{equation}

    And we must assure that $\phi_1(v_1)=1, \phi(v_2)=0$ and $\phi_2(v_2)=0$ and $\phi_2(v_2) = 1$. Using the facts that $v_1 = e_1 + e_2$ and $v_2 = 2 e_1 + e_2$, we get
    \begin{equation}
      \left | \;
      \begin{aligned}
        1 \phi_1 (e_1) + 1 \phi_1 (e_2) &= 1 \\
        2 \phi_1 (e_1) + 1 \phi_1 (e_2) &= 0
      \end{aligned} \;
      \right  | \; \implies \;
      \left | \;
        \begin{aligned}
          1 \phi_1 (e_1) + 1 \phi_1 (e_2) &= 1 \\
                         - 2 \phi_1 (e_2) &= -2
        \end{aligned} \;
      \right | \; ,
    \end{equation}

    and hence, $\phi_1(e_1) = -1$ and $\phi_1(e_2) = 2$. Likewise,
    \begin{equation}
    \left | \;
    \begin{aligned}
      1 \phi_2 (e_1) + 1 \phi_2(e_2) &= 0 \\
      2 \phi_2 (e_1) + 1 \phi_2 (e_2) &= 1
    \end{aligned} \;
    \right  | \; \implies \;
    \left | \;
    \begin{aligned}
      1 \phi_2 (e_1) + 1 \phi_2 (e_2) &= 0 \\
      - 2 \phi_2 (e_2) &= 1
    \end{aligned} \;
    \right | \; .
    \end{equation}

    Therefore, $\phi_2(e_1) = 1$ and $\phi_2(e_2) = -1$. To write $\phi_1$ and $\phi_2$ as members of $\real^2 \to \real$, we have $\phi_1(x,y) = -x + 2y$ and $\phi_2 (x,y) = x-y$. Now,
    \begin{equation}
      \begin{aligned}
        T'(\phi_1) &= \phi_1 \circ T = -(4x + 5y + 6z) + 2          (7x + 8y + 9z) = 10x + 11y + 12z \; , \;  \text{ and }  \\
        T'(\phi_2) &= \phi_2 \circ T = 4x + 5y + 6z - (7x + 8y + 9z) = -3x -3y -3z = -3(x+y+z).
      \end{aligned}
    \end{equation}

    With the same approach, we can calculate the dual basis for $w_1, w_2, w_3$. We get $\psi_1(x,y,z) = -\frac{1}{2}x + y$, $\psi_2(x,y,z) = \frac{1}{2}x - \frac{1}{2}y$ and $\psi_3(x,y,z) = z$. To answer question (b), we use the theorem \eqref{eq: equation for psi} from question 11 and write
    \begin{equation}
      \begin{aligned}
        \dual{T} (\varphi_1) =  \sum_{k=1}^{3} \dual{T} (\varphi_1) (w_k) \psi_k &= (20+22+0) \psi_1 + (40+22+0) \psi_2 + (0+0+12 )\psi_3 \\
        & = 42 \psi_1 + 62 \psi_2 + 12 \psi_3  \; , \text{ and }  \\
        \dual{T} (\varphi_2) =  \sum_{k=1}^{3} \dual{T} (\varphi_2) (w_k) \psi_k &= -12 \psi_1 -18 \psi_2 -3 \psi_3. \;
      \end{aligned}
    \end{equation}
  \end{xsol}
\end{xrcs}

\exercise{16}
\begin{xrcs}
  Suppose $W$ is finite-dimensional and $T \in \linmap (V,W)$. Prove that
  \begin{equation}
    \dual{T} = 0 \iff T = 0.
  \end{equation}
  \begin{xprf}
    \Leftarrowdirection If $T = 0$, then for each $\varphi \in \dual{W}$ we have
    \begin{equation}
      \dual{T} (\varphi) (v) = \varphi(T(v)) = \varphi(0) = 0.
    \end{equation}

    Hence, $\dual{T} (\varphi)$ is always sent to a map in $\dual{W}$ that is the zero map.

    \Rightarrowdirection Assume $\dual{T} = 0$. By definition, $\dual{T} = \varphi \circ T = 0$ for every $\varphi \in \dual{W}$. If $W = \{0\}$, then necessarily $T = 0$, so $\dual{T} = 0$ holds trivially.

    Now, assume $W \neq 0$ and let $w_1, \ddd, w_n$ be a basis of $W$. Suppose for the sake of contradiction that $T \neq 0$. Hence, there exists a nonzero $v \in V$ and a nonzero $w \in W$ such that $w=T(v)$. Express $w$ as $a_1 w_1 + \cdots + a_n w_n$. Since $w \neq 0$, pick $k$ such that $a_k \neq 0$. Define $\varphi \in \dual{W}$ by
    \begin{equation}
      \varphi(b_1 w_1 + \cdots + b_n w_n) = \frac{b_k}{a_k},
    \end{equation}

    where $b_k$ is the $k$-th coordinate of $\varphi$'s input. Hence, $\varphi(w) =1$.
    But then,
    \begin{equation}
      \dual{T}(\varphi) (v) = \varphi(T(v)) = \varphi(w) = 1 \neq 0.
    \end{equation}

    Thus, $\dual{T}(\varphi) \neq 0$, contradicting $\dual{T} = 0$. Hence, $T = 0$.
  \end{xprf}
\end{xrcs}