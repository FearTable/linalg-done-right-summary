\subsection*{Exercises about Null Spaces and Ranges 3B}
\hrule
\phantom{.}

\setcounter{xrcs}{8}
\begin{xrcs}
  Suppose $T \in \linmap(V,W)$ is injective and $v_1, \ddd, v_n$ is linearly independent in $V$. Prove that $Tv_1, \ddd, Tv_n$ is linearly independent in $W$.
\end{xrcs}
\begin{prf}
  Let $a_1, \ddd, a_n \in \myF$ such that $a_1 v_1 + \cdots + a_n v_n  = 0$. We know that following equivalences hold:
  \begin{equation}
    \begin{aligned}
              a_1 v_1 + \cdots + a_n v_n  &= 0 \\
      \iff T (a_1 v_1 + \cdots + a_n v_n) &= 0 \\
      \iff a_1 T v_1 + \cdots + a_n T v_n &= 0 \\
    \end{aligned}
  \end{equation}

  Since we know this only happens if $a_1 = 0, \ddd, a_n = 0$, the last equation shows that $T v_1, \ddd, T v_n$ is linearly independent. We have used the fact that $T$ is injective for the first equivalence.
\end{prf}

\begin{xrcs} % ChatGTP did a lot of changes
  Suppose $T \in \linmap(V,W)$ is injective and $v_1, \ddd, v_n$ is linearly independent in $V$. Prove that $Tv_1, \ddd, Tv_n$ is linearly independent in $W$.
\end{xrcs}
\begin{prf}
  Let $a_1, \ddd, a_n \in \myF$ such that
  \[
  a_1 Tv_1 + \cdots + a_n Tv_n = 0.
  \]
  By the linearity of $T$, this implies:
  \[
  T(a_1 v_1 + \cdots + a_n v_n) = 0.
  \]
  Since $T$ is injective, and $T(0)=0$ by \ref{thm: linear maps take 0 to 0}, \( T(u) = 0 \) if and only if \( u = 0 \). Thus:
  \[
  a_1 v_1 + \cdots + a_n v_n = 0.
  \]
  Because $v_1, \ddd, v_n$ are linearly independent in $V$, we conclude:
  \[
  a_1 = 0, \ddd, a_n = 0.
  \]
  Thus, $Tv_1, \ddd, Tv_n$ are linearly independent in $W$.
\end{prf}

\begin{xrcs}
  Suppose $v_1, \ddd, v_n$ spans $V$ and $T \in \linmap(V,W)$. Show that $Tv_1, \ddd, Tv_n$ spans $\myrange T$.
\end{xrcs}
\begin{prf}
  Let $v = a_1 v_1 + \cdots + a_n v_n \in V$, where $a_1, \ddd, a_n \in \myF$.
  \begin{equation}
    \implies Tv = a_1 T v_1 + \cdots + a_n T v_n \in \myrange T.
  \end{equation}

  Thus, every element of $Tv \in \myrange T$ can be written as a linear combination of $Tv_1, \ddd, Tv_n$. This shows that
  \begin{equation}
    \myrange T \subseteq \myspan{T v_1, \ddd, T v_n}.
  \end{equation}

  Conversely, \( Tv_1, \ddd, Tv_n \in \myrange T \) and $\myrange T$ is a vector space which is closed under addition and multiplication. Therefore, $\myrange T = \myspan{Tv_1, \ddd, Tv_n}$.
\end{prf}

