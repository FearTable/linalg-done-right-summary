\subsection*{Exercises about Vector Space of Linear Maps 3A}
\hrule
\phantom{.}

\setcounter{xrcs}{11}
%\begin{xrcs}
%  Suppose $U$ is a subspace of $V$ with $U \neq V$. Let $S \in \linmap(U, W)$ and $S \neq 0$, which means that $Su \neq 0$ for some $u \in U$. Define $T: V \to W$ by
%  \begin{equation}
%    Tv :\equiv
%    \begin{cases}
%      Sv & \mytext{if} v \in U, \\
%      0  & \mytext{if} v \in V \mytext{and} v \notin U \quad (v \in V \backslash U)
%    \end{cases}
%  \end{equation}
%
%  Prove that $T$ is not a linear map on $V$.
%\end{xrcs}
%\begin{prf}
%  Let $u \in U$ and $v \in V$ with $v \notin U$ ($v \in V \backslash U$) Consider the sum $u + v$. We claim that $u+v \notin U$, because if $u+v \in U$ then by closure of $U$, we would have $(u+v)-u = v \in U$, which contradicts $v \notin U$. Therefore, it holds that
%  \begin{equation}
%    T(\underbrace{u+v}_{\notin \; U}) = 0 \neq Su + 0 = Tu + Tv.
%  \end{equation}
%
%  The inequality above demonstrates that $T$ is not a linear map, because it fails to satisfy additivity. The failure of $T$ to be a linear map arises because $T$ is defined differently on $U$ and $V \backslash U$, breaking the additivity required for linearity.
%\end{prf}

\begin{xrcs}
  Suppose $U$ is a subspace of $V$ with $U \neq V$. Let $S \in \linmap(U, W)$ and $S \neq 0$, which means that $Su \neq 0$ for some $u \in U$. Define $T: V \to W$ by
  \begin{equation}
    Tv :\equiv
    \begin{cases}
      Sv & \mytext{if} v \in U, \\
      0  & \mytext{if} v \in V \mytext{and} v \notin U \quad (v \in V \backslash U)
    \end{cases}
  \end{equation}

  Prove that $T$ is not a linear map on $V$.
\end{xrcs}
\begin{prf}
  Let $u \in U$ and $v \in V$ with $v \notin U$ ($v \in V \backslash U$). Consider the sum $u + v$. We claim that $u+v \notin U$, because if $u+v \in U$ then by closure of $U$, we would have $(u+v)-u = v \in U$, which contradicts $v \notin U$. Therefore, it holds that
  \begin{equation}
    T(\underbrace{u+v}_{\notin \; U}) = 0 \neq Su + 0 = Tu + Tv.
  \end{equation}

  The inequality above demonstrates that $T$ is not a linear map, because it fails to satisfy additivity. The failure of $T$ to be a linear map arises because $T$ is defined differently on $U$ and $V \backslash U$, breaking the additivity required for linearity.
\end{prf}
