\subsection*{Exercises about Vector Space of Linear Maps 3A}
\hrule
\phantom{.}

\setcounter{xrcscount}{11}
%\begin{xrcs}
%  Suppose $U$ is a subspace of $V$ with $U \neq V$. Let $S \in \linmap(U, W)$ and $S \neq 0$, which means that $Su \neq 0$ for some $u \in U$. Define $T: V \to W$ by
%  \begin{equation}
%    Tv :\equiv
%    \begin{cases}
%      Sv & \mytext{if} v \in U, \\
%      0  & \mytext{if} v \in V \mytext{and} v \notin U \quad (v \in V \backslash U)
%    \end{cases}
%  \end{equation}
%
%  Prove that $T$ is not a linear map on $V$.
%\end{xrcs}
%\begin{prf}
%  Let $u \in U$ and $v \in V$ with $v \notin U$ ($v \in V \backslash U$) Consider the sum $u + v$. We claim that $u+v \notin U$, because if $u+v \in U$ then by closure of $U$, we would have $(u+v)-u = v \in U$, which contradicts $v \notin U$. Therefore, it holds that
%  \begin{equation}
%    T(\underbrace{u+v}_{\notin \; U}) = 0 \neq Su + 0 = Tu + Tv.
%  \end{equation}
%
%  The inequality above demonstrates that $T$ is not a linear map, because it fails to satisfy additivity. The failure of $T$ to be a linear map arises because $T$ is defined differently on $U$ and $V \backslash U$, breaking the additivity required for linearity.
%\end{prf}

\begin{xrcs}
  Suppose $U$ is a subspace of $V$ with $U \neq V$. Let $S \in \linmap(U, W)$ and $S \neq 0$, which means that $Su \neq 0$ for some $u \in U$. Define $T: V \to W$ by
  \begin{equation}
    Tv :\equiv
    \begin{cases}
      Sv & \mytext{if} v \in U, \\
      0  & \mytext{if} v \in V \mytext{and} v \notin U \quad (v \in V \backslash U)
    \end{cases}
  \end{equation}

  Prove that $T$ is not a linear map on $V$.

  \begin{xprf}
    Let $u \in U$ and $v \in V$ with $v \notin U$ ($v \in V \backslash U$). Consider the sum $u + v$. We claim that $u+v \notin U$, because if $u+v \in U$ then by closure of $U$, we would have $(u+v)-u = v \in U$, which contradicts $v \notin U$. Therefore, it holds that
    \begin{equation}
      T(\underbrace{u+v}_{\notin \; U}) = 0 \neq Su + 0 = Tu + Tv.
    \end{equation}

    The inequality above demonstrates that $T$ is not a linear map, because it fails to satisfy additivity. The failure of $T$ to be a linear map arises because $T$ is defined differently on $U$ and $V \backslash U$, breaking the additivity required for linearity.
  \end{xprf}
\end{xrcs}

\exercise{13}
\begin{xrcs}
  Suppose $V$ is finite-dimensional. Prove that every linear map on a subspace of $V$ can be extended to a linear map on $V$. In other words, show that if $U$ is a subspace of $V$ and $S \in \linmap (U,W)$, then $\exists T \in \linmap(L, W)$ such that
  \begin{equation}
    T u = S u, \; \; \forall u \in U.
  \end{equation}

  \begin{xprf}
    Let  $U$ be a subspace of a finite dimensional vector space $V$, $S \in \linmap(U,W)$, and $T \in \linmap(V,W)$. Let $X \subseteq V$ such that $V = U \oplus X$, and such that $u_1, \ddd, u_m$ is a basis of $U$ and $x_{m+1}, \ddd, x_n$ is a basis of $X$. This makes
    \begin{equation}
      u_1, \ddd, u_m, x_{m+1}, \ddd, x_{n}
    \end{equation}

    a basis of $V$. So we define $T \in \linmap(V,W)$ for $v \in V$ as follows:
    \begin{equation}
      \begin{aligned}
        T(v)  &\phantom{:}= T(c_1 u_1 + \cdots + c_m u_m + c_{m+1} x_{m+1} + \cdots + c_n x_n) \\
              &:=           c_1 S u_1 + \cdots + c_m S u_m
      \end{aligned}
    \end{equation}

    Clearly, $T u = S u$ for all $u \in U$.
  \end{xprf}
\end{xrcs}
