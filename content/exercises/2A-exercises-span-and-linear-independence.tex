\section*{Exercises about Span and Linear Independence}

\begin{xrcs}
  We want to find a list of four distinct vectors in $\myF^3$ whose span equals
  \begin{equation}
    \{(x,y,z) \in \myF^3 \mid x + y + z = 0 \}.
  \end{equation}
\end{xrcs}

\begin{xrcs}
  Claim: If $v_1, v_2, v_3, v_4$ spans $V$, then the list $v_1 - v_2, v_2 -v_3, v_3 -v_4, v_4$ also spans $V$.

  \begin{xprf}
    Let $u \in V$ and $a_i, b_i \in \myF$ \st
    \[
    \begin{aligned}
      u
      &= a_1 (v_1 -v_2) + a_2 (v_2-v_3) + a_3 (v_3-v_4)-a_4 v_4 \\
      &= a_1 v_1 - a_1 v_2 + a_2v_2 - a_2 v_3 + a_3 v_3 - a_3 v_4 + a_4 v_4 \\
      &= \underbrace{a_1}_{b_1} v_1 + \underbrace{(a_2 -a_1)}_{b_2} v_2 + \underbrace{(a_3 -a_2)}_{b_3} v_3 + \underbrace{(a_4 -a_3)}_{b_4} v_4 \\
      \implies b_1 & = a_1 \\
      b_2 &= a_2 -a_1 = a_2 - b_1 \iff a_2 = b_1 + b_2 \\
      b_3 &= a_3 - a_2 = a_3 - b_1 -b_2 \iff a_3 = b_1 + b_2 + b_3 \\
      b_4 &= a_4 - a_3 = a_4 - b_1 - b_2 - b_3 \iff a_4 = b_1 + b_2 + b_3 + b_4
    \end{aligned}
    \]

    So if we have an arbitrary vector $u \in V$ such that
    \[
    u = b_1 v_1 +  b_2 v_2 + b_3 v_3 + b_4 v_4
    \]

    it can be rewritten as a linear combination
    \[
    u = b_1 (v_1 - v_2) + (b_1-b_2)(v_2-v_3) + (b_1+b_2+b_3)(v_2-v_4)+(b_1+b_2+b_3+b_4)v_4
    \]

    So the list $v_1 - v_2, v_2 -v_3, v_3 -v_4, v_4$ also spans $V$.
  \end{xprf}
\end{xrcs}


\begin{xrcs}
  Claim: Let $v_1, \ddd, v_m \in V$ and
  \[
    w_k :\equiv v_1 + \cdots + v_k \quad \forall k\in \{1, \ddd, m\}.
  \]

  Then $\myspan{v_1, \ddd, v_m} = \myspan{w_1, \ddd, w_m}$.

  \begin{xprf}
    Observe that
    \[
    \begin{aligned}
      w_1 &= v_1
      & \qquad v_1 &= w_1       \\
      w_2 &= v_1 + v_2
      & \qquad v_2 &= w_2-v_1 = w_2 - w_1 \\
      w_3 &= \boxed{v_1 + v_2 + v_3}
      & \qquad v_3 &= w_3-v_1-v_2 = w_3-w_1-w_2+w_1 = w_3 - w_2 \\
      w_4 &= v_1 + v_2 + v_3 + v_4
      & \qquad v_4 &= w_4-\boxed{(v_1 + v_2 + v_3)}=w_4 - w_3 \\
      &\;\;\vdots
      &            &\;\;\vdots \\
      w_m &= v_1 + \cdots + v_m
      & \qquad v_m &= w_m - w_{m-1}
    \end{aligned}
    \]

    \StepOne So if $u \in \myspan{v_1, \ddd, v_m}$ we have for some $a_1, \ddd, a_m \in \myF$
    \[
    \begin{aligned}
      u &= a_1 v_1 + \cdots + a_m v_m \\
      &= a_1 w_1 + a_2 (-w_1 + w_2) + a_3(-w_2 + w_3) + \cdots + a_m (-w_{m-1} + w_m) \\
      &= (a_1 - a_2) w_1 + (a_2-a_3) w_2 + \cdots + (a_{m-1} + a_m) w_{m-1} + a_m w_m \\
    \end{aligned}
    \]

    Thus, $u \in \myspan {w_1, \ddd, w_m}$ and therefore
    \begin{equation}
      \label{eq: fact1}
      \myspan{v_1, \ddd, v_m} \subseteq \myspan{w_1, \ddd, w_m}.
    \end{equation}


    \StepTwo If $u \in \myspan{w_1, \ddd, w_m}$ we have for some $a_1, \ddd, a_m \in \myF$
    \[
    \begin{aligned}
      u &= a_1 w_1 + \cdots + a_m w_m \\
      &= a_1 v_1 + a_2 (v_1+v_2) + \cdots + a_m(v_1 + \cdots + v_m) \\
      &= a_1 v_1 + (a_2 v_1+ a_2 v_2) + \cdots + (a_m v_1 + \cdots + a_m v_m) \\
      &= (a_1 + \cdots + a_m) v_1 + (a_2 + \cdots + a_m) v_2 + \cdots + a_m v_m
    \end{aligned}
    \]

    Thus, $u \in \myspan{v_1, \ddd, v_m}$ and therefore
    \begin{equation}
      \label{eq: fact2}
      \myspan{w_1, \ddd, w_m} \subseteq \myspan{v_1, \ddd, v_m}.
    \end{equation}

    \eqref{eq: fact1} and \eqref{eq: fact2} together imply that $\myspan{w_1, \ddd, w_m} = \myspan{v_1, \ddd, v_m}$.
  \end{xprf}
\end{xrcs}


\setcounter{xrcscount}{15}
\begin{xrcs}
  $\polyn_4(\myF)$ has $5$ degrees of freedom ans looks like this:
  \[
    \polyn_4(\myF) = \{a_0 + a_1 z + a_2z^2 + a_3 z^3 + a_4 z^4 \mid a_0, \ddd, a4 \in \myF \}.
  \]

  So we have $5$ degrees of freedom. If you span $4$ polynomials, you can not get $5$ degrees of freedom.
\end{xrcs}

\setcounter{xrcscount}{19}
\begin{xrcs}
  Suppose $p_0, p_1, \ddd, p_m$ are polynomials in $\polyn_m(\myF)$ such that $p_k(2) = 0$ for $0\leq k \leq m$. \\
  Prove that $p_0, p_1, \ddd, p_m$ is not linearly independent in $\polyn_m(\myF)$.

  \begin{xprf}
    Define an evaluation map $\varphi : \polyn_m(\myF) \to \myF$ such that
    \[
    \varphi(p) :\equiv p(2) \quad \forall p \in \polyn(\myF).
    \]

    The polynomials that vanish at $z=2$ lies in the subspace $\mynull(\varphi)$ of $\polyn_m(\myF)$. So  we have
    \[
    \mynull(\varphi) \subseteq \polyn_m(\myF).
    \]

    Any polynomial $p \in \mynull(\varphi)$ can be written as
    \[
    p(z) = (z-2) q(z) \where q(z) \in \polyn_{m-1} (\myF).
    \]

    Therefore, $\mydim \mynull(\varphi) = m$. Since $p_0, p_1, \ddd, p_m \in \mynull(\varphi)$, the given list is not linearly independent by \ref{thm: length of linearly dependent list less or equal to length of spanning list}, because it has length $m+1$.
  \end{xprf}

\end{xrcs}

%\begin{xrcs}
%  Suppose $p_0, p_1, \ddd, p_m$ are polynomials in $\polyn_m(\myF)$ such that $p_k(2) = 0 \quad \forall k \in \{0, \ddd, m\}$. Prove that $p_0, p_1, \ddd, p_m$ is not linearly independent in $\polyn_m(\myF)$.
%\end{xrcs}
%\begin{prf}
%  Define the polynomials $p_0, p_1, \ddd, p_m$ for all $a's$ as an element of $\myF$ like this:
%  \[
%  \begin{aligned}
  %    p_0(z) &:\equiv a_{0,0}  \\
  %    p_1(z) &:\equiv a_{1,0} + a_{1,1} z  \\
  %    p_2(z) &:\equiv a_{2,0} + a_{2,1} z + a_{2,2} z^2  \\
  %    p_3(z) &:\equiv a_{3,0} + a_{3,1} z + a_{3,2} z^2 + a_{3,3} z^3  \\
  %    \vdots & \quad \\
  %    p_m(z) &:\equiv a_{m,0} + a_{m,1} z + a_{m,2} z^2 + a_{m,3} z^3 + \cdots + a_{m,m} z^m  \\
  %  \end{aligned}
%  \]
%
%  By assumption, \( p_k(2) = 0 \) for \( k = 1, \ddd, m \), which gives the following constraints:
%  \[
%  \begin{aligned}
  %    p_0(2) &= a_{0,0} = 0 \\
  %    p_1(2) &= a_{1,0} + a_{1,1} 2 = 0 \\
  %    p_2(2) &= a_{2,0} + a_{2,1} 2 + a_{2,2} 2^2  = 0  \\
  %    p_3(2) &= a_{3,0} + a_{3,1} 2 + a_{3,2} 2^2 + a_{3,3} 2^3  = 0  \\
  %    \vdots & \quad \\
  %    p_m(2) &= a_{m,0} + a_{m,1} 2 + a_{m,2} 2^2 + a_{m,3} 2^3 + \cdots + a_{m,m} 2^m  = 0  \\
  %  \end{aligned}
%  \]
%
%  From the definition of linear independence, we examine the equation:
%  \[
%    b_0 p_0 + b_1 p_1 + b_2 p_2 + \cdots + b_m p_m = 0 \iff b_0 0 + b_1 p_1 + b_2 p_2 + \cdots + b_m p_m = 0
%  \]
%
%  Substituting \( p_0(2) = 0 \), this reduces to:
%  \[
%  b_0 \cdot 0 + b_1 p_1 + b_2 p_2 + \cdots + b_m p_m = 0.
%  \]
%
%  Any choice of $b_0$ will solve this equation and give us a non-trivial solution. So the list of polynomials $p_0, p_1, \ddd, p_m$ is not linearly independent.
%\end{prf}


\clearpage