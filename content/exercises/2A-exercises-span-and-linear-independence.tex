\subsection*{Exercises about Span and Linear Independence 1B}
\hrule
\phantom{.}

\begin{xrcs}
  We want to find a list of four distinct vectors in $\myF^3$ whose span equals $\{(x,y,z) \in \myF^3 \mid x + y + z = 0 \}$.
\end{xrcs}

\begin{xrcs}
  Claim: If $v_1, v_2, v_3, v_4$ spans $V$, then the list $v_1 - v_2, v_2 -v_3, v_3 -v_4, v_4$ also spans $V$.
\end{xrcs}
\begin{prf}
  Let $u \in V$ and $a_i, b_i \in \myF$ \st
  \[
    \begin{aligned}
      u
      &= a_1 (v_1 -v_2) + a_2 (v_2-v_3) + a_3 (v_3-v_4)-a_4 v_4 \\
      &= a_1 v_1 - a_1 v_2 + a_2v_2 - a_2 v_3 + a_3 v_3 - a_3 v_4 + a_4 v_4 \\
      &= \underbrace{a_1}_{b_1} v_1 + \underbrace{(a_2 -a_1)}_{b_2} v_2 + \underbrace{(a_3 -a_2)}_{b_3} v_3 + \underbrace{(a_4 -a_3)}_{b_4} v_4 \\
      \implies b_1 & = a_1 \\
      b_2 &= a_2 -a_1 = a_2 - b_1 \iff a_2 = b_1 + b_2 \\
      b_3 &= a_3 - a_2 = a_3 - b_1 -b_2 \iff a_3 = b_1 + b_2 + b_3 \\
      b_4 &= a_4 - a_3 = a_4 - b_1 - b_2 - b_3 \iff a_4 = b_1 + b_2 + b_3 + b_4
    \end{aligned}
  \]

  So if we have an arbitrary vector $u \in V$ such that
  \[
    u = b_1 v_1 +  b_2 v_2 + b_3 v_3 + b_4 v_4
  \]

  it can be rewritten as a linear combination
  \[
    u = b_1 (v_1 - v_2) + (b_1-b_2)(v_2-v_3) + (b_1+b_2+b_3)(v_2-v_4)+(b_1+b_2+b_3+b_4)v_4
  \]

  So the list $v_1 - v_2, v_2 -v_3, v_3 -v_4, v_4$ also spans $V$.
\end{prf}

\begin{xrcs}
  Claim: Let $v_1, \ddd, v_m \in V$ and
  \[
    w_k :\equiv v_1 + \cdots + v_k \quad \forall k\in \{1, \ddd, m\}.
  \]

  Then $\myspan{v_1, \ddd, v_m} = \myspan{w_1, \ddd, w_m}$.
\end{xrcs}
\begin{prf}
  Observe that
  \[
    \begin{aligned}
    w_1 &= v_1
    & \qquad v_1 &= w_1       \\
    w_2 &= v_1 + v_2
    & \qquad v_2 &= w_2-v_1 = w_2 - w_1 \\
    w_3 &= \boxed{v_1 + v_2 + v_3}
    & \qquad v_3 &= w_3-v_1-v_2 = w_3-w_1-w_2+w_1 = w_3 - w_2 \\
    w_4 &= v_1 + v_2 + v_3 + v_4
    & \qquad v_4 &= w_4-\boxed{(v_1 + v_2 + v_3)}=w_4 - w_3 \\
    &\;\;\vdots
    &            &\;\;\vdots \\
    w_m &= v_1 + \cdots + v_m
    & \qquad v_m &= w_m - w_{m-1}
    \end{aligned}
  \]

  \qt{Step 1:} So if $u \in \myspan{v_1, \ddd, v_m}$ we have for some $a_1, \ddd, a_m \in \myF$
  \[
  \begin{aligned}
      u &= a_1 v_1 + \cdots + a_m v_m \\
    &= a_1 w_1 + a_2 (-w_1 + w_2) + a_3(-w_2 + w_3) + \cdots + a_m (-w_{m-1} + w_m) \\
    &= (a_1 - a_2) w_1 + (a_2-a_3) w_2 + \cdots + (a_{m-1} + a_m) w_{m-1} + a_m w_m \\
  \end{aligned}
  \]

  Thus, $u \in \myspan {w_1, \ddd, w_m}$ and therefore
  \begin{equation}
    \label{eq: fact1}
    \myspan{v_1, \ddd, v_m} \subseteq \myspan{w_1, \ddd, w_m}.
  \end{equation}


  \qt{Step 2:} If $u \in \myspan{w_1, \ddd, w_m}$ we have for some $a_1, \ddd, a_m \in \myF$
  \[
  \begin{aligned}
    u &= a_1 w_1 + \cdots + a_m w_m \\
    &= a_1 v_1 + a_2 (v_1+v_2) + \cdots + a_m(v_1 + \cdots + v_m) \\
    &= a_1 v_1 + (a_2 v_1+ a_2 v_2) + \cdots + (a_m v_1 + \cdots + a_m v_m) \\
    &= (a_1 + \cdots + a_m) v_1 + (a_2 + \cdots + a_m) v_2 + \cdots + a_m v_m
  \end{aligned}
  \]

  Thus, $u \in \myspan{v_1, \ddd, v_m}$ and therefore
  \begin{equation}
    \label{eq: fact2}
    \myspan{w_1, \ddd, w_m} \subseteq \myspan{v_1, \ddd, v_m}.
  \end{equation}

  \eqref{eq: fact1} and \eqref{eq: fact2} together imply that $\myspan{w_1, \ddd, w_m} = \myspan{v_1, \ddd, v_m}$.
\end{prf}