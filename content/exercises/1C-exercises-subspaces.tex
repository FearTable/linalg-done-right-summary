\section*{Exercises about Subspaces}

\begin{xrcs}
  For each of the following subsets of $\myF^3$, determine wether is is a subspace of $\myF^3$.
  \begin{enumerate}
    \item $\{(x_1, x_2, x_3) \in \myF \mid x_1 + 2x_2 + 3 x_3 = 0\}$
    \item $\{(x_1, x_2, x_3) \in \myF \mid x_1 + 2x_2 + 3 x_3 = 4\}$
    \item $\{(x_1, x_2, x_3) \in \myF \mid x_1 x_2 x_3 = 0\}$
    \item $\{(x_1, x_2, x_3) \in \myF \mid x_1 = 5x_3\}$
  \end{enumerate}

  \begin{xsol}
    \begin{enumerate}
    \item Let $E :\equiv \{(x_1, x_2, x_3) \in \myF \mid x_1 + 2x_2 + 3 x_3 = 0\}$, then $E$ is a subspace of $\myF^3$ because it is a plane. We have the additive identity  $0 \in \{(x_1, x_2, x_3) \in \myF \mid x_1 + 2x_2 + 3 x_3 = 0\}$. For $u,v \in E$ with their coordinates satisfying $u_1 + 2u_2 + 3u_3 = 0$ and $v_1 + 2v_2 + 3v_3 = 0$, we also have the condition that $(u_1 + v_1) + 2 (u_2 + v_2) + 3(u_3 + v_3)= 0$. So $u+v \in E$. Hence $E$ is closed under addition. If $w \in E$ and $\lambda \in \myF$, then if $u_1 + 2u_2 + 3u_3 = 0$, it also holds that $\lambda u_1 + 2\lambda u_2 + 3\lambda u_3 = 0$. Hence $\lambda u \in E$. So $E$ is closed under scalar multiplication. Therefore, $E$ is a subspace of $\myF^3$ by \ref{thm: conditions for a subspace}.

    \item It is not a subspace because is does not contain $0$. It is also not closed under addtition and scalar multiplication.

    \item Let $U :\equiv \{(x_1, x_2, x_3) \in \myF \mid x_1 x_2 x_3 = 0\}$. Then $U$ is actually the union of the $xy$-, the $xz$- and the $yz$-plane, because its description tells us that at least one of its coordinates has to be zero. Thus we have
    \[
    (1,1,0) + (0,1,1) = (1,2,1) \notin U.
    \]

    Therefore, $U$ is not closed under addition. But it is closed under scalar multiplication and we also have $0 \in U$. Therefore, $U$ is not a subspace of $\myF^3$.

    \item Let $U :\equiv \{(x_1, x_2, x_3) \in \myF \mid x_1 = 5x_3\}$. Then $0 \in U$, so we have checked for additive identity. Let $u,v \in U$. Then $u+v= (5u_3+5v_3, u_2+v_2, u_3+v_3) = (5(u_3+v_3), u_2+v_2, u_3+v_3) \in U$, so its closed under addition. Simmilarly for $\lambda \in \myF$ and $w \in U$ we have $\lambda u = (\lambda 5u_3, \lambda u_2, \lambda u_3) = (5 \lambda u_3, \lambda u_2, \lambda u_3) \in U$, so $U$ is a subspace of $\myF^3$.
  \end{enumerate}
  \end{xsol}

\end{xrcs}


\begin{xrcs}
  Verify all assertions about subspaces in example 1.35.

  \begin{enumerate}
    \item {
      If $b \in \myF$ then $U :\equiv \{(x_1, x_2, x_3, x_4) \in \myF \mid x_3 = 5x_4+b \}$ is a subspace of $\myF^4$ $\iff$ $b=0$.

      \begin{xsol}
        If $v,u \in U$, then
        \begin{equation}
          \begin{aligned}
            v+u & = (v_1+u_1, v_2+u_2, v_3+u_3, v_4+u_4, v_5+u_4) \\
            & = (v_1+u_1, v_2+u_2, v_3+u_3, 5v_4+b+5u_4+b, v_5+u_4) \\
            & = (v_1+u_1, v_2+u_2, v_3+u_3, 5(v_4 + u_4) + 2b, v_5+u_4)
          \end{aligned}
        \end{equation}

        Only if $b=0$, it can hold that the third coordinate of $v+u$ is $5$ times the fourth coordinate of $v+u$ plus an element of $\myF$ acting as $b$. %, given the constraints of $v$ and $u$ themselfes.
        Also note that if $\lambda \in \myF$ and $w \in U$, then $\lambda w = (\lambda w_1, \lambda w_2, \lambda u_4 + \lambda b, \lambda u_4)$. In this case, we have $\lambda w \in U$ if and only if $b=0$.
      \end{xsol}
    }

    \item{
      The set of continious real-valued functions on the intervall $[0,1]$ is a subspace of $\real^{[0,1]}$ (The set of real valued functions on the invervall $[0,1]$)

      \begin{xsol}
        The constant function $0$ is continious. Adding two continious functions together yields a continious function. Multiplying a continious function with a scalar also gives us a continious function.
      \end{xsol}
    }

    \item{
      The set of differentiable real-valued functions of $\real$ is a subspace of $\real^\real$.

      \begin{xsol}
        We check for all three properties. \\
        \emph{Additive identity:} $f(x) :\equiv 0 \quad \forall x \in \real$ is in the set of differentiable real-valued functions since $f'(x) = 0 \quad \forall x \in \real$ is also contained.

        \emph{Closededness under addition:} Let $h(x) :\equiv f(x) + g(x)$ for two differentiable real-valued functions $f$ and $g$. Then $h'(x) = \frac{d}{dx} (f(x)+g(x)) = f'(x) + g'(x)$, so this sum $h$ also belongs to set of differentiable real-valued functions.

        \emph{Closedednes under multiplication:} Let $h(x) :\equiv c f(x)$. Then $h'(x) = \frac{d}{dx} (c f(x)) = c f'(x)$. So this product $h$ also belongs to the set of differentiable real-valued functions.
      \end{xsol}
    }

    \item{
      The set $S$ of differentiable real-valued functions $f$ on the inverfall $(0,3)$ such that $f'(2) = b$ is a subspace of $\real^{(0,3)}$ if and only if $b=0$.

      \begin{xsol}
        For $f(x) :\equiv 0$ we have $f'(x) = 0$. The condition $f'(2) =  0 = b$ only holds if and only if $b=0$.

      For $h(x) :\equiv f(x) + g(x)$ we have $h'(x) = f'(x) + g'(x)$. The condition $h'(2) = f'(2) + g'(2) = 2b = b$  holds if and only if $b=0$.

      For $h(x) :\equiv cf(x)$ we have $h'(x) = cf'(x)$. $h'(2) = cf'(2) = cb = b$ can only hold for all constants $c$ if and only if $b=0$.
      \end{xsol}
    }
  \end{enumerate}
\end{xrcs}


\begin{xrcs}
  Show that the set $S$ of differentiable real-valued functions $f$ on the invervall $(-4,4)$ such that $f'(-1) = 3f(2)$ is a subspace of $\real^{(-4,4)}$.

  \begin{xsol} We check for all three properties. \\
    \emph{Additive identity:} For $f(x) :\equiv 0, f'(x) = 0$ we have $f'(-1) = 0 = 3 \cdot 0 = 3f(2)$, so $f'$ belongs so $S$ as well.

    \emph{Closed under addition:} For $h(x) :\equiv f(x) + g(x), h'(x) = f'(x) + g'(x)$ we got $h'(-1) = f'(-1) + g'(-1) = 3f(2) + 3g(2) = 3(f(2)+g(2)) = 3h(2)$. Therefore, $h \in S$.

    \emph{Closed under scalar multiplication:} For $h(x) :\equiv cf(x), h'(x) = cf'(x)$ we got $h'(-1) = cf'(-1) = c3f(2) = 3cf(2) = 3h(x)$. So if $h' \in S$ whenever $h \in S$.

    Hence $S$ is a subspace of $\real^{(-4,4)}$.
  \end{xsol}
\end{xrcs}

\setcounter{xrcscount}{7}
\begin{xrcs}
  Prove or give a counterexample: If $U$ is a nonempty subset of $\real^2$ such that $U$ is closed under addition and under taking additive inverses (meaning $-u \in U$ whenever $u \in U$), then $U$ is subspace of $\real^2$.

  \begin{xsol}
    This statement is wrong. Take $U :\equiv \{(x,y) \in \real^2 \mid (x,y) \in \integer^2 \} \subset \real^2$. Then $U$ is closed under addtion since $\integer$ is closed under addition. Example: $(2,1)+(-2,1)+(1,1) = (1,1) \in U$. But if $\lambda \in \real$ such that $\lambda \notin \integer$ we have for $a,b \in \integer$:
  \[
  \lambda(a,b) = (\lambda a, \lambda b) \notin U.
  \]

  For example: $\pi (3,4) \notin U$. Therefore $U$ is not a subspace of $\real^2$.
  \end{xsol}
\end{xrcs}



\setcounter{xrcscount}{11}
\begin{xrcs}
  Prove that the union $U \cup W$ of of two subspaces $U, W$ of $V$ is a subspace of $V$ $\iff$ one of the subspaces is contained in the other. So $U\subseteq W$ or $W \subseteq U$.

  \begin{xprf}
    \Leftarrowdirection Let $U$ and $W$ be two subspaces of $V$. If $U \subseteq W$ then $U \cup W = W \subseteq V$ and if $W \subseteq U$ then $U \cup W = U \subseteq V$.

    \Rightarrowdirection We can prove this either \underline{directly} or \underline{by contradiction}. In both proofs, let us assume that $U \cup W$ is a subspace of $V$, given that $U$ and $W$ are subspaces of $V$ as well. \\
    \prooffont{$\Rightarrow$ direction proof by contradiction:} Suppose $U \subsetneq W$ and $W \subsetneq U$. Note that this would imply that $U \neq \{0\}$ and $W \neq \{0\}$. Therefore, we have that
    \[
    \begin{aligned}
      &\exists u \in U: u \notin W \myand \\
      &\exists w \in W: w \notin U.
    \end{aligned}
    \]

    Since $u,w \in U \cup W \implies u+w \in U \cup W $ we have that either $u+w \in U$ or $u+w \in W$ (or both).

    \begin{itemize}
      \item \prooffont{Case $\mathbf{u+w \in U:}$} In this case, let $\widetilde{u} := u + w \in U$. Hence, $w = \widetilde{u} - u \in U$, a contradiction, because we had $w \notin U$
      \item \prooffont{Case $\mathbf{u+w \in W:}$} Let $\widetilde w := u + w \in W$. Hence $u = \widetilde{w} - u \in W$, again a contradiction, because we had $u \notin W$.
    \end{itemize}

    Therefore, the assumption that $U \subsetneq W$ and $W \subsetneq U$ can not hold. Hence, the opposite is true. By de Morgan's rule, it must hold that $U \subseteq W$ or $W \subseteq U$.

    \prooffont{Direct $\Rightarrow$ direction proof:} We have $2$ cases. $W \subseteq U$ or $W \subsetneq U$. If $W \subseteq U$, then the disjunction
    $
    W \subseteq U $ or $ U \subseteq W
    $
    is automatically true. So let us assume that $W \subsetneq U$ and prove that this implies that $U \subseteq W$.
    Let $u \in U$ and $w \in W \backslash U$. By the definition of union, it also holds that $u,w \in U \cup W$. Since $U \cup W$ is a subspace of $V$, we write
    \[
    u+w \in U \cup W.
    \]

    By the defintion of union, there are two possible cases. $u+w \in U$ or $u+w \in W$. The first case would imply
    \[
    w = (u+w) + (-u) \in U,
    \]

    which is impossible. Hence $u+w \in W$ and therefore,
    \[
    u= (u+w) -w \in W
    \]

    Since $u$ was arbitrary, $U \subseteq W$.
  \end{xprf}
\end{xrcs}


\setcounter{xrcscount}{22}
\begin{xrcs}
  Prove or give a counterexample: If $V_1, V_2, U$ are subspaces of $V$ such that $V=V_1 \oplus U$ and $V=V_2 \oplus U$, then $V_1 = V_2$.

  \begin{xsol}
    The statement is not ture. One counterexample would be $V :\equiv \real^2$ and
    \begin{equation}
      U :\equiv \{ (x_1, x_2) \mid x_2=0 \},
    \end{equation}

    where $U$ is the $x$-axis. Let $V_1$ and $V_2$ be defined as follows:
    \[
      V_1 :\equiv \underbrace{\{(x_1, x_2) \in \real^2 \mid x_1 = 0 \}}_{\text{$y$-axis}} \quad $ and $
      V_2 :\equiv \underbrace{\{ (x_1, x_2) \in \real^2 \mid x_2 = x_1 \}}_{\text{straight line with slope $1$}}
    \]

    We can see that
    \begin{equation}
      V=\real^2=V_1 \oplus U = V_2 \oplus U,
    \end{equation}

    because each $(x,y)\in \real^2$ can be uniquely represented as \begin{equation}
      \underbrace{(a,0)}_{\in U}+\underbrace{(0,b)}_{\in V_1}$ or $\underbrace{(c,0)}_{\in U}+\underbrace{(k,k)}_{\in V_2}=(c+k,k)$ where $a,b,c,k \in \real
    \end{equation}

    So the statement is not true.
  \end{xsol}
\end{xrcs}