\subsection*{Exercises about Bases 2B}
\hrule
\phantom{.}

\setcounter{xrcscount}{5}
% E 6
\begin{xrcs}
  Claim: If $v_1, v_2, v_3, v_4$ is a basis of $V$, then
  \begin{equation}
    v_1 + v_2, v_2 + v_3, v_3+v_4, v_4 $ is also a basis of $V.
  \end{equation}

  \begin{xprf}
    \StepOne We first show linear independence. Let $a_1, a_2, a_3, a_4 \in \myF$ such that
    \begin{equation}
      a_1 (v_1+v_2) + a_2(v_2+v_3) + a_3(v_3+v_4)+a_4 v_4 = 0.
    \end{equation}

    This is the case if and only if
    \[
    a_1 v_1 + a_1 v_2 + a_2 v_2 + a_2 v_3 + a_3 v_3 + a_3 v_4 + a_4 v_4
    = a_1 v_1 + (a_1 + a_2) v_2 + (a_2 + a_3) v_3 + (a_3 + a_4)v_4=0
    \]

    Since $v_1, \ldots, v_4$ are linearly independent, this is only the case if $0 = a_1 = a_1 + a_2 = a_2 + a_3 = a_3 + a_4$ and therefore, since $a_1=0$, $\iff$ $a_1 = \cdots = a_4 = 0$. So the list $v_1 + v_2, v_2 + v_3, v_3+v_4, v_4$ is linearly independent as well.

    \StepTwo Now we also have to show, that $v_1 + v_2, v_2 + v_3, v_3+v_4, v_4$ spans $V$. Let $u \in V$. With our old basis, $u$ would have been written as
    \begin{equation}
      \label{eq: equation of u}
      u = b_1 v_1 + \cdots + b_4 v_4.
    \end{equation}

    So let $v \in V$ such that
    \[
      \begin{aligned}
        v
        & = a_1 (v_1+v_2) + a_2(v_2+v_3) + a_3(v_3+v_4)+a_4 v_4 \\
        & = a_1 v_1 + (a_1 + a_2) v_2 + (a_2 + a_3) v_3 + (a_3 + a_4)v_4=0
      \end{aligned}
    \]

    Comparing this with the equation of $u$ \eqref{eq: equation of u}, we see that $v$ can be indeed any vector in $V$, since $v_1, v_2, v_3, v_4$ is a basis.

    If we want to rewrite vector $u$ we would write $a_1 = b_1.$ Next we would have $a_1 + a_2 = b_2$ and therefore $a_2 = b_2 - a_1 = b_2-b_1$. Next we would have $a_2+a_3=b_3$ and thus $a_3= b_3-a_2 = b_3 - (b_2-b_1)$. Lastly $a_4 = b_4-a_3 = b_4-(b_3 - (b_2-b_1))$. Putting this all together we have
    \[
      \begin{aligned}
        u
        & = b_1 \cdot (v_1+v_2) + (b_2-b_1)\cdot(v_2+v_3) + (b_3 - (b_2-b_1)) \cdot (v_3 + v_4) + (b_4-(b_3 - (b_2-b_1))) \cdot v_4 \\
        & = b_1  (v_1+v_2) + (b_2-b_1)(v_2+v_3) + (b_3 - b_2+b_1)  (v_3 + v_4) + (b_4-b_3+b_2-b_1)  v_4 \\
        & = b_1 v_1 + b_2 v_2 + b_3 v_3 + b_4 v_4.
      \end{aligned}
    \]

    which can be verified by comparing the coefficents of the $v$'s. For example, adding the two coefficients $(b_3 - b_2+b_1)$ and $(b_4-b_3+b_2-b_1)$ of $v_4$ together yields $b_4$. So every $u \in V$ can be expressed in either of the two lists $v_1, v_2, v_3, v_4$ or $v_1 + v_2, v_2 + v_3, v_3+v_4, v_4$. Since we have shown linear independence as well, $v_1 + v_2, v_2 + v_3, v_3+v_4, v_4$ is a basis of $V$.
  \end{xprf}
\end{xrcs}

% E 8
\begin{xrcs}
  Claim: If $v_1, v_2, v_3, v_4$ is a basis of $V$ and $U \subseteq V$, such that $v_1, v_2 \in U$ and $v_3 \notin U$ and $v_4 \notin U$, then $v_1, v_2$ is a basis of $U$.

  \begin{xprf}
    Any vector in $U$ can be represented as a unique linear combination of $v_1, v_2, v_3, v_4$ since $U \subseteq V$. Let $u\in U$ such that $u=a_1 v_1 + a_2 v_2 + a_3 v_3 + a_4 v_4$. In this case, $a_3$ must equal to $0$, because otherwise, due to closedness under addition and multiplication, we could subtract $a_1 v_1 + a_2 v_2+a_4v_4$ from $u$ and multiply by $\frac{1}{a_3}$ to get $v_3$, which is not in $U$. The same goes for $a_4$, which must be $0$ as well. So every vector in $U$ is of the form $u=a_1 v_1 + a_2 v_2$, since $v_1, v_2 \in U$ and $v_3, v_4 \notin U$. Since every basis has the same length and $v_1$ and $v_2$ are linearly independent, the list $v_1, v_2$ forms a basis of $U$.
  \end{xprf}
\end{xrcs}


\begin{xrcs}
  Suppose $v_1, \cdots, v_m$ is a list of vectors in $V$. For $k \in \{1, \ldots, m\},$ let
  \begin{equation}
    w_k = v_1 + \cdots + v_k
  \end{equation}

  Show that $v_1, \ldots, v_m$ is a basis of $V$ if and only if $w_1, \ldots, w_m$ is a basis of $V$.

  \begin{xprf}
    Let us first observe the following two patterns
    \begin{equation}
      \begin{aligned}
        w_1 &= v_1                    & \qquad v_1 &= w_1       \\
        w_2 &= v_1 + v_2              & \qquad v_2 &= w_2 - w_1 \\
        w_3 &= v_1 + v_2 + v_3        & \qquad v_3 &= w_3 - w_2 \\
        w_4 &= v_1 + v_2 + v_3 + v_4  & \qquad v_4 &= w_4 - w_3 \\
        &\;\;\vdots               &            &\;\;\vdots \\
        w_n &= v_1 + \cdots + v_n     & \qquad v_n &= w_n - w_{n-1}
      \end{aligned}
    \end{equation}
  \end{xprf}
\end{xrcs}


\phantom{.}

\hrule
