\subsection*{Exercises about Definition of a Vector Space 1B}
\hrule
\phantom{.}

%1
\begin{xrcs}
  Prove that $-(-v) = v \quad \forall v \in V.$
\end{xrcs}
\begin{prf}
  Let $v \in V$. By definition, we know that $v + (-v) = 0$. We also know by definition, that the additive inverse of $(-v)$, which is unique by \ref{thm: unique additive identity}, is denoted by $-(-v)$. The first equations shows that $-(-v) = v$, making $-(-v) + (-v) = 0$.
\end{prf}

%2
\begin{xrcs}
  Suppose $a \in \myF, v \in V$ and $a \cdot v = 0$. Prove that $a=0$ or $v=0$.
\end{xrcs}
\begin{prf}
  There are two cases for $a$. Either $a=0$ or $a\neq 0$. If $a=0$, we are done. If $a \neq 0$, then
  \[
    av=0 \iff \tfrac{1}{a}(av)=\tfrac{1}{a}0 \iff \left(\tfrac{1}{a} \right)v = 0 \iff v = 0.
  \]

  So either $a=0$ or $v=0$.
\end{prf}

%3
\begin{xrcs}
  Suppose $v,w \in V$. Explain why there exists a unique $x \in V$ such that $v + 3x = w$.

  Explaination: To solve this equation for $x$, we first need to subtract $v$ from $w$ and aftwewards devide by $3$. The additive inverse of $v$ is unique. But also the result of multyplying a vector by a number is unique.
\end{xrcs}

\begin{xrcs}
  The only requirement for a vector space the empty set fails to satisfy is the existence of a additive idenity.
\end{xrcs}

\begin{xrcs}
  Show that in the definition of a vector space, the additive inverse condition can be replace with the condition that
  \[
    0v = 0 \quad \forall v \in V.
  \]
\end{xrcs}
\begin{prf}
  By the distributive properties we get $\forall v \in V:$
  \[
  \begin{aligned}
    0 = 0v &= (1 + (-1))v
    &= 1v + (-1)v
    &= v + (-1)v
  \end{aligned}
  \]

  So $\forall v \in V: 0= v + (-1)v$ means that for every $v$ we have an additive inverse $(-1)v$.
\end{prf}
\phantom{.}
\hrule

