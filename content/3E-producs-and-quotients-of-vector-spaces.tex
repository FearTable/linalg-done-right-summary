\section{Products and Quotients of Vector Spaces}
\subsection{Products of Vector Spaces}

% 3.87
\setcounter{thm}{86}
\begin{mydef} [product of vector spaces]
  The product, addition and scalar multiplication of a list of vector spaces $V_1, \cdots, V_m$ is defined as follows: 
  \begin{equation}
    \begin{aligned}
      V_1 \times \cdots \times V_m &:\equiv \{ (v_1, \dots, v_m) \mid v_1 \in V_1, \dots, v_m \in V_m\} \\
      (u_1, \dots, u_m) + (v_1, \dots, v_m) &:\equiv (u_1+v_1, \dots, u_m+v_m) \\
      \lambda (v_1, \dots, v_m) &:\equiv (\lambda v_1, \dots, \lambda v_m)
    \end{aligned}
  \end{equation}
\end{mydef}

% 3.89
\setcounter{thm}{88}
\begin{thm} [product of vector space is a vector space]
  $V_1 \times \cdots \times V_m$ together with addition and scalar multiplication is a vector space over $\mathbb{F}$.
\end{thm}

% 3.93
\setcounter{thm}{91}
\begin{thm} [dimension of a product is the sum of dimensions]
  $\dim (V_1 \times \cdots \times V_m) = \dim V_1 + \cdots + \dim V_m$
\end{thm}

% 3.93
\begin{thm}
  Let $\Gamma: V_1 \times \cdots \times V_m \to V_1 + \cdots + V_m$ such that
  \begin{equation}
    \Gamma(v_1, \cdots, v_m) \mapsto v_1 + \cdots + v_m
  \end{equation}
  Then $v_1 + \cdots + v_m$ is a direct sum $\iff$ $\Gamma$ is injective.
\end{thm}
\begin{prf}
  By \ref{thm: injectivity iff null space equals zero-set}, $\Gamma$ is injective $\iff v1 = \cdots = v_m = 0$. This means, the only way to write $0$ as a sum $v_1 + \cdots + v_m$, where each $v_k \in V_k$, is by taking each $v_k=0$. Thus \ref{thm: condition for a direct sum} shows that $\Gamma$ is injective $\iff$ $V_1 + \cdots + V_m$ is a direct sum, as desired.
\end{prf}

\begin{thm}[a sum is a direct sum if and only if the dimensions add up]$V_1 + \cdots + V_1$ is a direct sum $\iff$
  \begin{equation}
    \dim (V_1+\cdots+V_m) = \dim V_1 + \cdots + \dim V_m
  \end{equation}
\end{thm}

\subsection{Quotient Spaces}

% 3.95
\setcounter{thm}{94}
\begin{mydef} [notation $v+U$]
  $v+U :\equiv \{v+u \mid u\in U\}$ for $v\in V$ and $U\subseteq V$.
\end{mydef}

% 3.97
\setcounter{thm}{96}
\begin{mydef} [translate]
  The set $v+U$ is said to be a ``translate'' of $U$, where $v\in V$ and $U \subseteq V$.
\end{mydef}

% 3.99
\setcounter{thm}{98}
\begin{mydef} [quotient space, $V/U$]
  Suppose $U$ is a subspace of $V$. Then the ``quotient space'' $V/U$ is the set of all translates of $U$. Thus
  $V/U :\equiv \{v+U \mid v\in V\}$.
\end{mydef}

\begin{example}
  If $U=\{ (x,2x)\in \mathbb{R}^2 \mid x\in \mathbb{R} \} \implies \mathbb{R}/U$ is the set of all lines with slope $2$.

  If $U$ is a plane in $\mathbb{R}^3$ $\implies$ $\mathbb{R}^3/U$ is the set of all planes parallel to $U$.
\end{example}

% 3.101
\setcounter{thm}{100}
\begin{thm} [two translates of a subspace are equal or disjoint]
  % v,w \in V is right!
  If $ U \subseteq V$ and $v,w\in V$ (not a typo!). Then
  \begin{equation}
    v-w \in U \iff v+U = w + U \iff (v+U)\cap (w+U) \neq \varnothing
  \end{equation}
\end{thm}

% 3.102
\begin{mydef} [addition and scalar multiplication on $V/U$]
  Suppose $U$ is a subspace of $V$. Then ``addition'' and ``scalar multiplication'' are defined on $V/U$ by
  \begin{equation}
    \begin{aligned}
      (v+U)+(w+U) & :\equiv (v+w) + U \\
      \lambda (v+U) & :\equiv (\lambda v) + U
    \end{aligned}
  \end{equation}
  $\forall v,w \in V$ and $\forall \lambda \in \myF$
\end{mydef}


% 3.103
\begin{thm} [quotient space is a vector space]
  $V/U$ is a vector space with additive identity $0+U$ which is equal to $U$ and the additive inverse $(-v)+U$.
\end{thm}

\begin{mydef} [quotient map, $\pi$]
  Suppose $U \subseteq V$. the ``quotient map'' $\pi: V \to V/U$ is the linear map defined by
  \begin{equation}
    \pi(v)=v+U \quad \forall v \in V.
  \end{equation}
\end{mydef}

\begin{thm}
  Suppose $\dim V < \infty$ and $U \subseteq V$.
  \begin{equation}
    \implies \dim V/U = \dim V - \dim U.
  \end{equation}
\end{thm}

\begin{mydef}[notation $\widetilde{T}$] 
  Suppose $T \in \linmap(V,W).$ Define $\widetilde{T}: V/(\mynull T) \to W$ by
  \begin{equation}
    \widetilde{T}(v+\mynull T) = Tv
  \end{equation}
\end{mydef}

\begin{thm}[null space and range of $\widetilde{T}$]
  Suppose $T\in \linmap(V,W)$ Then
  \begin{enumerate}
    \item $\widetilde{T} \circ \pi = T,$ where $\pi$ is the quotient map $V$ onto $V/(\mynull T)$;
    \item $\widetilde{T}$ is injective;
    \item $\myrange \widetilde{T} = \myrange T$;
    \item $V/(\mynull T)$ and $\myrange T$ are isomorphic vector spaces.
  \end{enumerate}
\end{thm}