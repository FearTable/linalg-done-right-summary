\section{Products and Quotients of Vector Spaces}
\subsection{Products of Vector Spaces}

% 3.87
\setcounter{thm}{86}
\begin{mydef} [product of vector spaces]
  The product, addition and scalar multiplication of a list of vector spaces $V_1, \cdots, V_m$ is defined as follows:
  \begin{equation}
    \begin{aligned}
      V_1 \times \cdots \times V_m &:\equiv \{ (v_1, \dots, v_m) \mid v_1 \in V_1, \dots, v_m \in V_m\} \\
      (u_1, \dots, u_m) + (v_1, \dots, v_m) &:\equiv (u_1+v_1, \dots, u_m+v_m) \\
      \lambda (v_1, \dots, v_m) &:\equiv (\lambda v_1, \dots, \lambda v_m)
    \end{aligned}
  \end{equation}
\end{mydef}

% 3.89
\setcounter{thm}{88}
\begin{thm} [product of vector space is a vector space]
  $V_1 \times \cdots \times V_m$ together with addition and scalar multiplication is a vector space over $\mathbb{F}$.
\end{thm}

% 3.93
\mce{92}
\begin{thm} [dimension of a product is the sum of dimensions]
  \label{thm: dimension of a product is the sum of dimensions}
  Suppose $V_1, \ddd, V_m$ are finite-dimensional vector spaces. Then $V_1 \times \cdots \times V_m$ is finite-dimensional and
  \[
    \dim (V_1 \times \cdots \times V_m) = \dim V_1 + \cdots + \dim V_m.
  \]
\end{thm}
\begin{prf}
  Choose a basis $v_{k_1}, \ddd, v_{k_{\dim V_k}}$ for each $V_k$. For each basis vector of each $V_k$, consider the element of $V_1 \times \cdots \times V_m$ that equals the basis vector in the $k$-th slot and $0$ in the other slots.
  \[
  \underbrace{
    \left (
    \begin{matrix}
      v_{1_1} \\ 0 \phantom{\vdots} \! \! \! \\ \vdots \\ \vdots \\ \vdots \\ 0 \phantom{\vdots} \! \! \!
    \end{matrix}
    \right )
    \; \mathlarger{,} \;
    \left (
    \begin{matrix}
      v_{1_2} \\ 0 \phantom{\vdots} \! \! \! \\ \vdots \\ \vdots \\ \vdots \\ 0 \phantom{\vdots} \! \! \!
    \end{matrix}
    \right )
    \mathlarger{,}
    \mathlarger{\ddd}
    \mathlarger{,}
    \left (
    \begin{matrix}
      v_{1_{\dim V_1}} \\ 0 \phantom{\vdots} \! \! \! \\ \vdots \\ \vdots \\ \vdots \\ 0 \phantom{\vdots} \! \! \!
    \end{matrix}
    \right )
  }_{\text{construced from basis vectors of $V_1$}}
  \;
  \mathlarger{,}
  \mathlarger{\ddd}
  \mathlarger{,}
  %%%%%%%%%%%%%%%%%%%%%%%%%%%%%%%%%%%%%%%%%%
  \underbrace{
    \left (
    \begin{matrix}
      0 \phantom{\vdots} \! \! \! \\ \vdots \\ v_{k_1} \\ 0 \phantom{\vdots} \! \! \! \phantom{\vdots} \! \! \!  \\ \vdots \\ 0 \phantom{\vdots} \! \! \!
    \end{matrix}
    \right )
    \; \mathlarger{,} \;
    \left (
    \begin{matrix}
      0 \phantom{\vdots} \! \! \! \\ \vdots \\ v_{k_{2}} \\ 0 \phantom{\vdots} \! \! \! \\ \vdots \\ 0 \phantom{\vdots} \! \! \!
    \end{matrix}
    \right )
    \mathlarger{,}
    \mathlarger{\ddd}
    \mathlarger{,}
    \left (
    \begin{matrix}
      0 \phantom{\vdots} \! \! \! \\ \vdots \\ v_{k_{\dim V_k}} \\ 0 \phantom{\vdots} \! \! \! \\ \vdots \\ 0 \phantom{\vdots} \! \! \!
    \end{matrix}
    \right )
  ´}_{\text{construced from basis vectors of $V_k$}}
  \mathlarger{,}
  \mathlarger{\ddd}
  \mathlarger{,}
  \underbrace{
      \left (
      \begin{matrix}
        0 \phantom{\vdots} \! \! \!  \\ \vdots \\ \vdots \\ \vdots \\ 0 \phantom{\vdots} \! \! \! \\ v_{m_1}
      \end{matrix}
      \right )
      \; \mathlarger{,} \;
      \left (
      \begin{matrix}
        0 \phantom{\vdots} \! \! \!  \\ \vdots \\ \vdots \\ \vdots \\ 0 \phantom{\vdots} \! \! \! \\ v_{m_2}
      \end{matrix}
      \right )
      \mathlarger{,}
      \mathlarger{\ddd}
      \mathlarger{,}
      \left (
      \begin{matrix}
        0 \phantom{\vdots} \! \! \!  \\ \vdots \\ \vdots \\ \vdots \\ 0 \phantom{\vdots} \! \! \! \\ v_{m_{\dim V_m}}
      \end{matrix}
      \right )
  }_{\text{construced from basis vectors of $V_m$}}
  \]

  The list of all such vector is linearly independent and spans $V_1 \times \cdots \times V_m$. Thus is is a basis of $V_1 \times \cdots \times V_m$ and the length of this basis is $(\dim V_1 + \cdots + \dim V_m)$.
\end{prf}

In the next result, the map $\Gamma$ is surjective by the definition of $V_1 + \cdots + V_m$ because of closedness. Thus the last word in the result below could be changed from \qt{injective} to \qt{invertible}.

% 3.93
\begin{thm}[products and direct sums]
  \label{thm: products and direct sums}
  Let $\Gamma: V_1 \times \cdots \times V_m \to V_1 + \cdots + V_m$ such that
  \begin{equation}
    \Gamma(v_1, \ddd, v_m) = v_1 + \cdots + v_m \quad \forall v_1 \in V_1, \ddd, v_m \in V_m.
  \end{equation}

  Then
  \begin{equation}
    V_1 + \cdots + V_m$ is a direct sum $\iff$ $\Gamma$ is injective/invertible.$
  \end{equation}

\end{thm}
\begin{prf}
  By \ref{thm: injectivity iff null space equals zero-set}, $\Gamma$ is injective $\iff v_1 = \cdots = v_m = 0$ [Stated correctly, $\Gamma$ is injective if and only if $\mynull \Gamma = \{ \vec 0 \}$]. This means, the only way to write $0$ as a sum $v_1 + \cdots + v_m$, where each $v_k \in V_k$, is by taking each $v_k=0$. Thus \ref{thm: condition for a direct sum} shows that $\Gamma$ is injective $\iff$ $V_1 + \cdots + V_m$ is a direct sum, as desired.
\end{prf}

\begin{thm}[a sum is a direct sum if and only if the dimensions add up]
  \label{thm: a sum is a direct sum if and only if the dimensions add up}
  Suppose $V$ is finite-dimensional and $V_1, \ddd,V_m$ are subspaces of $V$. Then
  \begin{equation}
    V_1 + \cdots + V_1$ is a direct sum $\iff
    \dim (V_1+\cdots+V_m) = \dim V_1 + \cdots + \dim V_m
  \end{equation}
\end{thm}

\begin{prf}
  The map $\Gamma: V_1 \times \cdots \times V_m \to V_1 + \cdots + V_m$ in \ref{thm: products and direct sums} is surjective. Thus by the rank-nullity theorem \ref{rank-nullity-theorem}, $\Gamma$ is injective $\iff$ $\dim \mynull \Gamma = 0$ and therefore
  \[
  \begin{aligned}
    &\dim (\underbrace{V_1 \times \cdots \times V_m}_{\text{domain}})=\underbrace{\dim \mynull \Gamma \phantom{V_m} \! \! \! \! \! \! \! \! \! \! \!}_{= \; 0} +  \dim(\underbrace{V_1 + \cdots + V_m}_{\text{codomain} \; \subseteq \; V
    }) \\
    &\dim (V_1 + \cdots + V_m) = \dim (V_1 \times \cdots \times V_m)
  \end{aligned}
  \]

  We can use \ref{thm: dimension of a product is the sum of dimensions} to conclude that
  \begin{equation}
    \begin{aligned}
      &\dim (V_1 \times \cdots \times V_m) = \dim V_1 + \cdots + \dim V_m \mytext{and therefore } \\
      &\dim (V_1 + \cdots + V_m) =  \dim V_1 + \cdots + \dim V_m
    \end{aligned}
  \end{equation}


  Since $\Gamma$ is injective, \ref{thm: products and direct sums} tells us that $V_1+\cdots+V_m$ is a direct sum.
  So we have have that
  \[\dim (V_1 + \cdots + V_m) =  \dim V_1 + \cdots + \dim V_m
  \]

  if and only if $V_1+\cdots+V_m$ is a direct sum.

  \prooffont{Special case for m=2:} For the special case where $m=2$ we can use \ref{thm: dimension of a subspace} where
  \[
    \dim (V_1 + V_2) = \dim V_1 + \dim V_2 - \dim (V_1 \cap V_2)
  \]

  and we get $\dim (V_1 \cap V_2) = 0$ $\iff$ $V_1 + V_2 = V_1 \oplus V_2$ by \ref{thm: dimension of a sum of subspaces}.
\end{prf}

\subsection{Quotient Spaces}

% 3.95
\setcounter{thm}{94}
\begin{mydef} [notation $v+U$]
  $v+U :\equiv \{v+u \mid u\in U\}$ for $v\in V$ and $U\subseteq V$.
\end{mydef}

% 3.97
\setcounter{thm}{96}
\begin{mydef} [translate]
  The set $v+U$ is said to be a \qt{translate} of $U$, where $v\in V$ and $U \subseteq V$.
\end{mydef}

% 3.99
\setcounter{thm}{98}
\begin{mydef} [quotient space, $V/U$]
  Suppose $U$ is a subspace of $V$. Then the \qt{quotient space} $V/U$ is the set of all translates of $U$. Thus
  \begin{equation}
    V/U :\equiv \{v+U \mid v\in V\}.
  \end{equation}
\end{mydef}

\begin{example}
  If $U=\{ (x,2x)\in \mathbb{R}^2 \mid x\in \mathbb{R} \} \implies \mathbb{R}/U$ is the set of all lines with slope $2$.

  If $U$ is a plane in $\mathbb{R}^3$ $\implies$ $\mathbb{R}^3/U$ is the set of all planes parallel to $U$.
\end{example}

% 3.101
\mce{101}
\begin{thm} [two translates of a subspace are equal or disjoint]
  \label{thm: two translates of a subspace are equal or disjoint}
  % v,w \in V is right!
  If $ U $ is a subspace of $V$ and $v,w\in V$ (not a typo!). Then
  \begin{equation}
    v-w \in U \iff v+U = w + U \iff (v+U)\cap (w+U) \neq \varnothing
  \end{equation}
\end{thm}
\begin{prf}
  \qt{Step 1:} First suppose $v-w\in U$. If $u \in U$, then
  \[
    v+u = (w - w + v + u) = w + \Big( \underbrace{(v-w)}_{\in \; U} + u\Big), $ thus $ v+u \in w+U.
  \]

  Similarly, because $v-w \in U$ we also have $w-v \in U$ and thus
  \[
    w+u = (v - v + w + u) = v+\Big( \underbrace{(w-v)}_{\in \; U}+u \Big), $ thus $ w+u \in v+U.
  \]

  So we have $\forall u \in U: v+u \in w+U$ which is equivalent to \[ v+U \subseteq w + U\]

  and $\forall u \in U: w+u \in v+U$ which is equivalent to \[ w+U \subseteq v+U.\]

  This implies $v+U = w+U$ given that $v-w \in U$, which proves the first equivalence.

  \qt{Step 2:} The equation $v+U = w+U$ implies that $(v+U) \cap (w+U) \neq \varnothing$.

  \qt{Step 3:} Now suppose $(v+U) \cap (w+U) \neq \varnothing$. Thus there exist $u_1, u_2 \in U$ such that
  \[
    v+u_1 = w+u_1, $ thus $ v-w = \underbrace{u_2-u_1}_{\in \; U}.
  \]
  Hence $v-w \in U$.
\end{prf}

% 3.102
\begin{mydef} [addition and scalar multiplication on $V/U$]
  Suppose $U$ is a subspace of $V$. Then \qt{addition} and \qt{scalar multiplication} are defined on $V/U$ by
  \begin{equation}
    \begin{aligned}
      (v+U)+(w+U) & :\equiv (v+w) + U \\
      \lambda (v+U) & :\equiv (\lambda v) + U
    \end{aligned}
  \end{equation}
  $\forall v,w \in V$ and $\forall \lambda \in \myF$.
\end{mydef}


% 3.103
\begin{thm} [quotient space is a vector space]
  $V/U$ is a vector space with additive identity $0+U$ which is equal to $U$ and the additive inverse $(-v)+U$.
\end{thm}

\begin{mydef} [quotient map, $\pi$]
  Suppose $U \subseteq V$. the \qt{quotient map} $\pi: V \to V/U$ is the linear map defined by
  \begin{equation}
    \pi(v)=v+U \quad \forall v \in V.
  \end{equation}
\end{mydef}

\begin{thm}
  Suppose $\dim V < \infty$ and $U \subseteq V$.
  \begin{equation}
    \implies \dim V/U = \dim V - \dim U.
  \end{equation}
\end{thm}

\begin{mydef}[notation $\widetilde{T}$]
  Suppose $T \in \linmap(V,W).$ Define $\widetilde{T}: V/(\mynull T) \to W$ by
  \begin{equation}
    \widetilde{T}(v+\mynull T) = Tv
  \end{equation}
\end{mydef}

\begin{thm}[null space and range of $\widetilde{T}$]
  Suppose $T\in \linmap(V,W)$ Then
  \begin{enumerate}[label=\textbf{(\alph*)}]
    \item $\widetilde{T} \circ \pi = T,$ where $\pi$ is the quotient map $V$ onto $V/(\mynull T)$;
    \item $\widetilde{T}$ is injective;
    \item $\myrange \widetilde{T} = \myrange T$;
    \item $V/(\mynull T)$ and $\myrange T$ are isomorphic vector spaces.
  \end{enumerate}
\end{thm}