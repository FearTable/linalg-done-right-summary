\section{Orthonormal Bases}

\subsection{Orthonormal Lists and the Gram-Schmidt Procedure}
\mce{22}
\begin{mydef}[orthonormal]
  A list of vectors is called \qt{orthonormal} if each vector in the list has norm $1$ and is orthogonal to all the other vector in the list. Since the square root of $1$ is $1$, we can use the the following definition as well:

  A list $e_1, \ddd, e_m$ of vectors in $V$ is \qt{orthonormal} if $\forall j,k \in \{1, \ddd, m\}$
  \begin{equation}
    \ip{e_j}{e_k} = \begin{cases}
      1,  & \text{if $j=k$} \\
      0, & \text{if $j \neq k$}
    \end{cases}
  \end{equation}

  Recall that $\norm{v} \equiv \sqrt{\ip{v}{v}}$.
\end{mydef}

\mce{24}
\begin{thm}[norm of an orthonormal linear combination]
  \label{thm: norm of an orthonormal linear combination}
  If $e_1, \ddd, e_m$ is an orthonormal list of vectors in $V$, Then
  \begin{equation}
    \norm{a_1e_1 + \cdots+ a_m e_m} ^2 = |a_1|^2 + \cdots + |a_m|^2 \quad \forall a_1, \ddd, a_m \in \myF.
  \end{equation}
\end{thm}
\begin{prf}
  Because $\forall k: \norm{e_k}=1$, this follows from repeated applications of the Pythagorean theorem \ref{thm: pythagorean theorem}
  \begin{equation}
    \begin{aligned}
      \norm{a_1e_1 + \cdots+ a_m e_m} ^2
      & = \norm{(a_1e_1 + \cdots + a_{m-1}e_{m-1}) + a_m e_m} ^2 \\
      & = \norm{(a_1e_1 + \cdots + a_{m-1}e_{m-1})}^2 + \norm{a_m e_m} ^2 \\
      & = \norm{(a_1e_1 + \cdots + a_{m-2}e_{m-2}) + a_{m-1}e_{m-1}}^2 + |a_m|^2 \\
      & = \norm{(a_1e_1 + \cdots + a_{m-2}e_{m-2})}^2 + \norm{a_{m-1}e_{m-1}}^2 + |a_m|^2\\
      & = \norm{(a_1e_1 + \cdots + a_{m-3}e_{m-3}) + a_{m-2}e_{m-2}}^2 + |a_{m-1}|^2 + |a_m|^2\\
      & = \; \cdots \\
      & = |a_1|^2 + |a_2|^2 + \cdots + |a_{m-1}|^2 + |a_m|^2
    \end{aligned}
  \end{equation}
  \vspace{-1em}
\end{prf}

\mce{25}
\begin{thm}[linear independence]
  \label{thm: orthonormal lists are linearly independent}
  Every orthornomal list of vectors is linearly independent.
\end{thm}
\begin{prf}
  Suppose $e_1, \ddd, e_m$ is an orthonormal list of vectors in $V$ and $a_1, \ddd, a_m \in \myF$ are \st\begin{equation}
    a_1e_1 + \cdots + a_m e_m = 0.
  \end{equation}

  $\implies |a_1|^2 + \cdots + |a_m|^2 = 0$ by \ref{thm: norm of an orthonormal linear combination}, which means that all the $a_k$'s are $0$. This is all that is needed to show.
\end{prf}

\mce{26}
\begin{thm}[Bessel's inequality]
  \label{thm: Bessel's inequality}
  Suppose $e_1, \ddd, e_m$ is an orthonormal list of vectors in $V$. If $v \in V$ then
  \begin{equation}
    |\ip{v}{e_1}|^2 + \cdots + |\ip{v}{e_m}|^2 \leq \norm{v}^2.
  \end{equation}
\end{thm}
\begin{prf}
  Suppose $v \in V$. Then
  \begin{equation}
    v= \underbrace{\ip{v}{e_1}e_1 + \cdots + \ip{v}{e_m} e_m}_u + \underbrace{v - \ip{v}{e_1}e_1 + \cdots + \ip{v}{e_m} e_m}_w
  \end{equation}

  Let $u$ and $w$ be defined as in the equation above. If $k \in \{1, \ddd, m\}$, then
  \begin{equation}
    \begin{aligned}
      \ip{w}{e_k}
      &=\ip{v - \underbrace{\ip{v}{e_1}}_{\in \; \myF}e_1 + \cdots + \underbrace{\ip{v}{e_m}}_{\in \; \myF} e_m}{ \,e_k \,} \\
      &=\ip{v}{e_k} - \ip{v}{e_1}\ip{e_1}{e_k} - \ip{v}{e_2}\ip{e_2}{e_k} - \, \cdots \, - \ip{v}{e_m}\ip{e_m}{e_k} \\
      &=\ip{v}{e_k} - \ip{v}{e_k}\ip{e_k}{e_k} \\
      &=0. \mytext{Hence} \\
      \ip{w}{u}
      & = \ip{w}{\underbrace{\ip{v}{e_1}}_{\in \; \myF}e_1 + \cdots + \underbrace{\ip{v}{e_m}}_{\in \; \myF} e_m} \\
      & = \ip{w}{\ip{v}{e_1}e_1} + \cdots + \ip{w}{\ip{v}{e_m}e_m} \\
      & = \overline{\ip{v}{e_1}}\ip{w}{e_1} + \cdots + \overline{\ip{v}{e_m}}\ip{w}{e_m} \\
      & = 0.
    \end{aligned}
  \end{equation}

  Since this means that $u$ and $w$ are orthogonal, we can use the well known Pythagorean theorem \ref{thm: pythagorean theorem} to conclude that
  \begin{equation}
    \begin{aligned}
      \norm{v}^2
      &=    \norm{u}^2 + \norm{w}^2 \\
      &\geq \norm{u}^2 \\
      &= |\ip{v}{e_1}|^2 + \cdots + |\ip{v}{e_m}|^2,
    \end{aligned}
  \end{equation}
  where the last line comes from \ref{thm: norm of an orthonormal linear combination}. If you look at the definition of $u=\ip{v}{e_1}e_1 + \cdots + \ip{v}{e_m} e_m$, the inner products are the coefficients in regards to the orthonormal basis $e_1, \ddd, e_m$.
\end{prf}

\begin{mydef}[orthonormal basis]
  An \qt{orthonormal} basis of $V$ is an orthonormal list of vectors in $V$ that is also a basis in $V$.
\end{mydef}

\begin{mydef}[orthonormal lists of the right length are orthonormal bases]
  Suppose $\dim V \neq \infty$.
\end{mydef}

\mce{30}
\begin{thm}[writing a vector as a linear combination of an orthonormal basis]
  \label{thm: writing a vector as a linear combination of an orthonormal basis}

  Supoose $e_1, \ddd, e_n$ is an orthonormal basis of $V$ and $u,v \in V$. Then
  \begin{enumerate}[label=\textbf{(\alph*)}]
    \item $v=\ip{v}{e_1}e_1 + \cdots + \ip{v}{e_n}e_n$;
    \item $\norm{v}^2 = |\ip{v}{e_1}|^2 + \cdots + |\ip{v}{e_m}|^2$
    \item $\ip{u}{v} = \ip{u}{e_1} \overline{\ip{v}{e_1}} + \cdots + \ip{u}{e_n} \overline{\ip{v}{e_n}}$
  \end{enumerate}
\end{thm}
\begin{prf}
  Let $v=a_1 e_1 + \cdots + a_n e_n$. Then $\forall k \in \{1, \ddd, n\}$ we have
  \begin{equation}
    \begin{aligned}
      \ip{v}{e_k}
      &=\ip{a_1 e_1 + \cdots + a_n e_n}{e_k} \\
      &=\ip{a_1 e_1}{e_k} + \cdots + \ip{a_n e_n}{e_k}\\
      &=\ip{a_ke_k}{e_k} \\
      &= a_k\ip{e_k}{e_k} \\
      &=a_k
    \end{aligned}
  \end{equation}
  Thus \textbf{(a)} holds.

  Now \textbf{(b)} follows immediately from \textbf{(a)} and \ref{thm: norm of an orthonormal linear combination}.

  Using conjugate linearity
  \begin{equation}
    \begin{aligned}
      \ip{u}{v}
      &=\ip{u}{\ip{v}{e_1}e_1 + \cdots + \ip{v}{e_n}e_n} \\
      &= \ip{u}{\ip{v}{e_1}e_1} + \cdots + \ip{u}{\ip{v}{e_n}e_n} \\
      &= \overline{\ip{v}{e_1}}\ip{u}{e_1} + \cdots + \overline{\ip{v}{e_n}}\ip{u}{e_n} \\
      &= \ip{u}{e_1}\overline{\ip{v}{e_1}} + \cdots + \ip{u}{e_n}\overline{\ip{v}{e_n}}
    \end{aligned}
  \end{equation}

  Therefore \textbf{(c)} holds.
\end{prf}

\subsection{Linear Functionals on Inner Product Spaces}