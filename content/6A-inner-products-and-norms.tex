\section{Inner Products and Norms}

\subsection{Inner Products}
We define the norm of $x=(x_1, \ldots, x_n) \in \real^n$ by
\begin{equation}
  \norm{x} = \sqrt{x_1^2 + \cdots + x_n^2}
\end{equation}
The norm is not lineaer. The dot product is.

\begin{mydef} [dot product]
  The \qt{dot product} of $x,y \in \real^n$ is defined by
  \begin{equation}
    x \cdot y :\equiv x_1 y_1 + \cdots + x_n y_n
  \end{equation}
\end{mydef}

The dot product on $\real^n$ has the following properties.
\begin{itemize}
  \item $x \cdot x = \norm{x}^2 \quad \forall x \in \real^n$
  \item $x \cdot x = 0 \iff x= 0$
  \item $y\in \real^n:$ The map from $\real^n$ to $\real$ that sends $x \in \real^n$ to $x\cdot y$ is linear. \\
  $T: \real^n \to \real, T(x) = x \cdot y$. Then $T$ is linear for a fixed $y \in \real^n$. %TODO: check.
  \item $x \cdot y = y \cdot x$.
\end{itemize}

For $z \in \compl$, we define the norm of $z$ by
\begin{equation}
  \begin{aligned}
    \norm{z} :\equiv \sqrt{ |z_1|^2 + \cdots + |z_n|^2} = \sqrt{ z_1 \overline {z_1} + \cdots z_n \overline {z_n}} \\
    \implies \norm{z}^2 = z_1 \overline{z_1} + \cdots + z_n \overline{z_n}
  \end{aligned}
\end{equation}

\begin{mydef} [inner product]
  The \qt{inner product} of $V$ is a function defined as follows:
  \begin{itemize}
    \item \bfemph{Positivity: } $
    {v,v} \geq 0 \quad \forall v \in V$
    \item \bfemph{Definiteness: } $\ip{v}{v} = 0 \iff v=0$
    \item \bfemph{Additifity in first slot: } $\ip{u+v}{w}=\ip{u}{w}+\ip{v}{w} \quad \forall u,v,w \in V$
    \item \bfemph{Conjugate symmetry:} $\ip{u}{v} = \overline{\ip{v}{u}} \quad \forall u, v \in V$
  \end{itemize}
\end{mydef}

\begin{example}
  (b) $\ip{w}{z} :\equiv c_1 w_1 \overline{z_1} + \cdots + c_n w_n \overline{z_n}$
\end{example}

\begin{mydef}[inner product space]
  An \qt{inner product space} is a vector space $V$ along with an inner product on $V$.
\end{mydef}

\setcounter{thm}{5}

\begin{thm}[basic properties of an inner product]
  $\forall u,v,w \in V$ and $\forall \lambda \in \myF$:
  \begin{enumerate}[label=(\alph*)]
    \item $V \ni x \mapsto \ip{u}{x} \in \myF$ is a linear map from $V$ to $\myF$.
    \item $\ip{0}{v} = 0 $
    \item $\ip{v}{0} = 0$
    \item $\ip{u}{v+w} = \ip{u}{v} + \ip{u}{w} $
    \item $\ip{u}{\lambda v} = \overline{\lambda} \ip{u}{v}$
  \end{enumerate}
\end{thm}

\subsection{Norms}

\begin{mydef} [norm]
  \begin{equation}
    \norm{v} :\equiv \sqrt{\ip{v}{v}}, \quad v \in  V
  \end{equation}
\end{mydef}

\begin{example}
  For $f \in C[-1,1]$ and with inner product given as in 6.3(c) we have $\norm{f} = \sqrt { \int_{-1}^{1} f^2}$

\end{example}

\begin{thm} [basic properties of the norm]
  Suppose $v \in V$
  \begin{enumerate}[label=(\alph*)]
    \item $\norm{v} = 0 \iff v = 0$
    \item $\norm{\lambda v} = |\lambda| \,\norm{v} \quad \lambda \in \myF$.
  \end{enumerate}
\end{thm}
\begin{prf}
  Let $\lambda \in \myF$ and $v \in V$. Then
  \begin{enumerate}[label=(\alph*)]
    \item $\ip{v}{v} = 0 \iff v=0$
    \item $\norm{\lambda v}^2 = \ip{\lambda v}{\lambda v} = \lambda \overline{\lambda} \ip{v}{v} = |\lambda|^2 \norm{v}^2$
  \end{enumerate}
  \vspace{-1em}
\end{prf}

\setcounter{thm}{9}
\begin{mydef}[orthogonal]
  $u,v \in V$ are called \qt{orthogonal} or \qt{perpendicular} if
  \begin{equation}
    \ip{u}{v} =0$. One also writes $u \perp v.
  \end{equation}
\end{mydef}

\begin{thm} [orthogonality and $0$]
  \phantom{.}
  \begin{enumerate}[label=(\alph*)]
    \item $0$ is orthogonal to every vector in $V$.
    \item $0$ is the only vector that is orthogonal to itself.
  \end{enumerate}
\end{thm}

\begin{thm}[Pythagorean theorem]
  \label{thm: pythagorean theorem}
  If $u,v \in V$ are orthogonal, then
  \begin{equation}
    \norm{u+v}^2 = \norm{u}^2 + \norm{v}^2.
  \end{equation}

  In this version of the theorem, $u$ and $v$ can equal to $0$.
\end{thm}
\begin{prf}
  $\norm{u+v}^2 = \ip{u+v}{u+v} = \ip{u}{u} + \underbrace{\ip{u}{v}}_{=0} + \underbrace{\ip{v}{u}}_{=0} + \ip{v}{v} = \norm{u}^2 + \norm{v}^2 $
\end{prf}

\setcounter{thm}{12}
\begin{thm}[an orthogonal decomposition]
  \label{thm: an orthogonal decomposition}
  Suppose $u,v \in V, v\neq 0.$ Set $c :\equiv \frac{\ip{u}{v}}{\norm{v}^2}$ and $w :\equiv u-\frac{\ip{u}{v}}{\norm{v}^2}v$. Then
  \begin{equation}
    u = cv+w = \frac{\ip{u}{v}}{\norm{v}^2} v + \left( u- \frac{\ip{u}{v}}{\norm{v}^2} v \right  ) \myand \ip{w}{v}=0
  \end{equation}
\end{thm}

\begin{thm}[Cauchy-Schwarz inequality]
  Suppose $u,v \in V$. Then
  \begin{equation}
    |\ip{u}{v}| \leq \norm{u} \, \norm{v}.
  \end{equation}
  Equality holds $\iff$ $u,v$ is a scalar multiple of the other. $u=\lambda v$
\end{thm}
\begin{proof}
  If $v=0$, both sides are $0$. Thus $v\neq0 $.

  Let $u :\equiv cv+w = \frac{\ip{u}{v}}{\norm{v}^2} v + w$ such that $v\perp w: \ip{w}{v} = 0$ like above (\ref{thm: an orthogonal decomposition}).

  By Pythagoras we have

  \begin{minipage}{\linewidth}
  \begin{equation}
    \begin{aligned}
      \norm{u}^2
      = \norm{ \frac{\ip{u}{v}}{\norm{v}^2} v}^2 + \norm{w}^2
      = \sqrt{ \ip{\frac{\ip{u}{v}}{\norm{v}^2}  v}{\frac{\ip{u}{v}}{\norm{v}^2} v}} + \norm{w}^2 \\
      = \sqrt{ \frac{1}{ \left(\norm{v}^2 \right  )^2}\ip{\ip{u}{v}v}{\ip{u}{v}v}} + \norm{w}^2
      = \frac{1}{\norm{v}^4} \ip{u}{v} \overline{\ip{u}{v}} + \norm{w}^2 \\
      = |\ip{u}{v}| \, \norm {v}^2 \frac{1}{\norm{v}^4} + \norm{w}^2
      = \frac{|\ip{u}{v}|}{\norm{v}^2} + \norm{w}^2 \\
      \geq \frac{|\ip{u}{v}|}{\norm{v}^2}
    \end{aligned}
  \end{equation}
  \end{minipage}

  Multiplying both sides with $\norm{v}^2$ yields the Cauchy-Schwarz inequality. Equality is only given, if $w=0$. So if $u=\lambda v$.
\end{proof}

\setcounter{thm}{16}
\begin{thm}[triangle inequality]
  $u,v \in V:$
  \begin{equation}
    \norm{u+v} \leq \norm{u} + \norm{v}
  \end{equation}
  This inequality is an equality $\iff$ $u=rv, r\in \real - \{0\}$.
\end{thm}
\begin{minipage}{\linewidth-30pt}
\begin{prf} Let $u,v \in V$.
    \begin{equation}
      \begin{aligned}
        \norm{u+v}^2
        &=    \ip{u+v}{u+v} & \\
        &=    \ip{u}{u+v}+\ip{v}{u+v} & \\
        &=    \ip{u}{u}+\ip{u}{v}+\ip{v}{u}+\ip{v}{v} & \\
        &=    \ip{u}{u}+\ip{v}{v}+\ip{u}{v}+\overline{\ip{u}{v}} & \text{conjugate symetrie}\\
        &=    \norm{u}^2+\norm{v}^2+2\operatorname{Re} \ip{u}{v} & \text{$z+\overline{z} = 2\operatorname{Re} z$}\\
        &\leq \norm{u}^2+\norm{v}^2+2|\ip{u}{v}| & \text{$z=\realpart z + (\impart z) i$} \\
        &\leq \norm{u}^2+\norm{v}^2+2\norm{u}\norm{v} & \text{Cauchy-Schwarz} \\
        &=    (\norm{u} + \norm{v})^2 &
      \end{aligned}
    \end{equation}
We have equality in the triangle inequality $\iff$ $\ip{u}{v} = \norm{u}\cdot\norm{v} \iff u = \lambda v$.
\end{prf}
\end{minipage}

\phantom{.}
