\section{Subspaces}

\begin{mydef} [subspace]
  A subset $U$ of $V$ ($U \subseteq V$) is called a \qt{subspace} of $V$ if $U$ is also a vector space with the same additive identity, addition, and scalar multiplication as on $V$.
\end{mydef}

\begin{thm} [conditions for a subspace]
  A subset $U$ of $V$ is a subspace of $V$ $\iff$ $U$ satisfies the following three conditions:
  \begin{itemize}
    \item \emph{Additive identity:}
    $0 \in U$
    \item \emph{Closed under addition:}
    $u,w \in U \implies u+w \in U$
    \item \emph{Closed under scalar multiplication:}
    $a \in \myF$ and $u \in U \implies au \in U$
  \end{itemize}
  The additive identity condition above could be replaced by the condition that $U$ is nonempty (because then taking $u \in U$ and multiplying it by $0$ would imply that $0\in U$).
\end{thm}
\begin{prf} We need to sho2 the proposition in two directions.
  \begin{description}
    \item{\qt{$\Rightarrow$ direction:}} If $U$ is a subspace of $V$, then $U$ satisfies the three conditions above by the definition of a vector space.
    \item{\qt{$\Leftarrow$ direction:}} The first condition ensures, that the additive identiy $0$ of $U$ is in $U$. The other two conditions make sure, that addition and scalar multiplication make sense on $U$.

    Thanks to \ref{thm: minus one times a vector} and the third condition, every $u \in U$ has an additive inverse $-u = (-1) u$.

    The other parts of the definition of a vector space, such associativity and commutativity, are automatically satisfied for $U$ because they hold on the larger space $V$. Thus $U$ is a vector space and hence a subspace of $V$.
  \end{description}
  \vspace{-1.1em}
\end{prf}

%\textbf{1.34} $U\subseteq V$ is a subspace of $V$ $\iff$
%\begin{itemize}
%	\item additive identity: $0 \in U$
%	\item closed under addition: $u,w\in U \implies u+w \in U$
%	\item closed under scalar multiplication: $a \in \myF$ and $u\in U$ $\implies$ $au \in U$
%\end{itemize}

\subsection{Sum of Subspaces}

% 1.36
\setcounter{thm}{35}
\begin{mydef}[sum of subspaces]
  If $V_1, \ldots, V_m$ are subspaces of $V$, the \qt{sum} of $V_1, \ldots, V_m$, denoted by $V_1 + \cdots + V_m$, is the set of all possible sums of elements of $V_1, \ldots, V_m$. More precisely,
\begin{equation}
    V_1 + \cdots + V_m :\equiv \{v_1 + \cdots + v_m \mid v_1 \in V_1, \, \dots \, , v_m \in V_m \}.
\end{equation}
\end{mydef}

%\textbf{1.40}
\setcounter{thm}{39}
\begin{thm} [sum of subspaces is the smallest containing subspace]
  \label{thm: sum of subspaces is the smallest containing subspace}
  $V_1 + \cdots + V_m$ is the smallest subspace of $V$ containing $V_1, \dots, V_m$.
\end{thm}
\begin{prf}
  First we need to show, that $V_1 + \cdots + V_m$ is a subspace of $V$. Clearly, the additive identity $0 = 0 + \cdots + 0 \in V_1 + \cdots + V_m \in V$.
  Let $u,w \in V_1 + \cdots + V_m$.  Then we have $u = v_1 + \cdots + v_m$ \st $v_1 \in V_1, \dots, v_m \in V_m$ and $w=v'_1 + \cdots + v'_m$ \st $v'_1, \dots, v'_m \in V_m$. Therefore
  \begin{equation}
    u+w = \underbrace{(v_1+v'_1)}_{\in V_1} + \cdots + \underbrace{(v_m+v'_m)}_{\in V_m} \in  V_1 + \cdots + V_m \in V.
  \end{equation}

  So $V_1 + \cdots + V_m$ is closed under addition. For $\lambda \in \myF$ we have that
  \begin{equation}
      \lambda u = \lambda(v_1 \cdots v_m) = \underbrace{\lambda v_1}_{\in V_1} + \cdots + \underbrace{\lambda v_m}_{\in V_m} \in V_1 + \cdots + V_m \in V,
  \end{equation}

  so it is closed under scalar multiplication. So we have shown that $V_1 + \cdots + V_m$ is a subspace of $V$. Now we have to show the minimality.
  The subspaces $V_1, \cdots, V_m$ are all contained in $V_1+\cdots+V_m$. Because $\forall \kInOneTillM:$
  \begin{equation}
    V_k = \{ v \mid v \in V_k \} \subseteq \{v_1 + \cdots + v_m \mid v_1 \in V_1,  \ldots , v_m \in V_m \} = V_1 + \cdots + V_m.
  \end{equation}

  Conversely, every subspace of $V$ containing $V_1, \ldots, V_m$ contains $V_1 + \cdots + V_m$, because its closed under addition. Thus $V_1 + \cdots + V_m$ is the smallest subspace of $V$ containing $V_1, \ldots, V_m$.
\end{prf}

\subsection{Direct Sums}

%\textbf{1.41}
\setcounter{thm}{40}
\begin{mydef}[direct sum, $\oplus$]
  \label{def: direct sum}
  %NO COMMA BEFORE IF ACC TO AXLER
  $V_1 + \cdots + V_m$ is called a \qt{direct sum} if each element of $V_1 +\cdots+V_m$ can be written in only one way as a sum $v_1 + \cdots + v_m$, where each $v_k \in V_k$. In this case we write:
  \begin{equation}
    V_1 \oplus \cdots \oplus V_m :\equiv V_1 + \cdots + V_m
  \end{equation}
\end{mydef}

\begin{example}
  Example: $\myF^3 =
  \left \{ \left ( x, y, 0 \right ) \in \myF^3 \mid x,y \in \myF \right \}
  \oplus
  \left \{ \left (  0, 0, z  \right ) \in \myF^3 \mid z \in \myF \right \}$
\end{example}


%\textbf{1.45}
\setcounter{thm}{44}
\begin{thm} [condition for a direct sum]
  \label{thm: condition for a direct sum}
  $V_1 + \cdots + V_m$ is a direct sum $\iff$ the only write $0$ as a sum $v_1 + \cdots + v_m$, where each $v_k \in V_k$, is by taking each $v_k$ equal to $0$.
\end{thm}
\begin{prf}
    Let  $V_1, \ldots, V_m$ be subspaces of $V$

    \qt{$\Rightarrow$ direction:} First suppose $V_1 + \cdots + V_m$ is a direct sum. Let $0 = v_1 + \cdots + v_m,$ where each $v_k \in V_k$.
    $\implies$ all the $v_k$'s are equal to $0$, because otherwise multiplying the equation above with a scalar $\lambda \in \myF$ would give a different sum (a different way to add them up).
    \begin{equation}
      0 = \lambda0= \lambda v_1 + \cdots + \lambda v_m, \whereEach \lambda v_k \in V_k
    \end{equation}

    Or a different way to argue, $0 = 0 + \cdots + 0$ is already a unique solution.

    \qt{$\Leftarrow$ direction:} Now suppose that the only way to write $0$ as a sum $v_1 + \cdots + v_m$ where each $v_k \in V_k$, is by taking each $v_k = 0$. Let
    \begin{equation}
      \begin{aligned}
        v&=v_1 + \cdots + v_m \where v_1 \in V_1, \ldots, v_m \in V_m \myand \\
        v&=u_1 + \cdots + u_m  \where u_1 \in V_1, \ldots, u_m \in V_m
      \end{aligned}
    \end{equation}

    Subtracting these two equations, we have
    \begin{equation}
      0=(v_1-u_1)+\cdots+(v_m-u_m).
    \end{equation}

    Because $v_1 - u_1 \in V_1, \ldots, v_m - u_m \in V_m$, the equation above implies that each $v_k -u_k$ equals $0$.
    \begin{equation}
      \implies v_1=u_1, \ldots, v_m=u_m
    \end{equation}

    Thus $u=v$, because $0$ has a unique representation.
\end{prf}

%\textbf{1.46}
\setcounter{thm}{45}
\begin{thm} [direct sum of two subspaces]
  \label{thm: intersection of direct sum of two subscpaces}
  $U+W$ is a direct sum $\iff$ $U \cap W = \{0\}$
\end{thm}
\begin{prf} Let $U$ and $W$ be subspaces of the same vector space.
  \begin{description}
    \item{\qt{$\Rightarrow$ direction:}} First suppose $U+W$ is a direct sum. If $v \in U \cap W$, then $v, -v \in U \myand v, -v \in W$. We also have that
    \begin{equation}
      0 = v + (-v), \where v\in U \myand (-v) \in W.
    \end{equation}
    $\implies v=0$, because $U+W$  is a direct sum. Therefore $U \cap W = \{0\}$

    \item{\qt{$\Leftarrow$ direction:}} Suppose $U\cap W = \{0\}$. Let  $u\in U$ and $w \in W$ such that
    \begin{equation}
      \label{eq: zero equals u plus w}
      0 = u+w
    \end{equation}
    By \ref{thm: condition for a direct sum} we know that our goal is to show that $u=w=0$. The equation \eqref{eq: zero equals u plus w} above implies that $u=-w\in W$  and $w=-u \in U$ for all arbitrary choices of $U$ and $W$. Thus $u,w \in U\cap W \implies u=w=0.$ Therefore, $U+W$ is a direct sum.
  \end{description}
  \vspace{-1em}
\end{prf}
