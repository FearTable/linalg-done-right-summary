\section{Bases}


\setcounter{thm}{25}
%\textbf{2.26}
\begin{mydef}
  A basis of $V$ is a list of vectors $\onetilln{v}$ in $V$ that is linearly independent and spans $V$.
  \begin{equation}
    V= \myspan{v_1, \dots, v_n}
  \end{equation}
\end{mydef}


\setcounter{thm}{27}
\begin{thm}
  $\onetilln{v}$ is a basis of $V$ $\iff$ every $v \in V$ can be written uniquely in the form
  \begin{equation}
    v=a_1 v_1 + \dots + a_n v_n, \quad \text{where} \; a_1, \dots, a_n \in \myF
  \end{equation}
  Note that the definition of linear dependence says nothing about uniqueness. That is just the motivation behind the definition!
\end{thm}

\setcounter{thm}{29}
\begin{thm} Every spanning list of a \vs can be reduced to a basis of the \vs.
\end{thm}

\setcounter{thm}{30}
\begin{thm} Every \findimvs has a basis.\end{thm}

\begin{thm} Every \lid list of vectors in a  \findimvs can be extended to a basis of the \vs. \end{thm}

\begin{thm} Suppose $V$ is \fd and $U$ is a subspace of $V$. Then there is a subspace $W$ of $V$ such that $V=U \oplus W$. \end{thm}
