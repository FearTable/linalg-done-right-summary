\section{Bases}


\setcounter{thm}{25}
%\textbf{2.26}
\begin{mydef} [Basis]
  A basis $\beta$ of $V$ is a list of vectors $\beta = \onetilln{v}$ in $V$ that is linearly independent and spans $V$.
  \begin{equation}
    V = \myspan{v_1, \dots, v_n} = \myspan{ \beta }
  \end{equation}
\end{mydef}

\setcounter{thm}{27}
\begin{thm} [Condition for a basis]
  $\onetilln{v}$ is a basis of $V$ $\iff$ every $v \in V$ can be written uniquely in the form
  \begin{equation}
    v=a_1 v_1 + \dots + a_n v_n $, where $ a_1, \dots, a_n \in \myF
  \end{equation}
  Note that the definition of linear dependence says nothing about uniqueness. That is just the motivation behind the definition!
\end{thm}

\setcounter{thm}{29}
\begin{thm} [every spanning list contains a basis] 
  \label{thm: every spanning list contains a basis}
  Every spanning list of a \vs can be reduced to a basis of the \vs.
\end{thm}
\begin{prf}
  Suppose $V=\myspan{v_1, \ldots, v_n}.$ Let $B_0 :\equiv (v_1, \ldots, v_n)$.
  
  \emph{\bfseries Step 1: } If $v_1 = 0$, delete $v_1$ from $B_0$: 
  \begin{equation}
    $Let $B_1 :\equiv (v_2, \ldots, v_n).
  \end{equation} 
  
  If $v_1 \neq 0$, then leave $B_0$ unchanged. $B_1 :\equiv B_0=(v_1, \ldots, v_n)$.
  
  \emph{\bfseries Step k: } If $v_k$ $\in$ $\myspan{v_1, \ldots, v_{k-1}}$, then delete $v_k$ from list $B_{k-1}$ sucht that:
  \begin{equation}
    $Let $B_k :\equiv B_{k-1} - v_k = (v_{1}^*, \ldots, v_{k-1}^*, v_{k+1}, \ldots, v_n)
  \end{equation} 
  
  where $v_1^*, \ldots, v_{k-1}^*$ are not necessarily the original elements $v_1, \ldots, v_{k-1}$.
  
  If $v_k \in \myspan{v_1, \ldots, v_{k-1}}$, leave the basis unchanged: $B_k :\equiv B_{k-1}$.
  
  Stop the process after step $n$, getting a list $B_n$. The list $B_n$ spans $V$ because our original list $B_0$ spanned $V$  have discarded only vectors that were already in the span of the previous vectors. The process ensures that no vector in $B$ is in the span of the previous ones. Thus $B$ is linearly independent. Hence $B$ is a basis of $V$.
  %TODO: Why does axler cite linear dependence lemma 2.19.
\end{prf}

\setcounter{thm}{30}
\begin{thm} [basis of finite\-/ dimensional vector space] 
  Every \findimvs has a basis.
\end{thm}
\begin{prf}
  By definition, a \fdvs has a spanning list. The previous result tells us that each spanning list can be reduced to a basis. 
\end{prf}

\begin{thm} 
  Every \lid list of vectors in a  \findimvs can be extended to a basis of the \vs. 
\end{thm}
\begin{prf}
  Let $u_1, \ldots, u_m \in V$ be linearly independent. Let $V=\myspan{w_1, \cdots, w_n}.$
  \begin{equation}
    \implies V=\myspan{u_1, \cdots, u_m, w_1, \cdots, w_n }
  \end{equation}
  Applying the procedure of the proof of \ref{thm: every spanning list contains a basis} to reduce this list to a basis of $V$ produces a basis consisting of vectors $u_1, \cdots, u_m$ and some $w$'s.
\end{prf}

\begin{thm} 
  If $V$ is \fd and $U$ is a subspace of $V$. Then there is a subspace $W$ of $V$ such that 
  \begin{equation}
    V=U \oplus W. 
  \end{equation}
\end{thm}
