\section{Null Spaces and Ranges}

\subsection{Null Space and Injectivity}

\begin{mydef} [null space, $\mynull T$]
  $\mynull T :\equiv \ker T :\equiv \{ v \in V \mid Tv=0\} \subseteq V$ for $T \in \lvw$
\end{mydef}

% 3.13
\setcounter{thm}{12}
\begin{thm} [null space is a subspace]
  \label{thm: null space is a subspace}
  For $T=\linmap(V, W)$, $\mynull T$ is a subspace of $V$
\end{thm}
\begin{prf}
  $0 \in \mynull T$, because $T(0) = 0$. Suppose $u,v \in \mynull T$ and $\lambda \in \myF.$ $\implies T(u+v)=Tu+Tv=0+0=0.$ Hence $u+v \in \mynull T$. We also have $T(\lambda u)= \lambda Tu = \lambda 0 = 0$, so $\lambda u \in \mynull T$.
\end{prf}

% 3.14
\setcounter{thm}{13}
\begin{mydef} [injectivity]
  \label{def: injectivity}
  A function $T: V \to W$ is called \qt{injective} if
  \begin{equation}
    Tu = Tv \implies u = v \quad \forall u,v \in V
  \end{equation}
  Or analogous:
  \begin{equation}
    u \neq v \implies Tu \neq Tv \quad \forall u,v \in V
  \end{equation}
\end{mydef}

% 3.15
\setcounter{thm}{14}
\begin{thm} [injectivity and null space property]
  \label{thm: injectivity iff null space equals zero-set}
  Let $T \in \linmap(V,W)$. Then
  \begin{equation}
    T \text{ is injective } \iff \mynull T = \{0\}.
  \end{equation}

\end{thm}
\begin{prf}
  \Rightarrowdirection First, assume that $T$ is injective.
  Since $\mynull T$  is a subspace of $V$ (by \ref{thm: null space is a subspace}), we have $\{0\} \subseteq \mynull T$. To prove that $\mynull T \subseteq \{0\}$, suppose $v \in \mynull T$. Consequently,
  \begin{equation}
    T(v) = 0.
  \end{equation}

  Moreover, since linear maps send $0$ to $0$ (by \ref{thm: linear maps take 0 to 0}), we have
  \begin{equation}
    T(0)=0.
  \end{equation}

  Since $T$ is injective, both equations imply that $v=0$. Thus, we can conclude that $\mynull T = \{0\}$, as desired.

  \Leftarrowdirection Second, assume $\mynull T = \{0\}$. Let $u,v \in V$ \st $Tu = Tv$. Then
  \begin{equation}
    0 = Tu - Tv = T(u-v).
  \end{equation}

  Thus, $u-v\in\mynull T$, so $u-v=0$, which gives $u=v$.  Therefore, $T$ is injective.
  %Thus, $u-v \in \mynull T = \{0\}$. Hence, $u-v=0$, which implies that $u=v$. Hence, $Tu = Tv$ implies that $u=v$, so $T$ is injective.
\end{prf}

\subsection{Range and Surjectivity}

% 3.16
\setcounter{thm}{15}
\begin{mydef} [range]
  $\operatorname{range}T :\equiv \{Tv \mid v \in V\} \subseteq W$ for $T \in \lvw$
\end{mydef}

% 3.18
\setcounter{thm}{17}
\begin{thm} [the range is a subspace]
  \label{thm: the range is a subspace}
  $T\in \linmap(V,W) \implies \myrange T$ is a subspace of $W$.
\end{thm}
\begin{prf}
  Let $T \in \linmap (V,W)$. Then $T(0)=0$ by \ref{thm: linear maps take 0 to 0}, which implies that $\boxed{0 \in \myrange T}\,$. So the additive identity $0$ is part of $\myrange T$.

  If $\boxed{w_1, w_2 \in \myrange T}\,$, then $\exists v_1, v_2\in V$ \st $Tv_1 = w_1$ and $Tv_2 =w_2$. Thus
  \[
    T(v_1 + v_2)=Tv_1 + Tv_2 = \boxed{w_1 + w_2 \in \myrange T}.
  \]

  Hence $\myrange T$ is closed under addition.

  If $\boxed{w \in \myrange T}$ and $\lambda \in \myF$, then $\exists v \in V$ \st $Tv = w$. Thus
  \[
    T(\lambda v) = \lambda Tv = \boxed{\lambda w \in \myrange T}.
  \]
  Hence $\myrange T$ is closed under scalar multiplication. Thus $\myrange T$ is a subspace of $W$ by \ref{thm: conditions for a subspace}.
\end{prf}

%3.19
\setcounter{thm}{18}
\begin{mydef} [surjectivity]
  \label{def: surjectivity}
  If $\myrange T=W$, $T:V\to W$ is called \qt{surjektive} or \qt{onto}.
\end{mydef}

\subsection{Fundamental Theorem of Linear Maps}

% Thm 3.21
\setcounter{thm}{20}
\begin{thm} [{\slshape \scshape Fundamental Theorem of Linear Maps or ``Rank Nullity Theorem''}]
  \label{rank-nullity-theorem}
  For $T \in \linmap(V,W)$, where $V$ is \fd, we have that $\myrange(T)$ is also \fd and
  \begin{equation}
    \dim V =
    \underbrace{ \dim\mynull T\phantom{g}\!\!\!\!\!}_{\text{nullity}}
    \; + \; \underbrace{\dim \myrange T}_{\text{rank}}.
  \end{equation}

  Note that $T \in \linmap(V,W)$ is not a typo.
\end{thm}
\begin{prf}
  Let $u_1, \ddd, u_m$ be a basis of $\mynull T$ with $\dim \mynull T = m$. Using \ref{thm: every linearly independent list of vectors in a finite-dimensional vector space can be extended to a basis of the vector space}, this basis which forms a linearly independent list inside $V$, can be extendend to a basis of $V$ which looks like this
  \[
    u_1, \ddd, u_m, v_1, \ddd, v_n \mytext{(we have been using the fact that $V$ is finite-dimensional.)}
  \]

  Thus $\dim V = m+n$. To complete the proof, we need to show that $\myrange T$ is finite-dimensional and $\dim \myrange T = n$. We will do this by proving that $Tv_1, \ddd, Tv_n$ is a basis of $\myrange T$.

  Let $v \in V$ sucht that for $a_1, \ddd, a_m \in \myF$ and $b_1, \ddd, b_n \in \myF$:
  \[
    v= (a_1 u_1 + \cdots + a_m  u_m) + (b_1 v_1 + \cdots + b_n v_n).
  \]

  If we apply $T$ we get $Tv=b_1Tv_1+ \cdots + b_nTv_n$. Since we can do this for every $v$, this implies that
  \[
    \myrange T = \myspan{Tv_1, \ddd, Tv_n} $ and that $\myrange T$ is finite dimensional.$
  \]

   To show what $Tv_1, \ddd, Tv_n$ is linearly independent, suppose $c_1, \ddd, c_n \in \myF$ such that
  \[
    c_1 Tv_1 + \cdots + c_n T v_n = 0.
  \]

  Then $T(c_1 v_1 + \cdots + c_n v_n) = 0.$ Hence $c_1 v_1 + \cdots + c_n v_n \in \mynull T$. Because $u_1, \ddd, u_m$ spans $\mynull T$, we can write
  \[
    c_1 v_1 + \cdots + c_n v_n = d_1 u_1 + \cdots +  d_m u_m \mytext{or}
  \]
  \[
  0 = c_1 v_1 + \cdots + c_n v_n - d_1 u_1 - \cdots - d_m u_m,
  \]

  where the $d$'s are in $\myF$. This last equation implies that all the $c$'s and $d$'s are $0$ because the $u$'s and $v$'s form a basis of $V$ and are therefore linearly-independent. Thus $Tv_1, \ddd, Tv_n$ is linearly independent (because all $c$'s are $0$) and hence is a basis of $\myrange T$, as desired.
\end{prf}


  %3.22
\setcounter{thm}{21}
\begin{thm} [linear map to a lower-dimensional space is not injective]
  \label{thm: linear-map-to-a-lower-dimensional-space-is-not-injective}
  Suppose $V$ and $W$ are finite-dimensional vector spaces such that $\dim V$ $>$ $\dim W$.
  Then no linear map from $V$ to $W$ is injective.
\end{thm}
\begin{prf} Let $T\in \linmap(V,W)$ such that $\dim V > \dim W$. Using \ref{rank-nullity-theorem}, we have
  \begin{align}
    \dim \mynull T &= \dim V - \dim \myrange T \\
                   &\geq \dim V - \dim W  \\
                   &> 0.
  \end{align}

  The first inequality follows from \ref{thm: dimension of a subspace}, because $\myrange T$ is a subspace of $W$. The second inequality holds because we assumed $\dim V > \dim W$. Now, the equation tells us that $\dim \mynull T > 0$, implying $\mynull T$ is not equal to $\{0\}$ (meaning that $\mynull T$ contains vectors other than $0$). Thus, $T$ is not injective by \ref{thm: injectivity iff null space equals zero-set}.
\end{prf}

% 3.24
\mce{24}
\begin{thm} [linear map to a higher-dimensional space is not surjective]
  \label{thm: linear map to a higher-dimensional space is not surjective}
  Suppose $V$ and $W$ are finite-dimensional vector spaces such that $\dim V < \dim W$. Then no linear map from $V$ to $W$ is surjective.
\end{thm}
\begin{prf}
  Let $T\in \linmap (V,W)$ such that $\dim V < \dim W$. Then
  \begin{align}
    \dim \myrange T & = \dim V - \dim \mynull T \\
                    & \leq \dim V  \\
                    & < \dim W.
  \end{align}

  Thus, we have $\dim \myrange T < \dim W$. So $\myrange T \neq W$. Hence, $T$ is not surjective by \ref{def: surjectivity}.
\end{prf}

%3.26
\mce{26}
\begin{thm} [homogenenous system of linear equations]
  A homogeneous system of linear equations with more variables then equations has nonzero solutions.
\end{thm}
\begin{prf} \qt{Homogeneous}, in this context, means, that the constant term on the right side of each equation below is $0$. Fix positive integers $m$ and $n$, let $A_{j,k} \in \myF$ for $j = 1, \ddd, m$ and $k = 1, \ddd, n$. Consider,
\[
\begin{aligned}
  A_{1,1}x_1 + A_{1,2}x_2 + &\cdots + A_{1,n}x_n = \sum_{k=1}^n A_{1,k}\,x_k = 0,\\
  A_{2,1}x_1 + A_{2,2}x_2 + &\cdots + A_{2,n}x_n = \sum_{k=1}^n A_{2,k}\,x_k = 0,\\
  &\; \; \vdots & \\
  A_{m,1}x_1 + A_{m,2}x_2 + &\cdots + A_{m,n}x_n = \sum_{k=1}^n A_{m,k}\,x_k = 0.
\end{aligned}
\]

Clearly, $x_1 = \cdots = x_n = 0$ is a solution. Define $T: \myF^n \to \myF^m$ by
\begin{equation}
  T(x_1, \ddd, x_n) := \left(\sum_{k=1}^n A_{1,k}\,x_k,\sum_{k=1}^n A_{2,k}\,x_k, \ddd, \sum_{k=1}^n A_{m,k}\,x_k \right).
\end{equation}

Thus, we want to know if $\mynull T$ is strictly bigger than $\{0\}$, which is equivalent to $T$ not beeing injective (by \ref{thm: injectivity iff null space equals zero-set}). By \ref{thm: linear-map-to-a-lower-dimensional-space-is-not-injective}, this is the case if $\dim \myF^n > \dim \myF^m$, so if $n>m$. This is the case if we have more variables than equations.
\end{prf}

%\textbf{3.28}
\mce{28}
\begin{thm} [system of lienar equations with more equations than variables]
  An inhomogeneous system of linear equations with more equations then variables has no solutions for some choice of constant terms.
\end{thm}

\begin{prf} We want to now whether a system of linear equations has no solutions for come choice of constant terms on the right hand side. Fix positive integers $m$ and $n$, let $A_{j,k} \in \myF$ for $j = 1, \ddd, m$ and $k = 1, \ddd, n$. For $c_1, \ddd, c_m \in \myF$, consider,
  \begin{equation}
    \label{eq: system of equations}
    \begin{aligned}
      A_{1,1}x_1 + A_{1,2}x_2 + &\cdots + A_{1,n}x_n = \sum_{k=1}^n A_{1,k}\,x_k = c_1,\\
      %A_{2,1}x_1 + A_{2,2}x_2 + &\cdots + A_{2,n}x_n = \sum_{k=1}^n A_{2,k}\,x_k = c_2,\\
      &\; \; \vdots & \\
      A_{m,1}x_1 + A_{m,2}x_2 + &\cdots + A_{m,n}x_n = \sum_{k=1}^n A_{m,k}\,x_k = c_m.
    \end{aligned}
  \end{equation}

  Define $T: \myF^n \to \myF^m$ by
  \begin{equation}
    T(x_1, \ddd, x_n) := \left(\sum_{k=1}^n A_{1,k}\,x_k,\sum_{k=1}^n A_{2,k}\,x_k, \ddd, \sum_{k=1}^n A_{m,k}\,x_k \right).
  \end{equation}

  Therefore, we can rewrite \eqref{eq: system of equations} as $T(x_1, \ddd, x_n) = (c_1, \ddd, c_m)$. Hence, we want to know if $\myrange T = \myF^m$ or $\myrange T \neq \myF^m$. What condition ensures that $T$ is not surjective? By \ref{thm: linear map to a higher-dimensional space is not surjective}, this is the case if $\dim \myF^n < \dim \myF^m$ or $n<m$. This means if we have more equations than variables in a system of linear equations, then there is no solutions for some constant terms.
\end{prf}
