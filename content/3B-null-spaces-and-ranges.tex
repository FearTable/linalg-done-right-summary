\section{Null Spaces and Ranges}

\subsection{Null Space and Injectivity}

\begin{mydef} [null space, null $T$]
  $\mynull T :\equiv \ker T :\equiv \{ v \in V \mid Tv=0\} \subseteq V$ for $T \in \lvw$
\end{mydef}

% 3.13
\setcounter{thm}{12}
\begin{thm} [null space]
  For $T=\linmap(V, W)$, $\mynull T$ is a subspace of $V$ 
\end{thm}
\begin{prf}
  $0 \in \mynull T$ because $T(0) = 0$. Suppose $u,v \in \null T$ and $\lambda \in \myF.$ $\implies T(u+v)=Tu+Tv=0+0=0.$ Hence $u+v \in \mynull T$. We also have $T(\lambda u)= \lambda Tu = \lambda 0 = 0.$
\end{prf}

% 3.14
\setcounter{thm}{13}
\begin{mydef} [injective]
  A function $T: V \to W$ is called injective if
  \begin{equation}
    Tu = Tv \implies u = v \quad \forall u,v \in V
  \end{equation}
  Or analogous:
  \begin{equation}
    u \neq v \implies Tu \neq Tv \quad \forall u,v \in V
  \end{equation}
\end{mydef}

% 3.15
\setcounter{thm}{14}
\begin{thm} [injectivity and null space]
  \label{thm: injectivity iff null space equals zero-set}
  Let $T \in \linmap(V,W)$. Then
  \begin{equation}
    T \text{ is injective } \iff \mynull T = \{0 \}.
  \end{equation}
\end{thm}

\subsection{Range and Surjectivity}

% 3.16
\setcounter{thm}{15}
\begin{mydef} [range]
  $\operatorname{range}T= \{Tv \mid v \in V\} \subseteq W$ for $T \in \lvw$
\end{mydef}

% 3.18
\setcounter{thm}{17}
\begin{thm} [range]
  \label{thm: the range is a subspace}
  $T\in \linmap(V,W) \implies \myrange T$ is a subspace of $W$.
\end{thm}

%3.19
\setcounter{thm}{18}
\begin{mydef} [surjectivity]
  \label{def: surjectivity}
  If $\myrange T=W$, $T:V\to W$ is called ``surjektive" or ``onto".
\end{mydef}

\subsection{Fundamental Theorem of Linear Maps}

  % 3.21
  \setcounter{thm}{20}
  \begin{thm} [fundamental theorem of linear maps or ``rank nullity theorem"]
    \label{rank-nullity-theorem}
    For $T \in \lvw$, where $T$ is finite-dimensional, we have
    \begin{equation}
      \dim V =
      \underbrace{ \dim \mynull T }_{\text{nullity}}
      + \underbrace{\dim \myrange T}_{\text{rank}}
    \end{equation}
  \end{thm}

  %3.22
  \setcounter{thm}{21}
  \begin{thm} [linear map to a lower-dimensional space is not injective]
    \label{thm: linear-map-to-a-lower-dimensional-space-is-not-injective}
    If $\dim V$ $>$ $\dim W$ \\ 
    $\implies$ No \lm from $V$ to $W$ is injective.
  \end{thm}
  \begin{prf} Let $T\in \linmap(V,W)$ such that $\dim V > \dim W$. Using \ref{rank-nullity-theorem} we have
    \begin{equation}
      \dim \mynull T = \dim V - \dim \myrange T \geq \dim V - \dim W > 0,
    \end{equation}
    The first inequality follows from \ref{dimension-of-a-subspace}, because $\myrange T \subseteq W$. Now we have that $\dim \mynull T > 0.$ Thus $T$ is not injective by \autoref{def: injectivity}.
  \end{prf}

  % 3.24
  \setcounter{thm}{23}
  \begin{thm} [linear map to a higher-dimensional space is not surjective]
    If $\dim V < \dim W$ $\implies$ No \lm from $V$ to $W$ is surjective.
  \end{thm}
  \begin{prf}
    Let $T\in \linmap (V,W)$ such that $\dim V < \dim W$. Then
    \begin{equation}
      \dim \myrange T = \dim V - \dim \mynull T \leq \dim < \dim W
    \end{equation}
    Thus we have $\dim \myrange T < \dim W$. So $\myrange T \neq W$. Thus $T$ is not surjective by \autoref{def:surjectivity}.
  \end{prf}
   
  %3.26
  \setcounter{thm}{25}
  \begin{thm} [homogenenous system of linear equations]
    A homogeneous system of linear equations with more variables then equations has nonzero solutions.
  \end{thm}

  %\textbf{3.28}
  \setcounter{thm}{27}
  \begin{thm} [system of lienar equations with more equations than variables]
    An inhomogeneous system of linear equations with more equations then variables has no solutions for some choice of constant terms.
  \end{thm}
