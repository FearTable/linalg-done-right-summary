\section{Span and Linear Independence}
\subsection{Linear Combinations and Span}

\setcounter{thm}{1}
\begin{mydef} [Linear combination]
  $a_1 v_1 + \cdots + a_m v_m$ is called a ``linear combination" of a list of vectors $v_1, \dots, v_m$
\end{mydef}

\setcounter{thm}{3}
\begin{mydef} [Span of a list of vectors]
	The ``span" of a list of vectors $\onetillm{v}$ is defined as follows:
  \begin{equation}
    \begin{aligned}
      \myspan {v_1, \dots, v_m} &:\equiv \{a_1 v_1 + \cdots + a_m v_m \mid a_1, \dots a_m \in \mathbb{F} \}\\
      \myspan{ } &:\equiv \varnothing
    \end{aligned}
  \end{equation}
\end{mydef}

\setcounter{thm}{5}
%\textbf{2.6}
\begin{thm} [The span of a list of vectors is a subspace]
  The  span of a list of vectors in $V$ is the smallest subspace of $V$ containing all vectors in the list.
\end{thm}

%\textbf{2.7}
\begin{mydef} [Spanning a vector space]
  If $\myspan{\oneTillM{v}} = V$, we say $\oneTillM{v}$ ``spans" $V$.
\end{mydef}

\setcounter{thm}{8}
%\textbf{2.9}
\begin{mydef} [Finite-dimensional vector space]
  Such a vector space is called finite\-/dimensional. We write $\dim V \neq \infty$ or $\dim V < \infty$
\end{mydef}

%\textbf{2.10, 2.11, 2.12}
\begin{mydef}
  The set of all polynomials is denoted by $\polyn (\myF) \subseteq \myF^\myF$.
\end{mydef}
\begin{mydef}
  $\polyn_m (\myF )=\myspan{1,z,\dots,z^m}$ denotes the set of all polynomials with coefficients in $\myF$ and degree at most $m$. The degree $m$ of a polynomial $p=a_0+a_1z+a_2z^2+\cdots+a_mz^m$ is denoted with $\degree p=m$.
  We also define for the zero polynomial $\degree 0 : \equiv -\infty$ \\
  (One should actually use the lambda notation $\lambda z.z^m$ to speak of functions)
\end{mydef}

\setcounter{thm}{12}
\begin{mydef}
  A vector space is called ``infinite-dimensional" if it is not finite-dimensional.
\end{mydef}

\pagebreak

\subsection{Linear Independence}

\setcounter{thm}{14}
\begin{mydef} [Linear independence]
  There are $2$ ways to think about ``linear independence". Lets first start with the motivation. A list of vectors $\oneTillM{v}$ is ``\lid", if every combination of these vectors $w = b_1v_1 + \dots+ b_mv_m$ $\in \myspan{\oneTillM{v}}$ has a unique choice of scalars $b_1, \dots, b_m \in \myF$.

  \bfemph{The actual definition of linear independence:} \\
  A list of vectors $\oneTillM{v}$ is also called \lid, if the only way to combine them together and yield zero, is if we choose every coefficient $\lambda_i$ to be zero.
  \begin{equation}
    \lambda_1v_1 + \dots + \lambda_mv_m = 0 \iff \lambda_1 = \cdots = \lambda_m = 0
  \end{equation}

  Another way to put it:\\
  Let $w\in \myspan{v_1, \dots, v_m}$ sucht that for $a_1, \dots, a_m, c_1, \dots, c_m \in \myF$:
  \begin{equation}
    \begin{aligned}
      w & =a_1 v_1 + \cdots + a_m v_m \myand \\
      w & =c_1 v_1 + \cdots + c_m v_m \\
      & \iff \\
      0 & = \underbrace{(a_1 - c_1)}_{= \, 0\text{?}} v_1 + \cdots + \underbrace{(a_m -c_m)}_{= \, 0\text{?}} v_m
    \end{aligned}
  \end{equation}

  if $a_1 = c_1, a_2 = c_2, \, \dots \, , a_m = c_m$ is the only solution, the representation of $w$ is unique and the only way to add them up together equaling $0$ is $0v_1+\cdots+0v_m=0$. Both of these definitions are equivalent.

  The empty list $()$ is also defined to be linearly independent.
\end{mydef}




%\textbf{2.16}
\begin{mydef} [Linear dependence]
  Otherwise, $\oneTillM{v}$ is called ``linearly dependent".
\end{mydef}

%	\textbf{2.19} Linear dependence lemma:\\
%	Suppose $\oneTillM{v}\in V$ is a linearly dependent list.
%	$\implies \exists \kInOneTillM$ : $v_k \in \myspan{\oneTill{v}{k-1}}$


%\textbf{2.29}

\setcounter{thm}{18}
%\textbf{2.19}
\begin{thm} [Linear dependence lemma]
  \label{linear-dependence-lemma}
  Suppose $v_{1}, \dots, v_{m}\in V$ is a linearly dependent list.
  \begin{equation}
    \implies \exists k \in \{ 1, \dots, m \} : v_k \in \myspan {v_1, \dots, v_{k-1}}.
  \end{equation}
  Furthermore, if $k$ satisfies the conditions above and the $k^{\text{th}}$-term is removed from $v_1, \dots, v_m$, then the span of the remaining list equals $\myspan {v_1, \dots, v_{m}}$:
  \begin{equation}
    \myspan {v_1, \dots, v_{k-1}, v_{k+1}, \dots, v_m} = \myspan {v_1, \dots, v_m}
  \end{equation}
  TODO: mistake
  %TODO : mistake
\end{thm}

\setcounter{thm}{21}
%\textbf{2.22}

\begin{thm}  [Length of linearly independent list $\mathbb{\leq}$ length of spanning list]
  \label{length-of-linearly-dependent-list-less-or-equal-length-of-spanning-list}
  In a finite dimensional vector space, the length of every linearly independent list of vectors is less then or equal to the length of every spanning list of vectors.
\end{thm}

\setcounter{thm}{24}
\begin{thm} [Finite-dimensional subspace]
  Every subspace of a \findimvs is finite\-/dimensional
\end{thm}
