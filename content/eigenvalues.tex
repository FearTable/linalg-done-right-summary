\chapter{Eigenvalues and Eigenvectors}
\section{Invariant Subspaces}
\subsection{Eigenvalues}

\begin{mydef}
    A \lm from a \vs to itself is called an ``operator". 
\end{mydef}

(Suppose $T\in \linmap(V)$, then may be $\left.T\right|_{V_{k}}$ is not an operator on a subspace $V_k$)

\begin{mydef}
    Let $T\in \linmap(V).$ $U \subseteq V$ is called ``invariant under $T$" if $\forall u \in U: Tu \in U.$ \\
    Thus $U$ is invariant under $T$ if $\left.T\right|_{U}$ is an operator on $U.$
\end{mydef}

\begin{example}
    Let $T\in \linmap(\mathcal{P}(\mathbb{R}))$ such that $Tp=p'.$ Let $U=\mathcal{P}_4(\mathbb{R})) \subseteq \mathcal{P}(\mathbb{R}).$ Then $U$ is invariant under $T$
    because if $p \in U$, $\deg p = 4$ and $\deg (p')=3$.
\end{example}

\begin{example}
    Let $T\in \linmap(V)$. Then $\{0\}, V, \operatorname{null} T, \operatorname{range} T$ are all invariant. \\
    (Sometimes, $\operatorname{null} T = \{0\}$ and $\operatorname{range} T=V$ if $T$ is invertible.)
\end{example}

\bfemph{Invariant subspaces of dimension one:} \\
Take any $v\in V, v\neq 0$ and let $U :\equiv \{  \lambda v \mid \lambda \in \myF \} = \myspan{v}$, then $U$ is a one-dimensional subspace of $V$. \\
If $U$ is invariant under an operator $T \in \linmap (V)$, then $Tv  \in U$. $\implies \exists \lambda \in \myF: Tv = \lambda v$. \\
Conversely if $Tv = \lambda v$, $\lambda \in \myF$, then $\myspan{v}$ is a one-dimensional subspace of $V$ invariant under $T$. 

\begin{mydef}
    Suppose $T\in \linmap (V)$. $\lambda \in \myF$ is called ``eigenvalue of $T$" if there exists $v \in V$ such that $v \neq 0$ and $Tv = \lambda v$
\end{mydef}

\setcounter{thm}{6}
\begin{thm}
    The following are equivalent:
    \begin{enumerate}[label=(\alph*)]
        \item $\lambda$ is an eigenvalue of $T$.
        \item $T-\lambda I$ is not injective.
        \item $T-\lambda I$ is not surjective.
        \item $T-\lambda I$ is not invertible.
    \end{enumerate}
\end{thm}

\setcounter{thm}{7}
\begin{thm}
    Let $T\in \linmap(V).$ A vector $v \in V$ is called ``an eigenvector" of $T$ corresponding to $\lambda$ if $v\neq 0$ and $Tv = \lambda v$.
    In other words: 
    \\A vector $v\in V, v \neq 0$ is an eigenvector corresponding to $\lambda \iff v \in \operatorname{null}(T-\lambda I_V)$
\end{thm}

\setcounter{thm}{10}
\begin{thm}
    Every list of eigenvectors of $T$ corresponding to distinct eigenvalues of $T$ is linearly independent.
\end{thm}
\begin{proof}
    Suppose the desired result is false. Then there exists a smallest list of length $m$ of linearly dependent eigenvectors $v_1, \dots, v_m$ with eigenvalues $\lambda_1, \dots, \lambda_m$ of $T$. Since an eigenvector is unequal to the zero vector, $m$ must be $\geq 2$.
    
    Because of the minimality of $m$ and becaue our list is linearly dependent: $\exists a_1, \dots, a_m \neq 0$ such that $a_1 v_1 + \cdots + a_m v_m = 0$. Now we apply $T-\lambda_m I$ on both sides of the equation and get  
    $a_1 \lambda_1 v_1 - a_1 \lambda_m v_1 + \cdots +
    a_{m-1} \lambda_{m-1} v_{m-1} - a_{m-1} \lambda_{m} v_{m-1} +
    \underbrace{a_m \lambda_m v_m -a_m \lambda_m v_m}_{=0}=0$
   
    From there it follows that:
   $a_1 \underbrace{(\lambda_1 - \lambda_m)}_{\neq 0} v_1 + \cdots + a_{m-1} \underbrace{(\lambda_{m-1}-\lambda_{m})}_{\neq 0} v_{m-1}=0$
   
   Which contradicts the minimality of $m$. Therefore, no such linearly dependent list of eigenvectors can exist.
\end{proof}

\begin{thm}
    Each operator on $V$ has at most $\dim V$ distinct eigenvalues.
    content
\end{thm}

\paragraph{}

\subsection{Polynomials applied to operators}

\setcounter{thm}{12}
\begin{mydef}
    Let $T \in \linmap(V)$ and $m\in \nat^{+}$ 
    \begin{itemize}
        \item $T^{m} \in \linmap(V), T^{m} = \underbrace{T \cdots T}_{\text{$m$ times}}$
        \item $T^0 :\equiv I_V$
        \item If $T$ is invertible with inverse $T^{-1}$ then $T^{-m}\in \linmap(V)$ is defined by $T^{-m} :\equiv (T^{-1})^m$ 
    \end{itemize}
\end{mydef}
$\implies T^m T^n = T^{m+n}$ and $(T^m)^n=T^{mn}$ when $m,n \in \mathbb{Z}$ when $T$ is invertible. And $m,n \in \mathbb{N}$ if $T$ is not invertible.

\begin{mydef}
    For $p \in \mathcal{P} (\myF)$, $p(z) = a_0+a_1z+a_2z^2+\cdots+a_mz^m$ 
    $\forall z \in \compl$\\
    For $T \in \linmap (V)$ we define: \\
    $p(T) :\equiv a_0 I + a_1 T + a_2 T^2 + \cdots a_m T^m,$ $p(T) \in \linmap(V)$
\end{mydef}

%TODO: example 5.15??

%TODO: example 5.16??

\setcounter{thm}{16}
\begin{thm}
    \label{multiplicative-properties}
    Suppose $p,q \in \mathcal{P} (\myF)$ and $T\in \linmap (V)$. Then $(p q)(T) = p(T) q(T) = q(T)p(T)$.
\end{thm}

\begin{thm}
    \label{null-space-and-range-of-p(T)-are-invariant-under-T}
    $T \in \linmap(V)$ and $p\in \mathcal{P} (\myF) \implies$
    $\operatorname{null} p(T)$ and $\operatorname{range} p(T)$ are invariant under $T$.
\end{thm}
\begin{proof}
    Suppose $u\in \operatorname{null} p(T) \implies p(T)u = 0$. Assoziativiy and distributivity of linear maps imply hat $(p(T))(Tu)=T(p(T)u)=T(0)=0$.$\implies Tu \in \mynull p(T)$.
    
    Suppose $u \in \myrange p(T) \implies \exists v\in V: u=p(T)v \implies Tu=T(p(T)v)=p(T)(Tv)$ \\ 
    $\implies Tu \in \myrange p(T)$.
\end{proof}

\subsection{The minimal polynomial}

\begin{thm}
    Every operator on a finite-dimensional nonzero complex vector space has an eigenvalue.
\end{thm}
\begin{proof}
    Suppose $\dim(V)=n>0$ and $T\in \linmap(V).$ Choose $v\in V, v\neq0$. Then $v, Tv, T^2v, \dots, T^nv$ is not linearly independent, because the list has length $n+1$. $\implies$ some linear combination of these vectors equals to $0$. $\implies$ there exists a non-constant polynomial $p$ of smallest degree such that $p(T)v = 0$. By the first version of the fundamental theorem of algebra (\ref{fundamental-theorem-of-algebra-first-version}) $\implies \exists \lambda \in \compl: p(\lambda) = 0.$\\
    (\ref{factororingOutZerosOfAPolynomial})$\implies \exists q \in \mathcal {P} (\compl): p(z) = (z-\lambda)  q(z) \; \forall z \in \compl$ \\
    (\ref{multiplicative-properties})$\implies 0=p(T)v=(T-\lambda I) (q(T)v)$. Because $q$ has a smaller degree than $p$, $q(T)v \neq 0$. \\
    $\implies$ $\lambda$ is an eigenvalue of $T$ with eigenvector $q(T)v$. 
\end{proof}

\subsection{Eigenvalues and the minimal polynomial}
\begin{mydef}
    A monic polynomial is a polynomial whose highest-degree coefficient equals $1$. 
\end{mydef}
\begin{example}
    $p(z)=2+9z^2+z^7, \deg p = 7$
\end{example}

\begin{thm}
    \label{unique-monic-polynomial-of-smallest-degree}
    Suppose $T\in \linmap(V)$. Then there exists a unique monic polynomial $p\in \mathcal{P}(\myF)$ of smallest degree such that $p(T)=0$. Furthermore $\deg p \leq \dim V$
\end{thm}
\begin{proof}
    If $\dim V=0$, $I$, is the zero-operator on $V$ and we let $p=1$ such that $1I\vec0=0$. \\
    Now we use induction on $\dim V$ and we assume $\dim V > 0$. Let $v\in V, v \neq 0$. The list $v, Tv, \dots, T^{\dim V}$ has length $1+\dim V.$
    $\implies$ linear dependence.
    
    By the linear dependence lemma (\ref{linear-dependence-lemma}), there is a smallest positive integer $m\leq \dim V$ such that $c_0 v + c_1 Tv + \cdots + c_{m-1} T^{m-1} v + T^m v = 0$ for some $c_0, c_1, \dots, c_{m-1} \in \myF$. \\
    
    Let $q(z) :\equiv c_0 + c_1z + \cdots + c_{m-1} z^{m-1} +z^{m} \in \mathcal{P} (\myF)$ \\
    $\implies q(T) v=0$. Not that $q(z)$ is a monic polynomial.
    
    If $k \in \nat$, then $q(T)(T^kv)=T^k(q(T)v) =T^k (0) =0$.\\
    By the linear dependence lemma (\ref{linear-dependence-lemma}) $\implies v, Tv, \dots, T^{m-1}v$ from before are linearly independent $\implies \dim \mynull q(T) \geq m$ \\
    $\implies \dim \myrange q(T) = \dim V - \dim \mynull q(T) \leq \dim V - m$.
    
    Because $\myrange q(T)$ is invariant under $T$ (by \ref{null-space-and-range-of-p(T)-are-invariant-under-T}), we can apply our induction hypothesis to the operator $\left.T\right|_{\myrange q(T)}$. \\
    So there exists monic $s \in \mathcal{P} (\myF): \deg s \leq \dim V - m$ and $s(\left.T\right|_{\myrange q(T)})=0$ \\
    $\implies \forall v \in V: (sq)(T)(v) = s(T) (q(T)v) = 0$, because $q(T)v \in \myrange q(T)$ and $\left.s(T)\right|_{\myrange q(T)}=s\left( \left.T\right|_{\myrange q(T)} \right )$.
    Therefore, $sq$ is a monic polynomial such that $\deg sq \leq \dim V$ and $(sq)(T)=0$.
    
    Proof of uniqueness: Let $p\in \mathcal{P} (\myF)$ a monic polynomial of smallest degree such that $p(T)=0$. Let $r\in \mathcal{P} ( \myF)$ another monic polynomial of same degree such that $p(T)=0$. \\
    $\implies (p-r) (T) = 0$(*) and also $\deg (p-r) < \deg p = \deg r$ \\
    If $p-r \neq 0$, we could devide $p-r$ by the coefficient of the highest order term in $p-r$ to get a monic polynomial that when applied to $T$ gives the $0$-operator(*). This polynomial would have a smaler degree than $p$ or $r$, which would be a contradiction. Therefore $p-r=0 \iff p = r$
\end{proof}

\setcounter{thm}{23}

\begin{mydef}
    Let $T\in \linmap (v)$. The ``minimal polynomial of $T$" is the unique monic polynomial $p\in \mathcal{P}(\myF)$ of smallest degree s.t. $p(T)=0$
\end{mydef}
\bfemph{Computation:} Find the smallest $m \in \nat$ such that: \\
$c_0I + c_1 T + \cdots + c_{m-1} T^{m-1} = -T^{m}$ has a solution $c_0, \dots, c_{m-1} \in \myF$. Solve for $m=1,2,\dots,\dim V$

Even faster (usually), pick $v \in V$ with $v \neq 0$ and consider the equation $c_0v + c_1Tv + \cdots + C_{\dim V-1}T^{\dim V-1}v=-T^{\dim V}v$. 
If this equation has a unique solution, as happens most of the time $c_0, c_1, c_2, \dots, c_{\dim V-1}, 1$ are the coefficients of the minimal polynomial of $T$.
%TODO: do more.

\setcounter{thm}{26}
\begin{thm}
    Let $T \in \linmap(V)$. Then 
    \begin{enumerate}[label=(\alph*)]
        \item The zeros of the minimal polynomial of $T$ are the eigenvalues of $T$.
        \item If $V$ is a complex vector space, the minimal polynomial has the form $(z-\lambda_1)\cdots(z-\lambda_m)$, where $\lambda_1, \dots, \lambda_m$ are the eigenvalues of $T$, possibly with repetitions. 
    \end{enumerate}
\end{thm}
\begin{proof} Let $p$ be the minimal polynomial of $T$.
    \begin{enumerate}[label=(\alph*)]
        \item Suppose $\lambda \in \myF$ is a zero of $p$. $\implies p(z)=(z-\lambda)q(z)$, whre $q$ is a monic polynomial with coefficients in $\myF$ (see \ref{factororingOutZerosOfAPolynomial}) \\
        $p(T)=0\implies 0=(T-\lambda I)(q(T)v) \; \forall v\in V$
        Because $q$ is of lesser degree than $p$, there exists at least one vector $v\in V$ sucht that $q(T)v \neq 0$, which makes $q(T)v$ an eigenvector with eigenvalue $\lambda$.
        
        Suppose $\lambda \in \myF$ is an eigenvalue of $T$. Thus there exists $v\in V, v \neq 0$ such that $Tv=\lambda v$. Repeated applications of $T$ on both sides of this equation show that $T^kv =\lambda^k v \; \forall k\in \nat$. 
        $\implies p(T)v=p(\lambda)v$. Because $p$ is the minimal polynomial of $T$, we have $p(T)v=0$. $\implies p(\lambda) = 0$. $\implies$ $\lambda$ is a zero of $p$.
        
        \item use (a) and the second version of the fundamental theorem of algebra. (\ref{fundamental-theorem-of-algebra-second-version})
     \end{enumerate}
\end{proof}